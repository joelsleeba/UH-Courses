% initial settings
\documentclass[12pt]{exam}
\usepackage{geometry}
\usepackage{graphicx}
\usepackage{enumitem}
\usepackage[usenames,dvipsnames]{xcolor}
\usepackage[backend=biber, style=alphabetic]{biblatex}
\usepackage{url,hyperref}

\usepackage{amsmath} % math symbols, matrices, cases, trig functions,
% var-greek symbols.
\usepackage{amsfonts} % mathbb, mathfrak, large sum and product symbols.
\usepackage{amssymb} % extended list of math symbols from AMS.
% https://ctan.math.washington.edu/tex-archive/fonts/amsfonts/doc/amssymb.pdf
\usepackage{amsthm} % theorem styling.
\usepackage{mathrsfs} % mathscr fonts.
\usepackage{yhmath} % widehat.
\usepackage{empheq} % emphasize equations, extending 'amsmath' and 'mathtools'.
\usepackage{bm} % simplified bold math. Do \bm{math-equations-here}

% geometry of paper
\geometry{
  a4paper, % 'a4paper', 'c5paper', 'letterpaper', 'legalpaper'
  asymmetric, % don't swap margins in left and right pages. as
  % opposed to 'twoside'
  centering, % to center the content between margins
  bindingoffset=0cm,
}

% hyprlink settings
\hypersetup{
  colorlinks = true,
  linkcolor = {red!60!black},
  anchorcolor = red,
  citecolor = {green!50!black},
  urlcolor = magenta,
}

% theorem styles
\theoremstyle{plain} % default; italic text, extra space above and below
\newtheorem{theorem}{Theorem}[section]
\newtheorem{proposition}{Proposition}[section]
\newtheorem{lemma}{Lemma}[section]
\newtheorem{corollary}{Corollary}[theorem]

\theoremstyle{definition} % upright text, extra space above and below
\newtheorem{definition}{Definition}[section]
\newtheorem{example}{Example}[section]

\theoremstyle{remark} % upright text, no extra space above or below
\newtheorem{remark}{Remark}[section]
\newtheorem*{note}{Note} %'Notes' in italics and without counter

% renewcommands for counters
\newcommand{\propositionautorefname}{Proposition}
\newcommand{\definitionautorefname}{Definition}
\newcommand{\lemmaautorefname}{Lemma}
\newcommand{\remarkautorefname}{Remark}
\newcommand{\exampleautorefname}{Example}

\addbibresource{~/Books/Research/research.bib}

\begin{document}

\title{MATH 6321 - Theory of functions of one real variable \\ Homework  IV}

% author list
\author{
  Joel Sleeba \\
}

\maketitle
\printanswers
\unframedsolutions

\begin{questions}
  \question
  \begin{solution}
    Let $e_0, e_1$ be the usual linearly independent unit norm vectors in
    $\ell^{1}(\mathbb{N})$. Let $ \mathcal{M} = \textrm{span}\{ e_0 \}$, and
    $\mathcal{N} = \textrm{span}\{ e_0, e_1 \}$. Let $ \lambda \in
    \mathbb{C}\setminus \{ 0 \}$ and define a linear functional
    \begin{align*}
      \phi :  \mathcal{M} \to  \mathbb{C} := t e_0 \mapsto t \lambda,
      \quad t \in \mathbb{C}
    \end{align*}
    Then by the definition of the operator norm, we see that
    $\|\phi\| = |\lambda|$
    Now, let $\phi_1, \phi_2: \mathcal{N} \to \mathbb{C}$ defined as
    \begin{align*}
      \phi_1(e_0) = \lambda = \phi_2(e_0) \\
      \phi_1(e_1) = -\frac{\lambda}{2}, \ \phi_2(e_1) = \frac{\lambda}{2}
    \end{align*}
    and then linearly extending to $\mathcal{N}$. We see that $\phi_1,
    \phi_2$ extend $\phi$. And since
    \begin{align*}
      |\phi_i(a e_0 + be_1)| &= |a \phi_i(e_0) + b \phi_i(e_1)| \\
      &= | a  \lambda  + (-1)^i\frac{\lambda}{2}b| \\
      &= |\lambda| |a + (-1)^i \frac{b}{2}| \\
      &\le |\lambda| \Big(|a| + \frac{|b|}{2} \Big) \\
      &\le |\lambda| (|a| + |b|) \\
      &= |\lambda| \|(a, b)\|_1
    \end{align*}
    we see that $\|\phi_i\| = |\lambda|$. Now by Hahn-Banach
    extension theorem, we see that both $\phi_1, \phi_2$ extends to
    linear functionals on $\ell^{1}(\mathbb{N})$. By an abuse of
    notation, call them $ \phi_1, \phi_2$. Then we see that $\phi_1,
    \phi_2$ are extensions of $\phi$, which preserve norm, but that
    the extension is not unique since $\phi_1(e_1) \neq \phi_2(e_1)$.
  \end{solution}

  \question
  \begin{solution}
    Let $(x_n)$ be as sequence in $X$.
    Assume that $(\|x_n\|) < M$. Since $\|x_n\|
    = \|i_{x_n}\|$, for any $f \in X^*$,
    \begin{align*}
      \|f(x_n)\| = \|i_{x_n}(f)\| \le \|i_{x_n}\|\|f\| =
      \|x_n\|\|f\| < M \|f\|
    \end{align*}
    shows that $\|f(x_n)\|$ is a bounded sequence. Since $f \in X^*$
    was arbitrary, this holds true for all $f \in X^*$.

    Conversely let $\sup_{n \in \mathbb{N}}\|i_{x_n}(f)\| = \sup_{n
    \in \mathbb{N}}\|f(x_n)\| < \infty$ for all
    $f \in X^*$. Then by a corollary to Banach-Steinhaus
    theorem, we see that
    \begin{align*}
      \sup_{  n \in \mathbb{N}} \|x_n\| = \sup_{n \in \mathbb{N}}
      \|i_{x_n}\| < N
    \end{align*}
    for some $N \ge 0$.
  \end{solution}

  \question
  \begin{solution}
    Let $\Lambda \in
    \textbf{c}_0^{*}$. We claim that the sequence $(y_n) =
    (\Lambda(e_n)) \in \ell^1$. Let
    $\theta_j \in [0, 2\pi)$ such that $e^{ i \theta_j}y_j =
    |y_j|$. Then for any $N \in \mathbb{N}$, we have
    \begin{align*}
      \sum_{j = 1}^{N} |y_n| = \sum_{j = 1}^{N} |\Lambda(e_j)| &=
      \sum_{j = 1}^{N} e^{i \theta_j} \Lambda(e_j) \\
      &= \Lambda\Big(\sum_{j = 1}^{N} e^{i \theta_j} e_j \Big) \\
      %&= \Bigg |\Lambda\Big(\sum_{j = 1}^{N} e^{i \theta_j} e_j
      % \Big) \Bigg | \\
      &\le \| \Lambda\| \Big \| \sum_{j = 1}^{N} e^{i \theta_j} e_j
      \Big \|_\infty \\
      &= \| \Lambda\|
    \end{align*}
    Since this is true for all $N \in \mathbb{N}$, taking the
    limits as $N \to \infty$, the inequality is preserved and we
    get that $(y_n) \in \ell^1$.

    Since any $x \in \textbf{c}_0$ can be written as $x = \sum_{n \in
    \mathbb{N}} x_i e_i$, where $x_i \to 0$, by linearity of
    $\Lambda$, we see that
    \begin{align*}
      \Lambda(x) &= \sum_{n \in \mathbb{N}} x_i \Lambda(e_i) =
      \sum_{n \in \mathbb{N}}  x_i y_i
    \end{align*}
  \end{solution}

\end{questions}
\printbibliography[heading=bibintoc]
\end{document}
