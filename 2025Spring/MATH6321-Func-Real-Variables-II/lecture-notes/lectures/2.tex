% TeX_root = ../main.tex

\marginnote{\scriptsize 16/01/2025 }

\begin{exercise}
  Find an orthonormal basis for $\ell^{2}(X)$
\end{exercise}
\begin{solution}
  We want an orthonormal family $(u_\alpha)_{\alpha \in A}$ such that
  \begin{align*}
    \|h\|^2 = \sum_{\alpha \in  A} |\langle h , u_\alpha \rangle |^2
  \end{align*}
  But here $\|h\|^2 = \sum_{x \in X} |h(x)|^{2}$, and
  \begin{align*}
    \langle h , g \rangle = \sum_{x \in X} h(x) \overline{g(x)}
  \end{align*}
  If we choose $u_x = \chi_{x}$, then we see that this satisfy our
  required properties.
\end{solution}

We formulate consequences of the characterization of orthonormal bases.

\begin{theorem}
  Let $(u_\alpha)_{\alpha \in A}$ be an orthonormal basis for a
  Hilbert space $\mathcal{H}$. Let
  \begin{align*}
    A_o = \{ \alpha  \ : \   \langle h , \alpha \rangle \neq 0 \}
  \end{align*}
  \marginnote{ \scriptsize \it Use the connection between inner
  product and evaluation from above to see that $A_o$ is at most countable.}
  Then enumerating  $A_o$ by $A_o = \{  \alpha_1 ,  \alpha_2 , \ldots
  \}$ and considering
  \begin{align*}
    h_n = \sum_{j = 1}^{n} \langle h , u_{\alpha_j} \rangle  u_{\alpha_j}
  \end{align*}
  gives $h_n \to h$.
\end{theorem}
\begin{proof}
  We saw that $h_n$ forms a cauchy sequence bases on the last oproof and choose
  \begin{align*}
    A_j = \{ \alpha_k  \ : \  1 \le k \le j \}
  \end{align*}
  Let $g = \lim_{n \to \infty} h_n$. Then the continuity of the inner
  product gives $ \langle h-g , u_\alpha \rangle = 0$ for each $\alpha \in A$.
  Thus
  \begin{align*}
    \|h - g\|^2 = \sum_{\alpha \in  A} |\langle h-g , u_\alpha \rangle |^2 = 0
  \end{align*}
  so $h = g$.
\end{proof}

\begin{corollary}
  If $(u_\alpha)_{\alpha \in \mathbb{N}}$ is an orthonormal basis for
  a Hilbert space $\mathcal{H}$,
  then for each $h \in  \mathcal{H}$,
  \begin{align*}
    h = \sum_{n \in \mathbb{N}} \langle h , u_n \rangle u_\alpha
  \end{align*}
  Conversely, if $\sum_{n \in \mathbb{N}} |c_n|^2 < \infty$, then
  \begin{align*}
    \sum_{n \in \mathbb{N}} c_n u_n \in \mathcal{H}
  \end{align*}
\end{corollary}
\begin{proof}
  \textcolor{red}{verify}
\end{proof}

\begin{remark}
  We can also take $  \mathbb{Z}$ instead of $\mathbb{N}$ in the above cases.
\end{remark}

\begin{example}
  Consider $L^2([-\pi, \pi])$, then
  \begin{align*}
    e_n(x) = \frac{1}{\sqrt{2\pi}} e^{inx}
  \end{align*}
  defines an orthonormal basis $(e_n)_{n \in \mathbb{Z}}$ for
  $L^{2}([-\pi, \pi])$
\end{example}
\begin{proof}
  It is clear that the above is an orthonormal family. We show that
  finite linear combinations of $e_n$s are dense in
  $L^{2}([-\pi, \pi])$.
  Find a $g \in C([-\pi, \pi])$, $\varepsilon/3$ away from $g$. Now
  find a $h$ periodic which is $ \frac{\varepsilon}{3}$ away from
  $g$. Now use Stone-Weierstrass theorem.
\end{proof}

\begin{corollary}[Reisz-Fischer Theorem]
  If $f \in L^{2}([-\pi, \pi])$, then $$f = \sum_{n \in \mathbb{Z}}
  \langle f , e_n \rangle e_n$$ and $\|f\|^2 = \sum_{n \in \mathbb{Z}}
  |\langle f , e_n \rangle ^2|$.
  Moreover, if $c \in \ell^{2}(\mathbb{Z})$, then
  \begin{align*}
    g = \sum_{n \in \mathbb{Z}} c_ne_n \in L^{2}([-\pi, \pi])
  \end{align*}
\end{corollary}

\begin{remark}
  For $f \in L^{2}([-\pi, \pi])$, $c_n = \langle f , e_n \rangle$ are
  called Fourier coefficients.
\end{remark}

\begin{theorem}
  Let $\mathcal{H}$ be a Hilbert space, then $\mathcal{H}$ has an
  orthonormal basis.
\end{theorem}
\begin{proof}
  We will show this using Zorn's lemma. Let $$\mathscr{S} = \{ U \subset H
  \ : \  U \textrm{ is an orthonormal set} \}$$
  It is easy to see that $\mathscr{S}$ is nonempty. Order
  $\mathscr{S}$ by set inclusion. Let $\mathscr{C}$ be a chain in
  $\mathscr{S}$, then
  \begin{align*}
    U_{\mathscr{C}} = \bigcup_{C \in \mathscr{C}} C
  \end{align*}
  will be an orthonormal set in $\mathscr{S}$. (Use the standard
  arguments to see this). Therefore $U_{\mathscr{C}}$ is the upper
  bound for the chain $\mathscr{C}$. Hence we see that $\mathscr{S}$ has a
  maximal element by the Zorn's lemma. Hence $\mathcal{H}$ has an
  orthonormal basis.
\end{proof}

\begin{definition}
  Let $\mathcal{H}, \mathcal{K}$ be Hilbert space. A linear map $V:
  \mathcal{H} \to \mathcal{K}$ is an isometry if
  \begin{align*}
    \langle Vh , Vg \rangle_{\mathcal{K}} = \langle h , g \rangle_{
    \mathcal{H}}
  \end{align*}
  for each $h, g \in \mathcal{H}$.
  If $V$ is onto, we say $V$ is unitary. Then we say $\mathcal{H}$
  and $\mathcal{K}$ are isomorphic.
  (From preservation of preservation of norm, $V$ is one-one).
\end{definition}

\begin{proposition}
  A linear map $V: \mathcal{H} \to \mathcal{K}$ is an isometry if and
  only if for every $h \in \mathcal{H}$,
  \begin{align*}
    \|Vh\|_{\mathcal{K}} = \|h\|_{\mathcal{H}}
  \end{align*}
\end{proposition}
\begin{proof}
  One way is easy. That is if the inner product is preserved, then
  the norm is preserved. Conversely, we use the polarization identity.
  \begin{align*}
    \langle h, g \rangle = \frac{1}{4}\sum_{j = 1}^{4} i^j\|h + i^j
    g\|^2,  \quad ( i = \sqrt{-1})
  \end{align*}
\end{proof}

\begin{example}
  Let $\mathcal{H} = \ell^{2}(\mathbb{N})$. Define
  \begin{align*}
    S : \mathcal{H} \to \mathcal{H} := (x_1 , x_2 , \ldots) \mapsto
    (0, x_1 , x_2 , \ldots , x_n)
  \end{align*}
  Then $S$ is clearly an isometry, but not unitary since nothing maps
  to $(1, x_1 , x_2 , \ldots)$
\end{example}
