% TeX_root = ../main.tex

\chapter{Complex Measures}

\marginnote{\scriptsize 20/03/2025 }

\section{Consequence of Radon-Nikodym Theorem}

\begin{theorem}
  If $\mu, \nu$ are positive $\sigma$-finite measures such that $\nu
  \ll \mu$, then there is a positive measurable function $h$ such that
  $d \nu = h \mu$
\end{theorem}

\begin{theorem}[Hahn-Decomposition Theorem]
  Let $\mu$ be a real-valued complex measure (signed measure) on a
  measurable space
  $(X, \mathcal{M})$. Then there are two sets $A, B$ such that $A
  \cup B = X, A \cap B = \emptyset$ and
  \begin{align*}
    \mu_+(E) := \mu( E \cap A), \quad \mu_-(E) = \mu(E \cap B)
  \end{align*}
  with $ \mu_+ \perp \mu_-$ and $ \mu_+ + \mu_- = \mu$, and $\mu_+ +
  \mu_- = |\mu|$.

  Moreover, if $   \mu = \mu_1 - \mu_2$ with $ \mu_1 , \mu_2$ being
  positive measures, then for any $E \in \mathcal{M}$ we have
  $\mu_1(E) \ge \mu_+(E), \mu_2(E) \ge \mu_-(E)$
\end{theorem}
\begin{proof}
  Since $\mu$ is a complex measure, $\mu \ll |\mu|$ and by
  Radon-Nikodym, there is a $h \in L^{1}(\mu)$ with $h(x) \in \{ 1, 2  \}$
  (polar decomposition) such that $d \mu = h d |\mu|$.

  Let $A = h^{-1}(1), B = X \setminus A$. We find that $d\mu_+ =
  \frac{1}{2}(d|\mu| + d\mu) = \frac{1}{2}(|h| d |\mu| + h d |\mu|) = h_+
  d |\mu|$,  and similarly $\mu_- = h_- d |\mu|$. The rest follows easily.
\end{proof}

\section{Bounded linear functionals on $L^p$}

\begin{note}
  Let $\mu$ be a positive measure, $1 \le p \le \infty$ and
  $\frac{1}{p} + \frac{1}{q} = 1$. Fixing $ g \in L^{1}(\mu)$.
  Holder's inequality gives that for any $ f \in L^{p}(\mu)$,
  \begin{align*}
    \big|\int fg \ d \mu\big| \le \|f\|_p \|g\|_q
  \end{align*}
  So that $\Lambda_g : L^{p}(\mu) \to \mathbb{C} := f \to \int fg \ d
  \mu$ is a bounded linear functional. Thus, we have a map $\Lambda :
  L^{q}(\mu) \to L^{p}(\mu)^* := g \mapsto \Lambda_g$.

  For $1 \le p < \infty$, the converse is true, too
\end{note}

\begin{lemma}
  If $\mu$ is $\sigma$-finite on $(X, \mathcal{M})$, then there is a
  $\omega \in L^{1}(\mu)$ such that $\forall x \in X : 0 <  \omega(x) < 1$.
\end{lemma}
\begin{proof}
  Choose a partition $E_j$ of $X$ such that $\mu(E_j) < \infty$ for
  each $j \in \mathbb{N}$. Let
  \begin{align*}
    \omega = \sum_{n \in \mathbb{N}} \frac{1}{2^n} \frac{1}{1 +
    \mu(E_n)} \chi_{E_n}
  \end{align*}
  Since $E_j$ is a partition, we get that
  \begin{align*}
    \int |\omega| \ d \mu &= \int \omega \ d \mu \\
    &= \int \sum_{n \in \mathbb{N}} \frac{1}{2^n} \frac{1}{1 +
    \mu(E_n)} \chi_{E_n} \ d \mu \\
    &= \sum_{n \in \mathbb{N}} \frac{1}{2^n} \frac{1}{1 +
    \mu(E_n)} \mu(E_n) \\
    &\le \sum_{n \in \mathbb{N}} \frac{1}{2^n}  = 1
  \end{align*}
  Hence $\omega \in L^{1}(\mu)$ is the required function.
\end{proof}

\begin{corollary}
  If $\mu$ is $\sigma$-finite, then $\tilde{ \mu}$ given by $d
  \tilde{\mu} = \omega d \mu$ is finite.
\end{corollary}

\begin{theorem}
  Let $\mu$ be a $\sigma$-finite measure, $1 \le p < \infty$, $q$ as
  usual. If $ \Lambda \in L^{p}(\mu)^*$, then there is $ g \in
  L^{q}(\mu)$ such that
  \begin{align*}
    \Lambda = \Lambda_g
  \end{align*}
  and $\|\Lambda\| = \|g\|_q$
\end{theorem}
\begin{proof}
  Begin by assuming $\mu$ is finite. Let $\Lambda : L^{p}(\mu) \to
  \mathbb{C}$ be a bounded linear functional. Notice that $\chi_E \in
  L^{p}(\mu)$ for each $E \in \mathcal{M}$. Consider $\lambda(E) =
  \Lambda(\chi_E)$. Let $\{E_j\}_{n = 1}^\infty$ be a partition of $E$. We find
  \begin{align*}
    \lambda \Big( \bigcup_{j = 1}^{n}E_j \Big) &= \Lambda \Big(
    \sum_{j = 1}^n \xi_{E_j} \Big) \\
    &= \sum_{j = 1}^{n} \Lambda(\chi_{E_j}) \\
    &= \sum_{ j = 1}^{n} \lambda(E_j)
  \end{align*}
  We conclude $\lambda$ is finitely additive. Note
  \begin{align*}
    \Big \| \chi_E - \chi_{\cup_{j = 1}^{n}E_j} \Big \|_p = \Big(
    \mu\big(\bigcup_{j = n+1}^{\infty} E_j\big)\Big)^{\frac{1}{p}}
  \end{align*}
  Using monotone convergence and boundedness of $\Lambda$, we get
  \begin{align*}
    \Lambda(  \chi_{\cup_{j = 1}^{n}E_j}) \to \Lambda(\chi_E)
  \end{align*}
  Thus $\lambda$ is a measure and $\lambda \ll \mu$ by the definition
  of $\lambda$. By
  Radon-Nikodym, we have $g \in L^{1}(\mu)$ with $d \lambda = g d \mu$.

  For $f$ simple,
  \begin{align*}
    \Lambda(f) = \int f \ d \lambda = \int fg \ d \mu := \Lambda_g(f)
  \end{align*}
  Now, consider $p = 1$. Then
  \begin{align*}
    \Big|\int f \ d \lambda\Big| = \Big|\int fg \ d \mu\Big| \le
    \|\Lambda\| \|f\|_1
  \end{align*}
\end{proof}
