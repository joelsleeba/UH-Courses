% TeX_root = ../main.tex

\chapter*{Course Information}
\marginnote{\scriptsize 14/01/2025 }

Office : Tuesday 11:30 - 12:30 PM, Wednesday 1-2PM\\
Midterm: 4th March 2025, in class

\chapter{Hilbert Spaces (continuation)}

\begin{exercise}[Warm up]
  \label{warm_up1}
  Show that if $f \in \ell^{2}(X)$, then utmost countably many values
  of $f$ are non-zero.
\end{exercise}
\begin{solution}
  Take the sets
  \begin{align*}
    E_n = \{ x \in X  \ : \  |f(x)|>1/n \}
  \end{align*}
  and consider their union.
\end{solution}

\section{Orthonormal Sets}

\begin{definition}
  A orthonormal family in a Hilbert space is a family
  $(u_\alpha)_{\alpha \in A}$ such that $\langle  u_\alpha ,  u_\beta
  \rangle  = 0$ if $\alpha \neq \beta$ and $\|u_\alpha\| = 1$ for
  each $ \alpha, \beta \in A$.
\end{definition}

\begin{theorem}[Bessel's Inequality]
  Given an orthonormal family $(u_\alpha)_{\alpha \in A}$ in a
  Hilbert space $\mathcal{H}$, and $h \in \mathcal{H}$, then
  \begin{align*}
    \|h\|^2 \ge \sum_{\alpha \in  A} |\langle h , u_\alpha \rangle |^2
  \end{align*}
\end{theorem}
\begin{proof}
  Let $B \subset A$ be finite, and let
  \begin{align*}
    g = \sum_{\alpha \in  B} \langle h , u_\alpha \rangle  u_\alpha
  \end{align*}
  We can easily show that, $\langle h-g , g \rangle  = 0$. Thus
  \begin{align*}
    \|h\|^2 &= \|h - g + g\|^2 \\
    &= \langle h-g +g , h-g +g \rangle \\
    &= \|h - g\|^2 + \|g\|^2 \\
    & \ge \|g\|^2
  \end{align*}
  Now the inequality follows form the definition of summation as the
  supremum of finite index sums.
\end{proof}

\begin{definition}
  Let $\mathcal{H}$ be a Hilbert space. An orthonormal family
  $(u_\alpha)_{\alpha \in A}$ is called complete, or an orthonormal
  basis, if for each $h \in H$,
  \begin{align*}
    \|h\|^2 = \sum_{\alpha \in  A} |\langle h , u_\alpha \rangle |^2
  \end{align*}
\end{definition}

\begin{definition}
  A set $U = \{ u_\alpha  \ : \   \alpha \in A \}$ is a maximal
  orthonormal set if for any $V$ with $V \supset U$ and $V$ is
  orthonormal, then $V = U$.
\end{definition}

\begin{theorem}
  Let $\mathcal{H}$ be a Hilbert space, $(u_\alpha)_{\alpha \in A}$
  an orthonormal family, then the following are equivalent.
  \begin{enumerate}
    \item $(u_\alpha)_{\alpha \in A}$ is an orthonormal basis
    \item $\textrm{span}\{ u_\alpha  \ : \   \alpha \in A \}$ is
      dense in $\mathcal{H}$
    \item $\{ u_\alpha \}$ is a maximal orthonormal set
  \end{enumerate}
\end{theorem}
\begin{proof}
  ($1 \implies 2$) Let $h$ be given. Consider for any $B \subset A$,
  \begin{align*}
    g = \sum_{\alpha \in  B} \langle h , u_\alpha \rangle  u_\alpha
  \end{align*}
  then we recall $\langle h- g , g \rangle  = 0$. And thus
  \begin{align*}
    \|h\|^2 =  \|h- g\|^2 + \|g\|^2
  \end{align*}
  We know from equality in Bessel's inequality that, for given
  $\varepsilon > 0$  we can choose $B$ such that $\|h\|^2 - \|g\|^2 <
  \varepsilon^2$. Hence
  \begin{align*}
    \|h - g\|^2 = \|h\|^2 - \|g\|^2 < \varepsilon^2
  \end{align*}
  Thus, we can find a $g$ that is arbitrarily close to $h$.

  ($\neg 3 \implies \neg 2$). Assuming $3$ is wrong, there exists $u
  \in \mathcal{H}$, such that $\|u\| = 1$, and $\langle u , u_\alpha
  \rangle = 0$ for each $\alpha \in A$. Next, for any finite $B
  \subset A$, and any $c_\alpha \in \mathbb{C}$, we look at
  \begin{align*}
    \|u - \sum_{\alpha \in  B} c_\alpha u_\alpha\| &= \|u\|^2 +
    \|\sum_{\alpha \in  B} c_\alpha u_\alpha\|^2 \\
    &= \|u\|^2 + \sum_{\alpha \in  B} |c_\alpha|^2 \\
    & \ge \|u\|^2 = 1
  \end{align*}
  Thus $u \not\in \overline{\textrm{span} \{ u_\alpha \}}$.

  $(\neg 1 \implies \neg 3)$. Assume there is $h \in \mathcal{H}$ such that
  \begin{align*}
    \|h\|^2 > \sum_{\alpha \in  A} |\langle h , u_\alpha \rangle|^2
  \end{align*}
  We know that $A_o = \{ \alpha \in A  \ : \  \langle h , u_\alpha
  \rangle \neq 0 \}$ is at most countable from \autoref{warm_up1}.
  We can find $A_1 \subset A_2 \subset \ldots$ each finite where
  \begin{align*}
    A_n = \{ \alpha \in A_o  \ : \  |\langle h , u_\alpha \rangle |
    \ge \frac{1}{n} \}
  \end{align*}
  and
  \begin{align*}
    A_o = \bigcup_{n = 1}^{\infty}A_n
  \end{align*}
  Let $g_n = \sum_{\alpha \in  A_n} \langle h , u_\alpha \rangle
  u_\alpha$. By monotone convergence theorem, given $\varepsilon> 0$,
  there exists $N \in \mathbb{N}$ such that
  \begin{align*}
    \sum_{\alpha \in  A} |\langle h , u_\alpha \rangle |^2 <
    \sum_{\alpha \in  A_N} |\langle h , u_\alpha \rangle |^2 + \varepsilon
  \end{align*}
  Thus, for $m \ge n \ge N$
  \begin{align*}
    \|g_m - g_n\|^2 &= \|\sum_{\alpha \in  A_m \setminus A_n} \langle
    h , u_\alpha \rangle  u_\alpha\|^2 \\
    &= \sum_{\alpha \in  A_m \setminus A_n} |\langle h , u_\alpha \rangle |^2 \\
    &= \sum_{\alpha \in  A_m} |\langle h , u_\alpha \rangle |^2 -
    \sum_{\alpha \in  A_n} |\langle h , u_\alpha \rangle |^2 \\
    &=  \sum_{\alpha \in  A} |\langle h , u_\alpha \rangle |^2 -
    \sum_{\alpha \in A_n} |\langle h , u_\alpha \rangle |^2 \\
    &< \varepsilon
  \end{align*}
  Therefore, we conclude that $(g_n)$ is a Cauchy sequence. By
  completeness of $\mathcal{H}$, $g_n \to g$ for some $g \in
  \mathcal{H}$. Let $\gamma = h - g$. If $\gamma \in A_o$, then
  $\gamma \in A_n$ for some $n \in \mathbb{N}$. Thus we'll get that
  \begin{align*}
    \langle h-g , u_\alpha \rangle &= 0
  \end{align*}
  \textcolor{red}{verify}

  If $\gamma \notin A_o$, then $\langle h , u_\gamma \rangle = 0$ and
  $\langle g_n , u_\gamma \rangle = 0$, so again $\langle  ,  \rangle
  $ \textcolor{red}{verify}

\end{proof}
