% TeX_root = ../main.tex

\marginnote{\scriptsize 30/01/2025 }

\begin{definition}
  A set $E \subset X$ is called \textbf{nowhere dense} if $\overline{E}$ does
  not contain a non-empty open set in $X$. A set is called
  \textbf{first category} if it is a union of nowhere dense sets,
  otherwise the set is called \textbf{second category}.
\end{definition}

\begin{theorem}[Baire category theorem, version II]
  Let $(X, d)$ be a  complete metric space. Then $X$ is not of first category.
\end{theorem}
\begin{proof}
  Let $E_1 , E_2 , \ldots$ be a sequence of nowhere dense sets. Then
  $\overline{E_n}$ has an empty interior for all $n \in \mathbb{N}$.
  Thus $\overline{E_n}^c$ is open and dense. Then by
  \autoref{thm:Baire-Category-Rudin} $\cap_{n =
  1}^{\infty}\overline{E_n}^c$ is a dense $G_\delta$ set. Thus
  $\cap_{n = 1}^{\infty} \overline{E_n}^c \neq \emptyset$. Thus
  taking complements, we get
  \begin{align*}
    \bigcup_{n = 1}^{\infty}E_n \subset \bigcup_{n = 1}^{\infty}
    \overline{E_n} \neq X
  \end{align*}
\end{proof}

\textcolor{red}{ \scriptsize \it missed something here}
