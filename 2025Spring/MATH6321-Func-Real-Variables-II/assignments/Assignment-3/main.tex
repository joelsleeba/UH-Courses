% initial settings
\documentclass[12pt]{exam}
\usepackage{geometry}
\usepackage{graphicx}
\usepackage{enumitem}
\usepackage[usenames,dvipsnames]{xcolor}
\usepackage[backend=biber, style=alphabetic]{biblatex}
\usepackage{url,hyperref}

\usepackage{amsmath} % math symbols, matrices, cases, trig functions,
% var-greek symbols.
\usepackage{amsfonts} % mathbb, mathfrak, large sum and product symbols.
\usepackage{amssymb} % extended list of math symbols from AMS.
% https://ctan.math.washington.edu/tex-archive/fonts/amsfonts/doc/amssymb.pdf
\usepackage{amsthm} % theorem styling.
\usepackage{mathrsfs} % mathscr fonts.
\usepackage{yhmath} % widehat.
\usepackage{empheq} % emphasize equations, extending 'amsmath' and 'mathtools'.
\usepackage{bm} % simplified bold math. Do \bm{math-equations-here}

% geometry of paper
\geometry{
  a4paper, % 'a4paper', 'c5paper', 'letterpaper', 'legalpaper'
  asymmetric, % don't swap margins in left and right pages. as
  % opposed to 'twoside'
  centering, % to center the content between margins
  bindingoffset=0cm,
}

% hyprlink settings
\hypersetup{
  colorlinks = true,
  linkcolor = {red!60!black},
  anchorcolor = red,
  citecolor = {green!50!black},
  urlcolor = magenta,
}

% theorem styles
\theoremstyle{plain} % default; italic text, extra space above and below
\newtheorem{theorem}{Theorem}[section]
\newtheorem{proposition}{Proposition}[section]
\newtheorem{lemma}{Lemma}[section]
\newtheorem{corollary}{Corollary}[theorem]

\theoremstyle{definition} % upright text, extra space above and below
\newtheorem{definition}{Definition}[section]
\newtheorem{example}{Example}[section]

\theoremstyle{remark} % upright text, no extra space above or below
\newtheorem{remark}{Remark}[section]
\newtheorem*{note}{Note} %'Notes' in italics and without counter

% renewcommands for counters
\newcommand{\propositionautorefname}{Proposition}
\newcommand{\definitionautorefname}{Definition}
\newcommand{\lemmaautorefname}{Lemma}
\newcommand{\remarkautorefname}{Remark}
\newcommand{\exampleautorefname}{Example}

\addbibresource{~/Books/Research/research.bib}

\begin{document}

\title{MATH 6321 - Theory of Functions of a Real Variable \\ Homework III}

% author list
\author{
  Joel Sleeba \\
}

\maketitle
\printanswers
\unframedsolutions

\begin{questions}

  \question
  \begin{solution}
    \begin{parts}
      \part Let
      \begin{align*}
        A^M = \{ x \in X  \ : \  |f_n(x)|<M, \ \forall n \in \mathbb{N} \}
      \end{align*}
      Since for all $x \in X$, $f_n(x) \to f(x) \in \mathbb{C}$,
      there is some $M \in \mathbb{N}$ such that $|f_n(x)| < M$ for
      all $n \in \mathbb{N}$. Then $x \in A^M$. Thus we see that
      \begin{align*}
        X = \bigcup_{M \in \mathbb{N}} A^M
      \end{align*}
      Then by the Baire category theorem, we get that for one of $M_0
      \in \mathbb{N}$, $\overline{A^{M_0}}$ has a nonempty interior.
      Let $V \subset \overline{A^{M_0}}$ be a non-empty open set.
      We'll show that $V$ is the set we're looking for.

      Let $x \in V$, and $n \in \mathbb{N}$ be given. Then by the
      continuity of $f_n$, there exists an $y \in A^{M_0} \cap V$
      such that $|f_n(x) - f_n(y)| < 1$. Thus
      \begin{align*}
        |f_n(x)| < |f_n(y)| + 1 < M_0 + 1
      \end{align*}
      Since $n \in \mathbb{N}$ was arbitrary, we get that this holds
      for all $ n \in \mathbb{N}$. Hence $V$ is the non-empty open
      set and $M_0 + 1$ is the integer we need.

      \part
      Let $\varepsilon > 0$ be given, and
      \begin{align*}
        A^N = \Big \{ x \in X  \ : \ |f_n(x) - f_m(x)| \le
        \frac{\varepsilon}{2}, \ \forall n, m \ge N \Big \}
      \end{align*}
      Since $f_n(x) \to f(x)$ pointwise, we get that
      \begin{align*}
        X = \bigcup_{N \in \mathbb{N}} A^N
      \end{align*}
      Moreover, since each
      \begin{align*}
        A^N = \bigcap_{ n, m > N \in \mathbb{N}} (f_n -
        f_m)^{-1}(\bar{B}_{\frac{\varepsilon}{2}}(0))
      \end{align*}
      we get that each $A^N$ is closed being the
      intersection of closed sets.
      Now by the same argument as for the previous part, there exist
      a non-empty open set $W \subset \overline{A^{N_0}} = A^{N_0}$ for some
      $N_0 \in \mathbb{N}$. We'll show that $W$ is the non-empty open
      set and $N_0$ is the integer we need.

      Let $x \in W$ and $n \ge N_0$. Since $f_n(x)
      \to f(x)$, there exists an $m \ge N_0$ such that
      \begin{align*}
        |f_m(x) - f(x)| < \frac{\varepsilon}{2}
      \end{align*}
      Then for that $m$, we get that
      \begin{align*}
        |f_n(x) - f(x)| &\le |f_n(x) - f_m(x)| + |f_m(x) - f(x)| \\
        &< \frac{\varepsilon}{2} + \frac{\varepsilon}{2} \\
        &= \varepsilon
      \end{align*}
      where $|f_n(x) - f_m(x)| \le \frac{\varepsilon}{2}$ since $ x
      \in W \subset A^{N_0}$ and $n, m \ge N_0$. Hence we are done.
    \end{parts}
  \end{solution}

  \question
  \begin{solution}
    \begin{parts}
      \part We'll first show that $\|M_f\| \le \|f\|_\infty$. To see
      this, notice that for any $g \in L^{2}(\mu)$, we have
      \begin{align*}
        \|fg\|_2^2 = \int |fg|^2 \ d \mu \le \int \|f\|_\infty^2 |g|^2 \ d \mu =
        \|f\|_\infty^2 \|g\|_2^2
      \end{align*}
      Thus $\|M_fg\|_2 = \|fg\|_2 \le \|f\|_\infty \|g\|_2$. Now it follows from
      the definition of the operator norm, that $\|M_f\| \le \|f\|_\infty$.

      \begin{definition}
        A measure $\mu$ on $(X, \Sigma)$ is called seminfinite
        whenever $\mu(A) = \infty$, there is a $\Sigma \ni B \subset
        A$ such that $0 < \mu(B) < \infty$.
      \end{definition}

      We'll show that seminiteness of the measure $\mu$ is sufficient
      for $\|M_f\| = \|f\|_\infty$ for all $f \in L^{\infty}(\mu)$.
      Let $f \in L^{\infty}(\mu)$ be given. Let
      \begin{align*}
        A_n = \Big \{ x \in X  \ : \  \|f\|_\infty^2 < |f(x)|^2 +
        \frac{1}{n} \Big \}
      \end{align*}
      By the definition of the essential supremum, we get that
      $\mu(A_n) > 0$ for each $n \in \mathbb{N}$. Let $B_n \subset
      A_n$ be a measurable set such that $0 < \mu(B_n) < \infty$.
      Now, let $g_n(x) = \frac{1}{\mu(B_n)} \chi_{B_n}$, clearly
      $\|g_n\|_2 = 1$, and
      \begin{align*}
        \|M_f(g_n)\|_2^2  &= \int |fg_n|^2 \ d \mu \\
        &= \frac{1}{\mu(B_n)}\int_{B_n} |f|^2 \ d \mu \\
        &\ge \frac{1}{\mu(B_n)}\int_{B_n} \|f\|_\infty^2 -
        \frac{1}{n} \ d \mu \\
        &= \frac{1}{\mu(B_n)}\int_{B_n} \|f\|_\infty^2 \ d \mu  \ - \
        \frac{1}{\mu(B_n)}\int_{B_n} \frac{1}{n} \ d \mu \\
        &= \|f\|_\infty^2 - \frac{1}{n}
      \end{align*}
      Hence
      \begin{align*}
        \sqrt{\|f\|_\infty^2 - \frac{1}{n}} \ \le \ \|M_f(g_n)\|_2 \ \le \
        \|f\|_\infty \|g\|_2 = \|f\|_\infty
      \end{align*}
      Thus, as $n \to \infty$, we see that $\|M_f(g_n)\|_2 \to
      \|f\|_\infty$. Hence $\|M_f\| = \|f\|_\infty$. Since $ f \in
      L^{\infty}( \mu)$ was arbitrary, this holds for every $ f \in
      L^{\infty}(\mu)$.
      \part We claim that the $M_f$ is an onto function from
      $L^{2}(\mu) \to L^{2}(\mu)$ if
      \begin{align*}
        0 \ \not\in \ R_f:= \{ w \in
          \mathbb{C}  \ : \  \mu(f^{-1}(B_\varepsilon(w)))>0, \ \forall
        \varepsilon > 0 \}
      \end{align*}
      the essential range of $f$. If $0 \not\in R_f$,
      then there exist $\varepsilon > 0$ such that
      $\mu(f^{-1}(B_\varepsilon(0))) = 0$. That implies $|f(x)| \ge
      \varepsilon$ almost everywhere. Then $\frac{1}{|f(x)|} <
      \frac{1}{\varepsilon}$ almost everywhere. Hence $\frac{1}{f}
      \in L^{\infty}(\mu)$. Then by the first part, for any $g \in L^{2}(\mu)$,
      \begin{align*}
        \frac{g}{f} \in L^{2}(\mu)
      \end{align*}
      Hence $M_f( \frac{g}{f}) = g$, which shows that $M_f$ is onto.
    \end{parts}
  \end{solution}

\end{questions}
\printbibliography[heading=bibintoc]
\end{document}
