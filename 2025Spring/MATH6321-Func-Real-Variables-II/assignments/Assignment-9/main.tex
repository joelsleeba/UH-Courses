% initial settings
\documentclass[12pt]{exam}
\usepackage{geometry}
\usepackage{graphicx}
\usepackage{enumitem}
\usepackage[usenames,dvipsnames]{xcolor}
\usepackage[backend=biber, style=alphabetic]{biblatex}
\usepackage{url,hyperref}

\usepackage{amsmath} % math symbols, matrices, cases, trig functions,
% var-greek symbols.
\usepackage{amsfonts} % mathbb, mathfrak, large sum and product symbols.
\usepackage{amssymb} % extended list of math symbols from AMS.
% https://ctan.math.washington.edu/tex-archive/fonts/amsfonts/doc/amssymb.pdf
\usepackage{amsthm} % theorem styling.
\usepackage{mathrsfs} % mathscr fonts.
\usepackage{yhmath} % widehat.
\usepackage{empheq} % emphasize equations, extending 'amsmath' and 'mathtools'.
\usepackage{bm} % simplified bold math. Do \bm{math-equations-here}

% geometry of paper
\geometry{
  a4paper, % 'a4paper', 'c5paper', 'letterpaper', 'legalpaper'
  asymmetric, % don't swap margins in left and right pages. as
  % opposed to 'twoside'
  centering, % to center the content between margins
  bindingoffset=0cm,
}

% hyprlink settings
\hypersetup{
  colorlinks = true,
  linkcolor = {red!60!black},
  anchorcolor = red,
  citecolor = {green!50!black},
  urlcolor = magenta,
}

% theorem styles
\theoremstyle{plain} % default; italic text, extra space above and below
\newtheorem{theorem}{Theorem}[section]
\newtheorem{proposition}{Proposition}[section]
\newtheorem{lemma}{Lemma}[section]
\newtheorem{corollary}{Corollary}[theorem]

\theoremstyle{definition} % upright text, extra space above and below
\newtheorem{definition}{Definition}[section]
\newtheorem{example}{Example}[section]

\theoremstyle{remark} % upright text, no extra space above or below
\newtheorem{remark}{Remark}[section]
\newtheorem*{note}{Note} %'Notes' in italics and without counter

% renewcommands for counters
\newcommand{\propositionautorefname}{Proposition}
\newcommand{\definitionautorefname}{Definition}
\newcommand{\lemmaautorefname}{Lemma}
\newcommand{\remarkautorefname}{Remark}
\newcommand{\exampleautorefname}{Example}

\addbibresource{~/Books/Research/research.bib}

\begin{document}

\title{MATH 6321 - Theory of functions of a real variable \\ Homework 9}

% author list
\author{
  Joel Sleeba \\
}

\maketitle
\printanswers
\unframedsolutions

\begin{questions}
  \question
  \begin{solution}
    Let $f \in L^{1}(\mathbb{R})$ and $x \in \mathbb{R}$ be a
    Lebesgue point. Then by the definition, we have
    \begin{align*}
      \lim_{r \to 0} \frac{1}{m(B_r(x))} \int_{B_r(x)} |f(x) - f(y)|
      \ d m( y) \ = \ 0
    \end{align*}
    Thus for every $\varepsilon > 0$, there is a $r_\varepsilon >0$
    such that for all $r < r_\varepsilon$,
    \begin{align*}
      \frac{1}{m(B_r(x))} \int_{B_r(x)} |f(x) - f(y)|
      \ d m( y) < \varepsilon
    \end{align*}
    Since $|\int f \ d \mu| < \int |f| \ d \mu$, we get
    \begin{align*}
      \Big|f(x) \ - \ &\frac{1}{m(B_r(x))} \int_{B_r(x)} f(y) \ d m(y)
      \Big| \\
      &= \Big|\frac{1}{m(B_r(x))} \int_{B_r(x)} f(x) \ d m(y) -
      \frac{1}{m(B_r(x))} \int_{B_r(x)} f(y) \ d m(y)\Big| \\
      &\le \frac{1}{m(B_r(x))} \int_{B_r(x)} |f(x) - f(y)| \ d m( y) \\
      &< \varepsilon
    \end{align*}
    and thus
    \begin{align*}
      |f(x)| - \Big|\frac{1}{m(B_r(x))}
      \int_{B_r(x)} f(y) \ d m(y) \Big|  \ < \varepsilon
    \end{align*}
    which gives
    \begin{align*}
      |f(x)| < \frac{1}{m(B_r(x))}
      \int_{B_r(x)} |f| \ d m + \varepsilon
    \end{align*}
    Since $\varepsilon >0$ was chosen arbitrarily, taking
    $\varepsilon \to 0$ and taking supremum over all $r > 0$ gives
    \begin{align*}
      |f(x)| \le \sup_{r > 0}\frac{1}{m(B_r(x))}
      \int_{B_r(x)} |f| \ d m = \mathcal{M}f(x)
    \end{align*}
  \end{solution}

  \question
  \begin{solution}
    For the sake of contradiction, assume that there exists $c_1 >
    c_2 > 0$ such that for $A_i := \{ x \in \mathbb{R} \ :
    \ |f(x)| \ge c_i\}$,  we have $c_i\mu(A_i) = \|f\|_1$.
    Then
    \begin{align*}
      \|f\|_1 \ge \int |f|\chi_{A_i} \ d \mu \ge \int  c_i \chi_{A_i}
      \ d \mu = c_i \mu(A_i) = \| f\|_1
    \end{align*}
    Thus
    \begin{align}
      \label{eq:2}
      \int_{A_i}  |f| - c_i \ d \mu  = 0
    \end{align}
    Since $\|f\|_1, c_i > 0$, by assumption, we see $\mu(A_i) > 0$.
    Moreover $|f(x)| - c_i > 0$ for all $x \in A_i$ by definition.
    Thus $\autoref{eq:2}$ forces $|f(x)| = c_i$ almost everywhere in
    $A_i$. But since by definition $A_1 \subset A_2$, this gives a
    contradiction as $|f|$ cannot be a.e equal to $c_1$ and $c_2$
    simultaneously in $A_2$ unless $c_1 = c_2$.
  \end{solution}

  \question
  \textcolor{red}{not finished}
  \begin{solution}
    Let $x \in \mathbb{R}$ and $g_x : [x - 1, x+ 1] \to \mathbb{R} :
    g_x(y) := |f(x) - f(y)|^2$. Note that since $f \in L^{2}([x- 1, x
    + 1])$, by Holder's inequality, $f \in L^{1}([x - 1, x + 1])$. Also
    \begin{align*}
      \int_{[x-1, x+1]}  |g_x| \ d m &= \int_{[x - 1, x + 1]}  |f(x)
      - f(y)|^2 \ d m(y) \\
      &\le \int_{[x-1, x +1]} |f(x)|^2 + 2|f(x)||f(y)| + |f(y)|^2 \ d \mu \\
      &\le 2|f(x)|^2 + 2|f(x)| \|f\|_1 + \|f\|_2 \\
      &< \infty
    \end{align*}
    Thus $g_x \in L^{1}([x - 1, x + 1])$. Thus almost every $y \in [x
    - 1, x + 1]$ is a Lebesgue point of $g_x$ by the Lebesgue
    differentiation theorem.

  \end{solution}

  \question
  \begin{solution}
    We know that $\mu(A):= \int_A  f \ d m$, defines a measure on
    $\mathbb{R}$. By the properties of the measure $\mu$, for any $x< y
    \in \mathbb{R}$,
    \begin{align*}
      \int_{(x, y]} f \ dm = \int_{(-\infty, y]} f \ dm -
      \int_{(-\infty,x]} f \ dm = 0 - 0 = 0
    \end{align*}
    Any open interval $(x, y) = \cup_{n = 1}^{\infty}(x,
    y- \frac{1}{n})$. By the continuity of the measure $\mu$ from
    below, we get
    \begin{align*}
      \int_{(x, y)}  f \ d m = \lim_{n \to \infty}   \int_{(x, y-
      \frac{1}{n}]}  f \ d m = 0
    \end{align*}
    Since $f \in L^{1}(m)$, we know that almost all $x \in
    \mathbb{R}$ are Lebesgue points of $f$. That is for almost every
    $x \in \mathbb{R}$,
    \begin{align*}
      f(x) = \lim_{r \to 0} \frac{1}{m(B_r(x))} \int_{B_r(x)} f \ d m
      = \lim_{r \to 0} 0 = 0
    \end{align*}
    Thus we are done.
  \end{solution}

\end{questions}
\printbibliography[heading=bibintoc]
\end{document}
