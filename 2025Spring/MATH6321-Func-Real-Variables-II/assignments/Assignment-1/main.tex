% initial settings
\documentclass[12pt]{exam}
\usepackage{geometry}
\usepackage{graphicx}
\usepackage{enumitem}
\usepackage[usenames,dvipsnames]{xcolor}
\usepackage[backend=biber, style=alphabetic]{biblatex}
\usepackage{url,hyperref}

\usepackage{amsmath} % math symbols, matrices, cases, trig functions,
% var-greek symbols.
\usepackage{amsfonts} % mathbb, mathfrak, large sum and product symbols.
\usepackage{amssymb} % extended list of math symbols from AMS.
% https://ctan.math.washington.edu/tex-archive/fonts/amsfonts/doc/amssymb.pdf
\usepackage{amsthm} % theorem styling.
\usepackage{mathrsfs} % mathscr fonts.
\usepackage{yhmath} % widehat.
\usepackage{empheq} % emphasize equations, extending 'amsmath' and 'mathtools'.
\usepackage{bm} % simplified bold math. Do \bm{math-equations-here}

% geometry of paper
\geometry{
  a4paper, % 'a4paper', 'c5paper', 'letterpaper', 'legalpaper'
  asymmetric, % don't swap margins in left and right pages. as
  % opposed to 'twoside'
  centering, % to center the content between margins
  bindingoffset=0cm,
}

% hyprlink settings
\hypersetup{
  colorlinks = true,
  linkcolor = {red!60!black},
  anchorcolor = red,
  citecolor = {green!50!black},
  urlcolor = magenta,
}

% theorem styles
\theoremstyle{plain} % default; italic text, extra space above and below
\newtheorem{theorem}{Theorem}[section]
\newtheorem{proposition}{Proposition}[section]
\newtheorem{lemma}{Lemma}[section]
\newtheorem{corollary}{Corollary}[theorem]

\theoremstyle{definition} % upright text, extra space above and below
\newtheorem{definition}{Definition}[section]
\newtheorem{example}{Example}[section]

\theoremstyle{remark} % upright text, no extra space above or below
\newtheorem{remark}{Remark}[section]
\newtheorem*{note}{Note} %'Notes' in italics and without counter

% renewcommands for counters
\newcommand{\propositionautorefname}{Proposition}
\newcommand{\definitionautorefname}{Definition}
\newcommand{\lemmaautorefname}{Lemma}
\newcommand{\remarkautorefname}{Remark}
\newcommand{\exampleautorefname}{Example}

\addbibresource{~/Books/Research/research.bib}

\begin{document}

\title{MATH 6321 - Functions of a real variable \\ Homework  I}

% author list
\author{
  Joel Sleeba \\
}

\maketitle
\printanswers
\unframedsolutions

\begin{questions}
  \question
  \begin{solution}
    For the sake of contradiction, assume that $M^\perp$ has
    dimension more than one. Then the Gram-Schmidt orthonormalization
    procedure guarantees the existence of orthonormal vectors $a, b
    \in M^\perp$. Now consider the vector $L(b)a - L(a)b$. Since
    $M^\perp$ is a subspace, we see that $L(b)a - L(a)b \in M^\perp$. Moreover
    \begin{align*}
      L( L(b) a - L(a) b) = L(b)L(a) - L(a)L(b) = \textbf{0}
    \end{align*}
    Hence $L(b) a - L(a) b \in M$. Thus $L(b) a  - L(a) b = 0$ and
    since $L(a) \neq 0 \neq L(b)$, as $a, b \in M^\perp$, we see that
    \begin{align*}
      b = \frac{L(b)}{L(a)}a
    \end{align*}
    But this contradicts our assumption that $ a, b$ are orthonormal.
    Hence we see that $M^\perp$ is a one dimensional subspace.
  \end{solution}

  \question

  \begin{solution}
    Let $f_k(t) = e^{ikt}$ for $k \in \mathbb{Z}$. Then we get that
    \begin{align*}
      \frac{1}{2\pi} \int_{- \pi}^{\pi}  f_k(t) \ dt  =
      \begin{cases}
        1, & \textrm{k = 0} \\
        0, & \textrm{otherwise}
      \end{cases}
    \end{align*}
    Also when $k \neq 0$,
    \begin{align*}
      \frac{1}{N}\sum_{n = 1}^{N} f_k(2 \pi n \alpha) &= \frac{1}{N}
      \sum_{n = 1}^{N} e^{i2 \pi \alpha n k } \\
      &= \frac{1}{N} \frac{e^{i2\pi \alpha Nk} - 1}{e^{i2\pi \alpha k} - 1} \\
      &\le \frac{1}{N} \frac{2}{e^{i2\pi \alpha k} - 1}
    \end{align*}
    Since $\alpha$ is irrational, the denominator above cannot be
    zero, and we get that the
    \begin{align*}
      \lim_{N \to \infty} \frac{1}{N} \sum_{n = 1}^{N} f_k(2 \pi n \alpha) = 0
    \end{align*}
    If $k = 0$, then $f_k(t) = e^{0} = 1$ and we get
    \begin{align*}
      \frac{1}{N} \sum_{n = 1}^{N} f_k(2 \pi n \alpha) = 1
    \end{align*}
    making the limit also equal to 1. Hence we have showed that the
    given equality
    \begin{equation}
      \label{eq:2}
      \lim_{N \to \infty} \frac{1}{N} \sum_{n = 1}^{N} f_k(2 \pi n
      \alpha) = \frac{1}{2\pi}
      \int_{- \pi}^{ \pi}  f_k(t) \ dt
    \end{equation}
    holds for all $f_k(t) = e^{ikt}$, where $k \in \mathbb{Z}$.

    Now, we know that the family of sets $f_k(t) = \frac{1}{\sqrt{2
    \pi}} e^{i k t}$ forms an orthonormal basis for $L^2([-\pi,
    \pi])$. Moreover, we know that every $2\pi$ periodic continuous
    function can be embedded into $L^2([-\pi, \pi])$. Hence if $f$ is
    any $2 \pi$ periodic continuous function, then there exists $a_j
    \in \mathbb{C}$ such that
    \begin{align*}
      f(t) = \sum_{j = 1}^{J} a_j f_{k_j}(t)
    \end{align*}

    Then by the properties of integration,
    \begin{align*}
      \frac{1}{2\pi} \int_{-\pi}^{ \pi}  f(t) \ dt \ = \ \frac{1}{2\pi}
      \int_{-\pi}^{ \pi}  \sum_{j = 1}^{J} a_j f_{k_j}(t) \ dt \ = \
      \sum_{j = 1}^{J} a_{j} \frac{1}{2 \pi} \int_{-\pi}^{\pi}
      f_{k_j}(t) \ dt \\
    \end{align*}

    Moreover by \autoref{eq:2}, for each $f_{k_j}$, and $\varepsilon
    > 0$, there is a $N_{k_j} \in \mathbb{N}$ such that for all $N > N_{k_j}$
    \begin{align*}
      \Bigg| \frac{1}{N} \sum_{n = 1}^{N} f_{k_j}(2\pi n \alpha) -
      \frac{1}{2\pi} \int_{-\pi}^{\pi}  f_{k_j}(t) \ dt\Bigg| <
      \frac{\varepsilon}{2^{j}|a_j|}
    \end{align*}
    Let $N_f = \max \{ N_{k_j}  \ : \  1 \le j \le J \}$, then for all $N > N_f$
    \begin{align*}
      \Bigg|\frac{1}{N} \sum_{n = 1}^{N} f(2\pi n \alpha) -
      \frac{1}{2\pi} \int_{-\pi}^{\pi}  f(t) \ dt\Bigg| &=
      \Bigg|\frac{1}{N}\sum_{n = 1}^{N} \sum_{j = 1}^{J} a_j
      f_{k_j}(2 \pi n \alpha) - \frac{1}{2\pi}
      \int_{-\pi}^{ \pi}  \sum_{j = 1}^{J} a_j f_{k_j}(t) \ dt \Bigg| \\
      &= \Bigg|\sum_{j = 1}^{J} a_j \frac{1}{N}\sum_{n = 1}^{N}
      f_{k_j}(2 \pi n \alpha) - \sum_{j = 1}^{J} a_j  \frac{1}{2\pi}
      \int_{-\pi}^{ \pi}  f_{k_j}(t) \ dt \Bigg| \\
      &\le \sum_{j = 1}^{J} |a_j| \Bigg|\frac{1}{N}\sum_{n = 1}^{N}
      f_{k_j}(2 \pi n \alpha) -  \frac{1}{2\pi}
      \int_{-\pi}^{ \pi}   f_{k_j}(t) \ dt\Bigg| \\
      &< \sum_{j = 1}^{J} \frac{\varepsilon}{2^{j}} \\
      &< \varepsilon
    \end{align*}
    Since $\varepsilon > 0$ was chosen arbitrary, this proves that
    \begin{align*}
      \lim_{N \to \infty} \frac{1}{N} \sum_{n = 1}^{N} f(2 \pi n
      \alpha) = \frac{1}{2\pi}
      \int_{- \pi}^{ \pi}  f(t) \ dt
    \end{align*}
  \end{solution}

  \question
  \begin{solution}
    Consider $P_2(\mathbb{R})$ the set of polynomials with real
    coefficients as a subspace of the inner product space $L^{2}([-1,
    1])$. Clearly $1, x, x^2$ is a basis for $P_2(\mathbb{R})$. Now
    by Gram-Schmidt orthonormalization, we see that $\big\{ \frac{1}{\sqrt 2},
    \sqrt{\frac{3}{2}}x, \sqrt{\frac{5}{2}}x^2 \big\}$ is an
    orthonormal basis for $P_2(\mathbb{R})$. Now we find the
    projection of the cubic polynomial $x^3$ to $P_2(\mathbb{R})$. We
    know that this projection will be
    \begin{align*}
      \Big \langle x^3 , \frac{1}{\sqrt{2}} \Big \rangle \frac{1}{\sqrt{2}} +
      \Big \langle x^3 , \sqrt{\frac{3}{2}} x \Big \rangle \sqrt{\frac{3}{2}} x
      + \Big \langle x^3 , \sqrt{\frac{5}{2}} x^2 \Big \rangle
      \sqrt{\frac{5}{2}} x^2 = 0 + \frac{3}{5}x + 0 x^2
    \end{align*}
    Also we know that $\|x^3 - f\|$ will be minimum for $f \in
    P_2(\mathbb{R})$, when $f$ is the projection of $x^3$ to
    $P_2(\mathbb{R})$. Hence
    \begin{align*}
      \min_{a, b, c \ \in \ \mathbb{R}} \int_{-1}^{1} |x^3 - a - bx -
      cx^2|^2 \ dx = \int_{-1}^{1} \Big|x^3 - \frac{3}{5}x\Big|^2
      \ dx = \frac{8}{175}
    \end{align*}
  \end{solution}

\end{questions}
\printbibliography[heading=bibintoc]
\end{document}
