% initial settings
\documentclass[12pt]{exam}
\usepackage{geometry}
\usepackage{graphicx}
\usepackage{enumitem}
\usepackage[usenames,dvipsnames]{xcolor}
\usepackage[backend=biber, style=alphabetic]{biblatex}
\usepackage{url,hyperref}

\usepackage{amsmath} % math symbols, matrices, cases, trig functions,
% var-greek symbols.
\usepackage{amsfonts} % mathbb, mathfrak, large sum and product symbols.
\usepackage{amssymb} % extended list of math symbols from AMS.
% https://ctan.math.washington.edu/tex-archive/fonts/amsfonts/doc/amssymb.pdf
\usepackage{amsthm} % theorem styling.
\usepackage{mathrsfs} % mathscr fonts.
\usepackage{yhmath} % widehat.
\usepackage{empheq} % emphasize equations, extending 'amsmath' and 'mathtools'.
\usepackage{bm} % simplified bold math. Do \bm{math-equations-here}

% geometry of paper
\geometry{
  a4paper, % 'a4paper', 'c5paper', 'letterpaper', 'legalpaper'
  asymmetric, % don't swap margins in left and right pages. as
  % opposed to 'twoside'
  centering, % to center the content between margins
  bindingoffset=0cm,
}

% hyprlink settings
\hypersetup{
  colorlinks = true,
  linkcolor = {red!60!black},
  anchorcolor = red,
  citecolor = {green!50!black},
  urlcolor = magenta,
}

% theorem styles
\theoremstyle{plain} % default; italic text, extra space above and below
\newtheorem{theorem}{Theorem}[section]
\newtheorem{proposition}{Proposition}[section]
\newtheorem{lemma}{Lemma}[section]
\newtheorem{corollary}{Corollary}[theorem]

\theoremstyle{definition} % upright text, extra space above and below
\newtheorem{definition}{Definition}[section]
\newtheorem{example}{Example}[section]

\theoremstyle{remark} % upright text, no extra space above or below
\newtheorem{remark}{Remark}[section]
\newtheorem*{note}{Note} %'Notes' in italics and without counter

% renewcommands for counters
\newcommand{\propositionautorefname}{Proposition}
\newcommand{\definitionautorefname}{Definition}
\newcommand{\lemmaautorefname}{Lemma}
\newcommand{\remarkautorefname}{Remark}
\newcommand{\exampleautorefname}{Example}

\addbibresource{~/Books/Research/research.bib}

\begin{document}

\title{MATH 6321 - Theory of functions on a real variable \\ Homework  6}

% author list
\author{
  Joel Sleeba \\
}

\maketitle
\printanswers
\unframedsolutions

\begin{questions}

  \question
  \begin{solution}
    Let $f : X \to \mathbb{C}$ be a function. Then
    \begin{align*}
      \|f\|_1 = |f(a)| \mu(\{ a \}) + |f(b)| \mu(\{ b \}) = |f(a)| +
      |f(b)|\cdot \infty
    \end{align*}
    and
    \begin{align*}
      \|f\|_\infty = \max \{ |f(a)|, |f(b)| \}
    \end{align*}
  \end{solution}
  Then it is clear that $f \in L^{1}(X)$ iff $f(b) = 0$ and $|f(a)| <
  \infty$. Then $\|f\|_1 = |f(a)|$. Thus $L^{1}(X) =
  \textrm{span}\{\delta_a\}$, where $\delta_a : X \to \mathbb{C}$
  such that $\delta_a(a) = 1$ and $\delta_a(b) = 0$. If $f \in
  L^{1}(X)$, then $f = f(a)\delta_a$. Thus $L^{1}(X)$ is a
  one-dimensional vector space.

  But $L^{\infty}(X) = \textrm{span}\{ \delta_a, \delta_b \}$, since
  any $f \in L^{\infty}(X)$ can be written as $f = f(a)\delta_a +
  f(b)\delta_b$. Thus $L^{2}(X)$ is a two-dimensional vector space.

  Since $L^1(X), L^\infty(X)$ are of different dimensions, their dual
  spaces will also be non-isomorphic.

  \question

  \begin{solution}
    By a warm-up exercise we did in the lecture, $f \in L^{1}(\mu)$
    forces $f$ to be zero except on a countable set. Thus $fg$ is
    zero everywhere except a countable set. Hence $fg$ is measurable.
    Moreover, since $|g(x)| < 1$
    \begin{align*}
      \Big|\int fg \ d \mu\Big| \le \int |fg| \ d \mu \le \int |f| \ d \mu
    \end{align*}
  \end{solution}
  Thus the map $\Lambda: f \to \int_E  f \ d \mu$ is a contraction.
  That $\Lambda$ is a linear functional follows form the linearity of
  the integration. Thus $\Lambda$ is a bounded linear map.

  Now to see that $g$ is not measurable, notice that $g^{-1}([0,
  \frac{1}{2}]) = [0, \frac{1}{2}]$ is uncountable with it's
  complement $(\frac{1}{2}, 1]$ also uncountable, while $[0,
  \frac{1}{2}]$ belongs to the Borel-sigma algebra of $\mathbb{R}$.

  \question
  \begin{solution}
    Recall that $C(I)$ is closed under the sup-norm. Let $f \in
    L^{\infty}(m)\setminus C(I)$ with $\|f\|_\infty = 1$. Define $\Lambda:
    \overline{\textrm{span}}(C(I) \cup \{ f \}) \to \mathbb{C}$ such
    that $\Lambda|_{C(I)} = 0$ and $\Lambda(f) = 1$ and linearily
    extending to the whole of $\overline{\textrm{span}}(C(I) \cup \{
    f \})$. Clearly $\|\Lambda\|  = 1$. Then by Hahn-Banach extension
    theorem, $\Lambda$ can be extended to a linear functional, (say
    $\Lambda$ by an abuse of notation) to the
    whole of $L^{\infty}(I)$ preserving the norm. Then $\Lambda$ is
    an example of a linear functional which is non-zero but vanishes
    on all of $C(I)$.

    For the sake of contradiction, assume that $\Lambda =  \Lambda_g$
    for some $g \in L^{1}(m)$. By the density of $C(I)$ in
    $L^{1}(I)$, there is a sequence of continuous functions $g_n$
    which converge to $ g$ in $L_1$ norm. Then $fg_n \to fg$ in $L_1$
    norm for all
    $f \in L^{\infty}( I)$, since
    \begin{align*}
      \int |fg_n - fg| \ d m & \le \|f\|_\infty \int |g_n - g| \ d m
    \end{align*}
    Specifically for $f = g_m$, we get that
    \begin{align*}
      \int g_m g_n \ d m \to \int g_m g \ d m = \Lambda(g_m) = 0
    \end{align*}
    This should force $g = 0$ almost everywhere which will give a contradiction.
  \end{solution}

\end{questions}
\printbibliography[heading=bibintoc]
\end{document}
