% initial settings
\documentclass[12pt]{exam}
\usepackage{geometry}
\usepackage{graphicx}
\usepackage{enumitem}
\usepackage[usenames,dvipsnames]{xcolor}
\usepackage[backend=biber, style=alphabetic]{biblatex}
\usepackage{url,hyperref}

\usepackage{amsmath} % math symbols, matrices, cases, trig functions,
% var-greek symbols.
\usepackage{amsfonts} % mathbb, mathfrak, large sum and product symbols.
\usepackage{amssymb} % extended list of math symbols from AMS.
% https://ctan.math.washington.edu/tex-archive/fonts/amsfonts/doc/amssymb.pdf
\usepackage{amsthm} % theorem styling.
\usepackage{mathrsfs} % mathscr fonts.
\usepackage{yhmath} % widehat.
\usepackage{empheq} % emphasize equations, extending 'amsmath' and 'mathtools'.
\usepackage{bm} % simplified bold math. Do \bm{math-equations-here}

% geometry of paper
\geometry{
  a4paper, % 'a4paper', 'c5paper', 'letterpaper', 'legalpaper'
  asymmetric, % don't swap margins in left and right pages. as
  % opposed to 'twoside'
  centering, % to center the content between margins
  bindingoffset=0cm,
}

% hyprlink settings
\hypersetup{
  colorlinks = true,
  linkcolor = {red!60!black},
  anchorcolor = red,
  citecolor = {green!50!black},
  urlcolor = magenta,
}

% theorem styles
\theoremstyle{plain} % default; italic text, extra space above and below
\newtheorem{theorem}{Theorem}[section]
\newtheorem{proposition}{Proposition}[section]
\newtheorem{lemma}{Lemma}[section]
\newtheorem{corollary}{Corollary}[theorem]

\theoremstyle{definition} % upright text, extra space above and below
\newtheorem{definition}{Definition}[section]
\newtheorem{example}{Example}[section]

\theoremstyle{remark} % upright text, no extra space above or below
\newtheorem{remark}{Remark}[section]
\newtheorem*{note}{Note} %'Notes' in italics and without counter

% renewcommands for counters
\newcommand{\propositionautorefname}{Proposition}
\newcommand{\definitionautorefname}{Definition}
\newcommand{\lemmaautorefname}{Lemma}
\newcommand{\remarkautorefname}{Remark}
\newcommand{\exampleautorefname}{Example}

\addbibresource{~/Books/Research/research.bib}

\begin{document}

\title{MATH 6303 - Modern Algebra II \\ Homework  I}

% author list
\author{
  Joel Sleeba \\
}

\maketitle
\printanswers
\unframedsolutions

\begin{questions}

  \question
  \begin{solution}
    Let $F$ be a finite field of characteristic $p$. Then $F_p$ can
    be embedded naturally to $F$. Hence we see that $F$ is a field
    extension over $F$. Then $F$ is a vector space over the field
    $F_p$. Moreover since $F$ is finite, there is an $F_p$-basis for
    $F$. Let $\beta = \{ f_1 , f_2 , \ldots , f_n   \}$ be an
    $F_p$-basis for $F$. Then $F \cong F_p^n$ by the fact that every
    vector space is isomorphic to its co-ordinate space. Thus we see
    that $|F| = |F_p|^n = p^n$.
  \end{solution}

  \question
  \begin{solution}
    We know that $[\mathbb{Q}(i): Q] = 2$ and
    $[\mathbb{Q}(\sqrt[3]{2}):\mathbb{Q}] = 3 =
    [\mathbb{Q}(\sqrt[3]{3}): \mathbb{Q}]$. Now if $x^3 - 2$ is
    irreducible over $\mathbb{Q}(i)$, then $\mathbb{Q}(i,
    \sqrt[3]{2}) = \mathbb{Q}(i)$ (This equality may be read as
      isomorphism and here we are considering $\sqrt[3]{2}$ to be a
    root of the polynomial $x^3 - 2$). But this would mean that
    $\mathbb{Q}(i)$ is a field extension of
    $\mathbb{Q}(\sqrt[3]{2})$. That is $\mathbb{Q}(\sqrt[3]{2})$ is a
    linear subspace of $\mathbb{Q}(i)$. But this is impossible as a
    vector space of dimension $3$ cannot be a subspace of dimension $2$.
  \end{solution}

  \question
  \begin{solution}
    Clearly $\mathbb{Q}(\sqrt{2}, \sqrt{3}) \supset
    \mathbb{Q}(\sqrt{2} + \sqrt{3})$ since $\sqrt{2} +
    \sqrt{3} \in \mathbb{Q}(\sqrt{2}, \sqrt{3})$. Now since
    $(\sqrt{2} + \sqrt{3})^{-1} = \sqrt{3} - \sqrt{2}$, we see that
    \begin{align*}
      \sqrt{2} = (\sqrt{2} + \sqrt{3}) - (\sqrt{3} - \sqrt{2}) \ \in \
      \mathbb{Q}(\sqrt{2} + \sqrt{3}) \\
      \sqrt{3} = (\sqrt{2} + \sqrt{3}) + (\sqrt{3} - \sqrt{2}) \ \in \
      \mathbb{Q}(\sqrt{2} + \sqrt{3})
    \end{align*}
    Thus $\mathbb{Q}(\sqrt{2}, \sqrt{3}) = \mathbb{Q}(\sqrt{2} +
    \sqrt{3})$. Moreover, $\sqrt{2} \notin \mathbb{Q}(\sqrt{3})$.
    Otherwise $\sqrt{2} = a + b \sqrt{3}$ for some $a, b \in
    \mathbb{Q}$, which is a contradiction. Thus we see that
    $[\mathbb{Q}(\sqrt{2}, \sqrt{3}): \mathbb{Q}(\sqrt{3})] = 2$
    since $x^2 -2$ is a polynomial in $\mathbb{Q}(\sqrt{3})[x]$ with
    root $\sqrt{2}$.
    Hence we see that
    \begin{align*}
      [\mathbb{Q}(\sqrt{2} + \sqrt{3}): \mathbb{Q}] =
      [\mathbb{Q}(\sqrt{2}, \sqrt{3}): \mathbb{Q}] =
      [\mathbb{Q}(\sqrt{2}, \sqrt{3}):
      \mathbb{Q}(\sqrt{3})][\mathbb{Q}(\sqrt{3}): \mathbb{Q}] = 2 \times 2 = 4
    \end{align*}
    Moreover by taking higher powers of $\sqrt{2} + \sqrt{3}$, we see
    that $x^4 - 10x^2 +1$ is an irreducible polynomial satisfied by
    $\sqrt{2} + \sqrt{3}$.
  \end{solution}

  \question
  \begin{solution}
    For the sake of contradiction, assume that $\sqrt[3]{2} \in
    \mathbb{Q}(\alpha_1 , \alpha_2 , \ldots , \alpha_n)$ and
    $\alpha_i \notin \mathbb{Q}(\alpha_1 , \alpha_2 , \ldots ,
    \alpha_{i-1})$ for all $1 \le i \le n$. Then there
    is a smallest $m \le n$ such that $\sqrt[3]{2} \in
    \mathbb{Q}(\alpha_1 , \alpha_2 , \ldots , \alpha_m)$. Then
    $\sqrt[3]{2} = a + \alpha_m b$, where $a, b \in
    \mathbb{Q}(\alpha_1 , \alpha_2 , \ldots, \alpha_{m-1})$. Hence
    \begin{align*}
      2 = (a + \alpha_m b)^3 =  a^3 + 3a^2 \alpha_m b + 3a \alpha_m^2
      b^2 + b^3 \in \mathbb{Q}(\alpha_1 , \alpha_2 , \ldots , \alpha_{m-1})
    \end{align*}
    But this forces $\alpha_m \in \mathbb{Q}(\alpha_1 , \alpha_2 ,
    \ldots , \alpha_{m-1})$. This is a contradiction
    since $\alpha_m \notin \mathbb{Q}(\alpha_1 , \alpha_2 , \ldots ,
    \alpha_{m-1})$ by our assumption.

  \end{solution}

\end{questions}
\printbibliography[heading=bibintoc]
\end{document}
