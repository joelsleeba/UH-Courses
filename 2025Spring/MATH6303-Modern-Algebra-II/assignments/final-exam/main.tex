% initial settings
\documentclass[12pt]{exam}
\usepackage{geometry}
\usepackage{graphicx}
\usepackage{enumitem}
\usepackage[usenames,dvipsnames]{xcolor}
\usepackage[backend=biber, style=alphabetic]{biblatex}
\usepackage{url,hyperref}

\usepackage{amsmath} % math symbols, matrices, cases, trig functions,
% var-greek symbols.
\usepackage{amsfonts} % mathbb, mathfrak, large sum and product symbols.
\usepackage{amssymb} % extended list of math symbols from AMS.
% https://ctan.math.washington.edu/tex-archive/fonts/amsfonts/doc/amssymb.pdf
\usepackage{amsthm} % theorem styling.
\usepackage{mathrsfs} % mathscr fonts.
\usepackage{yhmath} % widehat.
\usepackage{empheq} % emphasize equations, extending 'amsmath' and 'mathtools'.
\usepackage{bm} % simplified bold math. Do \bm{math-equations-here}

% geometry of paper
\geometry{
  a4paper, % 'a4paper', 'c5paper', 'letterpaper', 'legalpaper'
  asymmetric, % don't swap margins in left and right pages. as
  % opposed to 'twoside'
  centering, % to center the content between margins
  bindingoffset=0cm,
}

% hyprlink settings
\hypersetup{
  colorlinks = true,
  linkcolor = {red!60!black},
  anchorcolor = red,
  citecolor = {green!50!black},
  urlcolor = magenta,
}

% theorem styles
\theoremstyle{plain} % default; italic text, extra space above and below
\newtheorem{theorem}{Theorem}[section]
\newtheorem{proposition}{Proposition}[section]
\newtheorem{lemma}{Lemma}[section]
\newtheorem{corollary}{Corollary}[theorem]

\theoremstyle{definition} % upright text, extra space above and below
\newtheorem{definition}{Definition}[section]
\newtheorem{example}{Example}[section]

\theoremstyle{remark} % upright text, no extra space above or below
\newtheorem{remark}{Remark}[section]
\newtheorem*{note}{Note} %'Notes' in italics and without counter

% renewcommands for counters
\newcommand{\propositionautorefname}{Proposition}
\newcommand{\definitionautorefname}{Definition}
\newcommand{\lemmaautorefname}{Lemma}
\newcommand{\remarkautorefname}{Remark}
\newcommand{\exampleautorefname}{Example}

\addbibresource{~/Books/Research/research.bib}

\begin{document}

\title{MATH 6303 - Modern Algebra II \\ Final Exam}

% author list
\author{
  Joel Sleeba \\
}

\maketitle
\printanswers
\unframedsolutions

\begin{questions}
  \question
  \begin{solution}
    \begin{parts}
      \part Given that $G(A) = \sum_{n \in \mathbb{N}} a_nA^n$, where
      the $(i, j)$-th entry of the partial sums converge to the $(i,
      j)$-th entry of $G(A)$. Let $G_N(A) = \sum_{i = 1}^{N} a_i
      A^i$, be the $N$-th partial sum. We'll show that
      $PG_N(At)P^{-1} = G_N(PAtP^{-1}) = G_N(PAP^{-1}t)$. The first
      equality follows easily by the distributivity of the matrix
      multiplication as,
      \begin{align*}
        PG_N(At)P^{-1} = P\Big(\sum_{i = 1}^{N} a_i A^i t\Big)P^{-1}
        = \sum_{i =
        1}^{N} a_i PA^itP^{-1} = G_N(PAtP^{-1})
      \end{align*}
      Since $t = tI$, where $I$ is the identity matrix, it commutes
      with $P^{-1}$, and we get
      \begin{align*}
        G_N(PAtP^{-1}) = \sum_{i=1}^{N} a_i PA^itP^{-1} = \sum_{i =
        1}^{N} a_i P A^i P^{-1} tI = G_N(PAP^{-1}t)
      \end{align*}
      Since $G$ is defined as the limit of the partial sums
      $G_N$, whenever the limit exists, the equality will be
      preserved for $G$ as well.
      \part Let $A = A_1 \oplus A_2 \oplus \ldots \oplus A_m$. Then
      \begin{align*} A =
        \begin{pmatrix}
          A_1 & & \\
          &  \ddots & \\
          & & A_m \\
        \end{pmatrix} \quad
        \textrm{and} \quad
        A^n =
        \begin{pmatrix}
          A_1^n & & \\
          &  \ddots & \\
          & & A_m^n \\
        \end{pmatrix}
      \end{align*}
      Thus by the definition of $G$,
      \begin{align}
        \label{eq:1}
        G(A) =
        \begin{pmatrix}
          G(A_1) & & \\
          &  \ddots & \\
          & &  G(A_m) \\
        \end{pmatrix}
      \end{align}
      which proves what we need since $At = A_1t \oplus A_2t \oplus \ldots
      \oplus A_mt$ and
      \begin{align*}
        G(At) =
        \begin{pmatrix}
          G(A_1t) & & \\
          &  \ddots & \\
          & &  G(A_mt) \\
        \end{pmatrix}
      \end{align*}
      \part This is the special case of part (b) by taking $A_i = [z_i]$.
    \end{parts}
  \end{solution}

  \question
  \begin{solution}
    By the definition and since $AB = BA$,
    \begin{align*}
      \exp(A)\exp(B) &= \Big( I + A + \frac{A^2}{2!} +
      \frac{A^3}{3!} \ldots \Big)\Big( I + B + \frac{B^2}{2!} +
      \frac{B^3}{3!}\ldots \Big) \\
    \end{align*}
    and
    \begin{align*}
      \exp(A + B) &= I + A+B + \frac{(A+B)^2}{2!} +
      \frac{(A+B)^3}{3!} + \ldots
    \end{align*}
    Thus we see that the $n$th term in the above summation is
    \begin{align*}
      \frac{(A+B)^n}{n!} &= \frac{1}{n!} \sum_{i = 0}^{n} {n \choose
      i} A^{n-i}B^i \\
      &= \sum_{i = o}^{n} \frac{1}{i! (n-i)!} A^{n-i} B^i \\
      &= \sum_{i = 0}^{n} \frac{A^{n-i}}{(n-i)!} \times \frac{B^{i}}{i!}
    \end{align*}
    which is precisely the term in $\exp(A)\exp(B)$ whose powers sum
    to $n$. Since this is true for all $n \in \mathbb{N}$, the power
    series agree and thus $\exp(A)\exp(B) = \exp( A + B)$.
  \end{solution}

  \question
  \begin{solution}
    Let $Nt$ be as given. Then we see that
    \begin{align*}
      (Nt)^2 =
      \begin{bmatrix}
        0 & 0 & t^2 & & &\\
        & 0 & 0  & t^2 & &\\
        &  & \ddots & \ddots & \ddots & \\
        &  &   & 0 & 0  & t^2 \\
        &  &   &  & 0  & 0 \\
        & &  & &   &  0\\
      \end{bmatrix}
    \end{align*}
    and similarly for the rest of the powers of $Nt$. Hence by the
    definition of $\exp(Nt)$, we get that
    \begin{align*}
      \exp(Nt) =
      \begin{bmatrix}%{c c c c c c}
        1 & t & \frac{t^2}{2!} & \ldots & \ldots &  \frac{t^{r-1}}{(r-1)!} \\
        & 1 & t &\frac{t^2}{2!}  & & \vdots \\
        & & \ddots & \ddots & \ddots & \vdots \\
        & & & \ddots & t & \frac{t^2}{2!} \\
        & & & & 1 & t \\
        & & & & & 1
      \end{bmatrix}
    \end{align*}
    Now if
    \begin{align*} J =
      \begin{bmatrix}%{c c c c c}
        \lambda  & 1 & 0 & \ldots & 0\\
        &  \lambda & 1 & \ldots & \vdots \\
        &  & \ddots & \ddots  & \\
        &  &  &  \lambda  & 1 \\
        &  &  &  &  \lambda
      \end{bmatrix}
    \end{align*}
    is an elementary Jordan matrix with eigenvalue $\lambda$, then
    since $Jt = \lambda I t + Nt$,
    \begin{align*}
      \exp(Jt) = \exp(\lambda It + Nt) = \exp(\lambda I t)\exp(Nt) =
      e^{\lambda t}\exp(Nt)
    \end{align*}
    which gives
    \begin{align}
      \label{eq:3}
      \exp(Jt) =
      e^{\lambda t}
      \begin{bmatrix}%{c c c c c c}
        1 & t & \frac{t^2}{2!} & \ldots & \ldots &  \frac{t^{r-1}}{(r-1)!} \\
        & 1 & t &\frac{t^2}{2!}  & & \vdots \\
        & & \ddots & \ddots & \ddots & \vdots \\
        & & & \ddots & t & \frac{t^2}{2!} \\
        & & & & 1 & t \\
        & & & & & 1
      \end{bmatrix}
    \end{align}
  \end{solution}

  \question
  \begin{solution}
    From example 3, we see that for the given matrices $P, D$
    \begin{align*}
      P^{-1}DP =
      \begin{bmatrix}%{c c c c}
        1 & 1 & 0 & 0\\
        0 & 1 & 0 & 0\\
        0 & 0 & 1 & 1\\
        0 & 0 & 0 & 1
      \end{bmatrix} =
      \begin{bmatrix}%{c c }
        J & 0\\
        0 & J
      \end{bmatrix}
    \end{align*}
    where $J =
    \begin{bmatrix}%{c c}
      1 & 1\\
      0 & 1
    \end{bmatrix}$ is the elementary Jordan matrix. From
    \autoref{eq:1}, we get that
    \begin{align*}
      \exp(P^{-1}DP) =
      \begin{bmatrix}%{ c c }
        \exp(J) & 0\\
        0 & \exp(J)
      \end{bmatrix}
    \end{align*}
    Moreover, \autoref{eq:3} for $\lambda = t = 1$, shows that
    \begin{align*}
      \exp(P^{-1}DP) =
      \begin{bmatrix}%{ c c }
        \exp(J) & 0\\
        0 & \exp(J)
      \end{bmatrix} =
      e^{1}
      \begin{bmatrix}%{c c c c}
        1 & 1 & 0 & 0\\
        0 & 1 & 0 & 0\\
        0 & 0 & 1 & 1\\
        0 & 0 & 0 & 1
      \end{bmatrix} =
      \begin{bmatrix}%{c c c c}
        e & e & 0 & 0\\
        0 & e & 0 & 0\\
        0 & 0 & e & e\\
        0 & 0 & 0 & e
      \end{bmatrix}
    \end{align*}
    Then
    \begin{align*}
      \exp(D) = P\exp(P^{-1}DP)P^{-1} &=
      \begin{bmatrix}%{c c c c}
        0 & 1 & 2 & 0\\
        2 & 0 & -2 & 1\\
        1 & 0 & 0 & 0\\
        0 & 0 & 1 & 0
      \end{bmatrix}
      \begin{bmatrix}%{c c c c}
        e & e & 0 & 0\\
        0 & e & 0 & 0\\
        0 & 0 & e & e\\
        0 & 0 & 0 & e
      \end{bmatrix}
      \begin{bmatrix}%{c c c c}
        0 & 0 & 1 & 0\\
        1 & 0 & 0 & -2\\
        0 & 0 & 0 & 1\\
        0 & 1 & -2 & 2
      \end{bmatrix} \\
      &=
      \begin{bmatrix}%{c c c c}
        e & 2e & -4e & 4e\\
        2e & -e & 4e & -8e\\
        e & 0 & e & -2e\\
        0 & e & -2e & 3e
      \end{bmatrix}
    \end{align*}
  \end{solution}

  \question
  \begin{solution}
    If $A = \textbf{0}$, then $A^n = \textbf{0}$ matrix for all $n
    \in \mathbb{N}$. Since $B + \textbf{0} = B$ for any matrix $B$,
    form the definition of $e^A$, we get that
    $e^{\textbf{0}} = I$. Moreover since we know that $\exp(A + B) =
    \exp(A)\exp(B)$, we get that
    \begin{align*}
      \exp(A)\exp(-A) = \exp(A - A) = \exp(0) = I\\
      I= \exp(0) = \exp(-A
      + A) = \exp(-A)\exp(A)
    \end{align*}
    Thus $\exp(A)$ is a nonsingular matrix with inverse $\exp(-A)$
    for all $A \in M_n(K)$.
  \end{solution}

  \question
  \begin{solution}
    Let $A = UTU^{-1}$, where $U$ is a invertible matrix and $T$ is the
    corresponding Jordan representation of $A$. Additionally assume
    that $ T = T_1 \oplus T_2 \oplus \ldots \oplus T_n$, where each
    $T_i$ are elementary Jordan blocks with eigenvalue $\lambda_i$. Then
    \begin{align*}
      e^A  = e^{UTU^{-1}} &= I + UTU^{-1} + \frac{(UTU^{-1})^2}{2!} +
      \frac{(UTU^{-1})^3}{3!} + \ldots \\
      &= U \Big( I + T + \frac{T^2}{2!} + \frac{T^3}{3!}  + \ldots\Big) \\
      &= U e^T U^{-1}
    \end{align*}
    Also notice that $\det(e^{T_i})= e^{\lambda_i}$ by setting $t = 1$
    in \autoref{eq:3}. Thus
    \begin{align*}
      \det(e^A) = \det(e^T) = \prod_{i = 1}^{n} \det(T_i) = \prod_{i
      = 1}^{n}  e^{\lambda_i} = e^{\sum_{i = 1}^{n} \lambda_i} =
      e^{\textrm{tr}(T)} = e^{\textrm{tr}(A)}
    \end{align*}
  \end{solution}

  \question
  \begin{solution}
    We know that $\exp(A + B) = \exp(A)\exp(B)$. For a fixed $A \in
    M_n(K)$, $\Phi: K \to GL_n(K):= t \to \exp(At)$ is a well defined map
    since $\exp(A) \in GL_n(K)$ for all $A \in M_n(K)$. Moreover
    \begin{align*}
      \Phi(t + r) = \exp(A(t + r)) = \exp(At + Ar) = \exp(At)\exp(Ar)
    \end{align*}
    shows that $\Phi$ is a group homomorphism.
  \end{solution}

\end{questions}
\printbibliography[heading=bibintoc]
\end{document}
