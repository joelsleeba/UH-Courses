% initial settings
\documentclass[12pt]{exam}
\usepackage{geometry}
\usepackage{graphicx}
\usepackage{enumitem}
\usepackage[usenames,dvipsnames]{xcolor}
\usepackage[backend=biber, style=alphabetic]{biblatex}
\usepackage{url,hyperref}

\usepackage{amsmath} % math symbols, matrices, cases, trig functions,
% var-greek symbols.
\usepackage{amsfonts} % mathbb, mathfrak, large sum and product symbols.
\usepackage{amssymb} % extended list of math symbols from AMS.
% https://ctan.math.washington.edu/tex-archive/fonts/amsfonts/doc/amssymb.pdf
\usepackage{amsthm} % theorem styling.
\usepackage{mathrsfs} % mathscr fonts.
\usepackage{yhmath} % widehat.
\usepackage{empheq} % emphasize equations, extending 'amsmath' and 'mathtools'.
\usepackage{bm} % simplified bold math. Do \bm{math-equations-here}

% geometry of paper
\geometry{
  a4paper, % 'a4paper', 'c5paper', 'letterpaper', 'legalpaper'
  asymmetric, % don't swap margins in left and right pages. as
  % opposed to 'twoside'
  centering, % to center the content between margins
  bindingoffset=0cm,
}

% hyprlink settings
\hypersetup{
  colorlinks = true,
  linkcolor = {red!60!black},
  anchorcolor = red,
  citecolor = {green!50!black},
  urlcolor = magenta,
}

% theorem styles
\theoremstyle{plain} % default; italic text, extra space above and below
\newtheorem{theorem}{Theorem}[section]
\newtheorem{proposition}{Proposition}[section]
\newtheorem{lemma}{Lemma}[section]
\newtheorem{corollary}{Corollary}[theorem]

\theoremstyle{definition} % upright text, extra space above and below
\newtheorem{definition}{Definition}[section]
\newtheorem{example}{Example}[section]

\theoremstyle{remark} % upright text, no extra space above or below
\newtheorem{remark}{Remark}[section]
\newtheorem*{note}{Note} %'Notes' in italics and without counter

% renewcommands for counters
\newcommand{\propositionautorefname}{Proposition}
\newcommand{\definitionautorefname}{Definition}
\newcommand{\lemmaautorefname}{Lemma}
\newcommand{\remarkautorefname}{Remark}
\newcommand{\exampleautorefname}{Example}

\addbibresource{~/Books/Research/research.bib}

\begin{document}

\title{MATH6303 - Modern Algebra II \\ Homework 5}

% author list
\author{
  Joel Sleeba \\
}

\maketitle
\printanswers
\unframedsolutions

\begin{questions}

  \question
  \begin{solution}
    By the properties of tensor product
    \begin{align*}
      2 \otimes 1 = 2(1 \otimes 1) = 1 \otimes 2
    \end{align*}
    Since $2 = 0$ in $\mathbb{Z}/ 2 \mathbb{Z}$, we get that $2
    \otimes 1 = 1 \otimes 2 = 1 \otimes 0 = 0$ in $\mathbb{Z} \otimes
    \mathbb{Z}/2 \mathbb{Z}$.

    Now let $1 \otimes 2k \in \mathbb{Z}/ 2 \mathbb{Z} \otimes 2
    \mathbb{Z}$ be an arbitrary non-zero tensor. By the properties of
    the tensor product, $1 \otimes
    2k = k( 1 \otimes 2)$, and thus $1 \otimes 2$ generate
    $\mathbb{Z}/ 2 \mathbb{Z} \otimes 2 \mathbb{Z}$.

    Now to show that $1 \otimes 2 \neq 0$ in $\mathbb{Z}/2 2\mathbb{Z}
    \otimes \mathbb{Z}$, by the universal property of the tensor
    products, we just need to find a $\mathbb{Z}$-module homomorphism,
    $\phi: \mathbb{Z}/2 \mathbb{Z} \times 2\mathbb{Z} \to
    \mathbb{Z}/2 \mathbb{Z}$
    such that $\phi(1, 2) \neq 0$. Define
    \begin{align*}
      \phi(x, y) = x \frac{y}{2} \mod 2
    \end{align*}
    Then for $a, b \in \mathbb{Z}$, $x, y \in \mathbb{Z}/2
    \mathbb{Z}$, and $p, q \in 2 \mathbb{Z}$,
    \begin{align*}
      \phi(ax+ y, bp + q) &= (a x + y) \frac{(bp+q)}{2} \mod 2 \\
      &= \Big(b(ax + y) \frac{p}{2} \mod 2 + (ax + y) \frac{q}{2}
      \mod 2\Big) \mod 2 \\
      &= \Big(abx \frac{p}{2} \mod 2 + by \frac{p}{2} \mod 2 + a x
      \frac{q}{2} \mod   2 + y \frac{q}{2} \mod 2\Big) \mod 2 \\
      &=  ab\phi(x, p) + b \phi(y, p) + a\phi(x, q) + \phi(y, q)
    \end{align*}
    where the last sum is in $\mathbb{Z}/ 2 \mathbb{Z}$. Thus, we see
    that $\phi$ is bilinear with $ \phi(1, 2) = 1 \in \mathbb{Z}/ 2
    \mathbb{Z}$. Hence $1 \otimes 2 \neq 0$ in $\mathbb{Z}/2
    \mathbb{Z} \otimes 2 \mathbb{Z}$.
  \end{solution}

  \question
  \begin{solution}
    For the sake of contradiction, assume that there exist $v, w \in
    \mathbb{R}^2$ such that $e_1 \otimes e_2 + e_2 \otimes e_1 = v
    \otimes w$. Since $e_1, e_2$ is a basis of $\mathbb{R}^2$, let $
    v = v_1 e_1 + v_2 e_2$ and $w = w_1 e_1 + w_2 e_2$ for $v_i, w_j
    \in \mathbb{R}$. Then
    \begin{align*}
      v \otimes w = (v_1 e_1 + v_2 e_2) \otimes (w_1 e_1 + w_2 e_2) &=
      v_1w_1 (e_1 \otimes e_1) + v_1 w_2 (e_1 \otimes e_2) \\
      &+ v_2w_1 ( e_2 \otimes e_1) + v_2w_2( e_2 \otimes e_2)
    \end{align*}
    Since $v \otimes w = e_1 \otimes e_2 + e_2 \otimes e_1$, and $\{
    e_i \otimes e_j \ : \ i , j \in \{ 1, 2 \} \}$ forms a basis for
    $ \mathbb{R}^2 \otimes \mathbb{R}^2$, this forces $v_1w_1 =
    v_2w_2 = 0$. Without loss of generality, assume that $v_1 = 0$
    and $w_2 = 0$. But this forces $v \otimes w = v_2w_1 (e_2 \otimes
    e_1)$, and we get a contradiction. We can show that the other
    cases also leads to a contradiction, and hence our assumption
    that $e_1 \otimes e_2 + e_2 \otimes e_1$ is a simple tensor is false.
  \end{solution}

  \question
  \begin{solution}
    Let $v = av^\prime$, then
    \begin{align*}
      v \otimes v^\prime = av^\prime \otimes v^\prime = a(v^\prime
      \otimes v^\prime) = v^\prime \otimes av^\prime = v^\prime \otimes v
    \end{align*}
    Now assume $v \neq a v^\prime$ for any $a \in F$. We need to show
    that $v \otimes v^\prime \neq v^\prime \otimes v$. By the
    universal property of the tensor products, this is equivalent to
    finding a bilinear map $\phi : V \times V \to \mathbb{C}$ such
    that $\phi(v, v^\prime) \neq \phi(v^\prime, v)$. Define $\phi$ such that
    \begin{align*}
      \phi(x, y) =
      \begin{cases}
        \alpha \beta, & \textrm{ if } x = \alpha v, y = \beta v^\prime \\
        0, & \textrm{elsewhere}
      \end{cases}
    \end{align*}
    Then  we can verify that $\phi$ is a bilinear map, such that
    $\phi(v, v^\prime) = 1$. But $(v^\prime, v) = (v^\prime, 0)
    + (0, v)$ cannot be represented as a linear combination of $(v,
    0), (0, v^\prime)$ by assumption that $v \neq a v^\prime$ for any
    $a \in F$. Thus $\phi(v^\prime, v) = 0$. This shows that $v
    \otimes v^\prime \neq v^\prime \otimes v$.
  \end{solution}

  \question
  \textcolor{red}{not finished}
  \begin{solution}
    \begin{parts}
      \part Let $p(x)
      = p_0 + p_1x, q(x) = q_0 + q_1 x, r(x) = r_0 + r_1 x$ be
      elements in $I$. Since $\phi$ one depends on the first two
      co-efficients, it is enough to prove that $\phi$ is a bilinear
      map on the linear polynomials. By rules of modular addition we get,
      \begin{align*}
        \phi(p + r, q) & = \frac{(p_0+r_0) q_1}{2}\mod 2 \\
        & =(\frac{p_0q_1}{2} \mod 2 + \frac{r_0q_1}{2} \mod 2 ) \mod 2  \\
        &= (\phi(p,q) + \phi(r, q) ) \mod 2
      \end{align*}
      Similarly
      \begin{align*}
        \phi(p , q + r) & = \frac{p_0(q_1 + r_1)}{2}\mod 2 \\
        & =(\frac{p_0q_1}{2} \mod 2 + \frac{p_0r_1}{2} \mod 2 ) \mod 2 \\
        &= (\phi(p,q) + \phi(p, r) ) \mod 2
      \end{align*}
      Now let $\psi: R/I \to \mathbb{Z}/ 2 \mathbb{Z}$ be the natural
      homomorphism. Then $s \in \mathbb{Z}/2 \mathbb{Z}$ can be
      identified with $\psi^{-1}(s) =  s + I \in R/I$, and will have the
      representative $s + x \in R$. Since $(s + x) p(x) = sp_0 +
      (sp_1 + p_0)x$, and $  (s + x) q(x) = sq_0 + (sq_1 + q_0)x$, by
      these identification,
      \begin{align*}
        \phi(sp, q) = \phi((s + x) p, q) &= \frac{sp_0q_1}{2} \mod 2 \\
        &=s\phi(p, q) \mod 2
      \end{align*}
      and
      \begin{align*}
        \phi(p, sq) = \phi(p, (s + x)q) &= \frac{p_0 (sq_1 + q_0)}{2} \mod 2 \\
        &= \frac{sp_0q_1}{2} \mod 2 + \frac{p_0q_0}{2} \mod 2
      \end{align*}
      Since we assumed $p, q \in I$, $p_0, q_0$ are multiples of $2$,
      therefore $4|p_0q_0$, and hence $\frac{p_0q_0}{2} \mod 2 = 0$.
      Thus we get
      \begin{align*}
        \phi(p, sq) = \frac{sp_0q_1}{2} \mod 2  = s \phi(p, q)
      \end{align*}
      Thus we get that $\phi$ is a bilinear map $ I \times I \to
      \mathbb{Z}/ 2 \mathbb{Z}$.

      \part Notice that $R$-module homomorphism $I \otimes I \to
      \mathbb{Z}/2 \mathbb{Z}$ corresponding to the bilinear map
      $\phi$ above does precisely this by the universal property of
      the tensor products.

      \part By the universal property of the tensor products, we just
      need to find a bilinear map $\phi: I \times I \to
      \mathbb{Z}/2 \mathbb{Z}$ such that $\phi(2, x) \neq \phi(x,
      2)$. Taking $\phi$ to be the bilinear map above in the first
      part of the problem, we see that $\phi(2, x) = 1 \neq 0 =
      \phi(x, 2)$. Hence $2 \otimes x \neq x \otimes 2$.
    \end{parts}
  \end{solution}

\end{questions}
\printbibliography[heading=bibintoc]
\end{document}
