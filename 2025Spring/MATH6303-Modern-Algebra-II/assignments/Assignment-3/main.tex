% initial settings
\documentclass[12pt]{exam}
\usepackage{geometry}
\usepackage{graphicx}
\usepackage{enumitem}
\usepackage[usenames,dvipsnames]{xcolor}
\usepackage[backend=biber, style=alphabetic]{biblatex}
\usepackage{url,hyperref}

\usepackage{amsmath} % math symbols, matrices, cases, trig functions,
% var-greek symbols.
\usepackage{amsfonts} % mathbb, mathfrak, large sum and product symbols.
\usepackage{amssymb} % extended list of math symbols from AMS.
% https://ctan.math.washington.edu/tex-archive/fonts/amsfonts/doc/amssymb.pdf
\usepackage{amsthm} % theorem styling.
\usepackage{mathrsfs} % mathscr fonts.
\usepackage{yhmath} % widehat.
\usepackage{empheq} % emphasize equations, extending 'amsmath' and 'mathtools'.
\usepackage{bm} % simplified bold math. Do \bm{math-equations-here}

% geometry of paper
\geometry{
  a4paper, % 'a4paper', 'c5paper', 'letterpaper', 'legalpaper'
  asymmetric, % don't swap margins in left and right pages. as
  % opposed to 'twoside'
  centering, % to center the content between margins
  bindingoffset=0cm,
}

% hyprlink settings
\hypersetup{
  colorlinks = true,
  linkcolor = {red!60!black},
  anchorcolor = red,
  citecolor = {green!50!black},
  urlcolor = magenta,
}

% theorem styles
\theoremstyle{plain} % default; italic text, extra space above and below
\newtheorem{theorem}{Theorem}[section]
\newtheorem{proposition}{Proposition}[section]
\newtheorem{lemma}{Lemma}[section]
\newtheorem{corollary}{Corollary}[theorem]

\theoremstyle{definition} % upright text, extra space above and below
\newtheorem{definition}{Definition}[section]
\newtheorem{example}{Example}[section]

\theoremstyle{remark} % upright text, no extra space above or below
\newtheorem{remark}{Remark}[section]
\newtheorem*{note}{Note} %'Notes' in italics and without counter

% renewcommands for counters
\newcommand{\propositionautorefname}{Proposition}
\newcommand{\definitionautorefname}{Definition}
\newcommand{\lemmaautorefname}{Lemma}
\newcommand{\remarkautorefname}{Remark}
\newcommand{\exampleautorefname}{Example}

\addbibresource{~/Books/Research/research.bib}

\begin{document}

\title{MATH6303 Modern Algebra II \\ Homework  III}

% author list
\author{
  Joel Sleeba \\
}

\maketitle
\printanswers
\unframedsolutions

\begin{questions}

  \question
  \begin{solution}
    We know that the roots of $x^p - 2$ are precisely,
    $2^{\frac{1}{p}}e^{i\frac{ 2\pi j}{p}}$, where $0 \le j \le p-1$.
    Hence the splitting field of $x^p - 2$ is
    $\mathbb{Q}(2^{\frac{1}{p}}, e^{i\frac{2\pi}{p}})$. Since
    $2^{\frac{1}{p}}$ is a root of the irreducible (by Eisenstein
    with 2) polynomial $x^p - 2$, and $e^{i \frac{2 \pi}{p}}$ is a
    root of the irreducible cyclotomic polynomial $x^{p-1} + x^{p-2}
    + \ldots + x + 1$, we see that the degree of the extension
    \begin{align*}
      [\mathbb{Q}(2^{\frac{1}{p}}, e^{i \frac{2 \pi}{ p}}) :
      \mathbb{Q}] = [\mathbb{Q}(2^{\frac{1}{p}}, e^{i \frac{2 \pi}{
      p}}) : \mathbb{Q}(2^{\frac{1}{p}})]
      [\mathbb{Q}(2^{\frac{1}{p}}) : \mathbb{Q}] = p(p-1)
    \end{align*}
    Hence the Galois group, $\textrm{Aut}(\mathbb{Q}(2^{\frac{1}{p}},
    e^{i \frac{2 \pi}{ p}})/ \mathbb{Q})$ has $p(p-1)$ elements,
    where each $\tau \in \textrm{Aut}(\mathbb{Q}(2^{\frac{1}{p}},
      e^{i \frac{2 \pi}{ p}})/ \mathbb{Q}$ is uniquely determined by the image
      of $2^{\frac{1}{p}}$, and $e^{i \frac{2 \pi}{p}}$.
      Also notice that for any $\tau \in
      \textrm{Aut}(\mathbb{Q}(2^{\frac{1}{p}},
      e^{i \frac{2 \pi}{ p}})/ \mathbb{Q})$,
      \begin{align*}
        \tau(2^{\frac{1}{p}})
        \in \{ 2^{\frac{1}{p}} e^{i \frac{2\pi j}{p}}  \ : \  0 \le j
        < p   \} \\
        \tau(e^{i \frac{2\pi}{p}})
        \in \{ e^{i \frac{2\pi j}{p}}  \ : \  0 < j < p   \}
      \end{align*}

      We claim that $\textrm{Aut}(\mathbb{Q}(2^{\frac{1}{p}},
      e^{i \frac{2 \pi}{ p}})/ \mathbb{Q})$ is generated by the two
      automorphisms, $\tau_1, \tau_2$, where
      \begin{align*}
        \tau_1(2^{\frac{1}{p}}) = 2^{\frac{1}{p}}e^{ i
        \frac{2\pi}{p}}, \ \tau_1(e^{ i \frac{2\pi}{p}}) = e^{ i
        \frac{2\pi}{p}} \\
        \tau_2(2^{\frac{1}{p}}) = 2^{\frac{1}{p}}, \ \tau_2(e^{ i
        \frac{2\pi}{p}}) = e^{ i \frac{4\pi}{p}}
      \end{align*}
      and then further extended to the whole field $\mathbb{Q}(2^{\frac{1}{p}},
      e^{i \frac{2 \pi}{ p}})$. To see this, if $\tau \in
      \textrm{Aut}(\mathbb{Q}(2^{\frac{1}{p}},
      e^{i \frac{2 \pi}{ p}})/ \mathbb{Q})$ such that
      $\tau(2^{\frac{1}{p}}) = 2^{\frac{1}{p}} e^{i \frac{2\pi
      j}{p}}$, and $\tau(e^{i \frac{2\pi}{p}}) = e^{i \frac{2\pi
      k}{p}}$, then we can verify that $\tau = \tau_1^j \tau_2^k$. It
      is easy to see that $\tau_1^p$ is the identity automorphism.
      We can also verify that, $\tau_2^{p-1}( e^{i \frac{2\pi}{p}}) =
      e^{i \frac{2\pi}{p}}$.
      Thus we get that
      \begin{align*}
        \textrm{Aut}(\mathbb{Q}(2^{\frac{1}{p}}, e^{i \frac{2 \pi}{
        p}})/ \mathbb{Q}) = \langle \tau_1, \tau_2 : \tau_1^p = 1 =
        \tau_2^{p-1} , \tau_1 \tau_2 = \tau_2 \tau_1^2 \rangle
      \end{align*}

      Now we notice that for the given matrix group, the matices $ A =
      \begin{pmatrix}%{c c}
        1 & 1 \\
        0 & 1
      \end{pmatrix}$, and $ B=
      \begin{pmatrix}%{c c}
        2 & 0\\
        0 & 1
      \end{pmatrix}$
      generate the whole matrix as
      \begin{align*}
        \begin{pmatrix}%{c c}
          1 & 1\\
          0 & 1
        \end{pmatrix}^n =
        \begin{pmatrix}%{c c}
          1 & n\\
          0 & 1
        \end{pmatrix},  \quad
        \begin{pmatrix}%{c c}
          2 & 0\\
          0 & 1
        \end{pmatrix}^m =
        \begin{pmatrix}%{c c}
          2^m & 0\\
          0 & 1
        \end{pmatrix}
      \end{align*}
      and since any element in $\mathbb{F}_p$ has a form $2^m$, we get
      \begin{align*}
        \begin{pmatrix}%{c c}
          1 & n\\
          0 & 1
        \end{pmatrix}
        \begin{pmatrix}%{c c}
          2^m & 0\\
          0 & 1
        \end{pmatrix} =
        \begin{pmatrix}%{c c}
          2^m & n \\
          0 & 1
        \end{pmatrix}
      \end{align*}
      Again we see that $|A| = p$, and Fermat's little theorem give
      $|B| = p-1$. Moreover, observing $AB = BA^2$, we see that
      $\langle  A, B \rangle $, satisfy all the relations of the
      above group and hence we see that they are isomorphic.
    \end{solution}

    \question
    \begin{solution}
      We notice that the roots of $x^2
      - 14x + 9$ are $7 \pm \sqrt{40}$
      using the quadratic formula. Hence the roots of $x^4 - 14x^2 + 9$
      are precisely $\pm \sqrt{7 \pm \sqrt{40}}$. Moreover, notice that
      \begin{align}
        \label{align:2}
        \sqrt{7 + \sqrt{40}} \times \sqrt{7 - \sqrt{40}} = \sqrt{7^2 -
        40} = \sqrt{9} = 3
      \end{align}
      Hence the splitting field of the polynomial is
      $K = \mathbb{Q}(\sqrt{7 + \sqrt{40}})$. Thus
      $[\mathbb{Q}(\sqrt{7 + \sqrt{10}}): \mathbb{Q}] \le 4$. Hence
      the Galois group of the splitting field is of order $\le 4$.
      Now let $\tau \in \textrm{Aut}(K/\mathbb{Q})$. Then $\tau(\sqrt{7
      + \sqrt{40}})$ can be mapped to one of $\pm \sqrt{7 \pm
      \sqrt{40}}$. Since each of them give
      distinct automorphisms, we see that the Galois group contain
      atleast $4$ distinct automorphisms. This combined with the
      above inference, we get that $\textrm{Aut}(K/\mathbb{Q})$ has
      exactly $4$ elements.

      Let's look at these by cases. If $\tau$ fixes
      $\sqrt{7 + \sqrt{40}}$, since $\sqrt{7 - \sqrt{40}} =
      \frac{3}{\sqrt{7 + \sqrt{40}}}$, it fixes $\sqrt{7 - \sqrt{40}}$
      and therefore the whole $K$, and becomes just the identity map.
      By the same reasoning, any element of
      $\textrm{Aut}(K/\mathbb{Q})$ which fixes either of $\pm \sqrt{7
      \pm \sqrt{40}}$ fixes the whole field $K$.

      Let $\tau(\sqrt{7 + \sqrt{40}}) =\sqrt{7 - \sqrt{40}}$, then by
      \autoref{align:2}, $\tau(\sqrt{7 - \sqrt{40}}) =\sqrt{7 +
      \sqrt{40}}$. This forces $\tau^2 = e$.

      Now let $\tau(\sqrt{7 + \sqrt{40}}) = -\sqrt{7 - \sqrt{40}}$,
      then again \autoref{align:2} forces $\tau(\sqrt{7 - \sqrt{40}}) =
      -\sqrt{7 + \sqrt{40}}$. Thus, again $\tau^2 = e$.

      By a similar reasoning, we'll get that $\tau(\sqrt{7 +
      \sqrt{40}}) = -\sqrt{7 + \sqrt{40}}$ also gives $\tau^2 = e$.

      Since we exhausted all the possible automorphisms, we see that
      $\tau^2 = e$ for all $\tau \in \textrm{Aut}(K/\mathbb{Q})$. Hence
      the Galois group must be isomorphic to $V_4$.
    \end{solution}

    \question
    \begin{solution}
      We get that $x^4 - 4x^2 + 2$ is an irreducible polynomial (
      Eisenstein with $p = 2$) with root $\sqrt{2 + \sqrt{2}}$. All
      roots of $x^4 - 4x^2 + 2$ are $\pm \sqrt{2 \pm \sqrt{2}}$. Moreover,
      \begin{align*}
        \sqrt{2 + \sqrt{2}} \times \sqrt{2 - \sqrt{2}} = \sqrt{2}
      \end{align*}
      shows that the splitting field of $x^4 - 4x^2 + 2$, is $
      K = \mathbb{Q}(\sqrt{2 + \sqrt{2}}, \sqrt{2})$. Moreover we notice
      that $\sqrt{2 + \sqrt{2}} \not\in \mathbb{Q}(\sqrt{2})$, while
      $\sqrt{2 + \sqrt{2}}$ is a root of the polynomial $x^2 - (2 +
      \sqrt{2})$, which is again irreducible (by having no roots) in
      $\mathbb{Q}(\sqrt{2})$.
      Thus we get that
      \begin{align*}
        [\mathbb{Q}( \sqrt{2 + \sqrt{2}}, \sqrt{2}) : \mathbb{Q}] =
        [\mathbb{Q}( \sqrt{2 + \sqrt{2}}, \sqrt{2}) :
        \mathbb{Q}(\sqrt{2})] [ \mathbb{Q}(\sqrt{2}): \mathbb{Q}] = 2
        \times 2 = 4
      \end{align*}
      Thus we see that the Galois group of $x^4 - 4x^2 + 2$ is of order
      $4$. Now consider $\tau \in \textrm{Aut}(K / \mathbb{Q})$ such that
      \begin{align*}
        \tau(\sqrt{2 + \sqrt{2}}) = \sqrt{2 - \sqrt{2}}, \ \textrm{and}
        \ \tau(\sqrt{2 - \sqrt{2}}) = - \sqrt{2 + \sqrt{2}}
      \end{align*}
      We notice that the above definition gives $\tau(2 + 2) = 2 + 2$,
      which in turn fixes every $r \in \mathbb{Q}$. Thus $\tau$ extends
      to a well defined automorphism of $K$, which fixes $\mathbb{Q}$.
      Then it is easy to verify that $|\tau| = 4$. Thus we see that
      $\textrm{Gal}(K/\mathbb{Q}) = \mathbb{Z}_4$.

    \end{solution}

    \question
    \begin{solution}
      Let $K$ be the splitting field of the polynomial $x^p - x - a$.
      Since this is a polynomial of $p$ degrees, it can have atmost
      $p$ roots in $K$. Let $\alpha \in K$ be a root. Then $\alpha^p
      - \alpha - a  = 0$. We observe that $\alpha + 1$ is also a root
      since $(\alpha + 1)^p = \alpha^p + 1$ in a field of
      characteristic $p$, and
      \begin{align*}
        (\alpha + 1)^p - (\alpha + 1) - a = (\alpha^p + 1) - (\alpha
        + 1) - a = \alpha^p - \alpha - a  = 0
      \end{align*}
      Thus we see that if $\alpha$ is a root of the polynomial, then
      $\alpha + i$ is a root for all $0 < i < p$. Moreover all of
      them are distict since $\alpha + i = \alpha + j$ iff $p|i-j$.
      Hence $ \alpha, \alpha + 1, \alpha + 2, \ldots \alpha+ p -1$
      are all the roots of the polynomial $ x^p - x - a$. Thus we see
      that the splitting field, $K = \mathbb{F}_p(\alpha)$.

      We know that the Galois group $ \textrm{Aut}(K/\mathbb{F}_p)$
      permutes the roots of the polynomial $x^p - x - a$. Since all the roots
      of $x^p - x - a$ are of the form $\alpha + i$, where $ i \in
      \mathbb{F}_p$, $\tau \in \textrm{Aut}(K/\mathbb{F}_p)$ is
      uniquely determined by $\tau(\alpha)$. Let $\tau_1 \in
      \textrm{Aut}(K/\mathbb{F}_p)$ such that $\tau(\alpha) = \alpha
      + 1$. We claim that $\textrm{Aut}(K/\mathbb{F}_p) = \langle
      \tau_1 : \tau_1^p = 1 \rangle$. If $\tau_i \in \textrm{Aut}(K/
      \mathbb{F}_p)$ such that $\tau_i(\alpha) = \alpha + i$, then
      $\tau_i = (\tau_1)^i$. Thus the automorphism group is generated
      by $\tau_1$, and therefore it is cyclic.
    \end{solution}

  \end{questions}
  \printbibliography[heading=bibintoc]
  \end{document}
