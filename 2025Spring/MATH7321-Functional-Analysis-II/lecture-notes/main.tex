\documentclass[12pt]{report}

\usepackage{geometry} % automatic papersizes, margins.
\usepackage{makeidx} % 'makeidx' make and show index
\usepackage{enumitem} % itemize, enumerate, description.
\usepackage{hyperref} % hyperlinks, cross-references.
\usepackage{xcolor} % foreground and background color management.
% Better color mixing compared to 'color'
\usepackage{graphicx} % provide options for \includegraphics. Builds
% on 'graphic'
\usepackage{caption} % better control over captions of figures and equations.
\usepackage{appendix} % extra control over appendix
\usepackage[backend=biber, style=alphabetic]{biblatex} % better than
% bibtex, people say.
\usepackage{tocbibind} % add ToC/Bibliography/Index to ToC
\usepackage{marginnote} % for notes in margin
\usepackage{amsmath} % math symbols, matrices, cases, trig functions,
% var-greek symbols.
\usepackage{amsfonts} % mathbb, mathfrak, large sum and product symbols.
\usepackage{amssymb} % extended list of math symbols from AMS.
% https://ctan.math.washington.edu/tex-archive/fonts/amsfonts/doc/amssymb.pdf
\usepackage{amsthm} % theorem styling.
\usepackage{mathrsfs} % mathscr fonts.
\usepackage{yhmath} % widehat.
\usepackage{empheq} % emphasize equations, extending 'amsmath' and 'mathtools'.
\usepackage{bm} % simplified bold math. Do \bm{math-equations-here}
\usepackage{tikz} % for tikz diagrams
% \usepackage{tikz-cd} % commutative diagrams.
\usepackage{verbatim} % for concealing solutions

\geometry{
  a4paper, % 'a4paper', 'c5paper', 'letterpaper', 'legalpaper'
  asymmetric, % don't swap margins in left and right pages. as
  % opposed to 'twoside'
  centering, % to center the content between margins
  bindingoffset=0cm,
}

\hypersetup{
  colorlinks,
  linkcolor={red!75!black},
  citecolor={blue!50!black},
  urlcolor={blue!80!black}
}

\theoremstyle{plain} % default; italic text, extra space above and below
\newtheorem{theorem}{Theorem}[section]
\newtheorem{proposition}{Proposition}[section]
\newtheorem{lemma}{Lemma}[section]
\newtheorem{corollary}{Corollary}[theorem]
\newtheorem{problem}{Problem}[section]
\newtheorem{solution}{Solution}[section]

\theoremstyle{definition} % upright text, extra space above and below
\newtheorem{definition}{Definition}[section]
\newtheorem{example}{Example}[section]
\newtheorem{exercise}{Exercise}[section]

\theoremstyle{remark} % upright text, no extra space above or below
\newtheorem{remark}{Remark}[section]
\newtheorem*{note}{Note} %'Notes' in italics and without counter

\newcommand{\propositionautorefname}{Proposition}
\newcommand{\lemmaautorefname}{Lemma}
\newcommand{\corollaryautorefname}{Corollary}
\newcommand{\problemautorefname}{Problem}
\newcommand{\definitionautorefname}{Definition}
\newcommand{\exampleautorefname}{Example}
\newcommand{\remarkautorefname}{Remark}
\newcommand{\noteautorefname}{Note}
\newcommand{\exerciseautorefname}{Exercise}
\newcommand{\solutionautorefname}{Solution}

\addbibresource{articles.bib}

\begin{document}
\title{Functional Analysis II - MATH7321}

%\showsolutionstrue
%\showsolutionsfalse %If need to hide solutions

\author{
  Joel Sleeba \\
  joelsleeba1@gmail.com \\
}

\maketitle

\pagenumbering{roman} \setcounter{page}{2}
\tableofcontents
\pagenumbering{arabic} \setcounter{page}{1}

% TeX_root = ../main.tex

\marginnote{\scriptsize 26/08/2025 }

Office Hours : Tuesday 10 - 11 AM, Wednesday 1 - 2 PM

\section{Introduction}

Let $A$ be a $m \times n$ matrix and $D$ a diagonal $n \times n$
matrix with entries $d_1 , d_2 , \ldots , d_n$,
\begin{align*}
  A =
  \begin{bmatrix}%{c c c c}
    | & | & \cdots &  | \\
    a_1 & a_2 & \cdots &  a_n \\
    | & | & \cdots &  | \\
  \end{bmatrix}
\end{align*}
Then
\begin{align*}
  AD =
  \begin{bmatrix}%{c c c c}
    | & | & &  | \\
    d_1a_1 & d_2a_2 & \cdots &  d_na_n \\
    | & | & &  | \\
  \end{bmatrix}
\end{align*}
and if
\begin{align*}
  B =
  \begin{bmatrix}%{c c c c}
    - & b_1 & - \\
    - & b_2 & - \\
    & \vdots &  \\
    - & b_n & - \\
  \end{bmatrix}
\end{align*}
then
\begin{align*}
  DB =
  \begin{bmatrix}%{c c c c}
    - & d_1b_1 & - \\
    - & d_2b_2 & - \\
    & \vdots &  \\
    - & d_nb_n & - \\
  \end{bmatrix}
\end{align*}

\textcolor{red}{Do the same for upper triangular matrices and add
some context to the multiplications}.

Notice that every time you left multiply, you play with column, and
when you right multiply, you play with the rows.

\begin{exercise}
  Let $A$ be an $n \times n$ matrix over $\mathbb{C}$. Let $\omega =
  e^{ \frac{2\pi i}{n}}$. Then prove that
  \begin{align*}
    A' = \frac{1}{n} \sum_{k = 0}^{n} (U^*)^kAU^k
  \end{align*}
  preserve all the diagonal entries of $A$ and kills the rest of
  entries. That is $A^\prime = \text{Diag}(A)$
\end{exercise}

\subsection{Review of Linear Algebra}

\begin{itemize}
  \item Rank-Nullity Theorem
  \item Orthogonality
  \item Orthogonal projection is the closest point on the subspace
    from the given vector.
\end{itemize}

% TeX_root = ../main.tex

\marginnote{\scriptsize 28/08/2025 }

\begin{definition}
  If $A \in M_n$, $A = (a_{i, j})_{i, j = 1}^n$, we let
  \begin{align*}
    \rm{trace}(A) = \sum_{j = 1}^{n} a_{jj}
  \end{align*}
  The determinant is
  \begin{align*}
    \textrm{det}(A) = \sum_{ \sigma \in  S_n} \textrm{sgn}(\sigma)
    a_{1, \sigma(1)} a_{2, \sigma(2)} \ldots a_{n, \sigma(n)}
  \end{align*}
\end{definition}

\begin{remark}
  If $A = [a_1  a_2  \ldots  a_n]$, then $det(A) = f( a_1 , a_2 ,
  \ldots , a_n)$ is the only function that is linear in each $a_i$,
  alternating (swapping columns doesn't alter the value), and
  normalized ($det(I) = 1$).

  This is useful to show that for $A, B \in M_n$, $det(AB) = det(A) det(B)$.

  Moreover if $A =
  \begin{bmatrix}
    B & C \\
    0 & D
  \end{bmatrix}$, then $det(A) = det(B) det(D)$.

  We also have
  \begin{align*}
    det(A) = \sum_{i, j = 1}^{n} (-1)^{i + j} a_{i, j} det(A_{i, j})
  \end{align*}
  Where $A_{i, j}$ is the submatrix with $i$th row and $j$th column
  removed from $A$.
\end{remark}

\subsection{Eigenvalues and Eigenvectors}

\begin{definition}
  Eigenvalue, Eigenvector, Spectrum of a matrix
\end{definition}

\subsection{Similarity}

\begin{definition}
  A matrix $B \in M_n$ is similar to $A \in M_n$, if there is an
  invertible $S \in M_n$ such that $B = S^{-1}AS$. This defines an
  equivalence relation.
\end{definition}

\begin{theorem}
  If $A, B \in M_n$ are similar. Then their characteristic polynomial
  $P_A = P_B$.
\end{theorem}

\begin{remark}
  characteristic polynomial i snot characteristic upto similarity, because
  \begin{align*}
    A =
    \begin{pmatrix}%{c c}
      0 & 1 \\
      0 & 0
    \end{pmatrix} \text{ and } B =
    \begin{pmatrix}%{c c}
      0 & 0\\
      0 & 0
    \end{pmatrix}
  \end{align*}
\end{remark}

% TeX_root = ../main.tex

\marginnote{\scriptsize 02/09/2025 }

\subsection{Diagonazability}

Can we find conditions for diagonalizability.

\begin{theorem}
  Let $A \in M_n(\mathbb{C})$, $p_A(t) = \prod_{j = 1}^{n} (t -
  \lambda_j)$, and $\lambda_i \neq \lambda_j$ for $j \neq k$, then
  $A$ is diagonalizable.
\end{theorem}
\begin{proof}
  We'll show that there's a linearly independent set of $n$
  eigenvectors. Let $x_j \in \mathbb{C}^n$ such that $Ax_j =
  \lambda_j x_j$. If $\{x_1 , x_2 , \ldots , x_n  \}$ were linearly
  dependent, then there is a linear combination
  \begin{align*}
    \alpha_1x_{j_1} + \alpha_2x_{j_2} + \ldots + \alpha_rx_{j_r} = 0
  \end{align*}
  with $ r \le n$, and all $\alpha_j \neq 0$. Let $r$ be smallest
  such $r \le n$, and assume with possible renumbering that $j_i =
  i$. Then applying $ A$ to the linear combination gives us
  \begin{align*}
    A( \alpha_1x_{1} + \alpha_2x_{2} + \ldots + \alpha_nx_{n})
    =\alpha_1 \lambda_1 x_{1} + \alpha_2 \lambda_2 x_{2} + \ldots +
    \alpha_n \lambda_n x_{n}  = 0
  \end{align*}
  multiplying the previous equation with $\lambda_r$ and then
  subtracting gives us
  \begin{align*}
    \alpha_1 (\lambda_1 - \lambda_r) x_{1} + \alpha_2 (\lambda_2 -
    \lambda_r) x_{2} + \ldots + \alpha_r (\lambda_r - \lambda_r) x_{r}  = 0
  \end{align*}
  which contradicts the minimality of $r$.
\end{proof}

Unfortunately this is just a sufficient condition, as it excludes the
following matrix.

\begin{align*}
  \begin{bmatrix}%{c c c}
    0 & 0 & 0\\
    0 & 0 & 0\\
    0 & 0 & 1
  \end{bmatrix}
\end{align*}

\begin{definition}
  If for $A \in M_n(\mathbb{C})$,
  \begin{align*}
    p_A(t) = (t - \lambda_1)^{m_1} ( t - \lambda_2)^{m_2} \ldots (t -
    \lambda_r)^{m_r}
  \end{align*}
  then we say that $\lambda_j$ has algebraic multiplicty $m_j$. We
  call $\textrm{null}(\lambda_jI - A)$, the geometric multiplicity of
  $\lambda_j$
\end{definition}

\begin{lemma}
  If $A \in M_n$ has eigenvalue $\lambda$, and $ p_A(t) = (t -
  \lambda)^m q(t)$, with $q(\lambda) = 0$, then $r =
\textrm{nul}(\lambda I - A)) \le m$
\end{lemma}
\begin{proof}
Choose a basis $\{ x_1 , x_2 , \ldots , x_r \}$ of veignevectors,
spanning $E_\lambda = \{ x \in \mathbb{C}^n \ : \ Ax = \lambda x \}$.
Complete it to a basis $\{ x_1 , x_2 , \ldots , x_n \}$ of $
\mathbb{C}^n$.Let $S = [x_1 , x_2 , \ldots , x_n]$.

Then $AS = [ \lambda x_1 , \lambda x_2 , \ldots  \lambda x_r,
y_{r+1}, \ldots y_n]$ with some vectors $y_{r+1} , \ldots , y_n$.
Then $S^{-1}AS = $\textcolor{red}{verify}, and we get
\begin{align*}
  \textrm{det}(tI - A) & = \textrm{det}(tI - S^{-1}AS) \\
  &= ( t - \lambda)^r \textrm{det}(t - C)
\end{align*}

Thus we conclude that algebraic multiplicity of $\lambda$ is at least
equal to $r$.
\end{proof}

\begin{remark}
See that the sum of all the algebraic multiplicity of the eigenvalues
of $A \in M_n(\mathbb{C})$ is $n$.
\end{remark}

\begin{theorem}
The matrix $A \in M_n(\mathbb{C})$ is diagonalizable if and only if
the algebraic and geometric multiplicities are equal for each eigenvalue.
\end{theorem}
\begin{proof}
We note that given two eigenvalues $\lambda_j \neq \lambda_k$, then
their eigenspaces $E_{i}, E_{j}$ intersect trivially. Thus if $\{ v_1
, v_2 , \ldots , v_{r_1} \}$ and $\{ u_1 , u_2 , \ldots , u_{r_2} \}$
form a basis for $E_{\lambda_1}$ and $E_{\lambda_2}$ respectively,
then $\{ v_1 , v_2 , \ldots , v_{r_1}, u_1 , u_2 , \ldots , u_{r_2}
\}$ is linearly independent. Iterating this way, we get a basis for
$E_1 + E_2 \ \ldots  + E_n$ with dimension $r = \sum_{i = 1}^{k} r_i$.

If algebraic and geometric multiplicities equal then $r = n$, and we
have a basis of eigenvectors. Otherwise if $r < n$, then we do not
have such a basis of eigenvectors. And since existence of a basis of
eigenvectors characterizes diagonalizability, this characterizes
diagonalizability.
\end{proof}

Next lecture, we'll look when multiple matrices can be simultaneously
diagonalizable with the same $S$ matrix.

% TeX_root = ../main.tex

\marginnote{\scriptsize 04/09/2025 }

\subsection*{Warm Up}
Assume we know that $A$ is diagonalizable. Let $p_0, p_1 , p_2 ,
\ldots , p_n \in \mathbb{C}$ and consider
\begin{align*}
  B: = P(A) = p_0 I + p_1 A + p_2A^2 + \ldots + p_nA^n
\end{align*}
Is $B$ diagonalizable?
\begin{proof}
  Yes. Because if $A = S^{-1}DS$, then $A^n = S^{-1}D^n S$, and therefore
  \begin{align*}
    B = S^{-1} (p_0 I  + p_1D + p_2D^2 + \ldots + p_nD^n)S
  \end{align*}
\end{proof}

\subsection{Simultaneous diagonalization}

\begin{theorem}
  Let $A, B$ be diagonalizable. Then $AB = BA$ if and only if they
  are simultaneously diagonalizable by the same $S$.
\end{theorem}
\begin{proof}
  Let $D_A = S^{-1} A S$, and $B^\prime = S^{-1} B S$, where $D_A$ is
  a diagonal matrix. Without loss of generality, assume that common
  eigenvalues appear together in $D_A$. If not choose $S$ with an
  additional permutation of the rows.

  Assuming $AB = BA$, we get
  \begin{align*}
    D_AB^\prime &= S^{-1} ASS^{-1}BS \\
    &= S^{-1}ABS \\
    &= S^{-1}BAS \\
    &= S^{-1}BSS^{-1}AS \\
    &= B^\prime D_A
  \end{align*}
  If $B^\prime = [b_{i, j}^\prime]_{i, j = 1}^n$, then by $D_A
  B^\prime = B^\prime D_A$, from the diagonal structure of $D_A$, we get
  \begin{align*}
    \tilde{\lambda}_ib_{i, j}^\prime = b_{i, j}^\prime \tilde{\lambda}_j
  \end{align*}
  where $\tilde{\lambda}_i$ is the $i$-th diagonal entry on $D_A$.
  So, we have
  \begin{align*}
    (\tilde{ \lambda}_i - \tilde{\lambda}_j) b_{i, j}^\prime = 0
  \end{align*}
  which shows that if $\tilde{ \lambda}_i \neq \tilde{\lambda}_j$,
  then $b_{i, j}^\prime = 0$. Thus we get that
  \begin{align*}
    B^\prime =
    \begin{bmatrix}%{c c c c}
      B_1^\prime &  &  & \\
      & B_2^\prime &   & \\
      &  &  & \ddots  & \\
      &  & &   & B_r^\prime \\
    \end{bmatrix}
  \end{align*}

  Since $B$ and $B^\prime$ are diagonalizable, so is each
  $B_i^\prime$. Taking matrices $T_1 , T_2 , \ldots , T_r$ that
  diagonalize $B^\prime_1 ,  B^\prime_2 , \ldots , B^\prime_r$
  respectively, let
  \begin{align*}
    T =
    \begin{bmatrix}%{c c c c}
      T_1&  &  &  \\
      & T_2 &  &  \\
      &  & \ddots &  \\
      &  &  & T_4 \\
    \end{bmatrix}
  \end{align*}
  Then,
  \begin{align*}
    T^{-1} B T =
    \begin{bmatrix}
      T_1^{-1}B_1^\prime T_1 &  &  & \\
      & T_2^{-1}B_2^\prime T_2 &   & \\
      &  &  & \ddots  & \\
      &  & &   & T_r^{-1} B_r^\prime T_r \\
    \end{bmatrix} =
    \begin{bmatrix}%{c c c c}
      D_1^\prime&  &  &  \\
      &  D_2^\prime&  &  \\
      &  &  \ddots&  \\
      &  &  & D_r^\prime \\
    \end{bmatrix}
  \end{align*} where each $D_i^\prime$ is a diagonal block.
  Also,
  \begin{align*}
    T^{-1} D_A T =
    \begin{bmatrix}
      T_1^{-1}\lambda_1 I T_1 &  &  & \\
      & T_2^{-1}\lambda_2 IT_2 &   & \\
      &  &  & \ddots  & \\
      &  & &   & T_r^{-1} \lambda_r I T_r \\
    \end{bmatrix} = D_A
  \end{align*}
  This implies $D_A = T^{-1} S^{-1} A S T$, and $D_B = T^{-1} S^{-1}
  B ST$ are both diagonal.

  \textcolor{red}{Converse is left as an exercise}
\end{proof}

Next, we consider simultaneous diagonalization for a family of matrices.
\begin{definition}
  A family $F \subset M_n$ is a commuting family if for each $A, B
  \in F$, $AB = BA$.
\end{definition}

\begin{definition}
  A subspace $W \subset \mathbb{C}^n$ is called an $A$-invariant
  subspace for some $A \in M_n$ if $Aw \in W$ for all $w \in W$. If
  $F \subset M_n$, then $W$ is called $F$-invariant if for each $A
  \in F$, $W$ is $A$-invariant.
\end{definition}

\begin{lemma}
  If $W \subset \mathbb{C}^n$ is $A$-invariant for some $A \in M_n$,
  and suppose that $\textrm{dim}(W) \ge 1$, then there is an $x \in W
  \setminus \{  \textbf{0} \}$ such that $Ax = \lambda x$.
\end{lemma}
\begin{proof}
  We consider $B:= A|_W$. Then $B: W \to W$ has an eigenvector since
  it has atleast one eigenvalue by the fundamental theorem of algebra.
\end{proof}

\begin{lemma}
  If $F \subset M_n$ is a commuting family, then there exists an $x
  \in \mathbb{C}^n$ such that for each $A \in F$, $Ax = \lambda_A x$.
\end{lemma}
\begin{proof}
  Choose $W$ to be an $F$-invariant subspace of minimum, non-zero
  dimension. Existence of $W$ is guaranteed since we can choose $W =
  \mathbb{C}^n$.

  Next, we show that any $ x \in W \setminus \{ \textbf{0} \}$ is an
  eigenvector for each $ A \in \mathbb{F}$. Assume this is not true.
  Then there is  a $ y\in W \setminus \{ \textbf{0} \}$, and an $ A
  \in F$, such that $Ay \not\in \mathbb{C}y$. Since $W$ is
  $A$-invariant by the setup, by previous lemma, we get that there is
  a $x \in W \setminus \{ \textbf{0} \}$ such that $Ax = \lambda_x x$
  for some $\lambda \in \mathbb{C}$.

  Let $ W_0 = \{ z \in W \ : \ Az = \lambda z \}$. By $y \notin W_0$,
  we get that $W_0 \neq W$. But for any $B \in F$, by invariance of
  $W_0$, $Bx \in W$, and for $u \in W_0$,
  \begin{align*}
    A(Bu) = B(Au) =  \lambda Bu
  \end{align*}

  We observe $Bu \in W_0$, thus $B$ maps $W_0$ to $W_0$, so $ W_0$ is
  $ F$-invariant. We have derived a contradiction with the minimality of $W$.
\end{proof}

\begin{remark}
  This implies that commuting families have at least one common eigenvector
\end{remark}

\begin{definition}
  A simultaneously diagonalizable family is a family $F \subset M_n$
  such that there exists $S \in M_n$ for which $S^{-1} A S$ is
  diagonal for each $A \in F$
\end{definition}

% TeX_root = ../main.tex

\marginnote{\scriptsize 30/01/2025 }

\begin{theorem}
  Let $X$ be a compact Hausdorff space and $\mathcal{H}$ be a
  Hilbert space, and $\Psi: C(X) \to B(\mathcal{H})$ is a
  $*$-homomorphism (representation). Then there exist a unique
  spectral measure on $(X, \mathcal{B}, \mathcal{H})$ ($\mathcal{B}$
  being the Borel sigma algebra on $X$), such that
  \begin{align*}
    \Psi(f) = \int_X  f \ d E
  \end{align*}
\end{theorem}
\begin{proof}
  For every $\xi, \eta \in \mathcal{H}$, the map $ \nu : C(X) \to
  \mathbb{C}  :=  f \mapsto \langle  \Psi(f) \xi ,  \eta \rangle $ is
  a bounded linear functional on $C(X)$, hence by the reisz
  representation theorem, there exists a unique measure $\mu_{\xi,
  \eta}$ on $X$ such that
  \begin{align*}
    \langle \Psi(f) \xi ,  \eta \rangle  = \int f \ d \mu_{\xi,
    \eta} \quad f \in C(X)
  \end{align*}
  Now for any $\xi, \eta \in \mathcal{H}$, the linear map
  \begin{align*}
    \mathcal{M}(X) \to  \mathbb{C} := \phi
    \mapsto \int \phi \ d \mu_{\xi, \eta}
  \end{align*}
  is a bounded sesquilinear form about $\xi, \eta$ with the operator
  norm $\|\xi\|\|\eta\|$, which is attained when $\phi = \chi_X$. Hence
  there exist a unique $T_\phi \in B(\mathcal{H})$ such that
  \begin{align*}
    \langle T_\phi \xi ,  \eta \rangle = \int \phi \ d \mu_{\xi, \eta}
  \end{align*}

  Now we show that the map $\mathscr{F}: \mathcal{M}(X)\to B(\mathcal{H})  :=
  \phi \mapsto    T_\phi$ is a $*$-representation of
  $\mathcal{M}(X)$. Linearity is obvious, though
  \textcolor{red}{verify}. Argument for $T_{\overline{ \phi}} =
  T_{\phi}^*$ is as in the last theorem.
  Observe that for every $f \in C(X)$, $T_f = \Psi(f)$.
  \textcolor{red}{Take inner product with $\xi, \eta$}.
  \marginnote{ \scriptsize \it \textcolor{red}{verify the rest}}

  To show multiplicativity of the above map, let $f \in C(X), \phi
  \in \mathcal{M}(X)$. Then $\forall \xi, \eta \in
  \mathcal{H}$,
  \begin{align*}
    \marginnote{ \scriptsize \it \textcolor{red}{verify the 2nd step}}
    \langle T_{\phi f}(\xi), \eta \rangle  &= \int \phi f \ d \mu_{\xi, \eta} \\
    &= \int \phi \ d \mu_{\Psi(f) \xi, \eta} \\
    &= \langle T_\phi \Psi(f) \xi ,  \eta \rangle\\
    &= \langle  T_\phi T_f \xi ,  \eta \rangle
  \end{align*}
  shows that $\mathscr{F}$ is multiplicative if one of the functions
  are in $C(X)$.
  More generally, if $\phi, \psi \in \mathcal{M}(X)$, choose a net
  $f_i \in C(X)$  such that $f_i$ converges to $\phi$ weak * in
  $C(X)^{**} = \mathcal{M}(x)$, and $\|f_i\| \to \|\phi\|$
  (\textcolor{red}{This is
      guaranteed by Goldstein's
  theorem}) and $\|f_i\| \le \|\phi\|$. Now for any $\xi, \eta \in
  \mathcal{H}$, we have
  \begin{align*}
    \langle T_{\phi, \psi} \xi ,  \eta \rangle = \int \phi \psi \ d
    \mu_{\xi, \eta} = \lim_{i} \int \phi f_i \ d \mu_{\xi, \eta} =
    \lim_i \langle T_{\phi f_i} \xi ,  \eta \rangle = \lim_i \langle
    T_\phi T_{f_i} \xi , \eta \rangle
  \end{align*}
  which converge to $\langle T_\phi T_\psi \xi ,  \eta \rangle$.
  \textcolor{red}{verify}. Thus $\mathscr{F}$ is a $*$-representation
  of $\mathcal{M}(X)$.

  Now define $E : \Sigma \to  B(\mathcal{H}) :=   A \mapsto
  T_{\chi_A}$. Obviously $E$ satisfy all the first three properties
  of a spectral measure easily.

  See also that the map $E$ is countably additive. Let $(A_n)$ be a
  family of pairwise disjoint measurable sets of $X$
  and let $A = \cup_{n = 1}^{\infty}A_n$. Then for every $\xi, \eta
  \in \mathcal{H}$,
  \begin{align*}
    \Big \langle E \big(\bigcup_{n \in \mathbb{N}} A_n \big) \xi ,
    \eta \Big \rangle &= \int  \chi_{\cup_{n = 1}^{\infty}A_n} \ d
    \mu_{ \xi, \eta} = \mu_{\xi, \eta}\Big( \bigcup_{n =
    1}^{\infty}A_n \Big) = \sum_{n \in \mathbb{N}} \mu_{\xi , \eta}(A_n)
  \end{align*}

\end{proof}

% TeX_root = ../main.tex

\chapter{}

\section{Integrals}

\begin{definition}
  Define the integral of a measurable simple function $s: X \to [0, \infty]$ defined in the standard form as \[
      s = \sum_{j = 1}^{n} \alpha_j \chi_{A_j}
  \]
  with $\{ \alpha_1, \alpha_2, \ldots, \alpha_n \}$ as the range of $S$ and $A_j = s^{-1}(\{ \alpha_j \})$ by \[
    \int s \ d \mu = \sum_{j = 1}^{n} \alpha_j \mu(A_j)
  \]
\end{definition}
We adopt the convention $0\times \infty = 0$ from now onwards.

\begin{lemma}
  Let $(X, \mathcal{M}, \mu)$ be a measure space. Let $A_1 , A_2 , \ldots , A_n \in \mathcal{M}$ and $B_1 , B_2 , \ldots , B_{n^\prime} \in \mathcal{M}$ with the $A_j$s are mutually disjoint, as well as $B_j$s, and \[
      \bigcup_{j = 1}^{n}A_j = X = \bigcup_{j = 1}^{n^\prime}B_j
  \]
  Let $\alpha_1 , \alpha_2 , \ldots , \alpha_n \in [0, \infty]$ and $  \beta_1 , \beta_2 , \ldots , \beta_n^\prime \in [0, \infty]$ such that \[
        t = \sum_{j = 1}^{n^\prime} \beta_j \chi_{B_j} \le s = \sum_{j = 1}^{n} \alpha_j \chi_{A_j}
  \]
  then \[
    \sum_{j = 1}^{n^\prime} \beta_j \mu(B_j) \le \sum_{j = 1}^{n} \alpha_j \mu(A_j)
  \]
\end{lemma}
\begin{proof}
  \begin{align*}
    \sum_{j = 1}^{n^\prime} \beta_j \mu(B_j) &=  \sum_{j = 1}^{n} \beta_j \mu\Big(B_j \bigcap \big(\bigcup_{l = 1}^{n} A_l\big)\Big) \\ 
    &= \sum_{j = 1}^{n^\prime} \beta_j \mu \Big(\bigcup_{l = 1}^{n}B_j \cap A_l\Big) \\ 
    &= \sum_{ j = 1}^{n^\prime} \sum_{l = 1}^{n} \beta_j\mu \Big( B_j \cap A_l\Big) \\ 
  \end{align*}
  By a similar deduction, we get that \[
    \sum_{l = 1}^{n} \alpha_j \mu(A_j) = \sum_{ l = 1}^{n} \sum_{j = 1}^{n^\prime} \alpha_l \mu(A_l \cap B_j)
  \]
  Since we know that $t \le s$, comparing the values of the function at $A_l \cap B_j$, we get that $\beta_j \le \alpha_l$. This immediately gives us our needed result.
\end{proof}

\begin{corollary}
  If a measurable simple function has two representations \[
      s = \sum_{j = 1}^{n} \alpha_j \chi_{A_j} = \sum_{j = 1}^{n^\prime} \beta_j \chi_{B_j}
  \]
  with disjoint measurable sets as before, then \[
    \int s \ d \mu = \sum_{j = 1}^{n} \alpha_j \mu(A_j) = \sum_{ j = 1}^{n^\prime} \beta_j \mu(B_j)
  \]
\end{corollary}
\begin{proof}
   Use the fact that $a = b$ is equivalent to $ a \le b$ and $ b \le a$ and use above lemma.
\end{proof}

\begin{definition}
  Let $(X, \mathcal{M}, \mu)$ be a mesurable space, $s: X \to [0, \infty]$ a measurable simple function, \[
      s = \sum_{j = 1}^{n} \alpha_j \chi_{A_j}
  \]
  with $\{ A_j \}_{j=1}^n$ disjoint, measurable, then we define for $E \in \mathcal{M}$ \[
    \int_E s \ d \mu = \sum_{j = 1}^{n} \alpha_j \mu(A_j \cap E)
  \]
\end{definition}

\begin{lemma}
   If $s, t$ are non-negative measurable, simple fucntions and $t \le s$ and $E \in \mathcal{M}$, then \[
       \int_E t \ d \mu \le \int_E s \ d \mu
   \]
\end{lemma}
\begin{proof}
  Proof is exactly like before lemma, just replacing $\mu(A_j)$ with $\mu(A_j \cap E)$.
\end{proof}

\begin{remark}
  If $s: X \to [0, \infty]$ is simple and measurable, then   \[
      \int s \ dx = \sup \{ \int_E t d \mu \ : \ 0 \le t \le s \textrm{ is measurable and simple.} \}
  \]
\end{remark}

\begin{definition}
  For $f: X \to [0, \infty]$ measurable, we define \[
    \int_E f d \mu = \sup_{\substack{ 0\le t\le f \\ t \textrm{ is simple}}} \int_E t \ d \mu
  \]
\end{definition}
\begin{example}
  We will give some examples of measurable functions.
  \begin{itemize}[]
    \item $X = \mathbb{N}, \mathcal{M} = P(\mathbb{N}), \mu$ is the counting measure. $f: \mathbb{N} \to [0, \infty]$. Then let \[
        s_N(n) = \begin{cases}
          f(n), & n \le N \\ 
          0, &\textrm{otherwise}
        \end{cases}
    \]
      Now if $\sum_{j = 1}^{\infty} f(j) \le \infty$, then $f(j) \to \infty$ as $j \to \infty$. Thus if $t \le f$ and $t$ is simple, then there is $ N \in \mathbb{N}$ such that $t(j) = 0$ for each $ j \ge N$. Then by comparison, $0 \le t \le s_n \le f$ and finally, we have \[
        \sum_{j = 1}^{\infty}  t(j) \le \sum_{ j = 1}^{\infty}  s_N(j) \le \sum_{ j = 1}^{\infty}  f(j)
      \]
      so taking supremums, we get \[
        \sup_{\substack{0\le t \le f\\ t \textrm{ is simple}}}  \sum_{j = 1}^{\infty}  t(j) = \sup_{N \in \mathbb{N}}\sum_{ j \in \mathbb{N}} s_N(j) = \sum_{ j = 1}^{\infty}  f(j)
      \]
  \end{itemize}
\end{example}


























% TeX_root = ../main.tex

\marginnote{\scriptsize 06/02/2025 }

\begin{lemma}
  Let $\mu, \mu^\prime$ be Radon probability measures on $X$, such
  that $ \mu \sim \mu^\prime$. Define $U: L^{2}(X, \mu) \to L^{2}(X,
  \mu^\prime)$ by sending
  \begin{align*}
    U : \xi \to \sqrt{\frac{d\mu}{d \mu^\prime}} \xi
  \end{align*}
  Then $U$ is a unitary, and $\forall f \in C(X)$, and
  \begin{align*}
    UM_fU^* = M^\prime_f
  \end{align*}
  where $M_f \in B(L^{2}(X, \mu))$ and $M_f^\prime \in B(L^{2}(X, \mu^\prime))$.
\end{lemma}
\begin{proof}
  To see that $U$ is an isometry, take the $L^{2}$ norms on both of
  these spaces. \textcolor{red}{verify the rest}.
\end{proof}

\begin{lemma}
  Let $\mu, \mu^\prime \in \textrm{Prob}(X)$. If $\exists$ unitary
  $U: L^{2}(\mu) \to L^{2}(\mu^\prime)$ such that $UM_fU^* = M_f$ for
  all $f \in C(X)$, then $\mu \sim \mu^\prime$.
\end{lemma}
\begin{proof}
  Let $h = |U 1|^2 \in L^{1}(\mu^\prime)$. $\forall f \in C(X)$,
  \begin{align*}
    \int f \ d \mu = \langle M_f 1 , 1 \rangle_{\mu} = \langle
    U^*M_f^\prime U 1 , 1 \rangle = \int f h \ d \mu^\prime
  \end{align*}
  Thus we get $\mu = h \mu'$. Hence we see that $\mu \sim \mu^\prime$.
\end{proof}

\begin{definition}[Strong operator topology]
  The strong operator topology is the locally convex topology on
  $B(\mathcal{H})$ for separable $\mathcal{H}$, defined by the family
  of seminorms $\{ \phi_x  \ : \  x \in \mathcal{H} \}$, where
  $\phi_x(T) := \|T x\|$.
  We have $T_i \stackrel{SOT}{\to} T$ if and only if $\|(T_i - T) x\|
  \to 0$ for all $x \in \mathcal{H}$.
  For any $x_1 , x_2 , \ldots , x_n \in \mathcal{H}$, and
  $\varepsilon > 0$, $T \in B(\mathcal{H})$, the set
  \begin{align*}
    \{  S \in B(\mathcal{H})  \ : \    \|(T - S) x_i\| < \varepsilon,
    \ \forall i = 1,2, \ldots n \}
  \end{align*}
  is an open neighborhood of $T$.
\end{definition}

\begin{definition}
  \marginnote{ \scriptsize \it $T_i \to T$ in WOT $T_i(x)
  \stackrel{w^*}{\to} T(x)$ for all $x \in \mathcal{H}$}
  The weak operator topology is the locally convex topology on
  $B(\mathcal{H})$ for separable $\mathcal{H}$, defined by the family
  of seminorms $\{ \phi_{x, y}  \ : \  x \in \mathcal{H} \}$, where
  $\phi_{ x, y}(T) := |\langle  Tx , y \rangle |$.
  We have $T_i \stackrel{WOT}{\to} T$ if and only if $|\langle (T_i -
  T) x , y \rangle |
  \to 0$ for all $x,y \in \mathcal{H}$.
  For any $x_1 , x_2 , \ldots , x_n, y_1 , y_2 , \ldots , y_n \in
  \mathcal{H}$, and
  $\varepsilon > 0$, $T \in B(\mathcal{H})$, the set
  \begin{align*}
    \{  S \in B(\mathcal{H})  \ : \  |\langle (T - S) x_i , y_i
    \rangle | < \varepsilon, \ \forall i = 1,2, \ldots n \}
  \end{align*}
  is an open neighborhood of $T$.
\end{definition}

\begin{example}
  Let $\{ f_i \} \subset  C([0, 1]) \ni f$ identified as
  multiplication operators on $  B(L^{2}([0, 1]))$. Then $M_{f_i}
  \stackrel{WOT}{\to} M_{f}$ if and only if $f_i \to f$ in weak * topology
  on $L^{\infty}([0, 1])$.
\end{example}
\begin{proof}
  \textcolor{red}{verify with intergrals and the fact with dual space
  of $L^{1}([0, 1])$}
\end{proof}

\begin{lemma}
  For any convex subset $K \subset B(\mathcal{H}),
  \overline{K}^{\textrm{SOT}} = \overline{K}^{\textrm{WOT}}$.
\end{lemma}
\begin{proof}
  Clearly $\overline{K}^{\textrm{SOT}} \subset
  \overline{K}^{\textrm{WOT}}$ since the topologies are relative.
  Let $ T \in \overline{K}^{\textrm{WOT}}$, $ x_1 , x_2 , \ldots ,
  x_n \in H$ and $ \varepsilon > 0$. Let $\mathcal{H}^n =
  \mathcal{H} \oplus \mathcal{H} \oplus \ldots \oplus \mathcal{H}$,
  and $T^n \in B(H^n)$ be defined as $T^n = T \otimes C_n$, then
  $n$-amplification of $T$. Similarly, let
  \begin{align*}
    K^n = \{ T^n  \ : \  T \in K \}.
  \end{align*}
  Then $K^n$ is a convex subset of $B(\mathcal{H}^n)$, and
  \begin{align*}
    K^n(x_1 , x_2 , \ldots , x_n) = \{ T^n(x_1 , x_2 , \ldots , x_n)
    \ : \ T \in K  \}
  \end{align*}
  is a convex subset of $\mathcal{H}^n$. Thus, the weak and norm
  closures of the set coincide by what we proved last semester. By
  assuming, $T^n(x_1 , x_2 , \ldots , x_n)$ in the weak closure of
  $K^n(x_1 , x_2 , \ldots , x_n)$, there is a $T_i \in K$ such that
  $T_i^n(x_1 , x_2 , \ldots , x_n) \to T^n(x_1 , x_2 , \ldots , x_n)$ weakly.
\end{proof}

% TeX_root = ../main.tex

\marginnote{\scriptsize 11/02/2025}

\begin{exercise}
  Explore the correspondance between states on $\ell^{\infty}(X)$ and
  the finitely additive probability measures on $X$.
\end{exercise}

\begin{theorem}
  Let $\mathcal{H}$ be a Hilbert space, and let $f: B(\mathcal{H})
  \to \mathbb{C}$ be linear. Then, the following are equivalent.
  \begin{enumerate}[label=(\arabic*)]
    \item $f$ is SOT continuous.
    \item $f$ is WOT continuous.
    \item There exists $\xi_1 , \xi_2 , \ldots , \xi_n, \eta_1 ,
      \eta_2 , \ldots , \eta_n \in \mathcal{H}$ such that
      \begin{align*}
        f(T) = \sum_{i = 1}^{n} \langle T \xi_i , \eta_i \rangle
      \end{align*}
  \end{enumerate}
\end{theorem}
\begin{proof}
  We just need to prove $1 \implies 3$. By a lemma from last semester,
  there must exist $ \xi_1 , \xi_2 , \ldots , \xi_n \in \mathcal{H}$
  such that $|f(T)| \le \sum_{i = 1}^{n}  \|T \xi_i\|$ for all $T \in
  B(\mathcal{H})$. Let $   \mathcal{K} = \overline{\{ T \xi_o  \ : \
  T \in B(\mathcal{H}) \}} \subset \mathcal{H}$. Then
  \begin{align*}
    \phi: \mathcal{K} \to \mathbb{C} := T \xi_0 \to f(T)
  \end{align*}
  is a well defined linear functional for $\mathcal{K}$. Now by Reisz
  representation, we see that $\phi(T \xi_0) = \langle  T \xi_0 ,
  \eta_0 \rangle $ for some $\eta_0 \in \mathcal{H}$.

  But here, for $\xi_1 , \xi_2 , \ldots, \xi_n$, consider them in
  $\mathcal{H}^n$ and do the same process to get a $(\eta_1 , \eta_2
  , \ldots , \eta_n) \in \mathcal{H}^n$.
\end{proof}

\begin{theorem}
  Let $X, Y$ be normed spaces. For $1 \le p \le \infty$, define
  \begin{align*}
    \|(x, y)\|_p = \Big( \|x\|^p + \|y\|^p\Big)^{\frac{1}{p}}
  \end{align*}
  Then $\|\cdot\|_p$ is a norm extending both norms in $X$ and $Y$.
  If $X$ and $Y$ are Banach, then so is $(X \oplus Y, \|\cdot\|_p)$.
  All these norms are equivalent for all $1 \le p \le \infty$.
\end{theorem}

\begin{proposition}
  We have that $(X \oplus_p Y)^* = X^* \oplus_q Y^*$.
\end{proposition}
\begin{proof}
  We first show when $p = 1, q = \infty$. Consider the map
  \begin{align*}
    \Xi: (X \oplus_1 Y)^* \to X^* \oplus_\infty Y^* := F \to F_X \oplus F_Y
  \end{align*}
  once we identify $X$ as $X \oplus 0$ and $Y$ similarly as subspaces
  of $X \oplus Y$. See that this is a bijective linear map. Now show
  that the norms are preserved in $\Xi$.
\end{proof}

% TeX_root = ../main.tex

\begin{definition}
  Let $X$ and $Y$ be normed spaces and $T \in B(X, Y)$. The adjoint
  of $T$, denoted by $T^{*} \in B(Y^{*}, X^{*})$, is the map $T^{*}:
  f \to f\circ T$
\end{definition}

\begin{proposition}
  $\|T\| = \|T^*\|$
\end{proposition}
\begin{proof}
  $|T^*(f)| \le \|f \circ T\| \le \|T\|\|f\|$ implies $ \|T^{*}\| \le \|T\|$
  \begin{align*}
    \|T^*\| & = \sup \{ \|T^{*}(\phi)\| \ : \ \phi \in Y^*, \|\phi\| \le 1 \} \\
    &= \sup \{ |\phi(T(x))| \ : \ \phi \in Y^*, x \in X, \|\phi\| \le
    1, \|x\| \le 1 \} \\
    &= \|T\|
  \end{align*}
  Note that the last equality is a consequence of HBT since it
  guarantees the existence of $\phi_y \in Y^*$ with $\|\phi_y\|\le 1$
  and $\phi_y(y) = |y|$.
\end{proof}

\begin{lemma}
  For any $T \in B(X, Y)$, $T^*: Y^* \to X^*$ is weak * continuous
  (in both spaces)
\end{lemma}
\begin{proof}
  Let $\phi_i \to \phi$ weakly in $Y^*$. Then by definition for all
  $y \in Y$, $\phi_i(y) \to \phi(y)$. Then for $x \in X$,
  $T^*(\phi_i)(x) = \phi_i(T(x)) \to \phi(T(x)) = (T^*(\phi))(x)$
  which shows the continuity of $T^*$.
\end{proof}

\begin{lemma}
  For any normed space $X$, $i_X(X)$ is weak * dense in $X^{**}$.
\end{lemma}
\begin{proof}
  \textcolor{red}{verify}
\end{proof}

\begin{example}
  Is $i_{X^*}$ weak * - weak * continuous.
\end{example}

\section{Locally Convex Topological Vector Spaces}

\begin{lemma}
  Let $X$ be a normed space and $x_1 , x_2 , \ldots , x_n \in X$ and
  $\epsilon_1 , \epsilon_2 , \ldots , \epsilon_n \ge 0$. Then the set \[
    \bigcup_{x_1 , x_2 , \ldots , x_n, \epsilon_1 , \epsilon_2 ,
    \ldots , \epsilon_n}(\phi)
  \]
  is convex.
  \marginnote{ \scriptsize Review convexity arguments in Arveson's
  `Subalgebras of C* Algebras'}
  Moreover any topological vector spaces with the topology induced by
  a family of seminorms is locally convex.
\end{lemma}
\begin{proof}
  \textcolor{red}{verify}
\end{proof}

\begin{definition}
  Let $X$ be a vector space and $E \subset X$ be a convex subset. An
  element $a \in E$ is called an extreme point of $E$ if whenever $x,
  y \in E$, $0 \le t \le 1$ with $a  = tx + (1-t) y$, then $x = y = a$.
\end{definition}
\begin{example}
  Let $\bar{D} = \{ \alpha \in \mathbb{C} \ : \ |\alpha| \le 1 \}$.
  Then $\bar{D}$ is convex with $\textrm{Ext}(\bar{D}) = S^1$
\end{example}

\begin{theorem}[Krein-Milman Theorem]
  Let $X$ be a locally convex space, and let $K$ be a compact convex
  subset of $X$. Then the $ \textrm{Ext}(K) \neq \emptyset$ and
  indeed $K = \overline{\textrm{co}}(\textrm{Ext}(K))$
\end{theorem}

\begin{definition}
  Let $V$ be a vector space and $S\subset V$. The convex hull of $S$
  is defined as \[
    \textrm{co}(S) = \Big \{ \sum_{i = 1}^{n} t_i x_i \ \Big| \ 0 \le
    t \le 1, \sum t_i = 1, x_i \in S \Big\}
  \]
\end{definition}


% TeX_root = ../main.tex

\begin{lemma}
  Let $K$ be a convex set. $x_0 \in \textrm{Ext}(K)$ if and only if
  $K \setminus \{ x_0 \}$ is convex.
  \label{lem:extreme_point_of_convex_set}
\end{lemma}
\begin{proof}
  If $K \setminus \{ x_0 \}$ is not convex, since $K$ is convex,
  $x_0$ can be written as the convex combination of elements in
  $K\setminus \{ x_0 \}$
  which makes $x_0 \notin \textrm{Ext}(K_0)$. Conversely if $x_0 \in
  \textrm{Ext}(K)$, then
  $x_0$ cannnot be written as the convex combination of elements of
  $K$. Hence $K \setminus \{ x_0 \}$
  is closed under convex combinations, making in convex.
\end{proof}

\begin{theorem}[Krein-Milman Theorem]
  Let $X$ be a locally convex space, and let $K$ be a compact convex
  subset of $X$. Then the $ \textrm{Ext}(K) \neq \emptyset$ and
  indeed $K = \overline{\textrm{co}}(\textrm{Ext}(K))$
\end{theorem}
\begin{proof}
  We first prove that the $\textrm{Ext}(K) \neq \emptyset$.
  Note that $K \setminus \{ x_0 \}$ is a
  relatively open subset of $K$ since $\{ x_0 \}$ is closed and $K
  \setminus \{ x_0 \} = \{ x_0 \}^c$ relative to $K$.

  Now let $ \mathcal{A}$ be the collection of all relatively open
  convex proper subsets of $K$. Note that $\emptyset \in
  \mathcal{A}$, therefore $\mathcal{A}$ is nonempty. Equip
  $\mathcal{A}$ with the partial order defined by the set inclusion.
  Let $\mathscr{C}$ be a chain in $\mathcal{A}$ and $F_{\mathscr{C}}
  = \cup_{C \in \mathscr{C}} C$. $F_{ \mathscr{C}}$ is relatively
  \marginnote{\scriptsize Zorn's Lemma to find a maximal proper open
  convex subset of $K$}
  open being the union of relatively open subsets of $K$. To see that
  $F_{\mathscr{C}}$ is convex, let $x,  y \in F_{\mathscr{C}}$. Then
  since $\mathscr{C}$ is a chain, there exist a $C \in \mathscr{C}$
  such that $x, y \in \mathscr{C}$. Then by the convexity of $C$, $tx
  + (1-t)y \in C
  \subset F_{\mathscr{C}}$ for all $t \in [0, 1]$.

  We claim that $F_{ \mathscr{C}}$ is a proper subset of $K$. For the
  sake of contradiction, assume $F_{ \mathscr{C}} = K$. Since $K$ is
  compact and $C$ is open in $K$ for all $C \in \mathscr{C}$, there
  are finitely many $C_1 \subset C_2 \subset \ldots \subset C_k \in
  \mathscr{C}$ which cover $K$ (i.e $K = \cup_{n = 1}^{k}C_n$). Hence
  we get $C_k = K$, which is absurd since $C_k$ must be a proper subset of $K$.
  Hence $F_{\mathscr{C}} \in \mathcal{A}$ and thus every chain must
  have an upper bound in $\mathcal{A}$. Now by Zorn's lemma,
  $\mathcal{A}$ has a maximal element $K_0$.

  Since $K$ is a connected space (path connected by a straight line,
  being convex), we know that the only
  clopen subsets are $\emptyset$ and $K$. Since we know that $K_0$ is
  \marginnote{\scriptsize Constructing and open convex subset containing $K_0$}
  open being in $\mathcal{A}$, we see that $K_0 \neq K$ and $K_0 \neq
  \emptyset$. Therefore $\overline{K_0} \neq K_0$. Let $x_0 \in
  \overline{K_0} \setminus K_0$, $y_o \in K_0$ and $0 < t <1$. Define
  $\varphi_{t, y_0}: K \to K$ such that $\varphi_{t, y_0}(z) = ty_0 +
  (1-t) z$. Then $\varphi_{t, y_0}$ is ($1-t$ Lipschitz) continuous
  relative to $K$ and thus $\varphi_{t, y_0}^{-1}(K_0)$ is open in
  $K$.
  By the convexity of $K_0$, we get $K_0 \subset \phi^{-1}_{t, y_0}(K_0)$.

  Also $\varphi_{t, y_0}^{-1}(K_0)$ is convex. Let $a, b \in
  \phi^{-1}_{t, y_0}(K_0)$. Then $ty_0 + (1-t)a, ty_0 + (1-t)b \in
  K_0$. By the convexity of $K_0$ we get $r(ty_0 + (1-t)a) +
  (1-r)(ty_0 + (1-t)b) = ty_0 + (1-t)(ra + (1-r)b) = \phi_{t, y_0}(ra
  + (1-r)b) \in K_0$ for all $r \in [0, 1]$. Thus $ra + (1-r)b \in
  \phi^{-1}_{t, y_0}(K_0)$ for all $r \in [0, 1]$. Hence we get
  $\phi^{-1}_{t, y_0}(K_0)$ is convex.

  We claim, $x_0 \in \varphi_{t, y_0}^{-1}(K_0)$, then the maximality
  of $K_0$ will force $\phi^{-1}_{t, y_0}(K_0) = K$.  Let $U$ be a
  convex neighborhood of $0 \in X$ containing $-x$ for all $x \in U$
  \marginnote{\scriptsize \textcolor{red}{I can't picturize the choice of $z$}}
  (just take $-U$ and intersect with $U$) such that $y_o + E
  \subset K_0$, where $E = K \cap U$. Let $w = \varphi_{t,
  y_0}(x_0)$. Since $x_0 \in
  \overline{K_0}$, for any $r>0$, there exists $x_r \in K_0$ such
  that $x_r \in (x_0 +
  rE) \cap K_0 \neq \emptyset$. In particular, let $r =
  \frac{t}{1-t}$. Then by linearity, we get $\big(x_0 +
  \frac{t}{1-t}E\big) \cap K_0 = ( \frac{t}{1-t}  )E\cap
  (K_0 - x_0) \neq \emptyset$. Choose $z$ in the above set. Then \[
    y_0 - \Big( \frac{1-t}{t} \Big)z \ \in \ y_0 + E \subset K_0
  \]
  and $x_0 + z \in K_0$. Since $K_0$ is convex, \[
    t\Big(y_0 - \frac{(1-t)}{t}z\Big) + (1-t)(x_0 + z)  = \phi_{t,
    y_0}(x_0) \in K_0
  \]
  Thus $\phi^{-1}_{t, y_0}(K_0) = K$.

  Now we claim that $K = K_0 \cup \{ x_0 \}$. For the sake of
  contradiction assume $\exists p \in K$ such that $p \notin K_0 \cup
  \{ x_0 \}$. Since the space is Hausdorff and locally convex, $x_0$ has an
  open convex neighborhood $E$ in $X$ such that $p \not\in E$. Let
  $E^\prime = E \cap K$, $a \in K_0, b \in E^\prime$ and $0 < r < 1$.
  Then since $\phi_{t, y_0}(K) = K_0$ for all $t \in [0, 1], y_0 \in K_0$, we
  get $\phi_{r, a}(b) = ra + (1-r)b \in K_0$. So $K_0 \cup
  E^\prime$ is convex (Sine we know that $K_0, E^\prime$ are convex,
    we only need to worry about $rx + (1-r)y$ for $x \in K_0, y \in
  E^\prime$. But $\phi_{r, x}$ takes care of that). $K_0 \cup
  E^\prime$ is also open in $K$. Hence by maximality, we get $K_0
  \cup E' = K$. But this is a contradiction since $ p \not\in K_0
  \cup E^\prime$. Thus by \autoref{lem:extreme_point_of_convex_set}, we see
  that $x_0 \in \textrm{Ext}(K)$.

  Next we prove $K = \overline{co}(\textrm{Ext}(K))$. Let $P =
  \overline{co}(\textrm{Ext}(K))$ and for the sake of contradiction
  assume $P \neq K$. Let $ x_0 \in K \setminus P$.
  %
  % Let $E$ be an open
  % convex neighborhood of $0 \in X$ such that $(x_0 + E^\prime) \cap P
  % = \emptyset$ for $E^\prime = E \cap K$. Existence of such an $E$
  % is guaranteed
  % since $P$ is compact and the space $X$ is Hausdorff.
  % \marginnote{\scriptsize \textcolor{red}{Don't know how to proceed after}}
  %
  Now by the geometric Hahn-Banach separation theorem, we get that
  there is a continuous linear functional
  $\phi: X \to \mathbb{R}$ and a number $\alpha, \epsilon \in
  \mathbb{R}$ such that \[
    \Re\phi(x_0) \le  \alpha < \alpha + \epsilon \le \Re\phi(p),
    \quad  \forall p \in P
  \]

  % Define $\phi: X \to
  % \mathbb{R}$ such that \[
  %   \phi(x) = \inf \{ 0 \le t \big | x \in tE \}
  % \]
  % Observe that $E = \{ x \in X \ : \phi(x) < 1 \}$. For every $r \ge
  % 0$ and $x \in X$, $\phi(rx) = r \phi(x)$, and for all $x, y \in X$,
  % $ \phi(x+y) \le \phi(x) + \phi(y)$. Define $f:  \mathbb{R}\{ x_0 \}
  % \to \mathbb{R}$, $f(rx_0) = r \phi(x_0)$, for all $r \in
  % \mathbb{R}$. For every $r \ge 0$, we have $f(rx_0) = r \phi(x_0) =
  % \phi(rx_0)$. For $r < 0$, we have $f(rx_0) = r \phi(x_0) =
  % -\phi(-rx_0) \le 0 \le \phi(rx_0)$. So $f \le \phi$ on $\mathbb{R}
  % x_0$. Then by Hahn-Banach separation theorem, there is an
  % extension $ \tilde{f}: X \to
  % \mathbb{R}$ such that $\tilde{f}(x) \le \phi(x) $ for all $x \in X$.

\end{proof}


% TeX_root = ../main.tex

\marginnote{\scriptsize 18/03/2025 }

% TeX_root = ../main.tex

\chapter{Complex Measures}

\marginnote{\scriptsize 20/03/2025 }

\section{Consequence of Radon-Nikodym Theorem}

\begin{theorem}
  If $\mu, \nu$ are positive $\sigma$-finite measures such that $\nu
  \ll \mu$, then there is a positive measurable function $h$ such that
  $d \nu = h \mu$
\end{theorem}

\begin{theorem}[Hahn-Decomposition Theorem]
  Let $\mu$ be a real-valued complex measure (signed measure) on a
  measurable space
  $(X, \mathcal{M})$. Then there are two sets $A, B$ such that $A
  \cup B = X, A \cap B = \emptyset$ and
  \begin{align*}
    \mu_+(E) := \mu( E \cap A), \quad \mu_-(E) = \mu(E \cap B)
  \end{align*}
  with $ \mu_+ \perp \mu_-$ and $ \mu_+ + \mu_- = \mu$, and $\mu_+ +
  \mu_- = |\mu|$.

  Moreover, if $   \mu = \mu_1 - \mu_2$ with $ \mu_1 , \mu_2$ being
  positive measures, then for any $E \in \mathcal{M}$ we have
  $\mu_1(E) \ge \mu_+(E), \mu_2(E) \ge \mu_-(E)$
\end{theorem}
\begin{proof}
  Since $\mu$ is a complex measure, $\mu \ll |\mu|$ and by
  Radon-Nikodym, there is a $h \in L^{1}(\mu)$ with $h(x) \in \{ 1, 2  \}$
  (polar decomposition) such that $d \mu = h d |\mu|$.

  Let $A = h^{-1}(1), B = X \setminus A$. We find that $d\mu_+ =
  \frac{1}{2}(d|\mu| + d\mu) = \frac{1}{2}(|h| d |\mu| + h d |\mu|) = h_+
  d |\mu|$,  and similarly $\mu_- = h_- d |\mu|$. The rest follows easily.
\end{proof}

\section{Bounded linear functionals on $L^p$}

\begin{note}
  Let $\mu$ be a positive measure, $1 \le p \le \infty$ and
  $\frac{1}{p} + \frac{1}{q} = 1$. Fixing $ g \in L^{1}(\mu)$.
  Holder's inequality gives that for any $ f \in L^{p}(\mu)$,
  \begin{align*}
    \big|\int fg \ d \mu\big| \le \|f\|_p \|g\|_q
  \end{align*}
  So that $\Lambda_g : L^{p}(\mu) \to \mathbb{C} := f \to \int fg \ d
  \mu$ is a bounded linear functional. Thus, we have a map $\Lambda :
  L^{q}(\mu) \to L^{p}(\mu)^* := g \mapsto \Lambda_g$.

  For $1 \le p < \infty$, the converse is true, too
\end{note}

\begin{lemma}
  If $\mu$ is $\sigma$-finite on $(X, \mathcal{M})$, then there is a
  $\omega \in L^{1}(\mu)$ such that $\forall x \in X : 0 <  \omega(x) < 1$.
\end{lemma}
\begin{proof}
  Choose a partition $E_j$ of $X$ such that $\mu(E_j) < \infty$ for
  each $j \in \mathbb{N}$. Let
  \begin{align*}
    \omega = \sum_{n \in \mathbb{N}} \frac{1}{2^n} \frac{1}{1 +
    \mu(E_n)} \chi_{E_n}
  \end{align*}
  Since $E_j$ is a partition, we get that
  \begin{align*}
    \int |\omega| \ d \mu &= \int \omega \ d \mu \\
    &= \int \sum_{n \in \mathbb{N}} \frac{1}{2^n} \frac{1}{1 +
    \mu(E_n)} \chi_{E_n} \ d \mu \\
    &= \sum_{n \in \mathbb{N}} \frac{1}{2^n} \frac{1}{1 +
    \mu(E_n)} \mu(E_n) \\
    &\le \sum_{n \in \mathbb{N}} \frac{1}{2^n}  = 1
  \end{align*}
  Hence $\omega \in L^{1}(\mu)$ is the required function.
\end{proof}

\begin{corollary}
  If $\mu$ is $\sigma$-finite, then $\tilde{ \mu}$ given by $d
  \tilde{\mu} = \omega d \mu$ is finite.
\end{corollary}

\begin{theorem}
  Let $\mu$ be a $\sigma$-finite measure, $1 \le p < \infty$, $q$ as
  usual. If $ \Lambda \in L^{p}(\mu)^*$, then there is $ g \in
  L^{q}(\mu)$ such that
  \begin{align*}
    \Lambda = \Lambda_g
  \end{align*}
  and $\|\Lambda\| = \|g\|_q$
\end{theorem}
\begin{proof}
  Begin by assuming $\mu$ is finite. Let $\Lambda : L^{p}(\mu) \to
  \mathbb{C}$ be a bounded linear functional. Notice that $\chi_E \in
  L^{p}(\mu)$ for each $E \in \mathcal{M}$. Consider $\lambda(E) =
  \Lambda(\chi_E)$. Let $\{E_j\}_{n = 1}^\infty$ be a partition of $E$. We find
  \begin{align*}
    \lambda \Big( \bigcup_{j = 1}^{n}E_j \Big) &= \Lambda \Big(
    \sum_{j = 1}^n \xi_{E_j} \Big) \\
    &= \sum_{j = 1}^{n} \Lambda(\chi_{E_j}) \\
    &= \sum_{ j = 1}^{n} \lambda(E_j)
  \end{align*}
  We conclude $\lambda$ is finitely additive. Note
  \begin{align*}
    \Big \| \chi_E - \chi_{\cup_{j = 1}^{n}E_j} \Big \|_p = \Big(
    \mu\big(\bigcup_{j = n+1}^{\infty} E_j\big)\Big)^{\frac{1}{p}}
  \end{align*}
  Using monotone convergence and boundedness of $\Lambda$, we get
  \begin{align*}
    \Lambda(  \chi_{\cup_{j = 1}^{n}E_j}) \to \Lambda(\chi_E)
  \end{align*}
  Thus $\lambda$ is a measure and $\lambda \ll \mu$ by the definition
  of $\lambda$. By
  Radon-Nikodym, we have $g \in L^{1}(\mu)$ with $d \lambda = g d \mu$.

  For $f$ simple,
  \begin{align*}
    \Lambda(f) = \int f \ d \lambda = \int fg \ d \mu := \Lambda_g(f)
  \end{align*}
  Now, consider $p = 1$. Then
  \begin{align*}
    \Big|\int f \ d \lambda\Big| = \Big|\int fg \ d \mu\Big| \le
    \|\Lambda\| \|f\|_1
  \end{align*}
\end{proof}

% TeX_root = ../main.tex

\marginnote{\scriptsize 01/04/2025 }

\begin{lemma}
  Let $X$ be locally compact Hausdorff and $\lambda: C_c(X) \to
  \mathbb{R}$ be bounded linear functional. Then there are positive
  bounded linear functionals $\lambda_+, \lambda_-$ such that
  $\lambda = \lambda_+ - \lambda_-$.
\end{lemma}
\begin{proof}
  For this, we find bounded linear functional $\rho$,
  \begin{align*}
    |\lambda(f)| \le \rho(|f|) \le C \| f\|_\infty
  \end{align*}
  and then let $\lambda_+ = \frac{1}{2}(\lambda + \rho)$ and
  $\lambda_- = \frac{1}{2}(\rho - \lambda)$

  We define the map $\rho :   C_c(X)^+ \to  \mathbb{C} := f \mapsto
  \sup \{ |\lambda(h) \ | \ h \in C_c(X), |h| \le f \}$, where
  $C_c(X)^+$ is the set of non-negative real valued functions in $C_c(X)$.
  Let $ f, g \in C_c(X)^+$, then there is a $ h_1, h_2 \in C_c(X)$
  such that $|h_1| \le f, |h_2| \le g$ with $\rho(f) \le |
  \lambda(h_1)| + \varepsilon$ and $\rho(g) \le |\lambda(h_2)| + \varepsilon$.
  So, $\rho(f) + \rho(g) \le |\lambda(h_1)| + |\lambda(h_2)| + 2 \varepsilon$.
  Let $\alpha_1, \alpha_2 \in \{ \pm 1 \}$ such that
  $\lambda(\alpha_i h_i) = \alpha_i \lambda(h_i) \ge 0$. Then,
  \begin{align*}
    |\lambda(\alpha_1h_1)| + |\lambda( \alpha_2 h_2)| =
    \lambda(\alpha_1 h_1) + \lambda(\alpha_2 h_2)  =
    \lambda(\alpha_1 h_1 + \alpha_2 h_2)
  \end{align*}
  So,
  \begin{align*}
    \rho(f) + \rho(h) & \le \lambda(\alpha_1 h_1 + \alpha_2 h_2) +
    \varepsilon \\
    & \le \rho(|\alpha_1h_1 + \alpha_2 h_2|) + 2 \varepsilon  \quad
    \textrm{since } \alpha_1h_2 +
    \alpha_2 h_2 \le |\alpha_1h_1 + \alpha_2h_2|\\
    & \le \rho(|h_1| + |h_2|) + 2 \varepsilon \quad \textrm{since }
    \rho \textrm{ is order preserving} \\
    & \le \rho(f + g) + 2 \varepsilon \quad \textrm{since }
    \rho \textrm{ is order preserving}
  \end{align*}
  Since this holds for any $\varepsilon > 0$, we get $\rho(f + g) \ge
  \rho(f) + \rho(g)$.

  To show the reverse inequality, let $f, g \in C_c(X)^+$, and $h \in
  C_c(X)$ be such that $|h| \le f + g$. We define
  \begin{align*}
    h_1(x) =
    \begin{cases}
      \frac{f(x)}{f(x) g(x)}h(x), &f(x) + g(x) > 0 \\
      0, &\textrm{else}
    \end{cases} \\
  \end{align*}
  and $h_2(x) = h(x) - h_1(x)$. Then $|h_1| \le f, |h_2| \le g$.
  Moreover, $h_1, h_2$ are continuous where $ f(x) + g(x) \ge 0$. Next,
  \begin{align*}
    |\lambda(x)| &= |\lambda(h_1 + h_2)| \\
    & \le | \lambda(h_1)| + | \lambda(h_2)| \\
    & \le \rho(f) + \rho(g)
  \end{align*}
  Taking supremum over $h$, we get $\rho(f + g) \le \rho(f) +
  \rho(g)$. We have established additivity of $\rho$ for $f, g \ge 0$.
  For general $ f, g \in C_c(X)$, split $f, g, h$ into differences of
  positive and negative parts and rearrange to apply $\rho$ with
  linearity. Thus we'll get $\rho(f +g) = \rho(f) + \rho(g)$.

  Now to show homogeneity, let $  c \in \mathbb{R}$ and $f \in
  C_c(X)$. If $c < 0$,
  \begin{align*}
    \rho(cf^+) & = - \rho((cf^+)^-) \\
    &= - \rho(|c|f^+) \\
    &= - |c| \rho(f^+) \\
    &= c \rho(f^+)
  \end{align*}
  Again by splitting $f = f^+ - f^-$, we get the homogeneity. Thus we
  get $\rho$ is linear.
\end{proof}

\begin{lemma}
  If $\nu$ is a $\sigma$-finite regular positive measure on a locally compact
  Hausdorff space, and $  \mu$ is a complex measure with $|\mu| \ll
  \nu$, then $\mu$ is regular.
\end{lemma}
\begin{proof}
  Using Radon-Nikodym theorem, for a measurable set $E$, we have
  \begin{align*}
    \mu(E) = \int_E h \ d \nu
  \end{align*}
  with $h \in L^{1}(\mu)$. Considering that $\mu$ is regular, there
  are sequences  of open sets $V_j \supset E$, $\nu(V_j \setminus E)
  \stackrel{ j \to \infty}{\longrightarrow} 0$ and compact sets $K_j
  \subset E$, such that $\nu(E \setminus K_j) \stackrel{j \to
  \infty}{\longrightarrow} 0$.

  Next, by dominated convergence theorem,
\end{proof}

% TeX_root = ../main.tex

\chapter{Hilbert Spaces}

\begin{definition}
  Recall that a complex inner product o a complex vector space is a
  map \[
    \langle \ , \  \rangle : X \times X \to \mathbb{C}
  \]
  such that
  \begin{enumerate}[label=(\arabic*)]
    \item $\langle x , x \rangle \ge 0$ for all $x \in X$
    \item $\langle  x , y \rangle  = \overline{\langle y , x \rangle }$
    \item $\langle \alpha x + z, y \rangle  = \alpha \langle x , y
      \rangle  + \langle z , y \rangle $
  \end{enumerate}
\end{definition}

Recall the norms induced by the inner product and the Cauchy-Schwarz inequality.

\begin{definition}
  Complete inner product spaces are called Hilbert spaces
\end{definition}

\begin{proposition}
  Let $X$ be an inner product space. Then the inner product of $X$
  extends to an inner product on the completion (unique metric space
  completion) of $X$, turns it into
  a Hilbert space.
\end{proposition}

\begin{definition}
  If $x, y \in H$, the Hilbert space, we say $x \perp y$ if $ \langle
  x , y \rangle  = 0$
\end{definition}

\begin{definition}
  Given a set $S \subset H$, $S^\perp = \{ y \in H  \ : \  \langle x
  , y \rangle  = 0 \}$
\end{definition}

\begin{proposition}
  Let $H, K$ be Hilbert spaces and $T: H \to K$ be linear. Then the
  following are equivalent.
  \begin{enumerate}[label=(\arabic*)]
    \item $T$ is isometry
    \item $\langle  Tx , Ty \rangle = \langle x , y \rangle $ for all
      $x, y \in H$
  \end{enumerate}
\end{proposition}
\begin{proof}
  See Homework-5
\end{proof}

\begin{proposition}
  For all $x, y \in H$, a Hilbert space, then \[
    \|x + y\|^2 + \|x-y\|^2  = 2(\|x\|^2 + \|y\|^2)
  \]
\end{proposition}

\begin{example}
  Show that $c_{00}$ under the usual inner product is not complete
  and its completion is $\ell^2(\mathbb{N})$.
\end{example}
\begin{proof}
  Consider the sequence $x_n = (1, \frac{1}{2}, \frac{1}{3}, \ldots,
    \frac{1}{n}, 0,
  \ldots)$. Then clearly $x_n \in \textbf{c}_0$. Moreover $x_n \to x
  = (1, \frac{1}{2}, \ldots \frac{1}{n}, \frac{1}{n+1}, \ldots)$ in
  the norm by the inner product. But $x \notin \textbf{c}_{00}$. But $ x
  \in \ell^2(\mathbb{N})$. Moreover, the same process can be used to
  approximate any  element in $\ell^2(\mathbb{N})$ by elements in
  $\textbf{c}_{00}$. Thus we see that $ \textbf{c}_{00}$ is dense in
  $\ell^{2}(\mathbb{N})$
\end{proof}

\begin{example}
  $L^2(\mathbb{R},  \mu)$ with \[
    \langle f , g \rangle  = \int_\mathbb{R}  f \overline{g} \ d \mu
  \]
  is a Hilbert space.
\end{example}

\begin{example}
  Let $J$ be any set $\ell^2(J) = \{  f: J \to \mathbb{C}  \ : \
  \sum_{j \in J} |f(j)|^2 < \infty  \}$ with the usual inner product
  is a Hilbert space.
\end{example}

\begin{definition}
  An orthonormal basis for $H$ is a maximal orthonormal set.
\end{definition}

\begin{theorem}
  Let $H$ be a Hilbert space and $J$ be an orthonormal basis for $H$.
  Then there exists a bijective linear isometry $T : H \to \ell^2(J)$.
\end{theorem}


% TeX_root = ../main.tex

\begin{theorem}
  Let $\mathcal{H}$ be a Hilbert space and let $C$ be a non-empty closed convex
  subset of $\mathcal{H}$. Then there exist a unique vector $x \in C$ such that
  $\|x\| \le \|\eta\|$ for all $\eta \in C$.
\end{theorem}
\begin{proof}
  \marginnote{ \scriptsize \textit{\textcolor{red}{construction of the
  proof is a bit tricky}}}
  Let $d = \inf \{ \|\eta\|  \ : \ \eta \in C  \}$ and choose a
  sequence $ \eta_n \in C$  such that $\|\eta_n\| \to d$. Let $
  \epsilon > 0$. Choose $N \in \mathbb{N}$ such that $\|\eta_n\|^2 <
  d^2 + \epsilon$ for all $n \ge N$. Then for all $m,n \ge N$, we have \[
    \|\eta_n - \eta_m\|^2 = 2(\|\eta_n\|^2 + \|\eta_m\|^2) - 4 \|
    \frac{1}{2}(\eta_n + \eta_m) \|^2 \le 4(d^2  + \epsilon) - 4d^2 = 4 \epsilon
  \]
  (Note that since $C$ is convex, $\frac{1}{2}\eta_n +
    \frac{1}{2}\eta_m \in C$ and therefore, by the definition of $d$,
  we get $\|\frac{1}{2}(\eta_n + \eta_m)\| \ge d$.) Hence the sequence
  $\eta_n$ is Cauchy and hence convergent since
  the space is complete. Let $\eta = \lim_{n \to \infty} \eta_n$.
  Since $C$ is closed $\eta \in C$ and clearly $\|\eta\| = d$.

  To see the uniqueness, assume $\alpha \in C$, and $\| \alpha\| = d$. Then
  \begin{align*}
    \|\eta - \alpha\|^2 &= 2(\|\eta\|^2 + \|\alpha\|^2) - 4 \|
    \frac{1}{2}(\eta + \alpha) \|^2 \\
    & \le 4d^2 - 4d^2 = 0
  \end{align*}
  Verify the second inequality.
\end{proof}

\begin{corollary}
  Let $\eta \in \mathcal{H}$ and $C$ be as before. Then there exist a unique
  vector $x \in C$ such that $d(\eta, C) = \|x - \eta\|$
\end{corollary}
\begin{proof}
  Apply above theorem to $C^\prime = C - \eta$. Since $C$ is closed
  and convex, so will be its translation $C - \eta$.
\end{proof}

\begin{proposition}[Pythagoras Theorem]
  \label{PythagorasTheorem}
  Let $ x, y \in \mathcal{H}$ an inner product space, and $x \perp y$, then
  $\|x + y\|^2 = \|x\|^2 + \|y\|^2$.
\end{proposition}

\begin{lemma}
  \label{lem:15}
  Let $\mathcal{H}$ be a Hilbert space and $K$ be a nontrivial closed subspace.
  Let $\eta \in \mathcal{H}$. Then $\xi \in K$ satisfy $\|\xi -
  \eta\| = d(\eta, K)$ iff
  $\xi - \eta \perp K$.
\end{lemma}
\begin{proof}
  Let $\xi - \eta \perp K$ and $k \in K$. Then $\eta - k = (\eta -
  \xi) + (\xi - k)$ and by Pythagoras theorem, we get
  \begin{align*}
    \|\eta - k\|^2 = \|\eta - \xi\|^2 + \|\xi - k\|^2 \ge \|\eta - \xi\|^2
  \end{align*}
  Thus we see that $\|\eta - \xi\| = d(\eta, K)$.

  \marginnote{ \scriptsize  \textit{\textcolor{red}{cool proof technique}}}
  Conversely, let  $\|\eta - \xi\|= d(\eta, K)$. Then for all $\rho
  \in K$ and $t > 0$, we have
  \begin{align*}
    \|\eta - \xi\|^2 &\le \|\eta - (\xi + t\rho)\|^2  \\
    & = \| \eta - \xi -  t\rho\|^2 \\
    & = \|\eta - \xi\|^2 + t^2\|\rho\|^2 - 2t\Re \langle \eta - \xi ,
    \rho \rangle
  \end{align*}
  Hence we see that $|2\Re \langle \eta - \xi , \rho \rangle| \le t\|\rho\|^2$
  Since $t>0$ was arbitrary, limiting it to zero, we get $\Re \langle
  \eta - \xi ,  \rho \rangle = 0$. Replacing $\rho$ with $-i \rho$
  will give the imaginary part is also zero.
\end{proof}

\begin{remark}
  \label{KplusKperpisX}
  If $K$ is a closed subspace of the Hilbert space $\mathcal{H}$,
  then $\mathcal{H} = K \oplus K^\perp$. To see this notice that $K
  \cap K^\perp = \{ 0 \}$. Now if $x \in \mathcal{H}$, then there is
  a $k_x \in K$ such that $d(x, K) = \|x- k_x\|$. Moreover, from
  \autoref{lem:15}, we see that $x - k_x \in K^\perp$. Hence $x =
  k_x + (x - k_x) \in K \oplus K^\perp$. Now replacing $K$ with
  $K^\perp$, we see that $(K^\perp)^\perp = K$.
\end{remark}

\begin{theorem}[Reisz Representation Theorem]
  Let $\mathcal{H}$ be a Hilbert space and $f \in \mathcal{H}^*$.
  Then there exists a
  unique $\eta_f \in \mathcal{H}$ such that $f(x) = \langle x , \eta_f \rangle $
  for all $ x \in \mathcal{H}$. The map $\phi:  \mathcal{H}^* \to
  \mathcal{H} := f \to \eta_f$ is
  conjugate linear isometric bijection.
\end{theorem}
\begin{proof}
  If $f$ is the zero linear functional, $n_f = 0$ and we're done. If
  not let $K  = \textrm{Ker}(f)$. Then $K$ has co-dimension $1$.
  Consider $f|_{K^\perp}$. Let $v \in K^\perp$ such that $f(v) = 1$.
  Clearly $  K^\perp = \textrm{span}(v)$. And therefore for any
  $\alpha  v \in K^\perp$, $f(\alpha  v) = \alpha = \langle \alpha v
  , \frac{v}{\|v\|^2} \rangle $. Choose $\eta_f = \frac{v}{\|v\|^2}$.
  Now verify.
\end{proof}

\section{Orthogonal Projections}

\begin{theorem}
  Let $\mathcal{H}_1, \mathcal{H}_2$ be Hilbert spaces, and $T:
  \mathcal{H}_1 \to  \mathcal{H}_2$ be a bounded
  linear map. Then there exists a unique bounded linear map $T^*: \mathcal{H}_2
  \to \mathcal{H}_1$ satisfying $\langle  Tx , y
  \rangle_{\mathcal{H}_2} = \langle x ,
  T^*y \rangle_{\mathcal{H}_1}$ for all $x \in \mathcal{H}_1, y \in
  \mathcal{H}_2$.
\end{theorem}
\begin{proof}
  For every given $y \in \mathcal{H}_2$ define a linear functional $f^y: H_1
  \to \mathbb{C}$ as $f^y(x) = \langle Tx , y \rangle$. Since $f^{y}$
  is bounded, $f^y \in \mathcal{H}_1^*$. Hence by Reisz representation, there
  is a unique $T^*(y) \in \mathcal{H}_1$ such that $\langle  Tx , y \rangle =
  \langle x , T^*y \rangle$.

  Uniqueness follows from the fact that in any inner product space
  $X$, if $x, y \in X$ such that $ \langle x , z \rangle = \langle  y
  , z \rangle $ for all $z \in X$, then $\langle  x-y , z \rangle =
  0$ for all $z \in X$, in particular $z = x-y$. Then by the positive
  definiteness of the inner product, we get $x = y$.
\end{proof}

\begin{theorem}
  \label{OrthogoalProjectionsAreSelfAdjointContractions}
  Let $\mathcal{H}$ be a Hilbert space and $K$ a closed subspace. For every
  $\eta \in \mathcal{H}$, denote by $P_K(\eta)$, the unique closest vector in
  $K$, closest to $\eta$. Then
  \begin{enumerate}[label=(\arabic*)]
    \item $P_K: \mathcal{H} \to \mathcal{H}$ is linear, bounded with $\|P_K\| =
      1$ and idempotent.
    \item $P_K^* = P_K$ (self-adjoint)
  \end{enumerate}
\end{theorem}
\begin{proof}
  \begin{enumerate}[label=(\arabic*)]
    \item Let $\eta_1, \eta_2 \in \mathcal{H}$ and $\alpha \in \mathbb{C}$.
      Then using \autoref{lem:15} for all $\xi \in K$, we have
      \begin{align*}
        \langle  \alpha \eta_1
        +\eta_2 - \alpha P_K( \eta_1) - P_K(\eta_2) , \xi \rangle &=
        \alpha \langle \eta_1 - P_K(\eta_1) , \xi  \rangle + \langle
        \eta_2 - P_K(\eta_2) , \xi  \rangle  \\
        &= 0
      \end{align*}
      Thus we get $P_K(\alpha \eta_1 + \eta_2) = \alpha P_K(\eta_1) +
      P_K(\eta_2)$ again by the same \autoref{lem:15}. Moreover since
      $P_K(x)$ is a vector in $K$ it the vector in $K$ closest to
      $P_K(x)$ is $P_K(x)$ itself. Thus we see that $P^2_K(x) =
      P_K(x)$. Hence idempotent. Moreover since $K$ is closed, we
      have $X = K \oplus K^\perp$ (\autoref{KplusKperpisX}). Thus for
      all $x \in X$, we can
      write $ x = k_x + k^\prime_x$, where $ k_x \in K$ and $
      k^\prime_x \in K^\perp$. Then $P_K(x) = k_x$ and triangle
      inequality shows $ \|P_K(x)\| = \|k_x\| \le \|x\|$. Thus $\|P_K\| \le 1$.

    \item Since $X = K \oplus K^\perp$, let $x = k_x + k^\prime_x$
      and $y = k_y + k^\prime_y$. Then
      \begin{align*}
        \langle P(x) ,  y \rangle = \langle k_x , k_y + k^\prime_y
        \rangle = \langle k_x , k_y \rangle = \langle k_x  +
        k^\prime_x , k_y \rangle = \langle x , P_K(y) \rangle
      \end{align*}
      and the uniqueness of the adjoint proves that $P_K^* = P_K$
  \end{enumerate}
\end{proof}


% TeX_root = ../main.tex

\marginnote{\scriptsize 20/03/2025 }

\begin{definition}
  A bounded linear map $T: X \to Y$ between normed spaces is called a
  quotient map if it is surjective and
  \begin{align*}
    \|y\| = \inf \{ \|x\|  \ : \ T(x) = y \}
  \end{align*}
  for all $y \in Y$
\end{definition}

\begin{proposition}
  Quotient map between normed spaces extends to a quotient map between
  their completion.
\end{proposition}
\begin{proof}
  \textcolor{red}{verify}
\end{proof}

\begin{theorem}
  Let $T : X \to W$ and $S: Y \to Z$ be quotient maps between normed
  spaces. Then
  \begin{align*}
    T \hat{\otimes} S : X \hat{\otimes} Y \to W \hat{\otimes} Z
  \end{align*}
  is also a quotient map.
\end{theorem}
\begin{proof}
  To see that $T \hat{\otimes} S$ is a surjection, first observe that
  $T \otimes S : X \otimes Y \to W \otimes Z$ is a surjection.
  See that $ T \otimes S$ is also a quotient map. Now use the above proposition.
\end{proof}

\begin{note}
  If $X$ is a Banach space, then there is a set $I$ such that there
  is a quotient map from $L^{1}(I) \to X$.
\end{note}
\begin{proof}
  Notice that we can take $I = X_1$, the unit ball of $X$.
\end{proof}

\begin{proposition}
  Let $X, Y $ be Banach spaces. Let $I$ be a set such that $T:
  \ell^{1}(I) \to X$ is a quotient map. Then
  \begin{align*}
    T \hat{\otimes} Id_Y : \ell^{1}(I, Y) \to X \hat{\otimes} Y
  \end{align*}
  is a quotient map.
\end{proposition}
\begin{proof}
  Let $z \in X \hat{\otimes} Y$. \textcolor{red}{verify}
\end{proof}

% TeX_root = ../main.tex
\section{Compact Operators}
\begin{definition}
  let $\mathcal{X}, \mathcal{Y}$ be Banach Spaces. A linear map $T : \mathcal{
  X} \to \mathcal{Y}$ is called a
  compact operator if $\overline{T(B_1^\mathcal{X})}$ is compact,
  where $B_1^\mathcal{X}$
  is the unit ball of $\mathcal{X}$. We denote by
  $\mathcal{K}(\mathcal{X}, \mathcal{Y})$, the set of all
  compact operators.
\end{definition}

\begin{example}
  If either $\mathcal{X}$ or $\mathcal{Y}$ is finite, then every linear map $T:
  \mathcal{X} \to \mathcal{Y}$
  is compact. If $\mathcal{X}$ be any infinite dimensional Banach space. Then
  $T = \textrm{Id} : \mathcal{X} \to X$ is not compact. This follows from
  \autoref{ClosedUnitBallisCompactiffFiniteDim}.
\end{example}

\begin{definition}
  $T: \mathcal{X} \to \mathcal{Y}$ is called a finite rank if the
  dimension of $T$ is
  finite. Then the dimension of the image is called the rank of the
  operator. Let $\mathcal{F}(\mathcal{X}, \mathcal{Y})$ denote finite
  rank operators.
\end{definition}

\begin{lemma}
  Every compact operator is bounded.
\end{lemma}
\begin{proof}
  Every compact set is bounded in any metric space.
\end{proof}

\begin{theorem}
  Let $\mathcal{H}$ be a Hilbert space. Then $\mathcal{K}(\mathcal{H}) =
  \overline{\mathcal{F}(\mathcal{H})}^{\|\cdot\|}$
\end{theorem}
\begin{proof}
  It is evident that $\mathcal{F}(\mathcal{H}) \subset
  \mathcal{K}(\mathcal{H})$. We'll
  now show that
  $\overline{\mathcal{F}(\mathcal{H})}\subset \mathcal{K}(\mathcal{H}) $
  Let $T_n$ be a sequence in $\mathcal{F}(\mathcal{H})$ and $T_n \to
  T \in B(\mathcal{H})$ (in
  norm). We'll show that $\overline{T(B_1^\mathcal{H})}$ is closed and
  totally bounded, which is equivalent to compactness in metric
  spaces.

  Let $\epsilon > 0$ be given. Then there exist some $N \in
  \mathbb{N}$ such that $\|T_N - T\| < \epsilon$. Moreover \[
    \overline{T_N(B_1^\mathcal{H})} \subset \bigcup_{\eta \in  B_1^\mathcal{H}}
    B_\epsilon(T_N(\eta))
  \]
  By the compactness of $\overline{T_N(B_1^\mathcal{H})}$ there exist
  $\eta_1 , \eta_2 , \ldots , \eta_k \in B_1^\mathcal{H}$ such that  \[
    \overline{T_N(B_1^\mathcal{H})} \subset \bigcup_{i =
    1}^{k}B_\epsilon(T_N(\eta_i))
  \]
  Thus for an arbitrary $\eta \in B_1^\mathcal{H}$, $\exists i \le  k$
  such that $\|T_N(\eta) - T_N(\eta_i)\| < \epsilon$. Then
  \begin{align*}
    \|T(\eta) - T(\eta_i)\| &\le \|T(\eta) - T_N(\eta)\| + \|T_N(\eta)
    - T_N(\eta_i)\| + \|T_N(\eta_i) - T(\eta_i)\| \\
    & < 3 \epsilon
  \end{align*}
  Thus we see that
  \begin{align*}
    T(B_1^\mathcal{H}) \subset \bigcup_{i = 1}^{k} B_{3\epsilon}(T(\eta_i))
  \end{align*}
  and hence
  \begin{align*}
    \overline{T(B_1^\mathcal{H})} \subset \bigcup_{i = 1}^{k}
    \overline{B_{3\epsilon}(T(\eta_i))}
  \end{align*}
  which gives total boundedness. Closure of $\overline{
  T(B_1^\mathcal{H})}$ is obvious. Hence we see that
  $\overline{T(B_1^\mathcal{H})}$. Thus $T$ is compact and we see
  $\overline{\mathcal{F}(\mathcal{H})} \subset \mathcal{K}(\mathcal{H})$.

  Conversely, let $T \in B(\mathcal{H})$ be compact and $\epsilon>0$ be given.
  So $\exists \eta_1, \eta_2, \ldots , \eta_m \in B_1(\mathcal{H})$ such that
  \begin{align*}
    \overline{T(B_1^\mathcal{H})} &\subset \bigcup_{i =
    1}^{n}B_\epsilon(T(\eta_i))
  \end{align*}
  Assume that $\mathcal{H}$ is separable. Let $\{ e_n  \ : \  n \in
  \mathbb{N} \}$ be an orthonormal basis for $\mathcal{H}$ and let $P_n =
  P_{\textrm{Span}\{e_1 , e_2 , \ldots , e_n\}}$. Then choose $ N>0$
  such that for all $n >N$, $\forall i = 1, 2, \ldots m$, \[
    \|(P_nT - T)(\eta_i)\| < \epsilon
  \]
  Then $\forall \xi \in \mathcal{H}$ with $\|\xi\| = 1$. Choose $1 \le j \le
  m$ such that $\|T(\xi) - T(\eta_j)\| < \epsilon$. Then
  \begin{align*}
    \|P_nT(\xi) - T(\xi)\| & \le \|P_nT(\xi) - P_nT(\eta_j)\| +
    \|P_nT(\eta_j) - T(\eta_j)\| + \|T(\eta_j) - T(\xi)\| \\
    &< 3\epsilon
  \end{align*}

  Which gives that $\|P_nT - T\| < 3\epsilon$ for all $n \in \mathbb{N}$.

  The generalization of the case when $\mathcal{H}$ is non-separable, will use
  the fact that $T(\mathcal{H})$ is separable and using an orthonormal basis
  for the pre-image of $T(\mathcal{H})$, which will again be separable.
\end{proof}

\begin{proposition}
  Let $\mathcal{K}$ be a separable Hilbert space. Let $\{ e_n \ : \ n \in
  \mathbb{N} \}$ be an orthonormal basis for $\mathcal{K}$.
  For each $n \in
  \mathbb{N}$, let $P_n$ be the projection to the $\textrm{Span}\{
  e_1 , e_2 , \ldots , e_n \}$. Then for all $ x \in \mathcal{K}$ \[
    P_n(x) \to x
  \]
  pointwise
\end{proposition}
\begin{proof}
  Direct application of \autoref{OrthonormalBaisEquiv}(iv)
\end{proof}


% TeX_root = ../main.tex

\begin{lemma}
  If $T \in B(\mathcal{H})$ is compact, then $T(\mathcal{H})$ is separable.
\end{lemma}
\begin{proof}
  For the sake of contradiction, assume that $T(\mathcal{H})$ is not
  separable. Then $T(\mathcal{H})$ must have an uncountable orthonormal set $E$.
  This is because if $E$ is a countable basis, then $\mathbb{Q}E + i
  \mathbb{Q}E$ will be a countable dense subset of
  $T(\mathcal{H})$ making it separable. \textcolor{red}{verify}
\end{proof}

\begin{corollary}
  The set $\mathcal{K}(\mathcal{H})$ of all compact operators on $
  \mathcal{H}$ is a closed
  two-sided ideal in $B(\mathcal{H})$.
\end{corollary}
\begin{proof}
  Use the fact that compact operators are
  the closure of finite rank operators.
\end{proof}

\begin{corollary}
  $T \in \mathcal{K}(\mathcal{H})$ implies $T^* \in \mathcal{K}(\mathcal{H})$.
\end{corollary}
\begin{proof}
  Use the fact that $T$ is finite rank
  implies $T^*$ is finite rank.
\end{proof}

\begin{example}
  Let \[
    T: L^2([0, 1], m) \to L^2([0, 1], m)
  \]
  be defined as $T(f)(x) = xf(x)$. Prove that $T = T^*$ and that $T$
  has no eigenvectors.

  Only for joseph: If $\xi = \eta$ almost everywhere, then show that  $f \xi = f
  \eta$ almost everywhere if $ f \in C([0, 1])$
\end{example}
\begin{proof}
  See homework-5
\end{proof}

\begin{example}
  Let $\alpha_n$ be a bounded sequence in $\mathbb{C}$. Consider $T :
  \ell^2 \to \ell^2$, such that \[
    T(\delta_n) = \alpha_n \delta_n
  \]
  for all $n \in \mathbb{N}$. Then $T$ is bounded with $\|T\| =
  \|(\alpha_n)\|_\infty$.
\end{example}

\begin{example}
  \label{CompactiffEigvalsconverge0}
  Prove that the $T$ above is compact if and only if $(\alpha_n) \in c_{\bf 0}$
  Prove that $T^*(\delta_n) = \overline{\alpha_n}\delta_n$
\end{example}
\begin{proof}
  See homework-5
\end{proof}

\section{Spectral Theorem for Compact Normal Operators}
We'll be working to prove the following theorem.
\begin{theorem}
  Let $\mathcal{H}$ be separable Hilbert space and $T\in
  \mathcal{K}(H)$ be normal. i.e
  $TT^* = T^*T$. Then there exist an orthonormal basis $\{ e_n  \ :
  \  n \in \mathbb{N} \}$ of $\mathcal{H}$, and a sequence $(\alpha_n) \in
  c_{\bf 0}$, such that $T(e_n) = \alpha_n e_n$.
\end{theorem}

\begin{lemma}
  Let $T \in \mathcal{K}(\mathcal{H})$ be self-adjoint. Then there
  exist a non-zero $\lambda \in \mathbb{C}$ such that $
  \textrm{Ker}(T - \lambda I) \neq \{ 0 \}$.
\end{lemma}
\begin{proof}
  See the first part of the proof of
  \autoref{SpectralTheoremforCompactSAOperators}.
\end{proof}

\begin{definition}
  Let $T \in B(\mathcal{H})$. A subspace $W \leqslant H$ is said to be
  invariant under $T$ if $T(W) \subset W$. We say $W$ reduces $T$, if
  $T(W) \subset W$ and $T(W^\perp) \subset W^\perp$
\end{definition}

\begin{example}
  Let $T \in B(\mathcal{H})$.
  Prove that $W \subset \mathcal{H}$ is invariant under $T$ if and
  only if $P_WT = TP_W$
  Prove that $W$ reduces $T$ if and only if $P_WT = TP_W$ and
  $P_{W^\perp}T = T P_{W^\perp}$.
\end{example}
\begin{proof}
  See homework-5
\end{proof}

\begin{proposition}
  \label{InvariantSubspacesofSelfAdjointOperatorsReduce} Let $T \in
  B(\mathcal{H})$ be self adjoint with $T(\mathcal{M}) \subset
  \mathcal{M}$, where $\mathcal{M}$ is a closed subspace of
  $\mathcal{H}$. Then $\mathcal{M}$ reduces $T$.
\end{proposition}
\begin{proof}
  Let $m \in \mathcal{M}$ and $x \in \mathcal{M}^\perp$. Then
  \begin{align*}
    \langle T(x) ,  m \rangle = \langle x , T(m) \rangle = 0
  \end{align*}
  shows that $T(\mathcal{M}^\perp) \subset \mathcal{M}^\perp$.
\end{proof}


% TeX_root = ../main.tex

\chapter{$L^{p}$ Spaces}

\begin{definition}
  A function $\phi: (a,  b) \to \mathbb{R}$ is called convex if
  $$\phi(tx + (1-t)y) \le t \phi(x) + (1-t) \phi(y)$$
  for all $x, y \in
  (a, b)$ and $0 \le t \le 1$.
\end{definition}

\begin{proposition}
  A function $\phi: (a, b) \to \mathbb{R}$ is convex if and only if
  for $u, s, t$ with $ a < u \le t \le s < b$, we have \[
    \phi(t) \le \phi(s) \frac{u-t}{u-s} + \phi(u) \frac{t-s}{u-s}
  \]
  or equivalently using \[
    \phi(t) - \phi(s) = \frac{t-s}{u-s} (\phi(u) - \phi(s))
  \]
  satisfies \[
    \frac{\phi(t)-\phi(s)}{t-s} \le \frac{\phi(u)-\phi(s)}{u-s}
  \]
\end{proposition}

\begin{theorem}
  A function $\phi: (a, b) \to \mathbb{R}$ that is convex is continuous.
\end{theorem}
\begin{proof}
  Let $S = (s, \phi(s)), X = (  x, \phi(x)), Y = (y, \phi(y))$, with
  $a < s \le x \le y < b$.

  Draw secands and refer Rudin.
\end{proof}

\begin{theorem}[Jensen's Inequality]
  Let $(X, \mathcal{M}, \mu)$ be a measure space with $\mu(X) = 1$.
  If $f \in L^1(\mu)$ and for each $x \in X$, $a < f(x)< b$ and $
  \phi$ is convex on $(a, b)$, then \[
    \phi \Bigg(\int  f \ d \mu \Bigg) \le \int (\phi \circ f)\ d \mu
  \]
\end{theorem}
\begin{proof}
  We know by convexity that for $u \le s \le t$, \[
    \frac{\phi(t)-\phi(s)}{t-s} \le \frac{\phi(u)-\phi(s)}{u-s}
  \]
  Then there is $\beta$ such that \[
    \frac{\phi(t)-\phi(s)}{t-s} \le \beta \le \frac{\phi(u)-\phi(s)}{u-s}
  \]
  Consider LHS Inequality to get
  \begin{align*}
    \phi(t) - \phi(s) & \le \beta (t-s) \\
    \phi(s) & \ge \phi(t) + \beta(s-t)
  \end{align*}
  for $s < t$, and similarly by the RHS we get\[
    \phi(u)-\phi(s) \ge \beta(u-s)
  \]
  Hence in both the cases $(t = f(x), \ u = f(x))$ \[
    \phi(f(x)) - \phi(s) - \beta(f(x) - s) \ge 0
  \]
  Now integrating this gives \[
    \int \phi \circ f\ d \mu - \phi(t) - \beta \Big( \int  f \ d \mu -
    s \Big) \ge 0
  \]
  Choosing $s = \int  f \ d \mu$ gives out inequality.
\end{proof}

\begin{example}
  Take $\mu$ to be the probablity measure on $X = \{ 1, 2, 3, \ldots
  n \}$, assume $\mu(\{ j \})= \alpha_j> 0$. Then for $b_1 , b_2 ,
  \ldots , b_n > 0$, we have \[
    b_1^{\alpha_1}b_2^{\alpha_2}\ldots b_n^{\alpha_n} \le \sum_{j =
    1}^{n} \alpha_jb_j
  \]
\end{example}
\begin{proof}
  Use the convexity of $x \to e^x$, and let $ b_j = e^{c_j}$.
\end{proof}

\begin{theorem}[Holder's Inequality]
  Let $(X, \mathcal{M},  \mu)$ be a measure space, $f, g : X \to [0,
  \infty]$ be measurable. Then for $1 < p < \infty$, with $1/p + 1/q
  = 1$, then \[
    \int  fg \ d \mu \le \Bigg(\int  f^p \ d \mu \Bigg)^{\frac{1}{p}}
    \Bigg(\int  g^q \ d \mu\Bigg)^{\frac{1}{q}} \equiv \| f\|_p \|g\|_q
  \]
  and \[
    \Bigg(\int(f+g)^p \ d \mu\Bigg)^{\frac{1}{p}} \le \|f\|_p + \|g\|_p
  \]
\end{theorem}



% TeX_root = ../main.tex

\marginnote{\scriptsize 03/04/2025 }

% TeX_root = ../main.tex

\marginnote{ \scriptsize 19/11/2024}

Given $T$ as above, denote by $\sigma(T)$, the set of eigenvalues of
$T$ and  for each eigenvalue $ \lambda \in \sigma(T)$, denote
$P_\lambda:= P_{\textrm{Ker}(T- \lambda I)}$. Then spectral theorem gives
\begin{align*}
  T = \sum_{\lambda \in  \sigma(T)}  \lambda P_\lambda
\end{align*}

% \begin{exercise}
%   Show $\sum_{\lambda \in  \Lambda}  f(\lambda) P_\lambda$
%   convergences pointwise and in norm. i.e
%   \begin{align*}
%     \Big( \sum_{i = 1}^{N} f(\lambda_i) P_{\lambda_i}\Big)
%   \end{align*}
% \end{exercise}
% \begin{proof}
%   \textcolor{red}{verify}
% \end{proof}

\begin{definition}
  Let $\Xi$ be a non-empty subset of $B(\mathcal{H})$. The commutant
  of $\Xi$, is the set $\Xi^\prime = \{ S \in B(\mathcal{H})  \ : \    ST
  = TS, \forall T \in \Xi \}$
\end{definition}

\begin{exercise}
  \begin{enumerate}
    \item  Prove that for any set $\Xi^\prime$ is a closed subspace
      of $B(\mathcal{H})$.
    \item $\Xi^\prime = \textrm{span}(\Xi)^\prime$
  \end{enumerate}
\end{exercise}
\begin{proof}
  \begin{enumerate}[label=(\arabic*)]
    \item Let $(S_n)$ be a sequence in $\Xi^\prime$ such that $S_n
      \to S$ in $ B(\mathcal{H})$. Then by the continuity of the map
      $m_T: B(\mathcal{H}) \to B(\mathcal{H}):= R \to TR$ as $
      \|m_T(R)\| \le \|T\| \|R\|$, we see that $S_nT \to ST$ and
      similarly by the continuity of $m^\prime_T: S \to TS$, we see that
      $TS_n \to TS$. Then by the algebra of limits, for $ T \in \Xi$,
      we see that
      \begin{align*}
        0 = S_nT - TS_n \to ST - TS
      \end{align*}
      which forces $S \in \Xi^\prime$.
    \item
  \end{enumerate}
\end{proof}

\section{Functional Calculus}

\begin{theorem}[Functional calculus for compact normal operators]
  Let $T \in B(\mathcal{H})$ be compact normal with spectral decomposition
  \begin{align*}
    T = \sum_{\lambda \in  \sigma(T)} \lambda P_\lambda.
  \end{align*}
  For each $f \in L^\infty(\mathbb{C})$, define $f(T) := \sum_{ \lambda \in
  \sigma(T)}  f(\lambda)   P_\lambda$. Then
  \begin{enumerate}[label=(\arabic*)]
    \item $\forall f \in \ell^\infty(\mathbb{C}), \|f(T)\| = \sup \{
      |f(\lambda)|  \ : \  \lambda \in \sigma(T) \}$
    \item The map $\ell^\infty(\mathbb{C}) \to B(\mathcal{H})$
      defines as $f \to f(T)$ is linear.
    \item $\forall f, g \in \ell^\infty(\mathbb{C}), (fg)(T) = f(T)g(T)$
    \item $\overline{f}(T) = f(T)^*$
    \item $ \{ f(T)  \ : \   f \in \ell^\infty(\mathbb{C}) \} = \{ T
      \}^{\prime \prime}$
  \end{enumerate}
\end{theorem}
\begin{proof}
  \begin{enumerate}[label=(\arabic*)]
    \item \textcolor{red}{exercise}
    \item \textcolor{red}{exercise}
    \item \textcolor{red}{exercise}
    \item \textcolor{red}{exercise}
    \item Let $f \in \ell^\infty(\mathbb{C})$, and let $S \in
      \{T\}^\prime$. Since $ ST = TS$, by
      \autoref{CommutingOperatorsPreserveInvariantSpaces} for every $
      \lambda \in \sigma(T)$, the eigenspace $\textrm{Ker}(T - \lambda I)$ is
      invariant under $S$. Hence they are all reducing for $S$. i.e
      $SP_\lambda = P_\lambda S$. Thus,
      \begin{align*}
        Sf(T) &= S \sum_{\lambda \in  \sigma(T)}  f(\lambda)   P_\lambda \\
        &= \sum_{\lambda \in  \sigma(T)} f( \lambda) S P_\lambda \\
        &= \sum_{\lambda \in  \sigma(T)} f(  \lambda) P_\lambda  S \\
        &= f(T) S
      \end{align*}
      So we get $f(T) \in \{ T \}^{\prime \prime}$

      Let $ R \in \{ T \}^{\prime\prime}$. Note that $\forall \lambda \neq
      \lambda^\prime \in \sigma(T)$, $P_\lambda P_{\lambda^\prime} = 0$ by
      \autoref{DistinctEigenvectorsAreOrthogonal}. Then for all $\gamma \in
      \sigma(T)$, we get $$P_\gamma T = P_\gamma \sum_{\lambda \in
      \sigma(T)}  \lambda P_\lambda  = \gamma P_\gamma$$
      So $P_\gamma \in \{ T \}^\prime$. Thus $RP_\gamma = P_\gamma R$ for
      all $ \gamma \in \sigma(T)$.

      Fix $\gamma \in \sigma(T)$. Given $A \in
      B(P_\gamma(\mathcal{H}))$. Let $    S:= P_\gamma A P_\gamma :
      \mathcal{H} \to \mathcal{H}$. Then
      \begin{align*}
        ST &= P_\gamma AP_\gamma \sum_{\lambda \in  \sigma(T)}
        \lambda P_\lambda \\
        &= \gamma P_\gamma AP_\gamma \\
        &= \gamma S \\
        &= TS
      \end{align*}
      Thus $RS = SR$. This shows that $R|_{P_\gamma(\mathcal{H})}$
      commutes with every $A \in B(P_\gamma(\mathcal{H}))$. Hence
      there exist such a scalar $    f(\gamma) \in \mathbb{C}$ such
      that $R|_{P_\gamma(\mathcal{H})} = f(\gamma) I_{
      P_\gamma(\mathcal{H})}$. Thus $R = \sum_{\lambda \in
      \sigma(T)}  f(\gamma) P_\gamma$ and $ f \in \ell^\infty(\mathbb{C})$.
  \end{enumerate}

  \begin{corollary}
    The mapping above restricted to $\sigma(T)$ is an isometric
    bijective linear multiplicative $*$-preserving map.
  \end{corollary}

  \begin{definition}
    Let $T \in B(\mathcal{H})$. Define $\sigma(T) = \{ \lambda \in
    \mathbb{C}  \ : \   T - \lambda I \textrm{ is not invertible} \}$
  \end{definition}
\end{proof}


% TeX_root = ../main.tex

\marginnote{\scriptsize 21/11/2024 }

\begin{definition}
  A Banach algebra $\mathcal{A}$ is a Banach space with a
  multiplicatit a ring with addition satisfying $\|ab\| \le
  \|a\|\|b\|$ for all $ a, b \in \mathcal{A}$. We say $\mathcal{A}$
  is unital if the ring above is unital with multiplicative identity
  $1_{\mathcal{A}}$. Units (invertible elements)
  may also exist similarly.
\end{definition}

\begin{definition}
  Given a Banach algebra $\mathcal{A}$, and $a \in \mathcal{A}$, the
  spectrum of $a$, denoted by
  \begin{align*}
    \sigma(a) = \{ \lambda \in \mathbb{C}  \ : \  \lambda
    1_{\mathcal{A}} - a \textrm{ is not invertible in } \mathcal{A} \}
  \end{align*}
\end{definition}

\begin{example}
  Let $(\alpha_n) \in \textbf{c}_0$ and $T: \ell^{2} \to \ell^{2}$
  such that $T((x_n)) = (\alpha_n x_n)$. We claim that
  \begin{align*}
    \sigma(T) = \{ \alpha_n  \ : \  n \in \mathbb{N} \} \cup \{ 0 \}
  \end{align*}
\end{example}
\begin{proof}
  Since $T$ is a compact operator (see Homework-5), we must have $0
  \in \sigma(T)$. Otherwise if $T$ is invertible, we'll get $I  = T
  \circ T^{-1}$ be also compact, which is a contradiction since
  $\ell^{2}$ is infinite dimensional. Moreover since $(T - \alpha_n I
  )(e_n) = 0$, we have $ \alpha_n \in \sigma(T)$.

  Now assume that $\beta \notin \{ \alpha_n  \ : \  n \in \mathbb{N}
  \} \cup \{ 0 \}$. Then let $S \in B(\ell^{2})$ defined by
  \begin{align*}
    S(e_n) = \frac{1}{\alpha-\beta} e_n
  \end{align*}
  Then \textcolor{red}{show that indeed $S \in B(\ell^{2})$} and $(T
  - \beta I)S = S(T - \beta I) = I$
\end{proof}

\begin{example}
  Let $T \in B(L^{2}([0, 1]))$ such that $T(f)(x) = xf(x)$ for all $
  f \in L^{2}([0, 1])$. Then $  \sigma(T) = [0, 1]$.
\end{example}
\begin{proof}
  First let us see that $0 \in \sigma(T)$. Suppose $  S = T^{-1}$
  exists. Then $\forall n \in \mathbb{N}$ if $ f = S \chi_{[0, 1]}$,
  then $Tf = \chi_{[0, 1]}$. So $x f(x) =
  \chi_{[0, 1]}$. But this is absurd, since $\frac{1}{x} \chi_{[0,
  1]} \notin L^2([0, 1])$.
\end{proof}

\begin{lemma}
  Let $S \in B(\mathcal{H})$ such that $\|S - T\| \le 1$. Then $S$ is
  invertible.
\end{lemma}
\begin{proof}
  Since $\|S  -I\| \le 1$,
  \begin{align*}
    \sum_{n \in \mathbb{N}} \|(S-I)^n\| \le \sum_{n \in \mathbb{N}} \|S - I\|^n
  \end{align*}
  Moreover since $B(\mathcal{H})$ is a Banach space, absolutely
  convergent sequences converge and this gives that
  $\sum_{n \in \mathbb{N}} (I - S)^n$ converges. Thus
  \begin{align*}
    R = \sum_{n=1}^\infty (I - S)^n
  \end{align*}
  exists. Now for each $N \in \mathbb{N}$, we have
  \begin{align*}
    S \Big(\sum_{n = 0}^{N} (I - S)^n\Big) = (I - (I -
    S))\Big(\sum_{n = 1}^{N} (I - S)^n\Big) &= \sum_{n = 0}^{n} (I -
    S)^n - \sum_{n = 0}^{N} (I-S)^{n+1}\\
    &= I - (I-S)^{N+1}
  \end{align*}
  which converges to $0$ as $n \to \infty$.
\end{proof}

\begin{corollary}
  The set of invertible operators is open in $B(\mathcal{H})$.
\end{corollary}
\begin{proof}
  Let $S$ be invertible. Then
  \begin{align*}
    \|T - S\| = \|S(S^{-1}T - I)\| \le \| S\| \|S^{-1}T - I\|
  \end{align*}
\end{proof}

\begin{theorem}
  For any $T \in B(\mathcal{H})$, $\sigma(T)$ is a non-empty, compact
  subspace of $\mathbb{C}$.
\end{theorem}
\begin{proof}
  Observe that the function $f: \lambda \to T - \lambda I$ is
  continuous. Then $f^{-1}(G(\mathcal{A}))$ is open, where
  $G(\mathcal{A})$ is the collection of all invertible elements of
  $B(\mathcal{H})$. But $\sigma(T)^{c} = f^{-1}(G(\mathcal{A}))$.
  So we see $\sigma(T)$ is closed. Moreover let $\lambda \in
  \mathbb{C}$ with $\|T\| < |\lambda|$. Then,
  \begin{align*}
    -T + \lambda I = \lambda(\frac{-T}{\lambda} + I)
  \end{align*}
  Then $\|S - I\| = \| \frac{T}{\lambda}\| = \frac{\|T\|}{|\lambda|} < 1$
  Hence show that $\sigma(T)$ is bounded.
\end{proof}

% TeX_root = ../main.tex

\marginnote{\scriptsize 26/11/2024 }

\begin{example}
  Let $T : \ell^{2}(\mathbb{N}) \to \ell^{2}(\mathbb{N})$ be defined as
  $T \delta_1 = 0$ and $T \delta_{n+1} = \delta_n$. Then clearly
  $(\delta_1, 0)$ is an eigenvector-eigenvalue pair. Moreover for any
  $\lambda \in \mathbb{C}$ with $|\lambda| < 1$. Then
  \begin{align*}
    T: (1, \lambda, \lambda^2, \lambda^3, \ldots) &\to  ( \lambda,
    \lambda^2, \lambda^3, \ldots) \\
    &= \lambda(1, \lambda, \lambda^2, \lambda^3, \ldots)
  \end{align*}
  Thus we see that the open unit disk $\mathbb{D} \subset \sigma(T)$.
  Moreover since $\|T\| = 1$, we see that $\sigma(T) = \overline{\mathbb{D}}$.
\end{example}

\begin{proposition}
  \begin{align*}
    \sigma(T^*) = \overline{\sigma(T)}
  \end{align*}
\end{proposition}
\begin{proof}
  Use the fact that $(T^{-1})^* = (T^*)^{-1}$ and resolvent.
\end{proof}

\begin{example}
  Now consider $S: \ell^{2}(\mathbb{N}) \to \ell^{2}(\mathbb{N})$,
  defined as $S \delta_n = \delta_{n+1}$. Then $\sigma(S) =
  \overline{\mathbb{D}} = \overline{\sigma(T)}$ since $S = T^*$ and
  $\sigma(S) = \overline{\sigma(T)}$.
\end{example}

\begin{theorem}[Spectral Mapping Theorem]
  Let $T \in B(\mathcal{H})$, and let $\phi: \mathbb{C} \to
  \mathbb{C}$ be holomorphic in an open neighborhood of $\sigma(T)$.
  Then, $\phi(\sigma(T)) = \sigma(\phi(T))$.
\end{theorem}

Let $T \in B(\mathcal{H})$ be self adjoint. Consider the set
\begin{align*}
  \mathcal{A} = \overline{\{ p(T)  \ : \   p(x) \textrm{ is a
  polynomial} \}}^{\|\cdot\|}
\end{align*}
Then observe that $\mathcal{A}$ is a closed algebra of
$B(\mathcal{H})$. Then for any $\textbf{0} \neq f \in \mathcal{A}^*$,
assume that $f(ab) = f(a)f(b)$ and $a \in \textrm{Ker}(f)$ and $b \in
\mathcal{A}$, then $ab \in \textrm{Ker}(f)$. In particular let $b =
f(a)I = a \in \mathcal{A}$. Then $f(b) = 0 \implies f(a) I - a \in
\textrm{Ker}(f)$. Since $f \neq 0$, $\textbf{1} \not\in
\textrm{Ker}(f)$ and hence $f(a) \in \sigma(a)$

Conversely, let $\lambda \in \sigma(a)$. Thus $\lambda I - a$ is not
invertible. Thus the ideal generated by $\lambda I - a$ is not the
whole of $\mathcal{A}$. Then by Zorn's lemma, there is a maximal
ideal $\mathcal{M}$, which contain $\lambda I - a$.

\begin{exercise}
  The set $\{ \textbf{0} \neq \phi \in \mathcal{A}^* \ : \
  \phi(ab) = \phi(a) \phi(b) \}$ is weak * compact.
\end{exercise}
\begin{proof}
  \textcolor{red}{Homework}
\end{proof}

\begin{exercise}
  Maximal ideals of $B(\mathcal{H})$ are closed
\end{exercise}
\begin{proof}
  \textcolor{red}{Homework}. Show that closure of an ideal is an ideal.
\end{proof}

Define $f: \mathcal{A} \to \mathcal{A}/\mathcal{M} \cong \mathbb{C}$,
defined as $f(a) = a + \mathcal{M}$.

\begin{theorem}[Gelfand Transform]

\end{theorem}

% TeX_root = ../main.tex

\marginnote{ \scriptsize 13/11/2024}

\begin{theorem}
  Let $\mathcal{H}$ be a vector space over $\mathbb{C}$ with a
  positive semidefinite sesquilinear $ \langle \cdot , \cdot \rangle
  $ form and the associated seminorm $ \|\cdot\|$, then for all $x, y
  \in \mathcal{H}$,
  \begin{align*}
    \|x+y\| \le \|x\| + \|y\|
  \end{align*}
\end{theorem}
\begin{proof}
  \textcolor{red}{verify}
\end{proof}

\begin{remark}
  If $\langle \cdot , \cdot \rangle $ is an inner product space, then
  $\|\cdot\|$ defines a norm in $ \mathcal{H}$.
\end{remark}

\begin{definition}
  If $ \mathcal{H}$ be an inner product. If $\mathcal{H}$ is complete
  with respect to the topology induced by the inner product, then it
  is called a Hilbert space.
\end{definition}

\begin{example}
  $L^2(\mu)$, with functions identified that agrees almost everywhere
  is a Hilbert space when endowed with the inner product
  \begin{align*}
    \langle f , g \rangle  = \int f \overline{g} \ d \mu
  \end{align*}
\end{example}

\begin{proposition}
  Let $\mathcal{H}$ be a Hilbert space. Then for $g \in \mathcal{H}$
  \begin{align*}
    \lambda_g : \mathcal{H} \to \mathbb{C} := f \to \langle f , g \rangle
  \end{align*}
  is a linear, uniformly continuous functional.
\end{proposition}
\begin{proof}
  Use Cauchy-Schwarz inequality.
\end{proof}

\begin{definition}
  Let $H$ be a Hilbert space. We say $x, y \in \mathcal{H}$ are
  orthogonal if $\langle  x , y \rangle  = 0$. We also write $x \perp y$.

  If $ S \subset \mathcal{H}$, define
  \begin{align*}
    S^\perp = \{ x \in \mathcal{H}  \ : \   x \perp s, \forall s \in S \}
  \end{align*}
\end{definition}

\begin{theorem}
  If $S \subset H$, then $S^\perp$ is a closed subspace of $\mathcal{H}$.
\end{theorem}
\begin{proof}
  Let $z \in \mathcal{H}$. Then $K_z = z^\perp =
  \textrm{Ker}(\lambda_z)$ is closed. Observe that
  \begin{align*}
    S^\perp = \bigcap_{s \in S}K_s
  \end{align*}
  is closed as well.
\end{proof}

\begin{lemma}
  Let $\mathcal{M}$ be a closed subspace of a Hilbert space
  $\mathcal{H}$, and $h \in \mathcal{H}$, then there is a unique $m
  \in \mathcal{M}$ that minimizes the distance to $h$
\end{lemma}
\begin{proof}
  We recall the parallelogram law
  \begin{align*}
    \|x + y\|^2 + \|x -y\|^2 = 2 ( \|x\|^2 + \|y\|^2)
  \end{align*}
  and write for $x, y \in \mathcal{H}$,
  \begin{align*}
    \|x - y\|^2 = 2 (\|x\|^2 + \|y\|^2) - \|x + y\|^2
  \end{align*}
  Let $\delta = \inf \{ \|m - h\|  \ : \  m \in \mathcal{M}  \}$.
  There there is a sequence of $m_j \in \mathcal{M}$ such that $\|m_j
  - h\| \to \delta$. To show that $m_j$ is a Cauchy seqeunce,
  consider $x = m_j -h, y = m_i - h$. Then
  \begin{align*}
    \frac{x+y}{2} = \frac{m_i + m_j}{2} - h
  \end{align*}
  and we see that
  \begin{align*}
    \Big\|\frac{x+y}{2}\Big\| = \Big|\frac{m_i+m_j}{2} - h\Big|
  \end{align*}
  Then by prarllelogram law,
  \begin{align*}
    \|m_j - m_i\|^2 &= 2(\|x\|^2 + \|y\|^2) - \|x + y\|^2 \\
    &=2(\|m_j - h\|^2 + \|m_i - h\|^2 - \|m_i + m_j - 2h\|^2) \\
  \end{align*}
  \textcolor{red}{verify}

  This shows that, we can make $\|m_i - m_j\|$ arbitrarily small by
  requiring $ i, j \in \mathbb{N}$ for a similarly large
  $\mathbb{N}$, meaning $ m_j $ is Cauchy. Since $ \mathcal{M}$ is
  closed and a closed and a closed subset of a complete metric space,
  $\mathcal{M}$ is complete, so there is a point $m \in \mathcal{M}$,
  where $m_j$ converges to.
  We'll prove the uniqueness in the next lecture.
\end{proof}



% TeX_root = ../main.tex

\marginnote{ \scriptsize 19/11/24}

\begin{theorem}[Orthogonal Projections]
  If $M$ is a closed subspace of a Hilbert space $\mathcal{H}$, then
  for each $h \in \mathcal{H}$, there is a unique pair $m \in M$ and
  $ n \in M^\perp$ such that $h = m + n$ and $\|h\|^2 = \|m\|^ +
  \|n\|^2$. Moreover, the maps $P(h) = m$, $Q(h) = n$ are linear and
  write $m = Ph, n = Qh$
\end{theorem}
\begin{proof}
  Fix $h \in \mathcal{H}$. Pick $m \in M$ that is nearest to $h$,
  using precedent lemma. Let $n = h-m$. We will show $ n \in
  M^\perp$. For any $ x \in M$, $\alpha \in \mathbb{C}$, consider
  \begin{align*}
    \|n - \alpha x\|^2 &= \|n\|^2 - 2 \Re (\alpha \langle x , n
    \rangle ) + |\alpha|^2 \|x\|^2
  \end{align*}
  and $\|h\|^2 = \|m\|^2 + \|n\|^2$. Suppose $\langle  x , n \rangle
  \neq 0$. Choose $\alpha = \frac{t}{\langle x , n \rangle }$ for $t \in
  \mathbb{R}$. Then
  \begin{align*}
    \|n - \alpha x\|^2 &= \|n\|^2 - 2t + \frac{t^2\|x\|^2}{|\langle x
    , n \rangle |^2}
  \end{align*}
  For sufficiently small $t$, we have $2t >
  \frac{t^2\|x\|^2}{|\langle x, n \rangle |^2}$. Then we'd get
  \begin{align*}
    \|n - \alpha x\|^2 < \|n\|^2
  \end{align*}
  Replacing $n$ with $h -m$, we get
  \begin{align*}
    \|h - (m + \alpha x)\|^2 < \| h -m\|^2
  \end{align*}
  which contradicts the optimality of $m$ for distance to $h$. We
  conclude $ \langle x , n \rangle = 0$.
  This is true for each $x \in M$. Thus, we get $ n \in M^\perp$.

  Now to see that the choice of $m$(and $n$) is unique, let $h =
  m^\prime + n^\prime$, with $m^\prime \in M, n^\prime \in M^\perp$.
  Then $m + n = m^\prime + n^\prime$, which implies
  \begin{align*}
    \underbrace{m - m^\prime}_{\in M} = \underbrace{n^\prime - n}_{\in M^\perp}
  \end{align*}
  which forces $ m = m^\prime, n = n^\prime$, since $ M \cap M^\perp = \{ 0 \}$

  Now for the linearity of $P, Q$, let $h = h_1 + \alpha h_2$, where
  $h_1 = m_1 + n_1, h_2 = m_2 + n_2$ for $ m_i \in M,
  n_i \in M^\perp$. Then
  \begin{align*}
    h = \underbrace{m_1 + \alpha m_2}_{\in M} + \underbrace{ n_1 +
    \alpha n_2}_{\in M^\perp}
  \end{align*}
  This shows $P: h \to m, Q: h \to n$ are linear maps.
\end{proof}

\begin{definition}
  The maps $P, Q$ above are called orthogonal projections onto $M$
  and $M^\perp$, respectively.
\end{definition}

\begin{corollary}
  Let $M$ be a proper closed subspace in a Hilbert space
  $\mathcal{H}$. Then $M^\perp \neq \{ 0\}$.
\end{corollary}

\begin{exercise}
  Let $M \subset L^2(\mathbb{R})$ such that
  \begin{align*}
    M = \{ f \in L^2(\mathbb{R})  \ : \  f(x) = \alpha_n  \textrm{
    for almost every } x \in [n, n+1) \}
  \end{align*}
\end{exercise}



\printbibliography[heading=bibintoc]

\end{document}
