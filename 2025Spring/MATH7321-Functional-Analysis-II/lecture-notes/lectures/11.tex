% TeX_root = ../main.tex

\marginnote{\scriptsize 25/02/2025 }

\begin{exercise}
  Show that if $\omega_{\xi, \eta} \in B(\mathcal{H})^*$ such that
  $\omega_{\xi, \eta}(T) = \langle T \xi ,  \eta \rangle $, then
  $T_{\omega_{\xi, \eta}} = P_{\xi, \eta}$.
\end{exercise}
\begin{solution}
  \textcolor{red}{verify}
\end{solution}

\begin{theorem}
  $T_\phi \in \mathcal{K}(\mathcal{H})$ for every $\phi \in
  \mathcal{K}(\mathcal{H})^*$
\end{theorem}
\begin{proof}
  First assume $T_\phi \ge 0$. Let $T_\phi = \int_{0}^{\infty}
  \lambda \ d E$ for a spectral measure $E$. Let $\epsilon > 0$ be
  given. Let $P = \int_{\varepsilon}^{\infty}  \ dE$. We'll show that
  $P$ is finite rank.

  For the sake of contradiction, assume
  otherwise. Let $\{ \xi_n \}_{n \in \mathbb{N}}$ be an orthonormal
  set such that $P \xi_n =  \xi_n$ for all $n \in \mathbb{N}$.
  $\forall N \in \mathbb{N}$, let $S_N = \sum_{n = 1}^{N} P_{\xi_n,
  \eta_n}$. Then $S_N \in \mathcal{K}(\mathcal{H})$ is a finite rank
  projection. In particular $\|S_N\| = 1$.

  Then $\forall N \in \mathbb{N}$,
  \begin{align*}
    \|\phi\| & \ge |\phi(S_n)| \\
    & = \Big| \sum_{ n = 1}^{N} \phi(P_{\xi_n, \xi_n}) \Big| \\
    & = \sum_{n = 1}^{N} \langle T_\phi \xi_n , \xi_n \rangle \\
    & = \sum_{n = 1}^{N} \int_{0}^{\infty}  \lambda \ d E_{\xi_n, \xi_n} \\
    & \ge \sum_{n = 1}^{N} \int_{\varepsilon}^{\infty}  \lambda \ d
    E_{\xi_n, \xi_n} \\
    & \ge \varepsilon \sum_{n = 1}^{N} \int_{\varepsilon}^{\infty}
    \ d E_{\xi_n, \xi_n} \\
    &= \varepsilon N
  \end{align*}
  which is absurd, since we can take $N \to \infty$.

  Now, we have
  \begin{align*}
    \|T_\phi - T_\phi P\| &= \Big \| \int_{0}^{\infty}  \lambda \ d E
    - \int_{\varepsilon}^{\infty}  \lambda \ d E \Big \| \\
    &= \Big \|\int_{0}^{\varepsilon}  \lambda \ d E \Big \| \\
    & \le \varepsilon
  \end{align*}
  Hence we see that $T_\phi \in \mathcal{K}(\mathcal{H})$.

  Now for the general case, let $T_\phi = V |T_\phi|$ be the polar
  decomposition of $T_\phi$. Define $\psi \in \mathcal{K}(\mathcal{H})^*$, by
  \begin{align*}
    \psi(T) := \phi(TV^*)
  \end{align*}
  for every $T \in \mathcal{K}(\mathcal{H})$. Then $\langle T_\psi
  \xi, \eta \rangle = \psi(P_{\xi, \eta}) = \phi(P_{\xi, \eta}V^*) =
  \phi(P_{ \xi, V\eta}) = \langle   T_\phi  \xi ,  V\eta \rangle =
  \langle V^* T_\phi \xi ,  \eta \rangle = \langle |T_\phi| \xi ,
  \eta \rangle$. So $T_\psi = |T_\phi|$, and by the first part of the
  proof, $|T_\phi| \in \mathcal{K}(\mathcal{H})$. Thus $T_\phi = V
  |T_\phi| \in \mathcal{K}(\mathcal{H})$.
\end{proof}

\begin{theorem}
  Let $\phi \in \mathcal{K}(\mathcal{H})^*$. Then there are
  orthonormal basis $\{ \xi_i \}, \{ \eta_i \}$ for $ \mathcal{H}$
  and $(\alpha_i) \in \ell^{1}(\mathbb{R})_{+}$ such that
  \begin{align*}
    \phi = \sum_{i \in \mathbb{N}} \alpha_i \omega_{\xi_i, \eta_i}
  \end{align*}
\end{theorem}
\begin{proof}
  For every $T \in B(\mathcal{H})$, observe that $|\sum \alpha_n
  \omega_{\xi_n, \eta_n}(T)| \le (\sum \alpha_i) \| T\|$.

  By the previous theorem, there exists a $\alpha_i \in
  \textbf{c}_0$, and orthonormal basis $\{ \xi_i \}, \{ \eta_i \}$
  for $\mathcal{H}$ such that
  \begin{align*}
    T_\phi = \sum_{i \in \mathbb{N}} \alpha_i P_{\xi_i, \eta_i}
  \end{align*}

  Now let $(\beta_i) \in \textbf{c}_0$ be arbitrary. Let $S = \sum_{i
  \in \mathbb{N}} \beta_i P_{\eta_i, \xi_i}$. Then $ S \in
  \mathcal{K}(\mathcal{H})$, and $\|S\| \le \|(\beta_i)\|_\infty$. So,
  \begin{align*}
    \|\phi\| \|(\beta_i)\|_\infty & \ge |\phi(S)| \\
    &= \Bigg | \phi \Big( \sum_{i \in \mathbb{N}} \beta_iP_{\eta_i,
    \xi_i}\Big) \Bigg | \\
    &= \Big | \sum_{i \in \mathbb{N}} \beta_i \langle T_\phi \eta_i ,
    \xi_i \rangle \Big | \\
    &= \Big | \sum_{i \in \mathbb{N}} \beta_i \big \langle \big(
    \sum_{j \in \mathbb{N}} \alpha_j P_{\xi_j, \eta_j}\bigg) \eta_i ,
    \xi_i \bigg \rangle \Big | \\
    &= \Big | \sum_{i, j \in \mathbb{N}} \beta_i \alpha_j \langle
    P_{\xi_j, \eta_j}(\eta_i) ,  \xi_i \rangle \Big | \\
    &= \Big | \sum_{i \in \mathbb{N}} \beta_i \alpha_i \Big |
  \end{align*}

  Since $(\beta_i)$ was chosen arbitrarily, we get that $(\alpha_i)
  \in \ell^{1}$. Thus the series $\sum_{i \in \mathbb{N}} \alpha_i
  \omega_{\xi_i, \eta_i}$ converges in norm in $\mathcal{K}(\mathcal{H})^*$.
\end{proof}

\begin{exercise}
  Show that every finite rank operator is in the span of $P_{\xi,
  \eta}$, for $  \xi, \eta \in \mathcal{H}$.
\end{exercise}

\begin{definition}
  An operator $T \in B(\mathcal{H})$ of the form $T = T_\phi$ for
  some $\phi \in   \mathcal{K}(\mathcal{H})^*$ is called a trace
  class operator. We denote by $\mathcal{T}(\mathcal{H})$, the space
  of all trace class operators.
\end{definition}

\begin{exercise}
  \begin{align*}
    \Big \| \sum_{i \in \mathbb{N}} \alpha_i \omega_{\xi_i, \eta_i}
    \Big \| = \|(\alpha_i)\|_1
  \end{align*}
  and $\|T_\phi\|\le \|\phi\|$
\end{exercise}
