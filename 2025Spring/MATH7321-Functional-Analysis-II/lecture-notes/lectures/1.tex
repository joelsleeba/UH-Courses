% TeX_root = ../main.tex

\chapter{Banach Algebras}
\marginnote{\scriptsize 14/01/2025}

Let $\mathcal{H}$ be a separable Hilbert space, and  $T \in
\mathcal{K}(\mathcal{H})$ be self adjoint, and $X$ be the
orthonormal basis for $\mathcal{H}$ consisting of the eigenvectors of
$T$. Then $\mathcal{H} \cong \ell^{2}(X)$ and then $T$ can be
identified with a multiplication operator $M_f$ with $f \in
\ell^{\infty}(X)$ such that $f(x) = \lambda_x$, where $\lambda_x$ is
the eigenvalue for the eigenvector, $x \in X$.

If $T \in B(\mathcal{H})$ is normal, then $\textrm{span}\{ T^mT^{*n}
\ : \   m, n \in \mathbb{N} \}$ is an involutive algebra.

\section{Preliminaries}

\begin{definition}
  An algebra over $\mathbb{C}$(or $\mathbb{R}$) is a vector space
  over $\mathbb{C}$(or $\mathbb{R}$) with a product that is
  associative and distributive.

  A normed algebra is an algebra $\mathcal{A}$ equipped with a norm
  \marginnote{ \scriptsize \it product is continuous}
  such that $\|ab\| \le \|a\|\|b\|$ for all $a, b \in \mathcal{A}$.

  If the norm is complete in a normed algebra, we call it a Banach algebra.

  An algebra is unital if has a neutral element $1$ with respect to
  the multiplication. It is called commutative if $ab = ba$ for all
  $a, b \in \mathcal{A}$

  An involution on an algebra is a map $*: \mathcal{A} \to
  \mathcal{A}$ such that
  \begin{enumerate}[label=(\arabic*)]
    \item $(\alpha a + b)^* = \overline{\alpha} a^* + b^*$
    \item $(ab)^* = b^*a^*$
    \item $(a^*)^* = a$
  \end{enumerate}
  for all $a, b \in \mathcal{A}$ and $\alpha \in \mathbb{C}$

  A normed $*$-algebra is a normed algebra with an involution $*$
  such that $\|a^*\| = \|a\|$.
\end{definition}

\begin{definition}
  Let $\mathcal{A}$ be a  unital Banach algebra. An element $a \in
  \mathcal{A}$ is invertible if there exists $b \in \mathcal{A}$ such
  that $ab = ba = 1$.

  The spectrum of $a$ is the set
  \begin{align*}
    \sigma(a) = \{ \lambda \in \mathbb{C}  \ : \  \lambda 1 - a
    \textrm{ is not invertible } \}
  \end{align*}
\end{definition}

\begin{lemma}
  In any unital Banach algebra, the set of invertible elements is open.
\end{lemma}
\begin{proof}
  If $\|x\| < 1$, then
  \begin{align*}
    \frac{1}{1 - x} = \sum_{n \in \mathbb{N}} x^n
  \end{align*}
  or equivalently if $\|1 - x\| < 1$, then
  \begin{align*}
    \frac{1}{x} = \sum_{n \in \mathbb{N}} (1-x)^n
  \end{align*}
  Now let $x_0$ be invertible in the unital Banach algebra and let $x
  \in B(x_0, \frac{1}{\|x_0^{-1}\|})$. Then $\|xx_0^{-1}  - 1\| \le
  \|x - x_0\| \|x_0^{-1}\| < 1$ and similarly $\|x_0^{-1}x - 1\| <
  1$. This shows that $xx_0^{-1}, x_0^{-1}x$ are invertible with
  inverses (assumed) $y, z$ respectively. Thus $xx_0^{-1}y = 1 =
  zx_0^{-1}x$. \textcolor{red}{verify}.
\end{proof}

\begin{definition}
  The spectral radius of $a \in \mathcal{A}$ is defined to be
  \begin{align*}
    r(a) = \sup \{ |\lambda|  \ : \   \lambda \in \sigma(a) \}
  \end{align*}
\end{definition}

\begin{lemma} For all $a \in \mathcal{A}$,
  \begin{align*}
    r(a) \le \|a\|
  \end{align*}
\end{lemma}

\begin{theorem}
  Let $\mathcal{A}$ be a unital Banach algebra and $a \in
  \mathcal{A}$. Then $\sigma(a)$ is a non-empty compact subset of $\mathbb{C}$.
\end{theorem}
\begin{proof}
  \textcolor{red}{proof incoming for non-empty.} See last semester
  notes for proof of compactness.
\end{proof}

\begin{lemma}
  Let $X$ be a normed space. Then there exist a compact set $K$ and
  an isometric linear map $X \to C(K)$.
\end{lemma}
\begin{proof}
  embed $X$ to $X^{**}$, with $K$ being the weak * closed unit ball of $X^*$
\end{proof}

\begin{definition}
  Let $\mathcal{A}$ be a commutative Banach algebra. The spectrum of
  $\mathcal{A}$ is the set $\textrm{sp}(\mathcal{A}) = \{ \tau \in
  \mathcal{A}^*  \ : \  \tau(ab) = \tau(a)\tau(b), a, b \in \mathcal{A} \}$.
\end{definition}

\begin{lemma}
  \label{lem:spectrum_is_compact}
  For every $\tau \in \textrm{sp}(\mathcal{A})$,
  \begin{enumerate}
    \item $\tau(1) = 1$
    \item $\|\tau\| = 1$
    \item $ \textrm{sp}(\mathcal{A})$ is a weak * closed subset of
      $\mathcal{A}^*$ (Hence compact)
  \end{enumerate}
\end{lemma}
\begin{proof}
  \begin{enumerate}[label=(\arabic*)]
    \item Since $\tau(a) = \tau(a \cdot \textbf{1}) =
      \tau(a)\tau(\textbf{1})$ for all $ a \in \mathcal{A}$, it is
      immediate that $\tau(\textbf{1}) = 1 \in \mathbb{C}$.
    \item Since $\tau(\textbf{1}) = 1$, it is clear that $\|\tau\|
      \ge 1$. The other inequality follows from the fact that
      $\tau(a) \in \sigma(a)$ (see that $a - \tau(a) \textbf{1} \in
      \textrm{Ker}(\tau)$) and thus $|\tau(a)| \le r(a) \le \|a\|$.
    \item Let $ \tau, \phi \in \textrm{sp}(A)$. Then for all $\hat{a}
      \in \hat{\mathcal{A}}$ (image of $\mathcal{A}$ in
      $\mathcal{A}^{**}$ under the natural inclusion), we have
      \begin{align*}
        |\hat{a}(\tau - \phi)| =  |\hat{a}(\tau) - \hat{a}(\phi)|
        = |\tau(a) - \phi(a)| =
        | (\tau - \phi)(a)| \le \|a\||\tau - \phi|
      \end{align*}
      Thus if $\tau_\alpha$ is any weak * Cauchy net, $\tau_\alpha(a)$
      converges for all $a \in \mathcal{A}$. Let $\tau(a) =
      \lim_{\alpha} \tau_\alpha(a)$. Then all we have left is to
      prove that $\tau \in \textrm{sp}(\mathcal{A})$. But since
      $\tau_(\alpha) \to \tau(a)$ and $\tau_\alpha(b) \to \tau(b)$,
      by the algebra of limits in $\mathbb{C}$, we get $ \tau(ab) =
      \tau(a) \tau(b)$ proving $\tau \in \textrm{sp}(\mathcal{A})$.

      Thus we see that $\textrm{sp}(\mathcal{A})$ is a weak * closed
      subspace of the unit ball of $\mathcal{A}^*$ which is closed by
      the Banach Alaoglu theorem.
  \end{enumerate}
\end{proof}
