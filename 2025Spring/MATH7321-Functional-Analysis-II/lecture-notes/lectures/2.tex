% TeX_root = ../main.tex

\marginnote{\scriptsize 16/01/2025 }

\begin{definition}
  An ideal $I \subsetneq \mathcal{A}$ is called maximal if for any
  ideal $I \subset J$, then either $J = I$ or $J = \mathcal{A}$
\end{definition}

\begin{proposition}
  Recall that $R/I$ is a field if and only if $I$ is a
  maximal ideal of the ring $R$.
\end{proposition}

\begin{remark}
  While we consider ideals of an algebra, since the algebra has a
  vector space structure, we demand the ideal to be subspace with
  respect to the underlying linear operations
\end{remark}

\begin{lemma}
  If $\mathcal{A}$ is a (unital) Banach algebra, and $I$ is a closed
  ideal, then $\mathcal{A}/I$ is a (unital) Banach algebra.
\end{lemma}
\begin{proof}
  Since $I$ is an ideal, the ring structure of $A/I$ is well defined.
  Moreover since $\mathcal{A}$ is a Banach space and $I$ is a closed
  subspace, we know that $\mathcal{A}/I$ is a Banach space. Thus, we
  just need to verify the norm inequality
  \begin{align*}
    \|ab + I\| \le \|a + I\|\|b + I\|
  \end{align*}
  By the definition of the quotient norm for any $\varepsilon > 0$,
  there exist a $i_a, i_b \in
  I$ such that
  \begin{align*}
    \|a + i_a\| \le \|a + I\| + \varepsilon, \quad \|b + i_b\| \le
    \|b + I\| + \varepsilon
  \end{align*}
  Then,
  \begin{align*}
    \|ab + I\| &\le \|ab + i_ab + i_b a + i_aib\|  \\
    &= \|(a + i_a)( b + i_b)\| \\
    &\le \| a + i_a\| \|b + i_b\| \\
    &\le (\|a + I\| + \varepsilon)(\|b + I\| + \varepsilon) \\
    &\le \|a + I\|\|b + I\| + \varepsilon \|b  + I\| + \varepsilon
    \|a + I\| + \varepsilon^2
  \end{align*}
  Since $\varepsilon$ was chosen arbitrarily, this gives our result.
\end{proof}

\begin{lemma}
  In a unital Banach algebra, every maximal ideal is closed.
\end{lemma}
\begin{proof}
  Take a maximal ideal, take its closure, then it must be either the
  ideal itself or the whole of the algebra. If it contains the whole
  of the algebra, then the unital element must be there. Then the
  original ideal must contain invertible elements by the openness of
  the set of invertible elements. This will make the original ideal,
  the whole of the algebra, which is a contradiction.
\end{proof}

\begin{lemma}
  If $\mathcal{A}$ is a Banach algebra that is a division ring (If
  every non-zero element has an inverse), then
  $\mathcal{A} = \mathbb{C}$.
  \label{2:A=C}
\end{lemma}
\begin{proof}
  Let $0 \neq a \in \mathcal{A}$. Let $\lambda \in \sigma(a)$. Then
  $\lambda1 - a$ is not invertible. Hence $\lambda1 - a = 0$. So $a =
  \lambda 1$. Hence $\mathcal{A} = \mathbb{C}$.
\end{proof}

\begin{corollary}
  Let $\mathcal{A}$ be a unital commutative Banach algebra, and $I$
  be a maximal ideal. Then $\mathcal{A}/I = \mathbb{C}$.
\end{corollary}

\begin{lemma}
  \label{specturm_of_comm_alg_and_spectrum_of_element}
  For every $a \in \mathcal{A}$, a commutative unital Banach algebra,
  \begin{align*}
    \sigma(a) = \{ \tau(a)  \ : \   \tau \in \textrm{sp}(\mathcal{  A}) \}
  \end{align*}
\end{lemma}
\begin{proof}
  Let $\tau \in \textrm{sp}(\mathcal{A})$, then $\tau(\tau(a)1 -
  a) = 0$ and therefore $\tau(a)1 - a \in \textrm{Ker}(\tau)$,
  hence it is not invertible. Hence $\tau(a) \in \sigma(a)$.

  Conversely, let $ \lambda \in \sigma(a)$, then $\lambda 1 - a$ is
  not invertible, thus the ideal $I = \langle \lambda1 - a \rangle$
  is a proper ideal, since $r(\lambda 1 - a)$ will not be invertible
  for any $r \in \mathcal{A}$. So $1 \notin \langle \lambda1 - a
  \rangle$. By Zorn's lemma, $\langle \lambda1 -a \rangle $ is
  contained in a maximal ideal $I_\lambda$.

  Define $\tau: \mathcal{A} \to \mathbb{C} = \mathcal{A}/I_\lambda :=
  x \mapsto x + I_\lambda$. Then $\tau \in \textrm{sp}(\mathcal{A})$,
  and $\tau(a) = a + I_\lambda = \lambda + I_\lambda$ since
  $\lambda1 - a \in I_\lambda$. Now by the identification of $A/I_\lambda$
  with $\mathbb{C}$ as in \autoref{2:A=C}, we see that $\tau(a) =
  \lambda$.
\end{proof}

\begin{definition}
  Let $\mathcal{A}$ be a commutative Banach algebra. Define
  \begin{align*}
    \Phi :  \mathcal{A} \to  C(\textrm{sp}(\mathcal{A})) :=
    \Phi(a)(\tau) = \tau(a)
  \end{align*}
  for all $a \in \mathcal{A}, \tau \in \textrm{sp}(\mathcal{A})$. The
  map $\Phi$ is called the \textbf{Gelfand transform}.
\end{definition}

\begin{theorem}
  $\Phi$ is a contractive algebra homomorphism with $\| \Phi(a)\| = r(a)$.
\end{theorem}
\begin{proof}
  That $\Phi$ is contractive follows easily from
  \begin{align*}
    \|\Phi(a)(\tau)\| = \|\tau(a)\| \le \|a\|
  \end{align*}
  since $\tau$ is a contraction as proved in
  \autoref{lem:spectrum_is_compact}. Linearity and multiplicativity
  of $\Phi$ follows form the fact that every element $\tau \in
  \textrm{sp}(\mathcal{A})$ is linear and multiplicative on $\mathcal{A}$.
\end{proof}

\begin{remark}[Maximal ideals of $\mathcal{A}$ and $\textrm{sp}(\mathcal{A})$]
  Let $\mathcal{A}$ be unital commutative Banach algebra. Let $\tau
  \in \textrm{sp}(\mathcal{A})$, then $\textrm{Ker}(\tau)$ is a
  closed ideal of $\mathcal{A}$, and $A/\textrm{Ker}(\tau) \cong
  \mathbb{C}$ by the first isomorphism theorem. So
  $\textrm{Ker}(\tau)$ is a maximal ideal. The converse of this is
  also true. Natural map to the quotient space of a maximal ideal
  (which is now a filed isomorphic to $\mathbb{C}$) gives an element
  of the $\textrm{sp}(\mathcal{A})$. Hence $\textrm{sp}(\mathcal{A})$
  can be identified with the maximal ideals of $\mathcal{A}$.
\end{remark}

\begin{remark}
  Suppose $\tau, \tau^\prime \in \textrm{sp}(\mathcal{A})$, with
  $\textrm{Ker}(\tau) =\textrm{Ker}(\tau^\prime)$. Let $a \in
  \mathcal{A}$, then $\tau(a)1 - a \in \textrm{Ker}(\tau) =
  \textrm{Ker}(\tau^\prime)$ implies $\tau(a) = \tau^\prime(a)$ for
  all $a \in \mathcal{A}$.
\end{remark}

\begin{remark}
  Combining both of the above, we see that $\textrm{Ker}(\Phi)$ is
  the intersection of all maximal ideals of $\mathcal{A}$, that is
  the radical of $\mathcal{A}$.
\end{remark}

\begin{theorem}
  \label{thm:beurling}
  Let $\mathcal{A}$ be a Banach algebra. Then $\forall a \in
  \mathcal{A}$, we have
  \begin{align*}
    r(a) = \lim_{n \to \infty} \|a^n\|^{1/n}
  \end{align*}
\end{theorem}
\begin{proof}
  \textcolor{red}{verify}
\end{proof}

\begin{corollary}
  If $\|a^2\| = \|a\|^2$, then $r(a) = \|a\|$ and the Gelfand
  transform will be halal.
  \marginnote{ \scriptsize \it \textcolor{red}{not sure. Might need
  C* algebra structure}}
\end{corollary}

\begin{example}
  Let $T \in B(\mathcal{H})$ be self-adjoint. Let $\mathcal{A} =
  \overline{\textrm{span}}\{ T^n  \ : \  n \in \mathbb{N} \cup {0}
  \}$. Then $\mathcal{A}$ is a unital Banach algebra. Moreover we have
  \begin{align*}
    \|T^2\| = \|T\|^{2}
  \end{align*}
  by the self adjointness of $T$. Thus the Gelfand transform $\Phi$
  is isometric on $\mathbb{R}$-$\overline{\textrm{span}}\{T^n\}$.
\end{example}
