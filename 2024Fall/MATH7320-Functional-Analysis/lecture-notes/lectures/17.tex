% TeX_root = ../main.tex
\section{Compact Operators}
\begin{definition}
  let $\mathcal{X}, \mathcal{Y}$ be Banach Spaces. A linear map $T : \mathcal{
  X} \to \mathcal{Y}$ is called a
  compact operator if $\overline{T(B_1^\mathcal{X})}$ is compact,
  where $B_1^\mathcal{X}$
  is the unit ball of $\mathcal{X}$. We denote by
  $\mathcal{K}(\mathcal{X}, \mathcal{Y})$, the set of all
  compact operators.
\end{definition}

\begin{example}
  If either $\mathcal{X}$ or $\mathcal{Y}$ is finite, then every linear map $T:
  \mathcal{X} \to \mathcal{Y}$
  is compact. If $\mathcal{X}$ be any infinite dimensional Banach space. Then
  $T = \textrm{Id} : \mathcal{X} \to X$ is not compact. This follows from
  \autoref{ClosedUnitBallisCompactiffFiniteDim}.
\end{example}

\begin{definition}
  $T: \mathcal{X} \to \mathcal{Y}$ is called a finite rank if the
  dimension of $T$ is
  finite. Then the dimension of the image is called the rank of the
  operator. Let $\mathcal{F}(\mathcal{X}, \mathcal{Y})$ denote finite
  rank operators.
\end{definition}

\begin{lemma}
  Every compact operator is bounded.
\end{lemma}
\begin{proof}
  Every compact set is bounded in any metric space.
\end{proof}

\begin{theorem}
  Let $\mathcal{H}$ be a Hilbert space. Then $\mathcal{K}(\mathcal{H}) =
  \overline{\mathcal{F}(\mathcal{H})}^{\|\cdot\|}$
\end{theorem}
\begin{proof}
  It is evident that $\mathcal{F}(\mathcal{H}) \subset
  \mathcal{K}(\mathcal{H})$. We'll
  now show that
  $\overline{\mathcal{F}(\mathcal{H})}\subset \mathcal{K}(\mathcal{H}) $
  Let $T_n$ be a sequence in $\mathcal{F}(\mathcal{H})$ and $T_n \to
  T \in B(\mathcal{H})$ (in
  norm). We'll show that $\overline{T(B_1^\mathcal{H})}$ is closed and
  totally bounded, which is equivalent to compactness in metric
  spaces.

  Let $\epsilon > 0$ be given. Then there exist some $N \in
  \mathbb{N}$ such that $\|T_N - T\| < \epsilon$. Moreover \[
    \overline{T_N(B_1^\mathcal{H})} \subset \bigcup_{\eta \in  B_1^\mathcal{H}}
    B_\epsilon(T_N(\eta))
  \]
  By the compactness of $\overline{T_N(B_1^\mathcal{H})}$ there exist
  $\eta_1 , \eta_2 , \ldots , \eta_k \in B_1^\mathcal{H}$ such that  \[
    \overline{T_N(B_1^\mathcal{H})} \subset \bigcup_{i =
    1}^{k}B_\epsilon(T_N(\eta_i))
  \]
  Thus for an arbitrary $\eta \in B_1^\mathcal{H}$, $\exists i \le  k$
  such that $\|T_N(\eta) - T_N(\eta_i)\| < \epsilon$. Then
  \begin{align*}
    \|T(\eta) - T(\eta_i)\| &\le \|T(\eta) - T_N(\eta)\| + \|T_N(\eta)
    - T_N(\eta_i)\| + \|T_N(\eta_i) - T(\eta_i)\| \\
    & < 3 \epsilon
  \end{align*}
  Thus we see that
  \begin{align*}
    T(B_1^\mathcal{H}) \subset \bigcup_{i = 1}^{k} B_{3\epsilon}(T(\eta_i))
  \end{align*}
  and hence
  \begin{align*}
    \overline{T(B_1^\mathcal{H})} \subset \bigcup_{i = 1}^{k}
    \overline{B_{3\epsilon}(T(\eta_i))}
  \end{align*}
  which gives total boundedness. Closure of $\overline{
  T(B_1^\mathcal{H})}$ is obvious. Hence we see that
  $\overline{T(B_1^\mathcal{H})}$. Thus $T$ is compact and we see
  $\overline{\mathcal{F}(\mathcal{H})} \subset \mathcal{K}(\mathcal{H})$.

  Conversely, let $T \in B(\mathcal{H})$ be compact and $\epsilon>0$ be given.
  So $\exists \eta_1, \eta_2, \ldots , \eta_m \in B_1(\mathcal{H})$ such that
  \begin{align*}
    \overline{T(B_1^\mathcal{H})} &\subset \bigcup_{i =
    1}^{n}B_\epsilon(T(\eta_i))
  \end{align*}
  Assume that $\mathcal{H}$ is separable. Let $\{ e_n  \ : \  n \in
  \mathbb{N} \}$ be an orthonormal basis for $\mathcal{H}$ and let $P_n =
  P_{\textrm{Span}\{e_1 , e_2 , \ldots , e_n\}}$. Then choose $ N>0$
  such that for all $n >N$, $\forall i = 1, 2, \ldots m$, \[
    \|(P_nT - T)(\eta_i)\| < \epsilon
  \]
  Then $\forall \xi \in \mathcal{H}$ with $\|\xi\| = 1$. Choose $1 \le j \le
  m$ such that $\|T(\xi) - T(\eta_j)\| < \epsilon$. Then
  \begin{align*}
    \|P_nT(\xi) - T(\xi)\| & \le \|P_nT(\xi) - P_nT(\eta_j)\| +
    \|P_nT(\eta_j) - T(\eta_j)\| + \|T(\eta_j) - T(\xi)\| \\
    &< 3\epsilon
  \end{align*}

  Which gives that $\|P_nT - T\| < 3\epsilon$ for all $n \in \mathbb{N}$.

  The generalization of the case when $\mathcal{H}$ is non-separable, will use
  the fact that $T(\mathcal{H})$ is separable and using an orthonormal basis
  for the pre-image of $T(\mathcal{H})$, which will again be separable.
\end{proof}

\begin{proposition}
  Let $\mathcal{K}$ be a separable Hilbert space. Let $\{ e_n \ : \ n \in
  \mathbb{N} \}$ be an orthonormal basis for $\mathcal{K}$.
  For each $n \in
  \mathbb{N}$, let $P_n$ be the projection to the $\textrm{Span}\{
  e_1 , e_2 , \ldots , e_n \}$. Then for all $ x \in \mathcal{K}$ \[
    P_n(x) \to x
  \]
  pointwise
\end{proposition}
\begin{proof}
  Direct application of \autoref{OrthonormalBaisEquiv}(iv)
\end{proof}

