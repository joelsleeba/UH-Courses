% TeX_root = ../main.tex

\begin{example}
  Let $X$ be an infinite dimensional normed space. Then the set $A = \{ x
  \in X  \ : \  \|x\| = 1 \}$ is norm closed. But the weak closure of
  $A$ is the set \[
    \overline{ A}^w  = \overline{\{ x \in X  \ : \  \|x\| = 1 \}}^{w}
    = \{ y \in X  \ : \  \|y\| \le 1 \}
  \]
  Hence $A$ is an example of a norm closed set, which is not weak closed.
\end{example}

\begin{theorem}
  Let $X$ be a normed space and let $K \subset X$ be convex subset of
  $X$. Then the norm and the weak closure of $K$ coincide.
\end{theorem}
\begin{proof}
  Since norm topology is stronger than weak topology, we get
  $\overline{K}^{\|\cdot\|} \subset \overline{ K}^{w}$.
  Let $ x \in X$ such that $x \notin \overline{K}^{\|\cdot\|}$. Now
  since $ \{ x \}, K$ are convex and compact, by Hahn-Banach
  separation theorem, there is a $f \in X^*, s \in \mathbb{R},
  \epsilon > 0$ such that \[
    |f(x)| \le s < s + \epsilon \le |f(y)|, \quad \forall y \in
    \overline{K}^{\|\cdot\|}
  \]

  Since the set $\{ z \in X  \ : \  |f(z)| \ge s + \epsilon \}$ is
  weakly closed, and contains $K$, it must contain $\overline{K}^w$.
  Hence $ x \notin \overline{K}^w$
\end{proof}

\begin{corollary}
  Let $X$ be a normed space, and $(x_i)_{i \in I}$ be a net in $X$
  such that $x_i \to x$ weakly in $X$. Then there exists a net
  $(y_j)_{j \in J}$ of finite convex combinations of $\{ x_i  \ : \ i
  \in I  \}$ such that $y_i \to x$ in norm.
\end{corollary}
\begin{proof}
  \textcolor{red}{verify}
\end{proof}

\begin{proposition}
  If $K$ is a convex subset of a LCTVS. Then $\overline{K}$ is also convex.
\end{proposition}
\begin{proof}
  \textcolor{red}{verify}
\end{proof}

\begin{proposition}
  Show that $\textbf{c}_{0}$ is weakly closed and weak * dense in $\ell_\infty$.
\end{proposition}
\begin{proof}
  \textcolor{red}{verify}
\end{proof}

\begin{theorem}[Krein-Smulian Theorem]
  Let $X$ be a Banach space, and let $C$ be a convex subset of $X^*$.
  Then $C$ is weak * closed if and only if $C \cap \{ f \in X^*  \ :
  \  \|f\| \le r \}$ is weak * closed for all $r \in \mathbb{R}^+$.

  \textit{This should even work if we just take $n \in \mathbb{N}$}.
  \textcolor{red}{verify}.
\end{theorem}

\begin{corollary}
  Let $Z$ be a subspace of $X^*$. Then $Z$ is weak * closed if and
  only if $\{ \phi \in Z  \ : \  \|\phi\| \le 1 \}$ is closed.
\end{corollary}

\begin{corollary}
  Let $X$ be a separable Banach space. Then a convex subset $Z$ of
  $X^*$ is weak * closed if and only if it is weak * sequentially closed.
\end{corollary}
\begin{proof}
  Since $X$ is separable, for every $r > 0$, the set $ \{ f \in X^*
  \ : \  \|f\| \le r \}$ is weak * metrizable. Thus $Z \cap \{ f \in
  X^*  \ : \  \|f\| \le 1 \}$ is weak * closed iff it is weak *
  sequentially closed.
\end{proof}

\begin{corollary}
  Let $X$ be a separable Banach space, and $\phi \in X^{**}$. Then
  $\phi$ is weak * continuous if and only if $\phi$ is sequentially
  continuous i.e $f_n \to f$ weak * in $X^*$ implies $\phi(f_n) \to \phi(f)$.
\end{corollary}
\begin{proof}
  Assume $\phi$ is sequentially weak * continuous. Let $C =
  \textrm{Ker}(\phi)$ be a subspace of $X^*$. Let $g_n \in C$ be a
  sequence such that $g_n \to g \in X^*$. Then by assumption, $0 =
  \phi(g_n) \to \phi(g)$ implies $\phi(g) = 0$ and therefore $g \in
  C$. This shows  that $ C$ is weak * sequentially closed, hence weak
  * closed by the separability of $X$. Hence $\phi$ is weak * continuous.
\end{proof}

