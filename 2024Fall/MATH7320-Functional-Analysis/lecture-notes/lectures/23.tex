% TeX_root = ../main.tex

\marginnote{\scriptsize 26/11/2024 }

\begin{example}
  Let $T : \ell^{2}(\mathbb{N}) \to \ell^{2}(\mathbb{N})$ be defined as
  $T \delta_1 = 0$ and $T \delta_{n+1} = \delta_n$. Then clearly
  $(\delta_1, 0)$ is an eigenvector-eigenvalue pair. Moreover for any
  $\lambda \in \mathbb{C}$ with $|\lambda| < 1$. Then
  \begin{align*}
    T: (1, \lambda, \lambda^2, \lambda^3, \ldots) &\to  ( \lambda,
    \lambda^2, \lambda^3, \ldots) \\
    &= \lambda(1, \lambda, \lambda^2, \lambda^3, \ldots)
  \end{align*}
  Thus we see that the open unit disk $\mathbb{D} \subset \sigma(T)$.
  Moreover since $\|T\| = 1$, we see that $\sigma(T) = \overline{\mathbb{D}}$.
\end{example}

\begin{proposition}
  \begin{align*}
    \sigma(T^*) = \overline{\sigma(T)}
  \end{align*}
\end{proposition}
\begin{proof}
  Use the fact that $(T^{-1})^* = (T^*)^{-1}$ and resolvent.
\end{proof}

\begin{example}
  Now consider $S: \ell^{2}(\mathbb{N}) \to \ell^{2}(\mathbb{N})$,
  defined as $S \delta_n = \delta_{n+1}$. Then $\sigma(S) =
  \overline{\mathbb{D}} = \overline{\sigma(T)}$ since $S = T^*$ and
  $\sigma(S) = \overline{\sigma(T)}$.
\end{example}

\begin{theorem}[Spectral Mapping Theorem]
  Let $T \in B(\mathcal{H})$, and let $\phi: \mathbb{C} \to
  \mathbb{C}$ be holomorphic in an open neighborhood of $\sigma(T)$.
  Then, $\phi(\sigma(T)) = \sigma(\phi(T))$.
\end{theorem}

Let $T \in B(\mathcal{H})$ be self adjoint. Consider the set
\begin{align*}
  \mathcal{A} = \overline{\{ p(T)  \ : \   p(x) \textrm{ is a
  polynomial} \}}^{\|\cdot\|}
\end{align*}
Then observe that $\mathcal{A}$ is a closed algebra of
$B(\mathcal{H})$. Then for any $\textbf{0} \neq f \in \mathcal{A}^*$,
assume that $f(ab) = f(a)f(b)$ and $a \in \textrm{Ker}(f)$ and $b \in
\mathcal{A}$, then $ab \in \textrm{Ker}(f)$. In particular let $b =
f(a)I = a \in \mathcal{A}$. Then $f(b) = 0 \implies f(a) I - a \in
\textrm{Ker}(f)$. Since $f \neq 0$, $\textbf{1} \not\in
\textrm{Ker}(f)$ and hence $f(a) \in \sigma(a)$

Conversely, let $\lambda \in \sigma(a)$. Thus $\lambda I - a$ is not
invertible. Thus the ideal generated by $\lambda I - a$ is not the
whole of $\mathcal{A}$. Then by Zorn's lemma, there is a maximal
ideal $\mathcal{M}$, which contain $\lambda I - a$.

\begin{exercise}
  The set $\{ \textbf{0} \neq \phi \in \mathcal{A}^* \ : \
  \phi(ab) = \phi(a) \phi(b) \}$ is weak * compact.
\end{exercise}
\begin{proof}
  \textcolor{red}{Homework}
\end{proof}

\begin{exercise}
  Maximal ideals of $B(\mathcal{H})$ are closed
\end{exercise}
\begin{proof}
  \textcolor{red}{Homework}. Show that closure of an ideal is an ideal.
\end{proof}

Define $f: \mathcal{A} \to \mathcal{A}/\mathcal{M} \cong \mathbb{C}$,
defined as $f(a) = a + \mathcal{M}$.

\begin{theorem}[Gelfand Transform]

\end{theorem}
