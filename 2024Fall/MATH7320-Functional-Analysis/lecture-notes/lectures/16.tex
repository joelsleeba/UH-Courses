% TeX_root = ../main.tex

\begin{proposition}
  $I-P_K = P_{K^\perp}$
\end{proposition}
\begin{proof}
  Let $x \in X$ and $k \in K$. Then
  \begin{align*}
    \langle (I - P_k)(x), k \rangle &= \langle  x - P_k(x) , k \rangle \\
    &= \langle  x , k \rangle - \langle P_k(x) ,  k \rangle  \\
    &= \langle x , k \rangle  - \langle x , P_k(k) \rangle \\
    &= \langle x , k \rangle  - \langle  x , k \rangle  \\
    &= 0
  \end{align*}
  Shows that
\end{proof}

\begin{proposition}
  Let $K$ be a closed subspace of $\mathcal{H}$. Let $E \subset K$ be
  an orthonormal basis
  for $K$. Extend $E$ to an orthonormal basis $\tilde{E}$ for
  $\mathcal{H}$. Then \[
    P_K|_E = I_K, \quad P_K|_{\tilde{E} \setminus E} = 0
  \]
\end{proposition}

\begin{remark}[Parserval's Inequality]
  Let $\mathcal{H}$ be a Hilbert space. Let $E$ be an orthonormal set. Then for
  every vector $\eta \in \mathcal{H}$, \[
    \|\eta\|^2 \ge \sum_{e \in E} |\langle \eta ,  e \rangle |^2
  \]
\end{remark}

\begin{lemma}
  \label{PerpPerpisSpan}
  Let $S$ be a nonempty subset of $\mathcal{H}$. Then \[
    (S^\perp)^\perp  = \overline{\textrm{span}}(S)
  \]
\end{lemma}
\begin{proof}
  Notice from the above proposition that $\textrm{Ker}(P_K) =
  K^\perp$. Thus $(S^\perp)^\perp = \textrm{Ker}(P_{S^\perp})$. Since
  $P_{S^\perp}(s) = 0$ for all $s \in S$, we see that $S \subset
  \textrm{Ker}(P_{S^\perp})$. Moreover, since
  $\textrm{Ker}(P_{S^\perp})$ is a closed subspace, we see that
  $\overline{ \textrm{\textrm{span}}}(S) \subset
  \textrm{Ker}(P_{S^\perp}) = ( S^\perp)^\perp$.

  Now see that if $A \subset B$, then $B^\perp \subset A^\perp$. Note
  that if $\langle  x , b \rangle  = 0$ for all $b \in B$, then
  $\langle  x , a \rangle =0$ for all $a \in A$ since $A \subset B$.
  Thus we see that $B^\perp \subset A^\perp$. Then since $S \subset
  \overline{\textrm{span}}(S)$, we see that
  $\overline{\textrm{span}}(S)^\perp \subset S^\perp$ and
  $(S^\perp)^\perp \subset
  (\overline{\textrm{span}}(S)^\perp)^\perp$. Thus by
  \autoref{KplusKperpisX}, we see that $(S^\perp)^\perp \subset
  \overline{\textrm{span}}(S)$.
\end{proof}

\begin{corollary}
  \label{OrthonormalBaisEquiv}
  Let $E$ be an orthonormal subset of $\mathcal{H}$. Then the
  following are equivalent.
  \begin{enumerate}[label=(\arabic*)]
    \item $E$ is an orthonormal basis
    \item $(E^\perp)^\perp = \mathcal{H}$
    \item $\overline{\textrm{span}}(E) = \mathcal{H}$
    \item $\|\eta\|^2 = \sum_{e \in E} |\langle \eta ,  e \rangle
      |^2, \ \forall \eta \in \mathcal{H}$
  \end{enumerate}
\end{corollary}
\begin{proof}
  $1 \implies 2 \implies 3$ follows easily from
  \autoref{PerpPerpisSpan}. To see $3 \implies 4$, since $E$ is
  orthonormal and $\overline{\textrm{span}}(E) = \mathcal{H}$,
  we claim that it will be an orthonormal basis. Because if $y \perp E$, then $y
  \perp \textrm{span}(E)$. If $x_n \in \textrm{\textrm{span}}(E)$
  such that $x_n \to x$, then since $\langle y , x_n \rangle = 0$ for
  each $n \in \mathbb{N}$, we get
  \begin{align*}
    |\langle x , y \rangle|  = |\langle x - x_n , y \rangle + \langle
    x_n , y \rangle| = |\langle x - x_n , y \rangle| \le \|x-x_n\|\|y\|
  \end{align*}
  Since $x_n \to x$, we see that $\langle x , y \rangle = 0$. Thus we
  see that $E$ is an orthonormal basis. Thus we see that for all
  $\eta \in \mathcal{H}$,
  \begin{align*}
    \eta = \sum_{e \in E} \langle \eta ,  e \rangle e
  \end{align*}
  and Parseval's equality follows.

\end{proof}

\begin{theorem}
  Let $P \in B(\mathcal{H})$ be an idempotent. Then the following are
  equivalent.
  \begin{enumerate}[label=(\arabic*)]
    \item $P = P_K$ for some closed subspace $K \leqslant \mathcal{H}$
    \item $P = P^*$
    \item $\|P\| = 1$
  \end{enumerate}
\end{theorem}
\begin{proof}
  $1 \implies 2, 3$ is proved in
  \autoref{OrthogoalProjectionsAreSelfAdjointContractions}.
  To see $2 \implies 1$. Let $K = \textrm{Im}(P)$. Let $\rho \in
  K^\perp$. Then for all $\eta \in H$, \[
    \langle P(  \rho) ,  \eta \rangle  = \langle \rho ,  P^*(\eta)
    \rangle  = \langle \rho ,  P(\eta) \rangle = 0
  \]
  So $P|_{K^\perp} = 0$. Since $P^2 = P$, $P|_K = I$.  Hence $P = P_k$.

  To see $3 \implies 1$, assume $P \neq P_K$. Then $\exists \rho \in K^\perp$
  such that $P(\rho) \neq 0$. For each $n \in \mathbb{N}$, we have
  $\|\rho + n P(\rho)\|^2 = \|\rho\|^2 + n^2 \|P(\rho)\|^2$ by
  \autoref{PythagorasTheorem}.
  And \[
    \|P(\rho + n P(\rho))\|^2 = (n+1)^2 \|P(\rho)\|^2
  \]
  \marginnote{ \scriptsize \textit{\textcolor{red}{nice technique}}}
  Hence we have
  \begin{align*}
    \|P(\rho + n P(\rho))\|^2 - \|\rho + n P(\rho)\|^2 = 2n
    \|P(\rho)\|^2 + \|P(\rho)\|^2 - \|\rho\|^2
  \end{align*}
  So for large $n$, we have \[
    \|P(\rho + n P(\rho))\| > \|\rho + P(\rho)\|
  \]
  so $\|P\| > 1$.
\end{proof}

