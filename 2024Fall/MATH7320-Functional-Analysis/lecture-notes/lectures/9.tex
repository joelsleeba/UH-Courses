% TeX_root = ../main.tex

\begin{definition}
  Let $X$ and $Y$ be normed spaces and $T \in B(X, Y)$. The adjoint
  of $T$, denoted by $T^{*} \in B(Y^{*}, X^{*})$, is the map $T^{*}:
  f \to f\circ T$
\end{definition}

\begin{proposition}
  $\|T\| = \|T^*\|$
\end{proposition}
\begin{proof}
  $|T^*(f)| \le \|f \circ T\| \le \|T\|\|f\|$ implies $ \|T^{*}\| \le \|T\|$
  \begin{align*}
    \|T^*\| & = \sup \{ \|T^{*}(\phi)\| \ : \ \phi \in Y^*, \|\phi\| \le 1 \} \\
    &= \sup \{ |\phi(T(x))| \ : \ \phi \in Y^*, x \in X, \|\phi\| \le
    1, \|x\| \le 1 \} \\
    &= \|T\|
  \end{align*}
  Note that the last equality is a consequence of HBT since it
  guarantees the existence of $\phi_y \in Y^*$ with $\|\phi_y\|\le 1$
  and $\phi_y(y) = |y|$.
\end{proof}

\begin{lemma}
  For any $T \in B(X, Y)$, $T^*: Y^* \to X^*$ is weak * continuous
  (in both spaces)
\end{lemma}
\begin{proof}
  Let $\phi_i \to \phi$ weakly in $Y^*$. Then by definition for all
  $y \in Y$, $\phi_i(y) \to \phi(y)$. Then for $x \in X$,
  $T^*(\phi_i)(x) = \phi_i(T(x)) \to \phi(T(x)) = (T^*(\phi))(x)$
  which shows the continuity of $T^*$.
\end{proof}

\begin{lemma}
  For any normed space $X$, $i_X(X)$ is weak * dense in $X^{**}$.
\end{lemma}
\begin{proof}
  \textcolor{red}{verify}
\end{proof}

\begin{example}
  Is $i_{X^*}$ weak * - weak * continuous.
\end{example}

\section{Locally Convex Topological Vector Spaces}

\begin{lemma}
  Let $X$ be a normed space and $x_1 , x_2 , \ldots , x_n \in X$ and
  $\epsilon_1 , \epsilon_2 , \ldots , \epsilon_n \ge 0$. Then the set \[
    \bigcup_{x_1 , x_2 , \ldots , x_n, \epsilon_1 , \epsilon_2 ,
    \ldots , \epsilon_n}(\phi)
  \]
  is convex.
  \marginnote{ \scriptsize Review convexity arguments in Arveson's
  `Subalgebras of C* Algebras'}
  Moreover any topological vector spaces with the topology induced by
  a family of seminorms is locally convex.
\end{lemma}
\begin{proof}
  \textcolor{red}{verify}
\end{proof}

\begin{definition}
  Let $X$ be a vector space and $E \subset X$ be a convex subset. An
  element $a \in E$ is called an extreme point of $E$ if whenever $x,
  y \in E$, $0 \le t \le 1$ with $a  = tx + (1-t) y$, then $x = y = a$.
\end{definition}
\begin{example}
  Let $\bar{D} = \{ \alpha \in \mathbb{C} \ : \ |\alpha| \le 1 \}$.
  Then $\bar{D}$ is convex with $\textrm{Ext}(\bar{D}) = S^1$
\end{example}

\begin{theorem}[Krein-Milman Theorem]
  Let $X$ be a locally convex space, and let $K$ be a compact convex
  subset of $X$. Then the $ \textrm{Ext}(K) \neq \emptyset$ and
  indeed $K = \overline{\textrm{co}}(\textrm{Ext}(K))$
\end{theorem}

\begin{definition}
  Let $V$ be a vector space and $S\subset V$. The convex hull of $S$
  is defined as \[
    \textrm{co}(S) = \Big \{ \sum_{i = 1}^{n} t_i x_i \ \Big| \ 0 \le
    t \le 1, \sum t_i = 1, x_i \in S \Big\}
  \]
\end{definition}

