% TeX_root = ../main.tex

\begin{lemma}
  Let $K$ be a convex set. $x_0 \in \textrm{Ext}(K)$ if and only if
  $K \setminus \{ x_0 \}$ is convex.
  \label{lem:extreme_point_of_convex_set}
\end{lemma}
\begin{proof}
  If $K \setminus \{ x_0 \}$ is not convex, since $K$ is convex,
  $x_0$ can be written as the convex combination of elements in
  $K\setminus \{ x_0 \}$
  which makes $x_0 \notin \textrm{Ext}(K_0)$. Conversely if $x_0 \in
  \textrm{Ext}(K)$, then
  $x_0$ cannnot be written as the convex combination of elements of
  $K$. Hence $K \setminus \{ x_0 \}$
  is closed under convex combinations, making in convex.
\end{proof}

\begin{theorem}[Krein-Milman Theorem]
  Let $X$ be a locally convex space, and let $K$ be a compact convex
  subset of $X$. Then the $ \textrm{Ext}(K) \neq \emptyset$ and
  indeed $K = \overline{\textrm{co}}(\textrm{Ext}(K))$
\end{theorem}
\begin{proof}
  We first prove that the $\textrm{Ext}(K) \neq \emptyset$.
  Note that $K \setminus \{ x_0 \}$ is a
  relatively open subset of $K$ since $\{ x_0 \}$ is closed and $K
  \setminus \{ x_0 \} = \{ x_0 \}^c$ relative to $K$.

  Now let $ \mathcal{A}$ be the collection of all relatively open
  convex proper subsets of $K$. Note that $\emptyset \in
  \mathcal{A}$, therefore $\mathcal{A}$ is nonempty. Equip
  $\mathcal{A}$ with the partial order defined by the set inclusion.
  Let $\mathscr{C}$ be a chain in $\mathcal{A}$ and $F_{\mathscr{C}}
  = \cup_{C \in \mathscr{C}} C$. $F_{ \mathscr{C}}$ is relatively
  \marginnote{\scriptsize Zorn's Lemma to find a maximal proper open
  convex subset of $K$}
  open being the union of relatively open subsets of $K$. To see that
  $F_{\mathscr{C}}$ is convex, let $x,  y \in F_{\mathscr{C}}$. Then
  since $\mathscr{C}$ is a chain, there exist a $C \in \mathscr{C}$
  such that $x, y \in \mathscr{C}$. Then by the convexity of $C$, $tx
  + (1-t)y \in C
  \subset F_{\mathscr{C}}$ for all $t \in [0, 1]$.

  We claim that $F_{ \mathscr{C}}$ is a proper subset of $K$. For the
  sake of contradiction, assume $F_{ \mathscr{C}} = K$. Since $K$ is
  compact and $C$ is open in $K$ for all $C \in \mathscr{C}$, there
  are finitely many $C_1 \subset C_2 \subset \ldots \subset C_k \in
  \mathscr{C}$ which cover $K$ (i.e $K = \cup_{n = 1}^{k}C_n$). Hence
  we get $C_k = K$, which is absurd since $C_k$ must be a proper subset of $K$.
  Hence $F_{\mathscr{C}} \in \mathcal{A}$ and thus every chain must
  have an upper bound in $\mathcal{A}$. Now by Zorn's lemma,
  $\mathcal{A}$ has a maximal element $K_0$.

  Since $K$ is a connected space (path connected by a straight line,
  being convex), we know that the only
  clopen subsets are $\emptyset$ and $K$. Since we know that $K_0$ is
  \marginnote{\scriptsize Constructing and open convex subset containing $K_0$}
  open being in $\mathcal{A}$, we see that $K_0 \neq K$ and $K_0 \neq
  \emptyset$. Therefore $\overline{K_0} \neq K_0$. Let $x_0 \in
  \overline{K_0} \setminus K_0$, $y_o \in K_0$ and $0 < t <1$. Define
  $\varphi_{t, y_0}: K \to K$ such that $\varphi_{t, y_0}(z) = ty_0 +
  (1-t) z$. Then $\varphi_{t, y_0}$ is ($1-t$ Lipschitz) continuous
  relative to $K$ and thus $\varphi_{t, y_0}^{-1}(K_0)$ is open in
  $K$.
  By the convexity of $K_0$, we get $K_0 \subset \phi^{-1}_{t, y_0}(K_0)$.

  Also $\varphi_{t, y_0}^{-1}(K_0)$ is convex. Let $a, b \in
  \phi^{-1}_{t, y_0}(K_0)$. Then $ty_0 + (1-t)a, ty_0 + (1-t)b \in
  K_0$. By the convexity of $K_0$ we get $r(ty_0 + (1-t)a) +
  (1-r)(ty_0 + (1-t)b) = ty_0 + (1-t)(ra + (1-r)b) = \phi_{t, y_0}(ra
  + (1-r)b) \in K_0$ for all $r \in [0, 1]$. Thus $ra + (1-r)b \in
  \phi^{-1}_{t, y_0}(K_0)$ for all $r \in [0, 1]$. Hence we get
  $\phi^{-1}_{t, y_0}(K_0)$ is convex.

  We claim, $x_0 \in \varphi_{t, y_0}^{-1}(K_0)$, then the maximality
  of $K_0$ will force $\phi^{-1}_{t, y_0}(K_0) = K$.  Let $U$ be a
  convex neighborhood of $0 \in X$ containing $-x$ for all $x \in U$
  \marginnote{\scriptsize \textcolor{red}{I can't picturize the choice of $z$}}
  (just take $-U$ and intersect with $U$) such that $y_o + E
  \subset K_0$, where $E = K \cap U$. Let $w = \varphi_{t,
  y_0}(x_0)$. Since $x_0 \in
  \overline{K_0}$, for any $r>0$, there exists $x_r \in K_0$ such
  that $x_r \in (x_0 +
  rE) \cap K_0 \neq \emptyset$. In particular, let $r =
  \frac{t}{1-t}$. Then by linearity, we get $\big(x_0 +
  \frac{t}{1-t}E\big) \cap K_0 = ( \frac{t}{1-t}  )E\cap
  (K_0 - x_0) \neq \emptyset$. Choose $z$ in the above set. Then \[
    y_0 - \Big( \frac{1-t}{t} \Big)z \ \in \ y_0 + E \subset K_0
  \]
  and $x_0 + z \in K_0$. Since $K_0$ is convex, \[
    t\Big(y_0 - \frac{(1-t)}{t}z\Big) + (1-t)(x_0 + z)  = \phi_{t,
    y_0}(x_0) \in K_0
  \]
  Thus $\phi^{-1}_{t, y_0}(K_0) = K$.

  Now we claim that $K = K_0 \cup \{ x_0 \}$. For the sake of
  contradiction assume $\exists p \in K$ such that $p \notin K_0 \cup
  \{ x_0 \}$. Since the space is Hausdorff and locally convex, $x_0$ has an
  open convex neighborhood $E$ in $X$ such that $p \not\in E$. Let
  $E^\prime = E \cap K$, $a \in K_0, b \in E^\prime$ and $0 < r < 1$.
  Then since $\phi_{t, y_0}(K) = K_0$ for all $t \in [0, 1], y_0 \in K_0$, we
  get $\phi_{r, a}(b) = ra + (1-r)b \in K_0$. So $K_0 \cup
  E^\prime$ is convex (Sine we know that $K_0, E^\prime$ are convex,
    we only need to worry about $rx + (1-r)y$ for $x \in K_0, y \in
  E^\prime$. But $\phi_{r, x}$ takes care of that). $K_0 \cup
  E^\prime$ is also open in $K$. Hence by maximality, we get $K_0
  \cup E' = K$. But this is a contradiction since $ p \not\in K_0
  \cup E^\prime$. Thus by \autoref{lem:extreme_point_of_convex_set}, we see
  that $x_0 \in \textrm{Ext}(K)$.

  Next we prove $K = \overline{co}(\textrm{Ext}(K))$. Let $P =
  \overline{co}(\textrm{Ext}(K))$ and for the sake of contradiction
  assume $P \neq K$. Let $ x_0 \in K \setminus P$.
  %
  % Let $E$ be an open
  % convex neighborhood of $0 \in X$ such that $(x_0 + E^\prime) \cap P
  % = \emptyset$ for $E^\prime = E \cap K$. Existence of such an $E$
  % is guaranteed
  % since $P$ is compact and the space $X$ is Hausdorff.
  % \marginnote{\scriptsize \textcolor{red}{Don't know how to proceed after}}
  %
  Now by the geometric Hahn-Banach separation theorem, we get that
  there is a continuous linear functional
  $\phi: X \to \mathbb{R}$ and a number $\alpha, \epsilon \in
  \mathbb{R}$ such that \[
    \Re\phi(x_0) \le  \alpha < \alpha + \epsilon \le \Re\phi(p),
    \quad  \forall p \in P
  \]

  % Define $\phi: X \to
  % \mathbb{R}$ such that \[
  %   \phi(x) = \inf \{ 0 \le t \big | x \in tE \}
  % \]
  % Observe that $E = \{ x \in X \ : \phi(x) < 1 \}$. For every $r \ge
  % 0$ and $x \in X$, $\phi(rx) = r \phi(x)$, and for all $x, y \in X$,
  % $ \phi(x+y) \le \phi(x) + \phi(y)$. Define $f:  \mathbb{R}\{ x_0 \}
  % \to \mathbb{R}$, $f(rx_0) = r \phi(x_0)$, for all $r \in
  % \mathbb{R}$. For every $r \ge 0$, we have $f(rx_0) = r \phi(x_0) =
  % \phi(rx_0)$. For $r < 0$, we have $f(rx_0) = r \phi(x_0) =
  % -\phi(-rx_0) \le 0 \le \phi(rx_0)$. So $f \le \phi$ on $\mathbb{R}
  % x_0$. Then by Hahn-Banach separation theorem, there is an
  % extension $ \tilde{f}: X \to
  % \mathbb{R}$ such that $\tilde{f}(x) \le \phi(x) $ for all $x \in X$.

\end{proof}

