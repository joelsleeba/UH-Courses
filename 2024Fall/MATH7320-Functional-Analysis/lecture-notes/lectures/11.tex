% TeX_root = ../main.tex

\begin{lemma}
  Let $K_1, K_2$ be compact convex subsets of a locally compact TVS $X$. Then \[
    \overline{\textrm{co}}(K_1 \cup K_2) = (\textrm{co})( K_1 \cup K_2)
  \]
  \label{lem:convex_hull_of_the union_of_compact_convex_sets_are_compact}
\end{lemma}
\begin{proof}
  \textcolor{red}{verify}.
  We'll show that $\textrm{co}(K_1 \cup K_2)$ is compact and hence
  closed. Let $x = \alpha_1a_1 , \alpha_2a_2 , \ldots , \alpha_na_n +
  \beta_1b_1 , \beta_2b_2 , \ldots , \beta_mb_n \in \textrm{ co}(K_1
  \cup K_2)$, where $\sum_{i = 1}^{n} \alpha_i + \sum_{i = 1}^{m}
  \beta_i = 1$. Then \[
    x = \big(\sum_{i = 1}^{n} \alpha_i\big) \underbrace{\Bigg(
        \sum_{i = 1}^{n} \Big( \frac{\alpha_i}{\sum_{i = 1}^{n}
    \alpha_i}\Big) a_i\Bigg)}_{\in \ K_1}+ \big(\sum_{i = 1}^{m}
    \beta_i \big) \underbrace{\Bigg( \sum_{i = 1}^{m} \Big(
    \frac{\beta_i}{\sum_{i = 1}^{m} \beta_i}\Big) b_i\Bigg)}_{\in \ K_2}
  \]
  Hence every element $x \in \textrm{co}(K_1 \cup K_2)$, can be
  written as $x = ta + (1-t)b$ where $ a \in K_1, b \in K_2$.

  Now let $x_\lambda = t_\lambda a_\lambda + (1-t_\lambda)b_\lambda$
  be a net in $\textrm{co}(K_1 \cup K_2)$, for $  \lambda \in
  \Lambda$, $a_\lambda \in K_1, b_\lambda \in K_2$. Since
  $(a_\lambda)$ is a net in the compact set $K_1$, there is a subnet
  $a_\sigma$ for $\sigma \in \Sigma \subseteq \Lambda$, such that
  $a_\sigma \to a \in K_1$. By similar reasoning $b_\sigma$ has a
  convergent subnet $b_\pi$ for $\pi \in \Pi \subseteq \Sigma$, such
  that $b_\pi \to b \in K_2$. Again $t_\pi$ is a net in the compact
  space $[0, 1]$, hence is has a convergent subnet $t_\omega$ for $
  \omega \in \Omega \subseteq \Pi$ such that $t_\omega \to t$ in $[0, 1]$.

  Now consider the subnet $x_\omega = t_\omega a_\omega +
  (1-t_\omega)\beta_\omega$ of $x_\lambda$. Since $ \Omega \subseteq
  \Pi \subseteq \Sigma$, $t_\omega \to t, \beta_\omega \to b$ and
  $a_\omega \to a$. Therefore by the continuity of the scalar product
  and addition in the TVS, we get $x_\omega \to t \alpha + (1-t)
  \beta \in \textrm{co}(K_1 \cup K_2)$. Hence we get $\textrm{co}(K_1
  \cup K_2)$ is compact.
\end{proof}

\begin{theorem}[Inverse Krein-Milman]
  Let $K$ be a compact convex subset of a locally convex topological
  vector space $X$. Let $A \subset K$ be a closed subset of $K$. If
  $K = \overline{\textrm{co}}(A)$, then $\textrm{Ext}(K) \subset A$.
\end{theorem}

Note that $\textrm{Prob}[0, 1]$, the collection of probability
measures identified as a subspace of a $C([0, 1])^*$ is convex, weak
* compact with $\textrm{Ext}(K) = \{ \delta_x  \ : \  x \in [0, 1] \}$
\begin{proof}
  FSTOC, assume $\exists x_0 \in \textrm{Ext}(K)$, $x_0 \not \in A$.
  Since $A$ is compact, $\exists y_1 , y_2 , \ldots , y_n \in A$ and
  an open convex neighborhood $B$ of $0$ such that $$A \subset
  \cup_{i = 1}^{n}(y_i + B)$$
  and $x_0 \not\in y_i + \overline{B}$ for all $i = 1, 2, \ldots, n$.
  Let $B_i = (y_i + \overline{B}) \cap K$. Then $B_i$ is a compact
  convex subset of $K$ for each $i$. Hence by the previous lemma, we get \[
    \textrm{co}(B_1 \cup B_2 \cup \ldots \cup B_n) =
    \overline{\textrm{co}}(B_1 \cup B_2 \cup \ldots \cup B_n) \supset
    \overline{\textrm{co}}(A) = K
  \]
  Thus $\exists b_i \in B_i$ and $0 \le t_i \le 1$, $\sum_{i = 1}^{n}
  t_i = 1$ such that
  \marginnote{\scriptsize \textcolor{ red}{I struggle at finding the
  contradiction}}
  \[
    x_0 = t_n b_1 + t_n b_2 + \ldots + t_n b_n
  \]
  Since $x_0 \in \textrm{Ext}(K)$, this forces $x_0 = b_j$ for some
  $1 \le j \le n$.
  This contradicts the assumption that $x_0 \not\in y_i +
  \overline{B}$. Hence $x_0 \in A$.
\end{proof}

Note that in the following attempt to prove the theorem, it is not
obvious why $U$ is convex. If we try to argue using arguments to the
proof of separating a compact set and a point using open sets in a
Hausdorff space, we will eventually need to show that the finite
open subcover of $A$ sits inside a closed convex set that does not
contain $x_0$, which again is not obvious.
\begin{proof}
  FTSOC, assume that $\exists x_0 \in \textrm{Ext}(K)\setminus A$.
  \marginnote{ \scriptsize \textcolor{red}{Why is $U$ convex?}}
  Since the TVS is Hausdorff, there exist convex open sets $U,
  V$ such that $A \subset U, x_0 \in V$, and $U \cap V = \emptyset$.
  Moreover we claim that $\overline{U} \cap V = \emptyset$. Otherwise if $x \in
  \overline{V} \cap U$, then for any net $(x_\lambda) \in V$ that
  converge to $x$, by the definition of convergence $x_{\lambda_n}
  \in U$ for all $\lambda_n$ greater that some $\lambda_N$. This
  would contradict the assumption that $U\cap V = \emptyset$. Hence
  we see that $A \subset \overline{V}$, and therefore
  $\overline{\textrm{co}}(A) \subset \overline{V}$. But this would
  again contradict the fact that $\overline{\textrm{co}}(A) = K$
  since $x_0 \notin \overline{V}$.
\end{proof}

