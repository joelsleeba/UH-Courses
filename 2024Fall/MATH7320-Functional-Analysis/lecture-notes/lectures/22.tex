% TeX_root = ../main.tex

\marginnote{\scriptsize 21/11/2024 }

\begin{definition}
  A Banach algebra $\mathcal{A}$ is a Banach space with a
  multiplicatit a ring with addition satisfying $\|ab\| \le
  \|a\|\|b\|$ for all $ a, b \in \mathcal{A}$. We say $\mathcal{A}$
  is unital if the ring above is unital with multiplicative identity
  $1_{\mathcal{A}}$. Units (invertible elements)
  may also exist similarly.
\end{definition}

\begin{definition}
  Given a Banach algebra $\mathcal{A}$, and $a \in \mathcal{A}$, the
  spectrum of $a$, denoted by
  \begin{align*}
    \sigma(a) = \{ \lambda \in \mathbb{C}  \ : \  \lambda
    1_{\mathcal{A}} - a \textrm{ is not invertible in } \mathcal{A} \}
  \end{align*}
\end{definition}

\begin{example}
  Let $(\alpha_n) \in \textbf{c}_0$ and $T: \ell^{2} \to \ell^{2}$
  such that $T((x_n)) = (\alpha_n x_n)$. We claim that
  \begin{align*}
    \sigma(T) = \{ \alpha_n  \ : \  n \in \mathbb{N} \} \cup \{ 0 \}
  \end{align*}
\end{example}
\begin{proof}
  Since $T$ is a compact operator (see Homework-5), we must have $0
  \in \sigma(T)$. Otherwise if $T$ is invertible, we'll get $I  = T
  \circ T^{-1}$ be also compact, which is a contradiction since
  $\ell^{2}$ is infinite dimensional. Moreover since $(T - \alpha_n I
  )(e_n) = 0$, we have $ \alpha_n \in \sigma(T)$.

  Now assume that $\beta \notin \{ \alpha_n  \ : \  n \in \mathbb{N}
  \} \cup \{ 0 \}$. Then let $S \in B(\ell^{2})$ defined by
  \begin{align*}
    S(e_n) = \frac{1}{\alpha-\beta} e_n
  \end{align*}
  Then \textcolor{red}{show that indeed $S \in B(\ell^{2})$} and $(T
  - \beta I)S = S(T - \beta I) = I$
\end{proof}

\begin{example}
  Let $T \in B(L^{2}([0, 1]))$ such that $T(f)(x) = xf(x)$ for all $
  f \in L^{2}([0, 1])$. Then $  \sigma(T) = [0, 1]$.
\end{example}
\begin{proof}
  First let us see that $0 \in \sigma(T)$. Suppose $  S = T^{-1}$
  exists. Then $\forall n \in \mathbb{N}$ if $ f = S \chi_{[0, 1]}$,
  then $Tf = \chi_{[0, 1]}$. So $x f(x) =
  \chi_{[0, 1]}$. But this is absurd, since $\frac{1}{x} \chi_{[0,
  1]} \notin L^2([0, 1])$.
\end{proof}

\begin{lemma}
  Let $S \in B(\mathcal{H})$ such that $\|S - T\| \le 1$. Then $S$ is
  invertible.
\end{lemma}
\begin{proof}
  Since $\|S  -I\| \le 1$,
  \begin{align*}
    \sum_{n \in \mathbb{N}} \|(S-I)^n\| \le \sum_{n \in \mathbb{N}} \|S - I\|^n
  \end{align*}
  Moreover since $B(\mathcal{H})$ is a Banach space, absolutely
  convergent sequences converge and this gives that
  $\sum_{n \in \mathbb{N}} (I - S)^n$ converges. Thus
  \begin{align*}
    R = \sum_{n=1}^\infty (I - S)^n
  \end{align*}
  exists. Now for each $N \in \mathbb{N}$, we have
  \begin{align*}
    S \Big(\sum_{n = 0}^{N} (I - S)^n\Big) = (I - (I -
    S))\Big(\sum_{n = 1}^{N} (I - S)^n\Big) &= \sum_{n = 0}^{n} (I -
    S)^n - \sum_{n = 0}^{N} (I-S)^{n+1}\\
    &= I - (I-S)^{N+1}
  \end{align*}
  which converges to $0$ as $n \to \infty$.
\end{proof}

\begin{corollary}
  The set of invertible operators is open in $B(\mathcal{H})$.
\end{corollary}
\begin{proof}
  Let $S$ be invertible. Then
  \begin{align*}
    \|T - S\| = \|S(S^{-1}T - I)\| \le \| S\| \|S^{-1}T - I\|
  \end{align*}
\end{proof}

\begin{theorem}
  For any $T \in B(\mathcal{H})$, $\sigma(T)$ is a non-empty, compact
  subspace of $\mathbb{C}$.
\end{theorem}
\begin{proof}
  Observe that the function $f: \lambda \to T - \lambda I$ is
  continuous. Then $f^{-1}(G(\mathcal{A}))$ is open, where
  $G(\mathcal{A})$ is the collection of all invertible elements of
  $B(\mathcal{H})$. But $\sigma(T)^{c} = f^{-1}(G(\mathcal{A}))$.
  So we see $\sigma(T)$ is closed. Moreover let $\lambda \in
  \mathbb{C}$ with $\|T\| < |\lambda|$. Then,
  \begin{align*}
    -T + \lambda I = \lambda(\frac{-T}{\lambda} + I)
  \end{align*}
  Then $\|S - I\| = \| \frac{T}{\lambda}\| = \frac{\|T\|}{|\lambda|} < 1$
  Hence show that $\sigma(T)$ is bounded.
\end{proof}
