% TeX_root = ../main.tex

\begin{lemma}
  If $T \in B(\mathcal{H})$ is compact, then $T(\mathcal{H})$ is separable.
\end{lemma}
\begin{proof}
  For the sake of contradiction, assume that $T(\mathcal{H})$ is not
  separable. Then $T(\mathcal{H})$ must have an uncountable orthonormal set $E$.
  This is because if $E$ is a countable basis, then $\mathbb{Q}E + i
  \mathbb{Q}E$ will be a countable dense subset of
  $T(\mathcal{H})$ making it separable. \textcolor{red}{verify}
\end{proof}

\begin{corollary}
  The set $\mathcal{K}(\mathcal{H})$ of all compact operators on $
  \mathcal{H}$ is a closed
  two-sided ideal in $B(\mathcal{H})$.
\end{corollary}
\begin{proof}
  Use the fact that compact operators are
  the closure of finite rank operators.
\end{proof}

\begin{corollary}
  $T \in \mathcal{K}(\mathcal{H})$ implies $T^* \in \mathcal{K}(\mathcal{H})$.
\end{corollary}
\begin{proof}
  Use the fact that $T$ is finite rank
  implies $T^*$ is finite rank.
\end{proof}

\begin{example}
  Let \[
    T: L^2([0, 1], m) \to L^2([0, 1], m)
  \]
  be defined as $T(f)(x) = xf(x)$. Prove that $T = T^*$ and that $T$
  has no eigenvectors.

  Only for joseph: If $\xi = \eta$ almost everywhere, then show that  $f \xi = f
  \eta$ almost everywhere if $ f \in C([0, 1])$
\end{example}
\begin{proof}
  See homework-5
\end{proof}

\begin{example}
  Let $\alpha_n$ be a bounded sequence in $\mathbb{C}$. Consider $T :
  \ell^2 \to \ell^2$, such that \[
    T(\delta_n) = \alpha_n \delta_n
  \]
  for all $n \in \mathbb{N}$. Then $T$ is bounded with $\|T\| =
  \|(\alpha_n)\|_\infty$.
\end{example}

\begin{example}
  \label{CompactiffEigvalsconverge0}
  Prove that the $T$ above is compact if and only if $(\alpha_n) \in c_{\bf 0}$
  Prove that $T^*(\delta_n) = \overline{\alpha_n}\delta_n$
\end{example}
\begin{proof}
  See homework-5
\end{proof}

\section{Spectral Theorem for Compact Normal Operators}
We'll be working to prove the following theorem.
\begin{theorem}
  Let $\mathcal{H}$ be separable Hilbert space and $T\in
  \mathcal{K}(H)$ be normal. i.e
  $TT^* = T^*T$. Then there exist an orthonormal basis $\{ e_n  \ :
  \  n \in \mathbb{N} \}$ of $\mathcal{H}$, and a sequence $(\alpha_n) \in
  c_{\bf 0}$, such that $T(e_n) = \alpha_n e_n$.
\end{theorem}

\begin{lemma}
  Let $T \in \mathcal{K}(\mathcal{H})$ be self-adjoint. Then there
  exist a non-zero $\lambda \in \mathbb{C}$ such that $
  \textrm{Ker}(T - \lambda I) \neq \{ 0 \}$.
\end{lemma}
\begin{proof}
  See the first part of the proof of
  \autoref{SpectralTheoremforCompactSAOperators}.
\end{proof}

\begin{definition}
  Let $T \in B(\mathcal{H})$. A subspace $W \leqslant H$ is said to be
  invariant under $T$ if $T(W) \subset W$. We say $W$ reduces $T$, if
  $T(W) \subset W$ and $T(W^\perp) \subset W^\perp$
\end{definition}

\begin{example}
  Let $T \in B(\mathcal{H})$.
  Prove that $W \subset \mathcal{H}$ is invariant under $T$ if and
  only if $P_WT = TP_W$
  Prove that $W$ reduces $T$ if and only if $P_WT = TP_W$ and
  $P_{W^\perp}T = T P_{W^\perp}$.
\end{example}
\begin{proof}
  See homework-5
\end{proof}

\begin{proposition}
  \label{InvariantSubspacesofSelfAdjointOperatorsReduce} Let $T \in
  B(\mathcal{H})$ be self adjoint with $T(\mathcal{M}) \subset
  \mathcal{M}$, where $\mathcal{M}$ is a closed subspace of
  $\mathcal{H}$. Then $\mathcal{M}$ reduces $T$.
\end{proposition}
\begin{proof}
  Let $m \in \mathcal{M}$ and $x \in \mathcal{M}^\perp$. Then
  \begin{align*}
    \langle T(x) ,  m \rangle = \langle x , T(m) \rangle = 0
  \end{align*}
  shows that $T(\mathcal{M}^\perp) \subset \mathcal{M}^\perp$.
\end{proof}

