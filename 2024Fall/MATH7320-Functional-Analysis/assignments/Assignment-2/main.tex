% initial settings
\documentclass[12pt]{exam}
\usepackage{geometry}
\usepackage{graphicx}
\usepackage{enumitem}
\usepackage[usenames,dvipsnames]{xcolor}
\usepackage[backend=biber, style=alphabetic]{biblatex}
\usepackage{url,hyperref}

\usepackage{amsmath} % math symbols, matrices, cases, trig functions, var-greek symbols.
\usepackage{amsfonts} % mathbb, mathfrak, large sum and product symbols.
\usepackage{amssymb} % extended list of math symbols from AMS. https://ctan.math.washington.edu/tex-archive/fonts/amsfonts/doc/amssymb.pdf
\usepackage{amsthm} % theorem styling.
\usepackage{mathrsfs} % mathscr fonts.
\usepackage{yhmath} % widehat.
\usepackage{empheq} % emphasize equations, extending 'amsmath' and 'mathtools'.
\usepackage{bm} % simplified bold math. Do \bm{math-equations-here}

% geometry of paper
\geometry{
  a4paper, % 'a4paper', 'c5paper', 'letterpaper', 'legalpaper'
  asymmetric, % don't swap margins in left and right pages. as opposed to 'twoside'
  centering, % to center the content between margins
  bindingoffset=0cm,
}

% hyprlink settings
\hypersetup{
  colorlinks = true,
  linkcolor = {red!60!black},
  anchorcolor = red,
  citecolor = {green!50!black},
  urlcolor = magenta,
  }

% theorem styles
\theoremstyle{plain} % default; italic text, extra space above and below
\newtheorem{theorem}{Theorem}[section]
\newtheorem{proposition}{Proposition}[section]
\newtheorem{lemma}{Lemma}[section]
\newtheorem{corollary}{Corollary}[theorem]

\theoremstyle{definition} % upright text, extra space above and below
\newtheorem{definition}{Definition}[section]
\newtheorem{example}{Example}[section]

\theoremstyle{remark} % upright text, no extra space above or below
\newtheorem{remark}{Remark}[section]
\newtheorem*{note}{Note} %'Notes' in italics and without counter 

% renewcommands for counters
\newcommand{\propositionautorefname}{Proposition}
\newcommand{\definitionautorefname}{Definition}
\newcommand{\lemmaautorefname}{Lemma}
\newcommand{\remarkautorefname}{Remark}
\newcommand{\exampleautorefname}{Example}

\addbibresource{articles.bib}


\begin{document}

\title{MATH 7320 Functional Analysis\\ Homework 2}

% author list
\author{
Joel Sleeba \\
}

\maketitle
\printanswers
\unframedsolutions

\begin{questions}
  
  \question
  \begin{solution}
    One side is easy using the conitnuity of the functional, for the other way, use the functional $d(x, \mathcal{M})$
  \end{solution}
  

  \question
  \begin{solution}
    \begin{parts}
      \part If $X$ is reflexive, the map $X \to X^{**}:= x \to \textrm{ev}_x$ is an isometric isomorphism. Moreover we know that $X^{**} = B(X^{*}, \mathbb{C})$ is complete since $\mathbb{C}$ is complete. Therefore by the isometric isomorphism, we get that $X$ is complete.
      \part Let $i_X: X \to X^{**}: x \to \textrm{ev}_x$ be the canonical injection map. Then first we show that $i_{X^{**}} = (i_X)^{**}$, that is the canonical injection of the double dual is the double dual of the canonical injection. 

      First we notice that since $i_X: X \to X^{**}$, the dual of it $i_X^*: X^{***} \to X^{*}$ and $i_X^{**}: X^{**} \to X^{****}$ as the usual dual of linear transformations. Moreover $i_{ X^{**}}: X^{**} \to X^{****}$ shows that the domain and codomain of the maps are same, therefore considering the equality of the maps makes sense. (Notice that actually $i_{ X^{**}}$ is a map from $X^{**} \to (X^{**})^{**}$, but by definition it follows that $(X^{**})^{**} := (X^{***})^* = X^{****}$).

      Let $f: X^{***} \to C$

      \part Let $M \subset X$ be a closed subset of a reflexive space $X$. Consider the subset $F \subset X^{*}$ such that $F = \{ f \in X^{*} \ : \ f|_M = \textbf{0} \}$. Show that $X^{*}/F \cong^{ \textrm{iso}} M^{*}$ (For this it is enough to show that the restriction map $f \to f|_M$ has its kernel $F$. And for isometry, play with the quotient norm).
    \end{parts}
  \end{solution}


\end{questions}
\printbibliography[heading=bibintoc]
\end{document}
