\documentclass[12pt]{report}

\usepackage{geometry} % automatic papersizes, margins.
\usepackage{makeidx} % 'makeidx' make and show index
\usepackage{enumitem} % itemize, enumerate, description.
\usepackage{hyperref} % hyperlinks, cross-references.
\usepackage{xcolor} % foreground and background color management.
% Better color mixing compared to 'color'
\usepackage{graphicx} % provide options for \includegraphics. Builds
% on 'graphic'
\usepackage{caption} % better control over captions of figures and equations.
\usepackage{appendix} % extra control over appendix
\usepackage[backend=biber, style=alphabetic]{biblatex} % better than
% bibtex, people say.
\usepackage{tocbibind} % add ToC/Bibliography/Index to ToC
\usepackage{marginnote}
\usepackage{amsmath} % math symbols, matrices, cases, trig functions,
% var-greek symbols.
\usepackage{amsfonts} % mathbb, mathfrak, large sum and product symbols.
\usepackage{amssymb} % extended list of math symbols from AMS.
% https://ctan.math.washington.edu/tex-archive/fonts/amsfonts/doc/amssymb.pdf
\usepackage{amsthm} % theorem styling.
\usepackage{mathrsfs} % mathscr fonts.
\usepackage{yhmath} % widehat.
\usepackage{empheq} % emphasize equations, extending 'amsmath' and 'mathtools'.
\usepackage{bm} % simplified bold math. Do \bm{math-equations-here}
\usepackage{tikz} % for tikz diagrams
% \usepackage{tikz-cd} % commutative diagrams.
\usepackage{verbatim} % for concealing solutions

\geometry{
  a4paper, % 'a4paper', 'c5paper', 'letterpaper', 'legalpaper'
  asymmetric, % don't swap margins in left and right pages. as
  % opposed to 'twoside'
  centering, % to center the content between margins
  bindingoffset=0cm,
}

\hypersetup{
  colorlinks,
  linkcolor={blue!50!black},
  citecolor={blue!50!black},
  urlcolor={blue!80!black}
}

\theoremstyle{plain} % default; italic text, extra space above and below
\newtheorem{theorem}{Theorem}[section]
\newtheorem{proposition}{Proposition}[section]
\newtheorem{lemma}{Lemma}[section]
\newtheorem{corollary}{Corollary}[theorem]
\newtheorem{problem}{Problem}[section]

\theoremstyle{definition} % upright text, extra space above and below
\newtheorem{definition}{Definition}[section]
\newtheorem{example}{Example}[section]
\newtheorem{exercise}{Exercise}[section]

\theoremstyle{remark} % upright text, no extra space above or below
\newtheorem{remark}{Remark}[section]
\newtheorem*{note}{Note} %'Notes' in italics and without counter

\newcommand{\propositionautorefname}{Proposition}
\newcommand{\lemmaautorefname}{Lemma}
\newcommand{\corollaryautorefname}{Corollary}
\newcommand{\problemautorefname}{Problem}
\newcommand{\definitionautorefname}{Definition}
\newcommand{\exampleautorefname}{Example}
\newcommand{\remarkautorefname}{Remark}
\newcommand{\noteautorefname}{Note}

% For exercise and solutions
% \newif\ifshowsolutions
% \showsolutionstrue % Change to \showsolutionsfalse to hide solutions

% Custom environments for problems and solutions
% \newcounter{exercise}[chapter]
% \newenvironment{problem}[1][]
%   {\par\noindent\textbf{#1.}}
%   {\par}
%
% \newenvironment{solution}
%   {\ifshowsolutions \expandafter\solutioncontent \else
% \expandafter\comment \fi}
%   {\ifshowsolutions \hfill \qedsymbol \else \expandafter\endcomment
% \fi \vspace{1em}}
%
% \newenvironment{solutioncontent}
%   {\par\noindent\textit{Solution.}}
%   {\par}

\addbibresource{articles.bib}

\begin{document}
\title{MATH6320 - Functions of a Real Variable}

% \showsolutionstrue
%\showsolutionsfalse %If need to hide solutions

\author{
  Joel Sleeba \\
  % Cochin University of Science and Technology\\
  joelsleeba1@gmail.com \\
}

\maketitle

\pagenumbering{roman} \setcounter{page}{2}
\tableofcontents
\pagenumbering{arabic} \setcounter{page}{1}

% TeX_root = ../main.tex

\chapter{Banach Spaces}

\textbf{Textbook :} A Course in Functional Analysis, John Conway

Functional analysis is the study of Topological Vector Spaces.

\begin{definition}
  Let $X$ be a vector space (over $\mathbb{R}$ or $\mathbb{C}$). A
  seminorm on $X$ is a map $\|\cdot \|: X \to [0, \infty)$ such that
  \begin{itemize}
    \item  $\|\alpha x\| = |\alpha|\| x \|, \forall \alpha \in
      \mathbb{F}, \forall x \in X$
    \item $\|x+y\| \le \|x\| + \|y\|$
  \end{itemize}
  In addition if $\forall x\neq 0, \|x\| \neq 0$, we say $\|\cdot\|$
  is a norm on $X$
\end{definition}

Norm induces a metric $ d(x, y) = \|x-y\|$

\begin{note}
  Let $X$ be a normed space. Then the maps
  \begin{itemize}
    \item $+: X \times X \to X : (x, y) \to x+y$
    \item $\cdot: \mathbb{F} \times X \to X: (\alpha, x) \to \alpha x$
  \end{itemize}
  are continuous.
\end{note}

Hence every normed space is a topological vector space.

\begin{example}
  $\mathbb{F}^n$ with $\ell_p$-norm defined as \[
    \|(\alpha_1, \alpha_2, \ldots, \alpha_n)\|_p = \Big(\sum_{i =
    1}^{n} |a_i|^p\Big)^{ \frac{1}{p} }
  \]
\end{example}

\begin{example}
  $\mathbb{F}^n$ with $\ell_\infty$-norm defined as \[
    \|(\alpha_1, \alpha_2, \ldots, \alpha_n)\|_\infty = \max \{ |a_i| \}
  \]
\end{example}

\begin{example}
  Consider $C_{00} = \{ (a_n)_{n \in \mathbb{N}}  \ : \   a_n \in
    \mathbb{F}, \forall n \in \mathbb{N}, a_n = 0  \textrm{ except for
  finitely many }  n \in \mathbb{N}\}$ which can be identified by
  collection of functions $  f: \mathbb{N} \to \mathbb{F}$ with finite support.

  Then $$\|(a_n)\|_p = \Big(\sum_{n = 1}^{\infty} |a_n|^p\Big)^{ \frac{1}{p} }$$
  is a norm on $C_{00}$
\end{example}

\begin{proposition}
  \label{EquivalentDefnsofContinuity}
  Let $X, Y$ be normed space, and let $T: X \to Y$ be linear. Then
  the following are equivalent.
  \begin{itemize}
    \item $T$ is continuous
    \item $T$ is continuous on 0
    \item $T$ is continuous on any point $x \in X$
    \item $\exists M >0$ such that $\|T(x)\|_Y \le M \|x\|_X$ for all $x \in X$
  \end{itemize}
\end{proposition}
\begin{proof}
  ($1 \implies 2$) It is clear that if $T$ is continuous, then it is
  continuous at $0$ from the definition of continuity.

  $(2 \implies 3)$ Let $x \in X$ and $\{ x_n \}_{n \in \mathbb{N}}$
  be any sequence in $X$ that converge to $ x$. Then the sequence $\{
  y_n = x_n - x \}$ converge to zero by the algebra of limits. By the
  continuity of $T$ at zero, $\{ T(y_n) = T(x_n) - T(x) \}$ converge
  to $0$. Therefore $T(x_n) \to T(x)$. And this shows $T$ is
  sequentially continuous at $x \in X$. Since the space is a metric
  space, sequential continuity is equivalent to continuity.

  $(4 \implies 2)$ Let $x \in X$. Then $\|T(0) - T(x)\| = \|T(x)\|
  \le M \|x\| = M \|0 - x\|$. Hence $T$ is continuous at $0$.

  $(3 \implies 1)$

  $(2 \implies 4)$

\end{proof}

\begin{example}
  Let $T: \mathbb{F}^n \to \mathbb{F}^n$ be defined as $T(\alpha_1,
  \alpha_2, \ldots, \alpha_n) = (\alpha_1, 0, \ldots, 0)$. Is $T$
  convergent for any norm $ \|\cdot\|_1, \|\cdot\|_2$ in the domain and range?

  % No. Let $\|\cdot\|_1 = \|\cdot\|_$
\end{example}
\begin{proof}
  \textcolor{red}{verify}
\end{proof}

\begin{example}
  Consider identity function $I: C_{00} \to C_{00}$. Let the norm in
  domain be $ \|\cdot\|_\infty$ and that in range be $\|\cdot\|_1$.
  Is the function continuous? What if the norms in domain and range
  are switched?
\end{example}
\begin{proof}
  \textcolor{red}{verify}
\end{proof}

\begin{note}
  Let $X$ be a space with two norms $\|\cdot\|_1, \|\cdot\|_2$. When
  is the two norms topologically equivalent?

  When $\exists M, M^\prime$ such that $\|x\|_1 \le M \|x\|_2$ and
  $\|x\|_2 \le M^\prime \|x\|_1$
  Equivalently, when the identity map between the two spaces with
  their respective norms are bi-continuous. (See 4th equivalent
  statement of previous proposition)
\end{note}
\begin{theorem}
  Let $X$ and $Y$ be normed spaces, and $T: X \to Y$ be linear.
  Assume $X$ is finite dimensional. Then $T$ is continuous.
\end{theorem}
\begin{proof}
  Since $T(X) \le Y$ is finite dimensional, we may assume without
  loss of generality that $Y$ is also finite dimensional and $T$ is
  onto. Let $\{ x_1, x_2, \ldots x_n \}$ be a basis for $X$. Define
  another norm on $X$ as follows. For every $x  = \sum_{i = 1}^{n}
  \alpha_i x_i \in X$, \[
    \|x\|^\prime = \sum_{i = 1}^{n} |\alpha_i| (\|T(x_i)\| + \|x_i\|)
  \]
  \textcolor{red}{verify that this is a norm}. Then for every $x \in
  X$, we have \[
    \|T(x)\| \le \sum_{i = 1}^{n} |\alpha_i|\|T(x_i)\| \le \|x\|^\prime
  \]

  Hence $T$ is bound with respect to the norm $\|\cdot\|^\prime$ on
  $X$, since all norms are equivalent on $X$. Therefore $T$ is
  continuous w.r.t to the original norm on $X$.
\end{proof}

\begin{corollary}
  Let $X$ be a finite dimensional vector space. Then any two norms in
  $X$ are equivalent.
\end{corollary}
\begin{proof}
  Let $\{ e_1, e_2, \ldots e_n \}$ be a basis for $X$. For each $x =
  \sum_{i = 1}^{n} \alpha_i e_i \in X$, define \[
    \|x\|_\infty = \max \{ |\alpha_i| \}
  \]
  Then $\|\cdot\|_\infty$ is a norm and we'll show every norm on $X$
  is equivalent to this norm. Let $\|\cdot\|$ be an arbitrary norm on
  $X$. For each $x = \sum_{i = 1}^{n} \alpha_i e_i \in X$, we have
  \begin{align*}
    \|x\| &= \|\sum_{i = 1}^{n} \alpha_i e_i\| \\
    & \le \sum_{i = 1}^{n} |\alpha_i|\|e_i\| \\
    &\le \max \{ |\alpha_i| \} \sum_{i = 1}^{n} \|e_i\| \\
    & \le \|x\|_\infty \sum_{i = 1}^{n} \|e_i\|
  \end{align*}

  % Moreover, we have \[
  %   \|x\|_\infty \le \Big(\sum_{i = 1}^{\infty} |\alpha_i|^2\Big)^{
  % \frac{1}{2} } = \|x\|_2
  % \]
  Therefore the identity map $I: (X, \|\cdot\|_\infty) \to (X,
  \|\cdot\|)$ is continuous. Since the set $ K = \{ x \in X \ :
  \ \|x\|_\infty \le 1 \}$ is compact,  K is also compact in $(X,
  \|\cdot\|)$. Moreover if $V \subset K$ is closed
  in $\|\cdot\|_\infty$, then $V$ will also be closed
  in $ \|\cdot\|$ being compact. Hence $I|_K$ will be an open map.
  Now its an easy application of \autoref{EquivalentDefnsofContinuity}.
\end{proof}


% TeX_root = ../main.tex

\marginnote{\scriptsize 16/01/2025 }

\begin{definition}
  An ideal $I \subsetneq \mathcal{A}$ is called maximal if for any
  ideal $I \subset J$, then either $J = I$ or $J = \mathcal{A}$
\end{definition}

\begin{proposition}
  Recall that $R/I$ is a field if and only if $I$ is a
  maximal ideal of the ring $R$.
\end{proposition}

\begin{remark}
  While we consider ideals of an algebra, since the algebra has a
  vector space structure, we demand the ideal to be subspace with
  respect to the underlying linear operations
\end{remark}

\begin{lemma}
  If $\mathcal{A}$ is a (unital) Banach algebra, and $I$ is a closed
  ideal, then $\mathcal{A}/I$ is a (unital) Banach algebra.
\end{lemma}
\begin{proof}
  Since $I$ is an ideal, the ring structure of $A/I$ is well defined.
  Moreover since $\mathcal{A}$ is a Banach space and $I$ is a closed
  subspace, we know that $\mathcal{A}/I$ is a Banach space. Thus, we
  just need to verify the norm inequality
  \begin{align*}
    \|ab + I\| \le \|a + I\|\|b + I\|
  \end{align*}
  By the definition of the quotient norm for any $\varepsilon > 0$,
  there exist a $i_a, i_b \in
  I$ such that
  \begin{align*}
    \|a + i_a\| \le \|a + I\| + \varepsilon, \quad \|b + i_b\| \le
    \|b + I\| + \varepsilon
  \end{align*}
  Then,
  \begin{align*}
    \|ab + I\| &\le \|ab + i_ab + i_b a + i_aib\|  \\
    &= \|(a + i_a)( b + i_b)\| \\
    &\le \| a + i_a\| \|b + i_b\| \\
    &\le (\|a + I\| + \varepsilon)(\|b + I\| + \varepsilon) \\
    &\le \|a + I\|\|b + I\| + \varepsilon \|b  + I\| + \varepsilon
    \|a + I\| + \varepsilon^2
  \end{align*}
  Since $\varepsilon$ was chosen arbitrarily, this gives our result.
\end{proof}

\begin{lemma}
  In a unital Banach algebra, every maximal ideal is closed.
\end{lemma}
\begin{proof}
  Take a maximal ideal, take its closure, then it must be either the
  ideal itself or the whole of the algebra. If it contains the whole
  of the algebra, then the unital element must be there. Then the
  original ideal must contain invertible elements by the openness of
  the set of invertible elements. This will make the original ideal,
  the whole of the algebra, which is a contradiction.
\end{proof}

\begin{lemma}
  If $\mathcal{A}$ is a Banach algebra that is a division ring (If
  every non-zero element has an inverse), then
  $\mathcal{A} = \mathbb{C}$.
  \label{2:A=C}
\end{lemma}
\begin{proof}
  Let $0 \neq a \in \mathcal{A}$. Let $\lambda \in \sigma(a)$. Then
  $\lambda1 - a$ is not invertible. Hence $\lambda1 - a = 0$. So $a =
  \lambda 1$. Hence $\mathcal{A} = \mathbb{C}$.
\end{proof}

\begin{corollary}
  Let $\mathcal{A}$ be a unital commutative Banach algebra, and $I$
  be a maximal ideal. Then $\mathcal{A}/I = \mathbb{C}$.
\end{corollary}

\begin{lemma}
  \label{specturm_of_comm_alg_and_spectrum_of_element}
  For every $a \in \mathcal{A}$, a commutative unital Banach algebra,
  \begin{align*}
    \sigma(a) = \{ \tau(a)  \ : \   \tau \in \textrm{sp}(\mathcal{  A}) \}
  \end{align*}
\end{lemma}
\begin{proof}
  Let $\tau \in \textrm{sp}(\mathcal{A})$, then $\tau(\tau(a)1 -
  a) = 0$ and therefore $\tau(a)1 - a \in \textrm{Ker}(\tau)$,
  hence it is not invertible. Hence $\tau(a) \in \sigma(a)$.

  Conversely, let $ \lambda \in \sigma(a)$, then $\lambda 1 - a$ is
  not invertible, thus the ideal $I = \langle \lambda1 - a \rangle$
  is a proper ideal, since $r(\lambda 1 - a)$ will not be invertible
  for any $r \in \mathcal{A}$. So $1 \notin \langle \lambda1 - a
  \rangle$. By Zorn's lemma, $\langle \lambda1 -a \rangle $ is
  contained in a maximal ideal $I_\lambda$.

  Define $\tau: \mathcal{A} \to \mathbb{C} = \mathcal{A}/I_\lambda :=
  x \mapsto x + I_\lambda$. Then $\tau \in \textrm{sp}(\mathcal{A})$,
  and $\tau(a) = a + I_\lambda = \lambda + I_\lambda$ since
  $\lambda1 - a \in I_\lambda$. Now by the identification of $A/I_\lambda$
  with $\mathbb{C}$ as in \autoref{2:A=C}, we see that $\tau(a) =
  \lambda$.
\end{proof}

\begin{definition}
  Let $\mathcal{A}$ be a commutative Banach algebra. Define
  \begin{align*}
    \Phi :  \mathcal{A} \to  C(\textrm{sp}(\mathcal{A})) :=
    \Phi(a)(\tau) = \tau(a)
  \end{align*}
  for all $a \in \mathcal{A}, \tau \in \textrm{sp}(\mathcal{A})$. The
  map $\Phi$ is called the \textbf{Gelfand transform}.
\end{definition}

\begin{theorem}
  $\Phi$ is a contractive algebra homomorphism with $\| \Phi(a)\| = r(a)$.
\end{theorem}
\begin{proof}
  That $\Phi$ is contractive follows easily from
  \begin{align*}
    \|\Phi(a)(\tau)\| = \|\tau(a)\| \le \|a\|
  \end{align*}
  since $\tau$ is a contraction as proved in
  \autoref{lem:spectrum_is_compact}. Linearity and multiplicativity
  of $\Phi$ follows form the fact that every element $\tau \in
  \textrm{sp}(\mathcal{A})$ is linear and multiplicative on $\mathcal{A}$.
\end{proof}

\begin{remark}[Maximal ideals of $\mathcal{A}$ and $\textrm{sp}(\mathcal{A})$]
  Let $\mathcal{A}$ be unital commutative Banach algebra. Let $\tau
  \in \textrm{sp}(\mathcal{A})$, then $\textrm{Ker}(\tau)$ is a
  closed ideal of $\mathcal{A}$, and $A/\textrm{Ker}(\tau) \cong
  \mathbb{C}$ by the first isomorphism theorem. So
  $\textrm{Ker}(\tau)$ is a maximal ideal. The converse of this is
  also true. Natural map to the quotient space of a maximal ideal
  (which is now a filed isomorphic to $\mathbb{C}$) gives an element
  of the $\textrm{sp}(\mathcal{A})$. Hence $\textrm{sp}(\mathcal{A})$
  can be identified with the maximal ideals of $\mathcal{A}$.
\end{remark}

\begin{remark}
  Suppose $\tau, \tau^\prime \in \textrm{sp}(\mathcal{A})$, with
  $\textrm{Ker}(\tau) =\textrm{Ker}(\tau^\prime)$. Let $a \in
  \mathcal{A}$, then $\tau(a)1 - a \in \textrm{Ker}(\tau) =
  \textrm{Ker}(\tau^\prime)$ implies $\tau(a) = \tau^\prime(a)$ for
  all $a \in \mathcal{A}$.
\end{remark}

\begin{remark}
  Combining both of the above, we see that $\textrm{Ker}(\Phi)$ is
  the intersection of all maximal ideals of $\mathcal{A}$, that is
  the radical of $\mathcal{A}$.
\end{remark}

\begin{theorem}
  \label{thm:beurling}
  Let $\mathcal{A}$ be a Banach algebra. Then $\forall a \in
  \mathcal{A}$, we have
  \begin{align*}
    r(a) = \lim_{n \to \infty} \|a^n\|^{1/n}
  \end{align*}
\end{theorem}
\begin{proof}
  \textcolor{red}{verify}
\end{proof}

\begin{corollary}
  If $\|a^2\| = \|a\|^2$, then $r(a) = \|a\|$ and the Gelfand
  transform will be halal.
  \marginnote{ \scriptsize \it \textcolor{red}{not sure. Might need
  C* algebra structure}}
\end{corollary}

\begin{example}
  Let $T \in B(\mathcal{H})$ be self-adjoint. Let $\mathcal{A} =
  \overline{\textrm{span}}\{ T^n  \ : \  n \in \mathbb{N} \cup {0}
  \}$. Then $\mathcal{A}$ is a unital Banach algebra. Moreover we have
  \begin{align*}
    \|T^2\| = \|T\|^{2}
  \end{align*}
  by the self adjointness of $T$. Thus the Gelfand transform $\Phi$
  is isometric on $\mathbb{R}$-$\overline{\textrm{span}}\{T^n\}$.
\end{example}

% TeX_root = ../main.tex

\marginnote{\scriptsize 23/01/2025 }

\begin{example}
  Given an infinite dimensional Hilbert space with orthonormal basis
  $(u_n)_{n \in \mathbb{N}}$, show that $\{ u_n \}$ is not compact.
\end{example}
\begin{proof}
  Since $\|u_\alpha - u_\beta\| = \sqrt{2}$, take
  $\frac{1}{\sqrt{2}}$ radius balls around each $u_\alpha$ to get a
  collection of open balls that cover the set with no finite subcover.

  Another way to see is to use the sequential compactness criterion
  and see that the  sequence $(u_n)$ does not have any convergent subsequence.
  Since this is a metric space, sequential compactness is equivalent
  to compactness.
\end{proof}

\begin{theorem}[Every Hilbert space is $\ell^{2}(A)$]
  Let $\mathcal{H}$ be a Hilbert space, $(u_\alpha)_{\alpha \in A}$
  is an orthonormal basis, then there is a unitary map $U:
  \mathcal{H} \to \ell^{2}(A)$ such that $U(u_\alpha) = \chi_\alpha$
\end{theorem}
\begin{proof}
  We first note that by linearity, if $p \in \textrm{span}\{ u_\alpha
  \ : \ \alpha \in A \}$, then $U(p)$ is determined by $\chi_\alpha$.
  Next, by Bessel's inequality,
  \begin{align*}
    \|U(p)\|_{\ell^{2}(A)} \le \|p\|
  \end{align*}
  Hence $U$ is bounded. Hence it can be continuously extended to
  $\mathcal{H} = \overline{\textrm{span} \{ u_\alpha \}}$ as a limit
  of sequences. Also, by the equivality in the Bessel's inequality,
  we get that $U$ is an isometry, hence one-to-one.

  Now it remains to show that $U$ is onto. Given $g \in \ell^{2}(A)$,
  we know that there exists at most a countable set $ \{ \alpha_1 ,
  \alpha_2 , \ldots \} = A_0$ such that $g(\alpha_i) \neq 0$. Consider
  \begin{align*}
    h_n = \sum_{j = 1}^{n} g(\alpha_j) u_{\alpha_j}
  \end{align*}
  then,
  \begin{align*}
    u(h_n)(\alpha) =
    \begin{cases}
      g(\alpha_j), & \alpha_j \in A_0 \\
      0, & \textrm{otherwise}
    \end{cases}
  \end{align*}
  Moreover,
  \begin{align*}
    \Big \| U(h_n) - g \Big \|_{\ell^{2}(A)}^2 = \sum_{j =
    n+1}^{\infty} |g(\alpha_j)|^2 \to 0
  \end{align*}
  Now if
  \begin{align*}
    h = \sum_{j = 1}^{\infty}  g(\alpha_j) u_{\alpha_j} \in
    \mathcal{H} \quad (\textrm{ since } g \in \ell^{2}(A))
  \end{align*}
  we get
  \begin{align*}
    \|h - h_n\|^2 = \|U(h_n) - g\|^2_{\ell^{2}(A)} \to 0
  \end{align*}
  and the injectivity of $U$ shows that $U(h) = g$.
\end{proof}

\chapter{Banach Space Techniques}
\begin{definition}
  If $X$ is a real or complex normed vector space with a norm, and
  the complete in the topology induced by the norm, it is called a Banach space.
\end{definition}

\begin{definition}
  If $X, Y$ are normed vector spaces over $\mathbb{R}$ or
  $\mathbb{C}$, $\Lambda: X \to Y$ linear, then the norm of the operator
  \begin{align*}
    \|\Lambda\| = \sup \{ \|\Lambda x\|  \ : \  \|x\|< 1 \}
  \end{align*}
  If $\|\Lambda\| < \infty$, then we say that $\Lambda$ is bounded.
\end{definition}

\begin{proposition}
  Given $\Lambda: X \to Y$, a linear map between normed linear
  spaces, the following are equivalent
  \begin{enumerate}[label=(\arabic*)]
    \item $\Lambda$ is bounded
    \item $\Lambda$ is continuous
    \item $\Lambda$ is continuous at some $x_o \in X$
  \end{enumerate}
\end{proposition}
\begin{proof}
  $(1 \implies 2)$
  \begin{align*}
    \|\Lambda(x - y)\| \le \|\Lambda\| \|x - y\|
  \end{align*}
  gives $\|\Lambda\|$-Lipschitz continuity.

  $(2 \implies 3)$ Follows from the definition.

  $(3 \implies 1)$ For each $\varepsilon > 0$, there is $\delta > 0$
  such that for each $x \in X$, with $\|x - x_o\| < \delta$, then
  $\|\Lambda x - \Lambda x_o\| < \varepsilon$.
  Thus for $ \|y \| < \delta$,  by linearity of $\Lambda$, we get
  \begin{align*}
    \|\Lambda y\| = \|\Lambda(x_o + y) - \Lambda x_o\| < \varepsilon
  \end{align*}
  Again using linearity, we get for $\|y^\prime\| < 1$,
  \begin{align*}
    \|\Lambda y^\prime\| < \frac{\varepsilon}{\delta} < \infty
  \end{align*}
  Now since $\overline{ B_1(0)} \subset B_2(0)$, we see that
  $\|\Lambda\| < \frac{2\varepsilon}{\delta} < \infty$.
\end{proof}

\section{Consequence of Baire category theorem}

\begin{theorem}[Baire Category Theorem]
  \label{thm:Baire-Category-Rudin}
  If $(X, d)$ is a complete metric space, and $V_1 , V_2 , \ldots$
  are dense subsets, then
  \begin{align*}
    \bigcap_{n = 1}^{\infty} V_n
  \end{align*}
  is dense in $X$.
\end{theorem}
\begin{proof}
  We show that for any non-empty open set $W \subset X$,
  \begin{align*}
    \bigcap_{n = 1}^{\infty} V_j \cap W \neq \emptyset
  \end{align*}
  We write $B_r(x) = \{ y \in X  \ : \  d(x, y)< r \}$. Since $V_1$
  is dense and open, $V_{1} \cap W$ is open and dense in $W$. Thus we
  can find an $r_{1} > 0$ such that $\overline{B_{r_{1}}(x_1)}
  \subset W \cap V_1$. (First find an $r^\prime > 0$ such that
    $B_{r^\prime}(x_1) \subset W \cap V_{1}$. Then take $r_{1} =
  \frac{r^\prime}{2}$).

  We inductively proceed by taking $x_n \in V_n \cap
  B_{r_{n-1}}(x_{n-1})$ such that $\overline{B_{r_n}(x_n)} \subset
  V_n \cap B_{r_{n-1}}(x_{n-1})$. Without loss of generality, choose
  $0 < r_n < \frac{1}{n}$. This gives a sequence which satisfies for
  $i, j > n$ that $x_i, x_j \in B_{r_n}(x_n) \implies d(x_i, x_j) <
  2r_n < \frac{2}{n}$. Hence $x_n$ is Cauchy. By completeness $x_n
  \to x \in \overline{B_{r_n}(x_n)} \subset V_n \cap W$ for all $n$.
  Thus $x \in W$ and
  \begin{align*}
    x \in \bigcap_{n = 1}^{\infty}V_n
  \end{align*}
\end{proof}

% TeX_root = ../main.tex

\marginnote{\scriptsize 28/01/2025 }

\begin{example}
  Let $(X, \Sigma, \mu)$ be a measure space. Define $E : \Sigma \to
  B(L^{2}(X, \mu)):= A \to M_{\chi_A}$. It is easy to see that $E$
  satisfies the first 3 properties of a spectral measure.
  To verify the fourth property, let $A_1 , A_2 , \ldots \in \Sigma$
  be disjoint collection. Then
  \begin{align*}
    M_{\chi_{\cup_{n = 1}^{\infty}A_n}}(f) &= \chi_{\cup_{n =
    1}^{\infty}A_n}f = \sum_{n = 1}^{\infty} \chi_{A_n} f = \sum_{ n
    = 1}^{\infty}  M_{\chi_{A_n}}  f
  \end{align*}
  shows that the fourth property is also satisfied.
\end{example}

\begin{proposition}
  \label{prop:spectral_measure_gives_complex_measure}
  Let $E$ be a spectral measure on $(X, \Sigma, \mathcal{H})$. Then
  for every $\xi, \eta \in \mathcal{H}$.
  \begin{align*}
    E_{\xi, \eta}(A) = \langle  E(A) \xi ,  \eta \rangle
  \end{align*}
  defines a finite measure on $(X, \Sigma)$ with $\|E_{\xi, \eta}\|
  \le \|\xi\| \|\eta\|$.
\end{proposition}
\begin{proof}
  That $E_{\xi, \eta}(\emptyset) = 0$ is evident. Hence we only need
  to verify the countable disjoint additivity to show that $E_{\xi,
  \eta}$ is a measure. Let $A_1 , A_2 , \ldots \in \Sigma$ be a
  mutually disjoint collection. Then
  \begin{align*}
    E_{\xi, \eta} \Big( \bigcup_{n = 1}^{\infty}A_n \Big) &= \Big
    \langle E \Big( \bigcup_{n = 1}^{\infty} A_n \Big) \xi ,  \eta
    \Big \rangle \\
    &= \Big \langle  \sum_{n = 1}^{\infty} E(A_n) \xi ,  \eta
    \Big \rangle \\
    &= \sum_{n = 1}^{\infty} \big \langle E(A_n) \xi, \eta \big \rangle  \\
    &= \sum_{n = 1}^{\infty}  E_{\xi, \eta}(A_n)
  \end{align*}
  Taking the summation outside the inner product is justified by the
  below lemma. Thus we see that $E_{\xi, \eta}$ is a complex measure.
  Moreover, since $|\langle E(A) \xi ,  \eta \rangle| \le \|\xi\|
  \|\eta\|$, as $E(A)$ is a projection, we see that the measure is
  finite (Although this is implicit in complex measures). To see that
  $ \|E_{\xi, \eta}\| \le \|\xi\| \|\eta\|$, use the definition of
  bounded variation and the lemma below.
\end{proof}

\begin{lemma}
  Let $\{ P_i \}_{i \in I}$ be a family of pairwise orthogonal
  projections on a Hilbert space $\mathcal{H}$. Then there exist a
  unique projection $P$ on $\mathcal{H}$ such that $\forall \xi, \eta
  \in \mathcal{H}$,
  \begin{align*}
    \sum_{i \in I} \langle P_i \xi ,  \eta \rangle = \langle P \xi ,
    \eta \rangle
  \end{align*}
\end{lemma}
This means the convergence of orthogonal projections is not in the
norm sense, but rather in the sense above. For example
\begin{example}
  Consider $P_n = P_{\delta_n}$ in $B(\ell^{2}(\mathbb{N}))$, and let
  $  Q_n = \sum_{i = 1}^{n} P_i$. Then clearly $Q_i$ doesn't converge
  in norm, but rather in the above sense to the identity map in
  $B(\ell^{2}(\mathbb{N}))$.
\end{example}

\begin{lemma}[Reisz representation for sesquilinear forms]
  \label{lem:resiz_representation_for_sesquilinear}
  \marginnote{ \scriptsize \it \textcolor{red}{I'm not convinced.
  Need to verify}}
  Let $B(\cdot, \cdot)$ be a bounded sesquilinear form on a Hilbert
  space $\mathcal{H}$. Then there exist a unique $T \in
  B(\mathcal{H})$ such that for all $\xi, \eta \in \mathcal{H}$,
  \begin{align*}
    B(\xi, \eta) = \langle T \xi ,  \eta \rangle
  \end{align*}
\end{lemma}

\begin{proposition}
  \label{prop:spectral_measure_and_integration}
  \marginnote{ \scriptsize \it \textcolor{red}{verify if $\phi$ is a
  linear functional}}
  Let $E$ be a spectral measure on $(X, \Sigma, \mathcal{H})$, and
  $\phi$ be a bounded linear functional on $X$. Then
  there exist a unique $  T_\phi \in B(\mathcal{H})$ such that for
  all $\xi, \eta \in \mathcal{H}$
  \begin{align*}
    \int_X \phi \ d  E_{\xi, \eta} = \langle T_\phi \xi ,  \eta \rangle
  \end{align*}
  and we denote $T_\phi:= \int_X \phi \ d  E$
\end{proposition}
\begin{proof}
  See that the integral is sesquilinear on $\xi, \eta$ for simple
  functions. Then use
  \autoref{lem:resiz_representation_for_sesquilinear}.
\end{proof}

Note that the set $\mathcal{M}(X)$ of all bounded measurable functions on $X$,
equipped with sup norm is a Banach space. Moreover with pointwise
product and complex conjugation, it turns into a commutative C$^*$ algebra.

\begin{theorem}
  Let $E$ be a spectral measure on $(X, \Sigma, \mathcal{H})$ and
  \marginnote{ \scriptsize \it \textcolor{red}{verify if $\phi$ is a
  linear functional}}
  $\phi$ be a bounded measurable function. The map
  \begin{align*}
    \mathcal{M}(X) \to B(\mathcal{H}):= \phi \mapsto \int_ X \phi \ d E
  \end{align*}
  is a contractive $*$-homomorphism.
\end{theorem}
\begin{proof}
  To show that it is a $*$-homomorphism, observe that
  $\overline{E_{\xi, \eta}} = E_{\eta, \xi}$. This follows since
  \begin{align*}
    \overline{E_{\xi, \eta}(A)} = \overline{ \langle E(A) \xi , \eta
    \rangle } = \langle \eta , E(A) \xi \rangle = \langle E(A)  \eta , \xi
    \rangle = E_{\eta,  \xi}(A)
  \end{align*}
  Now let $T_{\overline{\phi}} \in B(\mathcal{H})$ corresponding to
  $\phi$ as in \autoref{prop:spectral_measure_and_integration}. Then
  by definition,
  \begin{align*}
    \int_X \overline{\phi} \ d  E_{\xi, \eta} &= \langle T_{\overline{
    \phi}}  \eta, \psi  \rangle
  \end{align*}
  Now the left integral is equal to
  \begin{align*}
    \overline{\int_X  \phi \ d E_{\eta, \xi}} = \overline{\langle
    T_{\phi} \eta ,  \xi \rangle } = \langle \xi ,  T_\phi \eta
    \rangle = \langle T_\phi^* \xi ,  \eta \rangle
  \end{align*}
  Thus we get that $T_{\overline{\phi}} = T_{\phi}^*$ and hence the
  map preserve the involution.

  To show that the map is multiplicative, we need to show that
  \begin{align*}
    \int_X \phi \psi \ d  E = \int_X \phi \ d  E \circ \int_X \psi \ d  E
  \end{align*}
  Notice that by \autoref{prop:spectral_measure_and_integration}, we'll
  done if we show that for all $\xi, \eta \in \mathcal{H}$,
  \begin{align*}
    \int_X \phi \psi \ d E_{\xi, \eta} = \int_X \phi \ d  E_{(\int_X
    \psi \ d E) \xi, \eta}
  \end{align*}
  But for this, it is enough to show the equivalence of the measures
  $\psi dE_{\xi, \eta}$ and $E_{(\int_X \psi \ d E)\xi, \eta }$. That
  is for all $A \in \Sigma$, we need to show that
  \begin{align}
    \label{eq:4}
    \Big \langle E(A)\Big(\int_X  \psi \ d  E\Big)\xi ,  \eta \Big
    \rangle = \int \psi \chi_A \ d  E_{\xi, \eta}
  \end{align}
  But again, the left hand side of the inner product is
  \begin{align*}
    \Big \langle \Big( \int_X \psi \ d E\Big) \xi ,  E(A) \eta \Big \rangle
  \end{align*}
  since $E(A)$ is a projection. Again by
  \autoref{prop:spectral_measure_and_integration}, we see that the
  above inner product is
  \begin{align*}
    \int_X \psi \ d E_{\xi, E(A) \eta}
  \end{align*}
  Then \autoref{eq:4} reduces to showing
  \begin{align*}
    \int_X \psi \ d E_{\xi, E(A) \eta} = \int \psi \chi_A \ d  E_{\xi, \eta}
  \end{align*}
  Again using the same reasoning, it is enough to show that the
  measures are the same. That is for any $B \in \Sigma$, we must have
  \begin{align*}
    E_{\xi , E(A) \eta}(B) =     \int_X \chi_A \chi_B \ dE_{\xi, \eta}
  \end{align*}
  But this is equivalent to
  \begin{align*}
    \langle E(A) \xi ,  E(B) \eta \rangle = \langle E(A)E(B) \xi ,
    \eta \rangle =  \langle  E(A \cap B) \xi , \eta \rangle = \int_X
    \chi_{A\cap B} \ d E_{ \xi, \eta} = E_{\xi, \eta}(A \cap B)
  \end{align*}
  which is true by a property of the Spectral measure.
\end{proof}

% \begin{definition}
%   A C$^*$ algebra is a Banach *-algebra satifying the norm equality
%   $\|a^*a\| = \|a\|^2$, for all $a \in \mathcal{A}$.
% \end{definition}

% TeX_root = ../main.tex

\marginnote{\scriptsize 09/09/2025 }

\begin{theorem}
  Let $F \subset M_n$ be a family of diagonalizable matrices, then
  $F$ is a commuting family if and only if it is simultaneously diagonalizable.
\end{theorem}
\begin{proof}
  It is an easy exercise to show that a simultaneously diagonalizable
  family is commuting.

  We prove the converse by induction over $n$. For $n = 1$, there's
  nothing to prove. Assume that this is true for all $F^\prime
  \subset M_k$, where $k < n$. If each $A \subset F$ is of the form
  $A = \lambda I$, again nothing to prove. Thus, assume $A$ is
  diagonalizable with eigenvalues $\lambda_1 , \lambda_2 , \ldots ,
  \lambda_r, r \ge 2$, and assume $AB = BA$, for each $B \in F$.
  Then $A, B$ are simultaneously diagonalizable by previous theorem
  and hence without loss of generality, assume that
  $A$ is diagonal.

  \textcolor{red}{verify rest from the lecture notes}
\end{proof}

\begin{remark}
  Given $C \in M_n(\mathbb{C})$, we can think of function associated
  with $C$, as $ \langle x , y \rangle   \to \langle Cx, y\rangle$.
  Notice that this function preserves all information about $C$,
  since we can find the individual matrix entries.
  We can also associate $Q_c : x \to \langle Cx , x \rangle$. We
  recall $Q_c$ determines $C$ as
  \begin{align*}
    \langle Cx, y \rangle = \frac{1}{4}\sum_{j = 1}^{4} i^j\|Q_c(x + i^j
    y)\|^2, \quad ( i = \sqrt{-1})
  \end{align*}
\end{remark}

\subsection{Hermitian, normal, and unitary matrices}
\begin{definition}
  Let $A \in M_{n, m}$, then the adjoint $A^* \in M_{m, n}$ satisfies
  $\langle  Ax , y \rangle  = \langle x , A^*y \rangle $ for each $x
  \in \mathbb{C}^m, y \in \mathbb{C}^n$.

  If for $A \in M_n$, $A = A^*$, then we say that $A$ is Hermitian or
  self-adjoint. If $A = - A^*$, the it is called skew-hermitian.

  If $A \in M_n$, then $A = B+ iC$, where $B = \frac{1}{2} (A + A^*),
  C = \frac{1}{2i} (A - A^*)$ have the property $B = B^*, C = C^*$.
  Here $ B$ is called the real part of $A$, and $iC$ is called the
  imaginary part of $A$.
\end{definition}

\begin{proposition}
  \begin{align*}
    A = A^* \iff (iA) = -(iA)^*
  \end{align*}
\end{proposition}

\begin{proposition}
  A matrix $A \in M_n$ is Hermitian if and only if for all $x \in
  \mathbb{C}^n$, $ \langle Ax , x \rangle  \in \mathbb{R}$.
\end{proposition}
\begin{proof}
  If $A$ is Hermitian, then
  \begin{align*}
    \overline{\langle Ax , x \rangle } & = \langle x , Ax \rangle =
    \langle A^*x , x \rangle  = \langle Ax , x \rangle
  \end{align*}
  shows that $\langle  Ax , x \rangle  \in \mathbb{R}$.

  Conversely, assume that $\langle  Ax , x \rangle  \in \mathbb{R}$
  for all $x \in \mathbb{C}^n$. Let $ A = B + iC$, where $B= B^*, C =
  C^*$. Then
  \begin{align*}
    \langle Ax , x \rangle  &= \underbrace{\langle Bx , x
    \rangle}_{\in \mathbb{R}}  + i \underbrace{\langle Cx , x
    \rangle}_{\in \mathbb{R}}  \\
  \end{align*}
  We conclude that $\langle  Cx , x \rangle  = 0$ for all $x$. Now
  using polarization identity, we get that $C = 0$.
\end{proof}

We also consider and equivalent of unimodular numbers.

\begin{definition}
  Let $A \in M_n$. $A$ is unitary if $A^*A = I = AA^*$.
\end{definition}

\begin{proposition}
  Let $A \in M_n$. The following are equivalent.
  \begin{enumerate}[label=(\arabic*)]
    \item $A$ is unitary.
    \item The columns of $A$ form an orthonormal basis.
    \item Rows of $A$ form an orthonormal basis.
    \item A preserves the norm for each $x \in \mathbb{C}^n$. That is
      $\|Ax\| = \|x\|$.
    \item $A$ preserve the inner product. That is $\langle  Ax , Ay
      \rangle = \langle x , y \rangle$
  \end{enumerate}
\end{proposition}
\begin{proof}
  You know,
\end{proof}

% TeX_root = ../main.tex

\marginnote{\scriptsize 04/02/2025 }

\begin{corollary}
  Let $X, Y$ be Banach spaces. If $T: X \to Y$ is linear, bounded map
  and $ T$ is one-to-one and onto, then $T^{-1}$ is bounded.
\end{corollary}
\begin{proof}
  Use the fact the open mapping theorem gives that $T^{-1}$ is
  continuous, and that continuity is boundedness in linear spaces.
\end{proof}

\begin{theorem}[Closed graph theorem]
  Let $X, Y$ be Banach spaces, then the graph of $T$, defined as
  $G(T) = \{ (x, Tx)  \ : \  x \in X \} \subset X \times Y$, under
  the norm $\|(x, y)\| = \| x\|_X + \|y\|_Y$ is closed if and only if
  $T$ is bounded.
\end{theorem}
\begin{proof}
  Refer back to functional analysis notes.
\end{proof}

\section{Applications of Banach-Steinhaus}

Let $C_{per}([-\pi, \pi])$ denote continuous functions $f: [-\pi,
\pi] \to \mathbb{C}$ such that $f(\pi) = f(-\pi)$. Since each
$C_{per}([-\pi, \pi]) \subset L^{2}([-\pi, \pi])$, each such $f$ has
a Fourier series. Let
\begin{align*}
  c_n = \frac{1}{2\pi} \int_{-\pi}^{ \pi}  f(t) e^{int} \ dt
\end{align*}
and
\begin{align*}
  s_n(t) = \sum_{ j = -n}^{n} c_j e^{ijt}
\end{align*}
We know that $s_n \to f$ in $L^2$.

But what about pointwise convergence. Let
\begin{align*}
  D_n = \sum_{j = -n}^{n} e^{ijt} =
  \frac{\sin((n+\frac{1}{2})t)}{\sin(\frac{t}{2})}
\end{align*}
Observe that
\begin{align*}
  \frac{1}{2\pi} \int_{-\pi}^{ \pi}  f_n(t)D_n(x-t) \ dx &= \sum_{j =
  -n}^{n} \Big( \frac{1}{2\pi} \int_{-\pi}^{ \pi} f(x) e^{-ijt} \ dt
  \Big) e^{ijx} \\
  &= \sum_{j = -n}^{n} c_j e^{ijx} = s_n(x)
\end{align*}
Choose linear functionals $\Lambda_n: C_{per}([-\pi, \pi]) \to
\mathbb{C}$ defined as
\begin{align*}
  \Lambda_n(f) = \frac{1}{2\pi} \int_{-\pi}^{ \pi}  f(t)D_n(-t) \ dt = s_n(0)
\end{align*}
Putting sup norm on $C_{per}([-\pi, \pi])$, we get that $\Lambda_n$
is linear, bounded, with
\begin{align*}
  |\Lambda_n(f)| &= \Big|\frac{1}{2\pi} \int_{-\pi}^{ \pi}
  f(t)D_n(-t) \ dt \Big| \\
  &\le \frac{1}{2 \pi}\|f\|_\infty\|D_n\|_1
\end{align*}
\textcolor{red}{Read the rest from Rudin}

% TeX_root = ../main.tex

\chapter{}

\begin{remark}
  \label{remark:measure_from_integral}
  Let $(X, \mathcal{M}, \mu)$ be a measure space, a simple function $s: X \to [0, \infty]$, then $\phi: \mathcal{M} \to [0, \infty]$ defined as \[
    \phi(E) = \int_E s \ d \mu
  \]
  is a measure.
\end{remark}
\begin{proof}
  Since our definiton demands that measure of some set should be finite, we verify this first. We see that \[
    \phi(\emptyset) = \int_\emptyset s \ d \mu = 0
  \]
  Now to prove countable disjoint additivity, consider the disjoint collection $\{ E_l \}_{l \in \mathbb{N}}$. And assume that $s = \sum_{j = 1}^{n} \alpha_j \chi_{A_j}$ with $ \alpha_j \in [0, \infty]$, with $A_j$s disjoint. Then for $E = \cup_{l = 1}^{\infty}E_l$, we have \begin{align*}
    \phi(E) &= \sum_{j = 1}^{n} \alpha_j \mu(A_j \cap E) \\ 
    &= \sum_{ j = 1}^{n} \sum_{l \in \mathbb{N}} \alpha_j \mu(A_j \cap E_l) \\ 
    &= \sum_{ l \in \mathbb{N}} \sum_{j = 1}^{n} \alpha_j \mu(A_j \cap E_l) \\ 
    &= \sum_{ l \in \mathbb{N}} \int_{E_l} s \ d \mu
  \end{align*}
\end{proof}


\section{Properties of Integrals}

\begin{theorem}
  \label{thm:properties_of_integrals}
  The interal of a non-negative measurable function from a measure space $(X, \mathcal{M}, \mu)$ has the following properties \begin{enumerate}[label=(\arabic*)]
    \item If $0 \le f \le g$, then $\int_ E f(x) \ dx \le \int_E g \ d \mu$
    \item If $A \subset B$, $A, B \in \mathcal{M}$, then $\int_A f \ d \mu \le \int_B f \ d \mu$
    \item If $c \in [0, \infty)$, $ E \in \mathcal{M}$, then $\int_E cf \ d \mu = c \int_E f \ d \mu$
    \item If $f = 0$, or $ \mu(E) = 0$, then $\int_E f \ d \mu = 0$
    \item For all $ E \in \mathcal{M}$, \[
        \int_E f \ d \mu = \int_X f \chi_{E} \ d \mu
    \]
  \end{enumerate}
\end{theorem}
\begin{proof}
  \begin{enumerate}[label=(\arabic*)]
    \item By definition \[
        \int f \ d \mu = \sup_{\substack{t \textrm{ is simple} \\ t \textrm{ is measurable} \\ 0 \le t \le f}} \int_E  t \ d \mu
    \]
      then the simple function  $t \le f$ is also $ t \le g$. Hence suping over simple functions under $g$, every simple function under $ f$ is included.
    \item Let $s = \sum_{i = 1}^{n} \alpha_i \chi_{A_i}$ be a simple function $0 \le s \le f$ with $\int s \ dx + \epsilon > \int f \ d \mu$.
      Using the inclusion $ A \subset B$, we get \begin{align*}
        \int_A s \ d \mu &= \sum_{n \in \mathbb{N}} \alpha_n
      \end{align*}

    \item Suppose $s = \sum_{j = 1}^{n} \alpha_j \chi_{A_j}$ is a simple function with disjoint $A_j$s. Then $s \chi_{E} = \sum_{j = 1}^{n} \alpha_j \chi_{A_j \cap E}$ is also simple (and measurable), and \[
        \int_E s \ dx = \sum_{j = 1}^{n} \alpha_j \mu(A_j \cap E) = \int  s \chi_{E} \ dx
    \]
      Hence the statement is true for simple measurable functions. Next, consider $f$ non-negative measurable, then for $\epsilon \ge 0$, we have a simple measurable function $s$ with $\int_E s \ d \mu + \epsilon > \int_E f \ d \mu$. Then by preceding part, \[
          \int s \chi_{E} \ d \mu + \epsilon > \int_E f \ d \mu
      \]
      Also $s \chi_E \le f \chi_E$. So \[
          \int f \chi_E \ d \mu + \epsilon \ge \sup_{t \textrm{ is simple}} \int s \chi_E \ d \mu + \epsilon > \int f \ d \mu
      \]
      Taking $\epsilon \to 0$ gives \[
          \int f \chi_E \ d \mu \ge \int_E f \ d \mu
      \]
      For the reverse inequaltiy, note that $f \chi_E \le f$, and use similar circus.
  \end{enumerate}
\end{proof}

\begin{theorem}[Monotone convergence theorem]
  Let $(X, \mathcal{M}, \mu)$ be a measure space, given a sequence $f_n: X \to [0, \infty]$ of measurable functions and they are monotone increasing, i.e for each $x \in X$, $0 \le f_1(x) \le f_2(x) \le \ldots $, then \[
       \lim_{n \to \infty} \int f_n \ d \mu = \int \lim_{n \to \infty} f_n \ d \mu
  \]
\end{theorem}
\begin{proof}
  Let $f = \lim_{n \to \infty} f_n$ be the pointwise limit. Then $ f$ is measurable. From $f_n \le f_{n+1}$, we get that \[
       \int f_n \ d \mu \le \int f_{n+1} \ d \mu
  \]
  so both sides of the claimed identity exist, and from $f_n \le f$, we also know that \[
       \int f_n \ d\mu \le \int f \ d \mu
  \]
  which taking the limits give us, \[
       \lim_{n \to \infty} \int f_n \ d\mu \le \int f \ d \mu
  \]

  Now let $s: X \to [0, \infty]$ be a simple measurable function $s \le f$. Choose $0 \le c < 1$, and define $ E_n = \{ x \in X \ : \ f_n(x) \ge cs(x) \} = (f_n - s)^{-1}([0, \infty])$. \textcolor{red}{Verify that difference between an extended real valued function and a real valued function is measurable, then $E_n$ is measurable}. This gives a nested sequence $E_1 \subset E_2 \subset \ldots $.
  If $ f(x) > 0$, then by $ f(x) > cs(x)$ and $  f_n(x) \to f(x)$, there is $n \in \mathbb{N}$ such that $x \in E_n$. 
  On the other hand if $f(x) = 0$, then $cs(x) = 0 = f(x)$, so $ x \in E_n$ for all $n \in \mathbb{N}$.
  We see that each $ x \in X$ is in the  union $\cup_{n = 1}^{\infty}E_n$. Hence $X = \cup_{n = 1}^{\infty}E_n$. Now we define $ \phi: \mathcal{M} \to [0, \infty]$ by \[
    \phi(E) = \int_E s \ d \mu
  \]
  which is a measure and $\phi(X) = \phi(\cup_{n = 1}^{\infty}E_n) =  \lim_{n \to \infty} \phi(E_n)$ by \autoref{thm:properties_of_integrals}. We rewrite this as \begin{align*}
    \int_X s \ d \mu &= \lim_{n \to \infty} \int_{E_n} s \ d \mu \\ 
    &= \lim_{n \to \infty} \int_X s \chi_{E_n} \ d \mu \\ 
    &\le \lim_{n \to \infty} \int_X \frac{1}{c} f_n \ d \mu
  \end{align*}
  Now take sup over all such simple (bounded) functions $s \le f$ and let $c \to 1$.
  \textcolor{red}{Finish this proof}.
\end{proof}

% TeX_root = ../main.tex

\marginnote{\scriptsize 11/02/2025}

\begin{exercise}
  Explore the correspondance between states on $\ell^{\infty}(X)$ and
  the finitely additive probability measures on $X$.
\end{exercise}

\begin{theorem}
  Let $\mathcal{H}$ be a Hilbert space, and let $f: B(\mathcal{H})
  \to \mathbb{C}$ be linear. Then, the following are equivalent.
  \begin{enumerate}[label=(\arabic*)]
    \item $f$ is SOT continuous.
    \item $f$ is WOT continuous.
    \item There exists $\xi_1 , \xi_2 , \ldots , \xi_n, \eta_1 ,
      \eta_2 , \ldots , \eta_n \in \mathcal{H}$ such that
      \begin{align*}
        f(T) = \sum_{i = 1}^{n} \langle T \xi_i , \eta_i \rangle
      \end{align*}
  \end{enumerate}
\end{theorem}
\begin{proof}
  We just need to prove $1 \implies 3$. By a lemma from last semester,
  there must exist $ \xi_1 , \xi_2 , \ldots , \xi_n \in \mathcal{H}$
  such that $|f(T)| \le \sum_{i = 1}^{n}  \|T \xi_i\|$ for all $T \in
  B(\mathcal{H})$. Let $   \mathcal{K} = \overline{\{ T \xi_o  \ : \
  T \in B(\mathcal{H}) \}} \subset \mathcal{H}$. Then
  \begin{align*}
    \phi: \mathcal{K} \to \mathbb{C} := T \xi_0 \to f(T)
  \end{align*}
  is a well defined linear functional for $\mathcal{K}$. Now by Reisz
  representation, we see that $\phi(T \xi_0) = \langle  T \xi_0 ,
  \eta_0 \rangle $ for some $\eta_0 \in \mathcal{H}$.

  But here, for $\xi_1 , \xi_2 , \ldots, \xi_n$, consider them in
  $\mathcal{H}^n$ and do the same process to get a $(\eta_1 , \eta_2
  , \ldots , \eta_n) \in \mathcal{H}^n$.
\end{proof}

\begin{theorem}
  Let $X, Y$ be normed spaces. For $1 \le p \le \infty$, define
  \begin{align*}
    \|(x, y)\|_p = \Big( \|x\|^p + \|y\|^p\Big)^{\frac{1}{p}}
  \end{align*}
  Then $\|\cdot\|_p$ is a norm extending both norms in $X$ and $Y$.
  If $X$ and $Y$ are Banach, then so is $(X \oplus Y, \|\cdot\|_p)$.
  All these norms are equivalent for all $1 \le p \le \infty$.
\end{theorem}

\begin{proposition}
  We have that $(X \oplus_p Y)^* = X^* \oplus_q Y^*$.
\end{proposition}
\begin{proof}
  We first show when $p = 1, q = \infty$. Consider the map
  \begin{align*}
    \Xi: (X \oplus_1 Y)^* \to X^* \oplus_\infty Y^* := F \to F_X \oplus F_Y
  \end{align*}
  once we identify $X$ as $X \oplus 0$ and $Y$ similarly as subspaces
  of $X \oplus Y$. See that this is a bijective linear map. Now show
  that the norms are preserved in $\Xi$.
\end{proof}

% TeX_root = ../main.tex

\marginnote{\scriptsize 13/02/2025 }

\begin{theorem}[Bicommutant Theorem]
  Let $A \subset B(\mathcal{H})$ be a unital $*$-subalgebra, where
  $\mathcal{H}$ is separable. Then $\overline{ \mathcal{A}}^{SOT} =
  \mathcal{A}^{\prime \prime}$
\end{theorem}
\begin{proof}
  Let $T \in \overline{\mathcal{A}}^{SOT}$, and $S \in
  \mathcal{A}^\prime$. Let $(T_i) \in \mathcal{A}$ be a net
  converging in SOT to $T$. Then for all $\xi \in \mathcal{H}$,
  \begin{align*}
    TS \xi = \lim_{i} T_iS \xi = \lim_{i} ST_i \xi = S \lim_i T_i \xi = ST \xi
  \end{align*}
  shows that $T \in \mathcal{A}^{\prime\prime}$.

  To see the converse, let $T \in \mathcal{A}^{\prime \prime}$, fix
  $\xi \in \mathcal{H}$, and $\varepsilon >
  0$ and let $K = \overline{\mathcal{A} \xi}$. Then notice that $K$
  is a reducing subspace for all operators in $\mathcal{A}$.
  \textcolor{red}{verify}. Then $P_K T = TP_K$ by our results on
  reducing subsapces in last semester. Thus $P_K \in \mathcal{A}^\prime$.

  %Prof version
  Let $\xi_1 , \xi_2 , \ldots , \xi_n \in \mathcal{H}$, $\varepsilon>
  0$ be given. Let
  \begin{align*}
    \mathcal{K} = \Big \{
      \begin{bmatrix}%{c}
        A \xi_1 \\
        \vdots \\
        A \xi_n
    \end{bmatrix}  \ : \  A \in \mathcal{A} \Big \}
  \end{align*}
  For each $A \in \mathcal{A}$, let $\mathcal{A}^{(n)} = I_n \otimes A\in
  B(\mathcal{H}^n)$. Let $\mathcal{A}^n = \{ A^{(n)}  \ : \  A \in
  \mathcal{A} \}$. Observe that $\mathcal{A}^{(n)}(\mathcal{K})
  \subset \mathcal{K}$. Also observe that $\mathcal{K}$ is reducing
  for all $A^{(n)} \in \mathcal{A}^{(n)}$. So $P_{\mathcal{K}} \in
  (\mathcal{A}^{(n)})^\prime = M_n(\mathcal{A}^\prime)$.
  \textcolor{red}{verify the rest}.
\end{proof}

\begin{definition}
  For every $\xi, \eta \in \mathcal{H}$, denote $\omega_{\xi, \eta}
  \in B(\mathcal{H})^*$ such that
  \begin{align*}
    \omega_{\xi, \eta}(T) = \langle T \xi ,  \eta \rangle
  \end{align*}
\end{definition}

\begin{theorem}
  $B(\mathcal{H}) = \overline{\textrm{span}}\{ \omega_{\xi, \eta}
  \ : \    \xi, \eta \in \mathcal{H} \}^*$
\end{theorem}

\begin{proposition}
  For $T \in B(\mathcal{H})$, the following are equivalent.
  \begin{enumerate}[label=(\arabic*)]
    \item $\langle T \xi ,  \xi \rangle \ge 0, \forall \xi
      \in \mathcal{H}$
    \item $\exists S \in B(\mathcal{H})$ such that  $T = S^*S$
    \item $\exists S \ge 0$ such that $T = S^2$. In this case we
      write $S = T^{\frac{1}{2}}$.
  \end{enumerate}
\end{proposition}
\begin{proof}
  $2 \implies 1$ is obvious.

  Now for the other one, it is clear that $T$ is self adjoint. So we
  get that $\Xi: C(\textrm{sp}(T)) \to B(\mathcal{H})$. We claim that
  $\textrm{sp}(T) \subset \mathbb{R}^{\ge 0}$, which then completes
  the proof by the identification of $T$ with the multiplication
  operator corresponding to the identity function in $C(\textrm{sp}(T))$.

  Let $\lambda < 0$. Then for all $\xi \in \mathcal{H}$,
  \begin{align*}
    \|(T - \lambda I) \xi\|^2 &= \langle (T - \lambda I) \xi , (T -
    \lambda I) \xi \rangle \\
    &= \|T \xi\|^2 - 2 \lambda \langle T \xi ,
    \xi \rangle + \lambda^2 \|\xi\|^2 \\
    & \ge \lambda^2 \|\xi\|^2
  \end{align*}
  which shows that $T - \lambda I$ is injective. Then it has a linear
  left inverse $S$. Then $S : (T - \lambda I) \xi \mapsto  \xi$. Thus
  we get $\|S\| \le \frac{1}{\lambda}$. Also $S(T- \lambda I) = I$
  implies $(T - \lambda I) S^* = I$, which shows that $T - \lambda I$
  is not invertible. Hence $\lambda \not\in \sigma(T)$.
\end{proof}

\begin{definition}
  $T \in B(\mathcal{H})$ is called positive if it satisfies any of
  the above conditions.
\end{definition}

\begin{definition}
  Let $T \in B(\mathcal{H})$. We define $|T| = (T^*T)^{ \frac{1}{2}}$
\end{definition}

\begin{definition}
  Given $\xi, \eta \in \mathcal{H}$, we denote $P_{\xi, \eta}:
  \mathcal{H} \to  \mathcal{H} := \rho \mapsto  \langle \rho ,  \eta
  \rangle  \xi$.
\end{definition}

\begin{lemma}
  $\forall T \in B(\mathcal{H})$, $\exists S \in B(\mathcal{H})$ such
  that $T = S|T|$.
\end{lemma}
\begin{proof}
  For every $\xi \in \mathcal{H}$, define the map $S_1 :
  \textrm{Image}(|T|) \to  \textrm{Image}(T) := |T|\xi \mapsto  T
  \xi$. If $|T|\xi = 0$, then $ \langle |T|\xi , |T|\xi \rangle =
  \langle |T|^2 \xi,  \xi \rangle = \langle T^*T \xi ,  \xi \rangle =
  \|T \xi\|^2 = 0$. Moreover \textcolor{red}{$S_1$ is linear}. Thus
  the map is well defined.

  We have $\|S_1 |T|(\xi)\|^2 = \|T \xi\|^2 = \langle T \xi ,  T \xi
  \rangle = \langle T^*T \xi ,  \xi \rangle  = \langle |T|\xi , |T|
  \xi \rangle = \||T|\xi\|^2$. Hence $S_1$ is an isometry. Thus it
  extends to an isometry from $\overline{\textrm{Image}(|T|)}$ onto
  $\overline{\textrm{Image}(T)}$. Now define $S \in B(\mathcal{H})$
  to be $S = S_1P_{\overline{ \textrm{Image}(|T|)}}$. Then $  S^*S =
  P_{\overline{ \textrm{Image}(|T|)}}$, and $SS^* = P_{\overline{
  \textrm{Image}(T)}}$.
\end{proof}

\begin{theorem}
  If $T \in \mathcal{K}(\mathcal{H})$, $\exists$ orthonormal basis
  $\{ \xi_n \}, \{ \eta_n \}$ for $\mathcal{H}$ and $(\alpha_n) \in
  \textbf{c}_0$ such that
  \begin{align*}
    T = \sum_{n \in \mathbb{N}} \alpha_n P_{\xi_n, \eta_n}
  \end{align*}
\end{theorem}
\begin{proof}
  By the lemma, $T = S|T|$. Observe $|T| \in
  \mathcal{K}(\mathcal{H})$. There exists an orthonormal basis
  $\xi_n$ and $(\alpha_n) \in \textbf{c}_0(\mathbb{R})_+$ such that
  $|T| = \sum_{n} \alpha_n P_{\xi_n}$. Then $T = S |T| = \sum_{n}
  \alpha_n SP_{\xi_n, \eta_n} = \sum_{n} a_n P_{S \eta_n, \xi_n}$.
  Work on with previous lemma to show that $S
  \xi_n$ is orthonormal basis.
\end{proof}

\begin{definition}
  Let $f \in \mathcal{K}(\mathcal{H})^*$. Define $T_f \in
  B(\mathcal{H})$ to be the unique operator satisfying $\langle T_f
  \xi ,  \eta \rangle = f(P_{\xi, \eta})$. Then the map $f \to T_f$ is linear.
\end{definition}

% TeX_root = ../main.tex

\chapter{}

\begin{theorem}[Fatou's Lemma]
  If $(f_n)$ is a sequence of measurable functions $f_n: X \to [0, \infty]$, then \[
        \int \lim_{n \to \infty} \inf f_n \ d \mu \le \lim_{n \to \infty} \inf \int  f_n \ d \mu
  \]
\end{theorem}
\begin{proof}
  Let $g_m(x) = \inf_{n \ge m} f_n(x)$. Then $0 \le g_1(x) \le g_2(x) \le \ldots$. Then by MCT, we get \[
       \int \lim_{m \to \infty} g_m \ d \mu = \lim_{n \to \infty} \int  g_m \ d \mu(x)
  \]
  Also see that if $n \ge m$, then $f_n \ge g_m$ and therefore, we get \[
       \int  f_n \ d \mu \ge \int  g_m \ d \mu
  \]
  So \[
    \inf_{n \ge m} \int  f_n \ d \mu \ge \int g_m \ d \mu
  \]
  Now taking $m \to \infty$ on both sides, we get \[
    \lim_{n \to \infty} \inf \int  f_n \ d \mu \ge \int \lim_{n \to \infty} \inf f_n\ d \mu
  \]
  which proves the theorem.
\end{proof}

\begin{example}
  Let $\mu$ be the counting measure on $X = \{ 0, 1 \}$. Let \[
    f_{2n}(x) = \begin{cases}
      0, & x = 0 \\ 
      1, & x = 1
    \end{cases} \quad 
    f_{2n+1} = \begin{cases}
      1, & x = 0 \\ 
      0, & x = 1
    \end{cases}
  \]
  Then $\int \lim_{n \to \infty} \inf f_n \ d \mu = 0 \le 1 = \lim_{n \to \infty} \inf \int  f_n \ d \mu$
\end{example}

\begin{theorem}[Lebesgue dominated convergence theorem]
  Let $(X, \mathcal{M}, \mu)$ be a measurable space. If $f_n: X \to \mathbb{C}$ defines a sequence of measurable functions pointwise converging to $f$, and there is a $g \in L^1(\mu)$ such that \[
      |f_n| \le g, \quad \forall n \in \mathbb{N}
  \]
  Then $f \in L^1(\mu)$ and  \[
      \int |f_n - f| \ d \mu \to 0
  \]
  So we exchange limits and integral and write \[
      \lim_{n \to \infty} \int  f_n \ d \mu = \int f \ d \mu
  \]
\end{theorem}
\begin{proof}
  We have $|f| \le g$ since $|f_n| \le g$ for all $n \in \mathbb{N}$ and $f_n \to f$ pointwise. Consider $h_n = 2g - |f_n - f| \ge 0$ (\textcolor{red}{Use triangle inequality to show that $h_n \ge 0$}). Fatou's lemma gives \begin{align*}
    \lim_{n \to \infty} \inf \int (2g - |f_n - f|) \ d \mu &\ge \int   \lim_{n \to \infty} (2g - |f_n - f|)  \ d \mu \\ 
    &= 2 \int  g \ d \mu + \int \lim_{n \to \infty} \inf (- |f_n - f|) \ d \mu \\ 
    &= 2 \int  g \ d \mu - \int \lim_{n \to \infty} \sup (|f_n - f|) \ d \mu \\ 
  \end{align*}

  But we also have \[
    \lim_{n \to \infty} \inf \int (2g - |f_n - f|) \ dx \le 2 \int g \ d \mu + \lim_{n \to \infty} \inf \int |f_n - f| \ d \mu
  \]
  \textcolor{red}{Hairy logic. Verify with Rudin.}
\end{proof}


\section{Measure Zero}

\begin{definition}
  We say that a property $P$ holds almost everywhere if \[
    \mu \big( \{ x \in X \ : \ \textrm{P does not hold at} x \}\big) = 0
  \]
\end{definition}

\begin{theorem}
  If $f: X \to [0, \infty]$ and $\int  f \ d \mu = 0$, then $f = 0$ almost everywhere. Conversely, if $ f = 0$ almost everywhere then $\int  f \ d \mu = 0$.
\end{theorem}
\begin{proof}
  Let $E_n = \{ s \in X \ : \ f(x) \ge \frac{1}{n} \}$ and $E = \cup_{n = 1}^{\infty}E_n = \{  x \in X \ : \ f(x) > 0 \}$. Note that $E$ is measurable since each of $E_i$ is measurable.
  So \begin{align*}
    0 = \int f \ d \mu & \ge \int f \chi_{E_n} \ d \mu \\ 
    & \ge \int \frac{1}{n}\chi_{E_n} \ dx \\ 
    &= \frac{1}{n} \mu(E_n) \ge 0
    \end{align*}
  Hence $\mu(E_n) = 0$ for each $n \in \mathbb{N}$. Hence $E$ is a measure zero set. Therefore $f$ is zero almost everywhere.

  Conversely if $f = 0$ almost everywhere, then let \[
    g(x) = \begin{cases}
      0, & f(x) = 0 \\
      \infty, & \textrm{otherwise}
    \end{cases}
  \]
  Then $g$ is a measurable simple function with $g > f$ and $\int g \ d \mu = $. Hence $ \int f \ d\mu = 0$.
\end{proof}

\begin{theorem}
  If $f_n: X \to \mathbb{C}$ defines a sequence of measurable functions and if \[
    \sum_{n \in \mathbb{N}} |f_n| \in L^1(\mu). 
  \]
  Then \[
    \sum_{n \in \mathbb{N}} f_n \in L^1(\mu)
  \]
  and the series $\sum_{n \in \mathbb{N}} f_n$ converges almost everywhere.
  \textcolor{red}{See theorem}
\end{theorem}
\begin{proof}
  We assume each $f_n$ is defined on $X \setminus S_n$ with $\mu(S_n) = 0$. We have to show that there exist a set $S$ with $\mu(S) = 0$ and $\forall x \notin S$,  $\sum_{n \in \mathbb{N}} f_n(x)$ converges.
  Let \[
    f(x) = \sum_{n \in \mathbb{N}} |f_n(x)|
  \]
  By MCT \[
      \sum_{n \in \mathbb{N}} \int |f_n| \ d \mu = \int f \ d\mu \le \infty
  \]
  This implies $\{ x \ : \ f(x) = \infty \}$ has measure zero. Hence if $x \notin S_n$ nad $x \notin \{ x \ : f(x) = \infty \}$, then $ \sum_{n \in \mathbb{N}} f_n(x)$ converges absolutely. Thus $S = \cup_{n = 1}^{\infty}S_n \cup \{ x \ : \ f(x) = \infty \}$ is measure zero and $ x \in S^c$
\end{proof}

\begin{definition}
  Let $(X, \mathcal{M}, \mu)$ be a measure space. If for any $E \in \mathcal{M}$ and $F \subset E$, $\mu(E) = 0$ implies $F \subset \mathcal{M}$, then $\mu$ is called complete.
\end{definition}






% TeX_root = ../main.tex

\marginnote{\scriptsize 18/03/2025 }

% TeX_root = ../main.tex

\chapter{Complex Measures}

\marginnote{\scriptsize 20/03/2025 }

\section{Consequence of Radon-Nikodym Theorem}

\begin{theorem}
  If $\mu, \nu$ are positive $\sigma$-finite measures such that $\nu
  \ll \mu$, then there is a positive measurable function $h$ such that
  $d \nu = h \mu$
\end{theorem}

\begin{theorem}[Hahn-Decomposition Theorem]
  Let $\mu$ be a real-valued complex measure (signed measure) on a
  measurable space
  $(X, \mathcal{M})$. Then there are two sets $A, B$ such that $A
  \cup B = X, A \cap B = \emptyset$ and
  \begin{align*}
    \mu_+(E) := \mu( E \cap A), \quad \mu_-(E) = \mu(E \cap B)
  \end{align*}
  with $ \mu_+ \perp \mu_-$ and $ \mu_+ + \mu_- = \mu$, and $\mu_+ +
  \mu_- = |\mu|$.

  Moreover, if $   \mu = \mu_1 - \mu_2$ with $ \mu_1 , \mu_2$ being
  positive measures, then for any $E \in \mathcal{M}$ we have
  $\mu_1(E) \ge \mu_+(E), \mu_2(E) \ge \mu_-(E)$
\end{theorem}
\begin{proof}
  Since $\mu$ is a complex measure, $\mu \ll |\mu|$ and by
  Radon-Nikodym, there is a $h \in L^{1}(\mu)$ with $h(x) \in \{ 1, 2  \}$
  (polar decomposition) such that $d \mu = h d |\mu|$.

  Let $A = h^{-1}(1), B = X \setminus A$. We find that $d\mu_+ =
  \frac{1}{2}(d|\mu| + d\mu) = \frac{1}{2}(|h| d |\mu| + h d |\mu|) = h_+
  d |\mu|$,  and similarly $\mu_- = h_- d |\mu|$. The rest follows easily.
\end{proof}

\section{Bounded linear functionals on $L^p$}

\begin{note}
  Let $\mu$ be a positive measure, $1 \le p \le \infty$ and
  $\frac{1}{p} + \frac{1}{q} = 1$. Fixing $ g \in L^{1}(\mu)$.
  Holder's inequality gives that for any $ f \in L^{p}(\mu)$,
  \begin{align*}
    \big|\int fg \ d \mu\big| \le \|f\|_p \|g\|_q
  \end{align*}
  So that $\Lambda_g : L^{p}(\mu) \to \mathbb{C} := f \to \int fg \ d
  \mu$ is a bounded linear functional. Thus, we have a map $\Lambda :
  L^{q}(\mu) \to L^{p}(\mu)^* := g \mapsto \Lambda_g$.

  For $1 \le p < \infty$, the converse is true, too
\end{note}

\begin{lemma}
  If $\mu$ is $\sigma$-finite on $(X, \mathcal{M})$, then there is a
  $\omega \in L^{1}(\mu)$ such that $\forall x \in X : 0 <  \omega(x) < 1$.
\end{lemma}
\begin{proof}
  Choose a partition $E_j$ of $X$ such that $\mu(E_j) < \infty$ for
  each $j \in \mathbb{N}$. Let
  \begin{align*}
    \omega = \sum_{n \in \mathbb{N}} \frac{1}{2^n} \frac{1}{1 +
    \mu(E_n)} \chi_{E_n}
  \end{align*}
  Since $E_j$ is a partition, we get that
  \begin{align*}
    \int |\omega| \ d \mu &= \int \omega \ d \mu \\
    &= \int \sum_{n \in \mathbb{N}} \frac{1}{2^n} \frac{1}{1 +
    \mu(E_n)} \chi_{E_n} \ d \mu \\
    &= \sum_{n \in \mathbb{N}} \frac{1}{2^n} \frac{1}{1 +
    \mu(E_n)} \mu(E_n) \\
    &\le \sum_{n \in \mathbb{N}} \frac{1}{2^n}  = 1
  \end{align*}
  Hence $\omega \in L^{1}(\mu)$ is the required function.
\end{proof}

\begin{corollary}
  If $\mu$ is $\sigma$-finite, then $\tilde{ \mu}$ given by $d
  \tilde{\mu} = \omega d \mu$ is finite.
\end{corollary}

\begin{theorem}
  Let $\mu$ be a $\sigma$-finite measure, $1 \le p < \infty$, $q$ as
  usual. If $ \Lambda \in L^{p}(\mu)^*$, then there is $ g \in
  L^{q}(\mu)$ such that
  \begin{align*}
    \Lambda = \Lambda_g
  \end{align*}
  and $\|\Lambda\| = \|g\|_q$
\end{theorem}
\begin{proof}
  Begin by assuming $\mu$ is finite. Let $\Lambda : L^{p}(\mu) \to
  \mathbb{C}$ be a bounded linear functional. Notice that $\chi_E \in
  L^{p}(\mu)$ for each $E \in \mathcal{M}$. Consider $\lambda(E) =
  \Lambda(\chi_E)$. Let $\{E_j\}_{n = 1}^\infty$ be a partition of $E$. We find
  \begin{align*}
    \lambda \Big( \bigcup_{j = 1}^{n}E_j \Big) &= \Lambda \Big(
    \sum_{j = 1}^n \xi_{E_j} \Big) \\
    &= \sum_{j = 1}^{n} \Lambda(\chi_{E_j}) \\
    &= \sum_{ j = 1}^{n} \lambda(E_j)
  \end{align*}
  We conclude $\lambda$ is finitely additive. Note
  \begin{align*}
    \Big \| \chi_E - \chi_{\cup_{j = 1}^{n}E_j} \Big \|_p = \Big(
    \mu\big(\bigcup_{j = n+1}^{\infty} E_j\big)\Big)^{\frac{1}{p}}
  \end{align*}
  Using monotone convergence and boundedness of $\Lambda$, we get
  \begin{align*}
    \Lambda(  \chi_{\cup_{j = 1}^{n}E_j}) \to \Lambda(\chi_E)
  \end{align*}
  Thus $\lambda$ is a measure and $\lambda \ll \mu$ by the definition
  of $\lambda$. By
  Radon-Nikodym, we have $g \in L^{1}(\mu)$ with $d \lambda = g d \mu$.

  For $f$ simple,
  \begin{align*}
    \Lambda(f) = \int f \ d \lambda = \int fg \ d \mu := \Lambda_g(f)
  \end{align*}
  Now, consider $p = 1$. Then
  \begin{align*}
    \Big|\int f \ d \lambda\Big| = \Big|\int fg \ d \mu\Big| \le
    \|\Lambda\| \|f\|_1
  \end{align*}
\end{proof}

% TeX_root = ../main.tex

\marginnote{\scriptsize 01/04/2025 }

\begin{lemma}
  Let $X$ be locally compact Hausdorff and $\lambda: C_c(X) \to
  \mathbb{R}$ be bounded linear functional. Then there are positive
  bounded linear functionals $\lambda_+, \lambda_-$ such that
  $\lambda = \lambda_+ - \lambda_-$.
\end{lemma}
\begin{proof}
  For this, we find bounded linear functional $\rho$,
  \begin{align*}
    |\lambda(f)| \le \rho(|f|) \le C \| f\|_\infty
  \end{align*}
  and then let $\lambda_+ = \frac{1}{2}(\lambda + \rho)$ and
  $\lambda_- = \frac{1}{2}(\rho - \lambda)$

  We define the map $\rho :   C_c(X)^+ \to  \mathbb{C} := f \mapsto
  \sup \{ |\lambda(h) \ | \ h \in C_c(X), |h| \le f \}$, where
  $C_c(X)^+$ is the set of non-negative real valued functions in $C_c(X)$.
  Let $ f, g \in C_c(X)^+$, then there is a $ h_1, h_2 \in C_c(X)$
  such that $|h_1| \le f, |h_2| \le g$ with $\rho(f) \le |
  \lambda(h_1)| + \varepsilon$ and $\rho(g) \le |\lambda(h_2)| + \varepsilon$.
  So, $\rho(f) + \rho(g) \le |\lambda(h_1)| + |\lambda(h_2)| + 2 \varepsilon$.
  Let $\alpha_1, \alpha_2 \in \{ \pm 1 \}$ such that
  $\lambda(\alpha_i h_i) = \alpha_i \lambda(h_i) \ge 0$. Then,
  \begin{align*}
    |\lambda(\alpha_1h_1)| + |\lambda( \alpha_2 h_2)| =
    \lambda(\alpha_1 h_1) + \lambda(\alpha_2 h_2)  =
    \lambda(\alpha_1 h_1 + \alpha_2 h_2)
  \end{align*}
  So,
  \begin{align*}
    \rho(f) + \rho(h) & \le \lambda(\alpha_1 h_1 + \alpha_2 h_2) +
    \varepsilon \\
    & \le \rho(|\alpha_1h_1 + \alpha_2 h_2|) + 2 \varepsilon  \quad
    \textrm{since } \alpha_1h_2 +
    \alpha_2 h_2 \le |\alpha_1h_1 + \alpha_2h_2|\\
    & \le \rho(|h_1| + |h_2|) + 2 \varepsilon \quad \textrm{since }
    \rho \textrm{ is order preserving} \\
    & \le \rho(f + g) + 2 \varepsilon \quad \textrm{since }
    \rho \textrm{ is order preserving}
  \end{align*}
  Since this holds for any $\varepsilon > 0$, we get $\rho(f + g) \ge
  \rho(f) + \rho(g)$.

  To show the reverse inequality, let $f, g \in C_c(X)^+$, and $h \in
  C_c(X)$ be such that $|h| \le f + g$. We define
  \begin{align*}
    h_1(x) =
    \begin{cases}
      \frac{f(x)}{f(x) g(x)}h(x), &f(x) + g(x) > 0 \\
      0, &\textrm{else}
    \end{cases} \\
  \end{align*}
  and $h_2(x) = h(x) - h_1(x)$. Then $|h_1| \le f, |h_2| \le g$.
  Moreover, $h_1, h_2$ are continuous where $ f(x) + g(x) \ge 0$. Next,
  \begin{align*}
    |\lambda(x)| &= |\lambda(h_1 + h_2)| \\
    & \le | \lambda(h_1)| + | \lambda(h_2)| \\
    & \le \rho(f) + \rho(g)
  \end{align*}
  Taking supremum over $h$, we get $\rho(f + g) \le \rho(f) +
  \rho(g)$. We have established additivity of $\rho$ for $f, g \ge 0$.
  For general $ f, g \in C_c(X)$, split $f, g, h$ into differences of
  positive and negative parts and rearrange to apply $\rho$ with
  linearity. Thus we'll get $\rho(f +g) = \rho(f) + \rho(g)$.

  Now to show homogeneity, let $  c \in \mathbb{R}$ and $f \in
  C_c(X)$. If $c < 0$,
  \begin{align*}
    \rho(cf^+) & = - \rho((cf^+)^-) \\
    &= - \rho(|c|f^+) \\
    &= - |c| \rho(f^+) \\
    &= c \rho(f^+)
  \end{align*}
  Again by splitting $f = f^+ - f^-$, we get the homogeneity. Thus we
  get $\rho$ is linear.
\end{proof}

\begin{lemma}
  If $\nu$ is a $\sigma$-finite regular positive measure on a locally compact
  Hausdorff space, and $  \mu$ is a complex measure with $|\mu| \ll
  \nu$, then $\mu$ is regular.
\end{lemma}
\begin{proof}
  Using Radon-Nikodym theorem, for a measurable set $E$, we have
  \begin{align*}
    \mu(E) = \int_E h \ d \nu
  \end{align*}
  with $h \in L^{1}(\mu)$. Considering that $\mu$ is regular, there
  are sequences  of open sets $V_j \supset E$, $\nu(V_j \setminus E)
  \stackrel{ j \to \infty}{\longrightarrow} 0$ and compact sets $K_j
  \subset E$, such that $\nu(E \setminus K_j) \stackrel{j \to
  \infty}{\longrightarrow} 0$.

  Next, by dominated convergence theorem,
\end{proof}

% TeX_root = ../main.tex

\marginnote{\scriptsize 18/03/2025 }

Recall that

\begin{proposition}
  Let $T \in B(X, W), S \in B(Y, Z)$. Then $\exists! T \otimes
  S : X \otimes_\alpha Y \to W \otimes_\alpha Z$ such that $(T
  \otimes S)(x \otimes y) = T(x) \otimes S(y)$.
\end{proposition}
\begin{proof}
  Let $z = \sum_{i = 1}^{n} x_i \otimes y_i$. By our definitions
  \begin{align*}
    \|(T \otimes S)(z)\|_\alpha & = \sup_{\phi \in W^\dagger_1, \psi \in
    Z^\dagger_1} \Bigg\{ \bigg|\sum_{i = 1}^{n} \phi(T(x_i))
    \psi(S(y_i))\bigg|\Bigg\} \\
    & \le \|T\|\|S\| \|z\|_\alpha
  \end{align*}

  Moreover, $\|T \otimes S\| = \|T\|\|S\|$ by taking $x \otimes y$,
  where $\|T(x)\| \le \|T\| - \varepsilon$ and similarly for $S$.
\end{proof}

\begin{exercise}
  Let $z \in \textbf{c}_{00} \otimes X$ such that $z = \sum_{i =
  1}^{n} \delta_{n_i} \otimes x_i$. Then
  \begin{align*}
    \Big \| \sum_{i = 1}^{n} \delta_{n_i} \otimes x_i \Big \|_\alpha
    &= \sup_{\phi \in \ell^{1}_1, \psi \in X^*_1} \Big \{
    \Big|\sum_{i = 1}^{n} \phi(\delta_{n_i})  \psi(x_i) \Big|\Big \}
    \le \max \{\|x_i\|\}
  \end{align*}
\end{exercise}

\begin{theorem}
  \begin{align*}
    \textbf{c}_0 \otimes_\alpha X \cong \textbf{c}_0(X)
  \end{align*}
\end{theorem}

For $z \in X \otimes Y$, define $B_z: X^* \times Y^* \to \mathbb{C}$
such that $B_z(\phi, \psi):= (\phi \otimes \psi)(z)$.

% TeX_root = ../main.tex

\chapter{}

\section{Vitali Sets}

\begin{theorem}
  If $\mathcal{M}$ is a $\sigma$-algebra on $\mathbb{R}$ and
  $\lambda: \mathcal{M} \to [0, \infty]$ is a translation invariant
  measure with $0 < \lambda([0, 1)) < \infty$, then there is $ E
  \subset [0, 1)$ such that $E \notin \mathcal{M}$.
\end{theorem}
\begin{proof}
  Endow $[0, 1)$ with an equivalence relation $a \sim b \iff a-b \in
  \mathbb{Q}$. This gives a partition of $[0, 1)$ by the equivalence classes.
  Now from each of these classes pick (by AOC) one representative element and
  build the set $E$. Observe that for $r, s \in \mathbb{Q}$, $(E + s)
  \cap (E + r) = \emptyset$ if and only if $r = s$.

  Also note that  \[
    [0, 1) \ \subset \ \cup_{r \in \mathbb{Q} \cap [-1, 1]}(E+r)
  \]
  Therefore \[
    E \ \subset [0, 1) \ \subset  \ \cup_{r \in \mathbb{Q} \cap [-1, 1]}(E+r)
    \ \subset \ [-1, 2)
  \]
  \textcolor{red}{verify the rest, its easy}.
\end{proof}

\begin{theorem}[Luzin's theorem]
  Let $X$ be a locally compact Hausdorff space.
  \begin{enumerate}[label=(\arabic*)]
    \item $\mu$ is a regular measure on a $\sigma$-algebra
      $\mathcal{M}$ containing $B(X)$
    \item $f: X \to \mathbb{C}$ is measurable
    \item there is a $A \in \mathcal{M}$ such that $\mu(A) < \infty$
      and $f = 0$ on $A^c$
  \end{enumerate}
  Given $\epsilon> 0$ there is a $g \in C_c(X)$ such that $
  \mu(\{ x \in X  \ : \  f(x) \neq g(x) \}) < \epsilon$
\end{theorem}



% TeX_root = ../main.tex

\marginnote{\scriptsize 25/03/2025 }

Show that if $z \in X \hat{\otimes} Y$, with $z = \sum_{n \in
\mathbb{N}} x_i \otimes y_i$, then $\sum_{n \in \mathbb{N}}
\|x_i\|\|y_i\| < \infty$.

\begin{proposition}
  Let $X, Y, Z$ be Banach spaces. Then $B(X \hat{\otimes} Y , Z)
  \cong \textrm{Bil}(X \times Y , Z)$.
\end{proposition}
\begin{proof}
  Let $\phi \in B(X \hat{\otimes} Y , Z)$. Define $\rho:
  \textrm{Bil}(X \times Y, Z)$ such that $\rho(x, y) = \phi(x \otimes
  y)$. Then the map $\phi \to \rho$ is a linear injective contraction.

  Conversely, given a bounded bilinear $ \rho: X \times Y \to Z$, let
  $\phi: X \otimes Y \to Z$ be the respective linear map, defined as
  $\phi(x \otimes y) = \rho(x, y)$. Let $\omega \in X \otimes Y$.
  Choose $x_1 , x_2 , \ldots , x_n \in X, y_1 , y_2 , \ldots , y_n
  \in Y$ such that $\omega = \sum_{i = 1}^{n} x_i \otimes y_i$ and
  $\|\omega\|_{\wedge} > \sum_{i = 1}^{n}  \|x_i\|\|y_i\| - \varepsilon$. Then
  \begin{align*}
    \|\phi(\omega)\| &= \|\sum_{i = 1}^{n} \rho(x_i, y_i)\| \\
    &\le \|\rho\| \sum_{i = 1}^{n} \|x_i\|\|y_i\| \\
    &\le \|\rho\|(\|\omega\|_{\wedge} + \varepsilon)
  \end{align*}
\end{proof}

\begin{corollary}
  $(X \hat{\otimes} Y)^* \cong \textrm{Bil}(X \times Y) \cong B(X,
  Y^*) \cong B(Y, X^*)$
\end{corollary}
\begin{proof}
  Notice that the last three equivalences follow easily since if
  $\rho: X \times Y \to \mathbb{C}$ is bilinear, then $\tilde{ \rho}:
  X \to Y^*$ defined as $\tilde{ \rho}(x)(y) = \rho(x, y)$, and
  similarly for $Y \to X^*$
\end{proof}

\begin{proposition}
  Let $\sum_{n \in \mathbb{N}} x_n \otimes y_n \in X \otimes Y$ with
  $ \sum_{n \in \mathbb{N}} \|x_i\|\|y_i\| < \infty$ and
  \begin{align*}
    \sum_{n \in \mathbb{N}} f(x_n)y_n = 0
  \end{align*}
  for all $f \in X^*$. Then
\end{proposition}
\begin{proof}
  \textcolor{red}{Notice that all the finite rank operators satisfy this.}

  Let $T \in B(Y, X^*)$. Assume that there's a net $T_j$ of finite
  rank operators such that $T_j \to T$ uniformly on compact subsets
  of $Y$. Let $\varepsilon > 0$ and choose $N \in \mathbb{N}$ such
  that $\sum_{i = N+1}^{\infty} \|x_i\|\|y_i\| < \varepsilon$.
\end{proof}

% TeX_root = ../main.tex

\marginnote{\scriptsize 27/03/2025 }

\begin{definition}
  A bounded linear map $T \in B(X, Y)$ is called a nuclear operator
  if $T$ is in the image of $ X^* \hat{\otimes} Y$.
\end{definition}

\begin{definition}
  Let $X$ be a Banach space. We say $X$ has the approximation
  property, if the identity map on $X$ can be approximated uniformly
  on compact subsets of $X$ by finite rank operators on $X$.
\end{definition}

\begin{theorem}
  If either $X^*$ or $Y$ has the approximation property, then $X^*
  \hat{\otimes} Y \cong \mathcal{N}(X, Y)$.
\end{theorem}

\begin{lemma}
  Let $X$ be a Banach space. Then $\mathcal{N}(X)$ is an ideal of $B(X)$.
\end{lemma}

% TeX_root = ../main.tex

\marginnote{\scriptsize 01/04/2025 }

\begin{definition}
  Let $V, W$ be normed spaces. A crossed norm on $V \otimes W$ is a
  norm satisfying $ \|v \otimes w\| \le \|v\| \|w\|$.
\end{definition}
\begin{definition}
  Let $X, Y$ be normed spaces. A tensor norm on $X \otimes Y$ is a
  crossed norm such that $ X^* \otimes Y^* \subset (X \otimes Y)^*$,
  and $ \|\phi \otimes \psi\| \le \|\phi\| \|\psi\|$ for all $ \phi
  \in X^*, \psi \in Y^*$.
\end{definition}

\begin{lemma}
  If $\|\cdot\|$ is a tensor norm on $X \otimes Y$, then
  \begin{align*}
    \|x \otimes y \| = \|x\| \|y\| \\
    \|\phi \otimes \psi \| = \|\phi\| \|\psi\|
  \end{align*}
  for all $x \in X, y \in Y, \phi \in X^*, \psi \in Y^*$.
\end{lemma}
\begin{proof}
  \textcolor{red}{Exercise}
\end{proof}

Recall that the projective norm is a tensor norm and it is the
largest. Also recall that injective norm is a tensor norm, which is
the smallest.

\begin{theorem}
  If $Z \leqslant X$, then
  \begin{align*}
    Z \check{\otimes} Y \leqslant X \check{\otimes} Y
  \end{align*}
\end{theorem}
\begin{proof}
  Immediate from the definition of the injective tensor norm.
\end{proof}

Moreover note that $X \otimes Y$ has an injection to
$\textrm{Bil}(X^* \times Y^*)$ as evaluations. Also note that
$\textrm{Bil}(X^* \times Y^*)$ has a norm where $\phi \in
\textrm{Bil}(X^* \times Y^*)$ has norm
\begin{align*}
  \|\phi\| = \sup\{|\phi(x, y)| \ : \ \|x\|, \|y\| \le 1\}
\end{align*}
Then the push-back norm into $X \otimes Y$ is the injective tensor norm.

If instead $X \otimes Y$ borrows the norm via the injection $X
\otimes Y \hookrightarrow \textrm{Bil}(X \times Y)^*$ again via the
\textcolor{red}{obvious injection}. This push-back gives the
projective tensor norm.

Recall that $(X \hat{\otimes} Y)^* = \textrm{Bil}(X \times Y)$

\begin{theorem}
  Let $\rho \in \textrm{Bil}(X \times Y)$. Then, $\rho \in (X
  \check{\otimes} Y)^*$ iff there is a radon measure (from Reisz
  representation) $\mu$ in
  $X^*_1 \times Y^*_1$ such that $$\rho(x \otimes y) = \int_{X^*_1 \times
  Y^*_1}  \phi(x) \psi(y) \ d  \mu(\phi, \psi)$$
  In this case, we say $\rho$ is an integral Bilinear map.
\end{theorem}
\begin{proof}
  Use $z \to f_z: $
\end{proof}

\begin{definition}
  Let $X, Y$ be Banach spaces. Let $\alpha$ be a tensor norm on $X
  \otimes Y$. We define the dual norm $\alpha^*$ on $ X^* \otimes Y^*
  $ via the embedding
  \begin{align*}
    X^* \otimes Y^* \hookrightarrow (X \otimes Y)^*
  \end{align*}
\end{definition}

\begin{lemma}
  $\alpha^*$ is a tensor norm.
\end{lemma}

\begin{definition}
  Let $\Xi : X \otimes Y \to Y \otimes X$ be the flip map. The flip
  of $\alpha$ denoted $\alpha_f$ is $\alpha \circ \Xi$.
\end{definition}

clearly the projective and injective norms are flip invariant.

\begin{exercise}
  Let $X, Y$ be finite dimensional normed spaces. Then
  \begin{enumerate}[label=(\arabic*)]
    \item $\wedge^* = \vee$ and $\vee^* = \wedge$
    \item $(\alpha^*)^* = \alpha$
  \end{enumerate}
\end{exercise}
\begin{proof}
  \begin{enumerate}[label=(\arabic*)]
    \item
    \item
      Let $\rho \in X^* \otimes Y^*$. By definition,
      \begin{align*}
        \|\rho\|_{\wedge^*} &= \sup \{ |\rho(z)| \ : \ z \in (X
        \hat{\otimes} Y)_1 \} \\
        &\ge \|\rho\|_{\vee} \\
        &= \sup \{ \rho(x \otimes y) \ : \ x \in X_1, y \in Y_1 \}
      \end{align*}
      Let $\varepsilon >0$, and let $z = \sum_{i = 1}^{N} x_i \otimes
      y_i \in (X \otimes Y)_1$
      be such that $\|z\|_{\wedge} > \sum_{i = 1}^{n} \|x_i\|\|y_i\|
      - \varepsilon$ and $\sum_{i = 1}^{n}  \rho(x_i \otimes y_i) = \rho(z)
      \ge \|\rho\|_{\wedge^*} - \varepsilon$. Then
      \begin{align*}
        \sum_{i = 1}^{n}  \rho(x_i \otimes y_i) \le \|\rho\|_{\wedge}
        \sum_{i = 1}^{n} \|x_i\|\|y_i\| \le \|\rho\|_{\wedge}()
      \end{align*}
  \end{enumerate}
\end{proof}

% TeX_root = ../main.tex

\marginnote{ \scriptsize 13/11/2024}

\begin{lemma}
  Every $T \in B(\mathcal{H})$ can be decomposed as a linear combination of two
  self-adjoint operators.
\end{lemma}
\begin{proof}
  \begin{align*}
    T = \frac{T+T^*}{2} + i \frac{(T-T^*)}{2i}
  \end{align*}
  Verify that $\frac{T+T^*}{2}$ and $\frac{T-T^*}{2i}$ are self adjoint.
\end{proof}

\begin{lemma}
  $T \in B(\mathcal{H})$ is normal if and only if $T + T^*$ and $T
  - T^*$ commutes
\end{lemma}
\begin{proof}
  \begin{align*}
    (T + T^*)( T - T^*) = T^2 + T^*T - TT^* - T^{*2} \\
    (T - T^*)( T + T^*) = T^2 - T^*T + TT^* - T^{*2}
  \end{align*}
\end{proof}

\begin{lemma}
  $T \in B(\mathcal{H})$ is normal if and only if $T = T_1 + i T_2$
  where $T_{1}, T_{2} \in S(\mathcal{H})$ and $T_{1}T_{2} = T_{2}T_1$
\end{lemma}
\begin{proof}
  Directly apply above lemma
\end{proof}

\begin{lemma}
  \label{CommutingOperatorsPreserveInvariantSpaces}
  Let $T \in B(\mathcal{H}), \lambda \in \mathbb{C}$ be an eigenvalue
  of $T$, and let $ S \in B(\mathcal{H})$ with $ST = TS$. Then $
  \textrm{Ker}(T - \lambda I)$ is invariant under $S$
\end{lemma}
\begin{proof}
  Let $M = \textrm{Ker}(T - \lambda I)$ and $ m \in M$. Then
  \begin{align*}
    (T - \lambda I)(S(m)) &= TS(m) - \lambda S(m) \\
    &= ST(m) - \lambda S(m)\\
    & = S(\lambda m) - \lambda S(m)\\
    &= \lambda S(m) - \lambda S(m) = 0
  \end{align*}
\end{proof}

\begin{lemma}
  \label{DistinctEigenvectorsAreOrthogonal}
  Let $T \in B(\mathcal{H})$, be self-adjoint. Let $\lambda_1 \neq
  \lambda_2 \in \sigma(T)$ and $\xi_1, \xi_2 \in \mathcal{H}$ their
  corresponding eigenvectors, then $ \langle \xi_1 , \xi_2 \rangle = 0$
\end{lemma}
\begin{proof}
  \begin{align*}
    0 = \langle T \xi_1 , \xi_2 \rangle - \langle \xi_1 , T\xi_2
    \rangle = \lambda_1 \langle  \xi_1 , \xi_2 \rangle - \lambda_2
    \langle \xi_1 , \xi_2 \rangle = (\lambda_1 - \lambda_2) \langle
    \xi_1 , \xi_2 \rangle
  \end{align*}
\end{proof}

\begin{theorem}[Spectral theorem for compact normal operators]
  Let $\mathcal{H}$ be a separable Hilbert space and $T \in
  \mathcal{K}(\mathcal{H})$ normal. Then there is an orthonormal
  basis $ \{ e_n \}_{n \in \mathbb{N}}$ and a sequence $(\alpha_n)
  \in \textbf{c}_{0}$ such that
  \begin{align*}
    T e_n = \alpha_n e_n
  \end{align*}
  for all $n \in \mathbb{N}$.
\end{theorem}
\begin{proof}
  Let $T \in \mathcal{K}(\mathcal{H})$ be normal. So $\exists S_1,
  S_{2} \in \mathcal{K}(\mathcal{H})\cap S(  \mathcal{H})$ such that
  $T = S_1 + i S_2$ and $S_{1}S_{2} = S_{1}S_{2}$. Then by spectral
  theorem for self-adjoint operators, we get $\sigma(S_{1})$, the set
  of eigenvalues of $S_{1}$ such that
  \begin{align*}
    H = \bigoplus_{\lambda \in \sigma(S_{1})}\textrm{Ker}(S_{1} - \lambda I)
  \end{align*}
  where each $\textrm{Ker}(S_{1} - \lambda I)$ is finite dimensional.

  Since $S_2$ commutes with $S_{1}$, $ \textrm{Ker}(S_{1} - \lambda
  I)$ is invariant under $S_2$ by
  \autoref{CommutingOperatorsPreserveInvariantSpaces}, and
  $S_2|_{\textrm{Ker}(S_{1} -
  \lambda I)}$ is self-adjoint by
  \autoref{InvariantSubspacesofSelfAdjointOperatorsReduce} and
  compact. Thus, by the first part
  of the proof, for each $\lambda \in \sigma(S_1)$, we can choose and
  orthonormal basis $E_\lambda$ for $\textrm{Ker}(S_1 - \lambda I )$
  consisting of eigenvectors of $S_2$.

  Observe that if $\xi \in \mathcal{H}$ is such that $S_{1} \xi
  =\lambda \xi$ and $S_2 \xi = \beta \xi$, then
  \begin{align*}
    T \xi = (\lambda + i \beta) \xi
  \end{align*}

  Now let $E = \cup_{\lambda \in  \sigma(S_1)} E_\lambda$. Then $E$
  is an orthonormal basis for $\mathcal{H}$ consisting of eigenvectors of $T$.
\end{proof}

\begin{lemma}
  Let $V$ be a vector space and $f_1 , f_2 , \ldots , f_n: V \to
  \mathbb{C}$ linear. Then $ f \in \textrm{span} \{ f_1 , f_2 ,
  \ldots , f_n \}$ iff
  \begin{align*}
    \bigcap_{k = 1}^{n}\textrm{Ker}(f_k) \subset \textrm{Ker}(f)
  \end{align*}
\end{lemma}
\begin{proof}
  If $f \in \textrm{span}\{ f_1 , f_2 , \ldots , f_n \}$ and if $x
  \in \textrm{Ker}(f_i)$ for each $1 \le i \le n$, then clearly $f(x) = 0$.
  Conversely \textcolor{red}{verify}.
\end{proof}

\begin{lemma}
  \begin{align*}
    (X, \textrm{weak})^* = X^*
  \end{align*}
\end{lemma}


% TeX_root = ../main.tex

\chapter{}

\begin{remark}
  Consider the counting measure $\mu$, on $\mathbb{N}$. Find a
  sequence of functions $f_n : \mathbb{N} \to [0, \infty)$, such that
  $\|f_n\|_1 \to 0$ and $g = \sup_{n} f_n \notin L^1(\mu)$.
\end{remark}

\begin{lemma}
  Let $(f_n) \in L^p( \mu)$ be a Cauchy sequence in $1 \le p \le
  \infty$. Then there exists a subsequence $(f_{n_j})$ which is convergent
  pointwise almost everywhere.
\end{lemma}
\begin{proof}
  First suppose, $p < \infty$. Starting from a Cauchy sequence,
  choose a subsequence $n_1 < n_2 < \ldots$ such that for each $ k
  \in \mathbb{N}$ \[
    \|f_{n_k} - f_{n_{k+1}}\| < \frac{1}{2^k}
  \]
  Let \[
    g_l = \sum_{k = 1}^{l} |f_{n_{k+1}} - f_{n_k}| \quad g = \sum_{k
    = 1}^{\infty} |f_{n_{k+1}} - f_{n_k}|
  \]
  Then $g_n^p \le g_{n+1}^p \le \ldots$ and $g_n^p \to g^p$. Then by
  monotone convergence theorem, \[
    \int g_n^p \ d \mu \to \int g^p \ d \mu
  \]
  Moreover, using Minkowski's inequality, we get
  \begin{align*}
    \|g_l\|_p &\le \sum_{k = 1}^{l} \|f_{n_{k+1}} - f_{n_k}\| \\
    &\le \sum_{k = 1}^{\infty}  \|f_{n_{k+1}} - f_{n_k}\| \\
    &\le 1
  \end{align*}
  By monotone convergence, we get $\|g\|_p \le 1$. In particular $g$
  is finite almost everywhere. Hence \[
    f = \sum_{k = 1}^{\infty} (f_{n_{k+1}} - f_{n_k})
  \]
  is absolutely convergent almost everywhere. So by telescoping
  series for almost every $x \in X$
  \begin{align*}
    f(x) &= \lim_{l \to \infty} \sum_{k = 1}^{l} (f_{n_{k+1}} - f_{  n_k})(x) \\
    &= \lim_{ l \to \infty} (f_{n_{l+1}}(x) - f_{n_1}(x))
  \end{align*}
  So $f_{n_l}$ converges for almost every $x \in X$.

  Next, we consider $p = \infty$.  For $n, k \in \mathbb{N}$, let
  \begin{align*}
    E_{n, k} = \{ x \in X  \ : \  |f_n(x) - f_k(x)|> \|f_n - f_k\|_\infty \}
  \end{align*}
  Then $\mu(E_{n, k}) = 0$, by the definition of essential supremum.
  Moreover $E = \cup_{n, k = 1}^{\infty} E_{n, k}$ also has measure $0$.
  On $E^c$, for each $k , n \in \mathbb{N}$, we have \[
    |f_n(x) - f_k(x)| \le \| f_n - f_k\|
  \]
  This means $f_n|_E^c$ converges uniformly.
\end{proof}

\begin{theorem}
  For $1 \le p \le \infty$, $L^p(\mu)$ is a complete metric space.
  (After identifying functions that are equal almost everywhere.)
\end{theorem}
\begin{proof}
  \begin{enumerate}[label=(\arabic*)]
    \item For $p = \infty$, the proof in the above lemma is the proof
    \item For the rest of  the $p$, consider the Cauchy sequence
      $f_n$ in $L^p(\mu)$, $p < \infty$. It has c pointwise almost
      everywhere converging subseqence converging to $f$. We need to
      show that $ f \in L^p(\mu)$ and convergence is in norm. That is
      $  \|f_n - f\|_p \to 0$.

      We apply Fatou's lemma to the function $g_k = |f_n - f_{n_k}|^p$ to get
      \begin{align*}
        \lim_{k \to \infty} \inf \int |f_n - f_{n_k}|^p \ d \mu &\ge
        \int \lim_{k \to \infty} \inf |f_n - f_{n_k}|^p \ d \mu \\
        & =  \|f_n - f\|^p
      \end{align*}

      Given $\epsilon > 0$, since $f_n$ is Cauchy in $ L^p(\mu)$,
      there is a $ N$ such that for $n, m \ge N$, we have \[
        \epsilon^p > \|f_n - f_m\|^p_p = \int |f_n - f_m|^p \ d \mu
      \]
      By taking $m = n_k \to \infty$, we then get  \[
        \epsilon^p \ge \|f_n - f\|_p^p
      \]
      This implies $f \in L^p(\mu)$, by \[
        \|f\|_p \le \|f - f_n\|_p  + \|f_n\|_p
      \]
      Now that fact that $\|f - f_n\|_p \to 0$, we get $f \in L^1(\mu)$.
  \end{enumerate}
\end{proof}



% TeX_root = ../main.tex

\marginnote{\scriptsize 21/11/2024 }

\begin{definition}
  A Banach algebra $\mathcal{A}$ is a Banach space with a
  multiplicatit a ring with addition satisfying $\|ab\| \le
  \|a\|\|b\|$ for all $ a, b \in \mathcal{A}$. We say $\mathcal{A}$
  is unital if the ring above is unital with multiplicative identity
  $1_{\mathcal{A}}$. Units (invertible elements)
  may also exist similarly.
\end{definition}

\begin{definition}
  Given a Banach algebra $\mathcal{A}$, and $a \in \mathcal{A}$, the
  spectrum of $a$, denoted by
  \begin{align*}
    \sigma(a) = \{ \lambda \in \mathbb{C}  \ : \  \lambda
    1_{\mathcal{A}} - a \textrm{ is not invertible in } \mathcal{A} \}
  \end{align*}
\end{definition}

\begin{example}
  Let $(\alpha_n) \in \textbf{c}_0$ and $T: \ell^{2} \to \ell^{2}$
  such that $T((x_n)) = (\alpha_n x_n)$. We claim that
  \begin{align*}
    \sigma(T) = \{ \alpha_n  \ : \  n \in \mathbb{N} \} \cup \{ 0 \}
  \end{align*}
\end{example}
\begin{proof}
  Since $T$ is a compact operator (see Homework-5), we must have $0
  \in \sigma(T)$. Otherwise if $T$ is invertible, we'll get $I  = T
  \circ T^{-1}$ be also compact, which is a contradiction since
  $\ell^{2}$ is infinite dimensional. Moreover since $(T - \alpha_n I
  )(e_n) = 0$, we have $ \alpha_n \in \sigma(T)$.

  Now assume that $\beta \notin \{ \alpha_n  \ : \  n \in \mathbb{N}
  \} \cup \{ 0 \}$. Then let $S \in B(\ell^{2})$ defined by
  \begin{align*}
    S(e_n) = \frac{1}{\alpha-\beta} e_n
  \end{align*}
  Then \textcolor{red}{show that indeed $S \in B(\ell^{2})$} and $(T
  - \beta I)S = S(T - \beta I) = I$
\end{proof}

\begin{example}
  Let $T \in B(L^{2}([0, 1]))$ such that $T(f)(x) = xf(x)$ for all $
  f \in L^{2}([0, 1])$. Then $  \sigma(T) = [0, 1]$.
\end{example}
\begin{proof}
  First let us see that $0 \in \sigma(T)$. Suppose $  S = T^{-1}$
  exists. Then $\forall n \in \mathbb{N}$ if $ f = S \chi_{[0, 1]}$,
  then $Tf = \chi_{[0, 1]}$. So $x f(x) =
  \chi_{[0, 1]}$. But this is absurd, since $\frac{1}{x} \chi_{[0,
  1]} \notin L^2([0, 1])$.
\end{proof}

\begin{lemma}
  Let $S \in B(\mathcal{H})$ such that $\|S - T\| \le 1$. Then $S$ is
  invertible.
\end{lemma}
\begin{proof}
  Since $\|S  -I\| \le 1$,
  \begin{align*}
    \sum_{n \in \mathbb{N}} \|(S-I)^n\| \le \sum_{n \in \mathbb{N}} \|S - I\|^n
  \end{align*}
  Moreover since $B(\mathcal{H})$ is a Banach space, absolutely
  convergent sequences converge and this gives that
  $\sum_{n \in \mathbb{N}} (I - S)^n$ converges. Thus
  \begin{align*}
    R = \sum_{n=1}^\infty (I - S)^n
  \end{align*}
  exists. Now for each $N \in \mathbb{N}$, we have
  \begin{align*}
    S \Big(\sum_{n = 0}^{N} (I - S)^n\Big) = (I - (I -
    S))\Big(\sum_{n = 1}^{N} (I - S)^n\Big) &= \sum_{n = 0}^{n} (I -
    S)^n - \sum_{n = 0}^{N} (I-S)^{n+1}\\
    &= I - (I-S)^{N+1}
  \end{align*}
  which converges to $0$ as $n \to \infty$.
\end{proof}

\begin{corollary}
  The set of invertible operators is open in $B(\mathcal{H})$.
\end{corollary}
\begin{proof}
  Let $S$ be invertible. Then
  \begin{align*}
    \|T - S\| = \|S(S^{-1}T - I)\| \le \| S\| \|S^{-1}T - I\|
  \end{align*}
\end{proof}

\begin{theorem}
  For any $T \in B(\mathcal{H})$, $\sigma(T)$ is a non-empty, compact
  subspace of $\mathbb{C}$.
\end{theorem}
\begin{proof}
  Observe that the function $f: \lambda \to T - \lambda I$ is
  continuous. Then $f^{-1}(G(\mathcal{A}))$ is open, where
  $G(\mathcal{A})$ is the collection of all invertible elements of
  $B(\mathcal{H})$. But $\sigma(T)^{c} = f^{-1}(G(\mathcal{A}))$.
  So we see $\sigma(T)$ is closed. Moreover let $\lambda \in
  \mathbb{C}$ with $\|T\| < |\lambda|$. Then,
  \begin{align*}
    -T + \lambda I = \lambda(\frac{-T}{\lambda} + I)
  \end{align*}
  Then $\|S - I\| = \| \frac{T}{\lambda}\| = \frac{\|T\|}{|\lambda|} < 1$
  Hence show that $\sigma(T)$ is bounded.
\end{proof}

% TeX_root = ../main.tex

\chapter{Inner Product Spaces}

\begin{definition}
  Let $\mathcal{H}$ be a vector space over $\mathbb{C}$. A
  sesquilinear form is a function
  \begin{align*}
    \langle \cdot , \cdot \rangle : \mathcal{H} \times \mathcal{H}
    \to \mathbb{C}
  \end{align*}
  satisfying
  \begin{itemize}[]
    \item $\langle  x , y \rangle  = \overline{\langle y , x \rangle }$
    \item $\langle  x + \alpha z , y \rangle  = \langle x , y \rangle
      + \alpha \langle x , z \rangle $
  \end{itemize}
  for all $x, y, z \in \mathcal{H}, \alpha \in \mathbb{C}$.
  It is said to be positive semidefinite (positive definite) if $
  \langle x , x \rangle
  \ge 0$ ($\langle  x , x \rangle > 0$ for all $x \in
  \mathcal{H}\setminus \{0\}$) for all $x \in \mathcal{H}$.

  A positive definite sesquilinear forms makes $\mathcal{H}$ an inner
  product space.
\end{definition}

\begin{example}
  Take $L^2(\mu)$ (functions identified almost everywhere) with the
  natural inner product is an inner product.
\end{example}

\begin{proposition}
  If $\mathcal{H}$ is a complex vector space with a positive
  semidefinite sesquilinear form and $\langle  x , x \rangle = 0$,
  then $\langle  x , y \rangle  = 0$ for all $y \in \mathcal{H}$.
\end{proposition}
\begin{proof}
  Take $\alpha \in \mathbb{C}$ and consider
  \begin{align*}
    \langle x + \alpha y , x  +\alpha y \rangle  & = \langle x , x
    \rangle  + \alpha \langle y , x \rangle  + \overline{\alpha}
    \langle y , y \rangle  \\
    &= 2 \Re \big(  \overline{\alpha} \langle x , y \rangle  \big) +
    |\alpha|^2 \langle y , y \rangle
  \end{align*}
  Now if $\langle  x , y \rangle \neq 0$, then either $\langle  y , y
  \rangle  = 0$ or nonzero. If $\langle  y , y \rangle = 0$, take
  $\alpha = - \langle x , y \rangle $ to get
  \begin{align*}
    \langle x +\alpha y , x + \alpha y \rangle = \underbrace{2\Re \big( -
    \overline{\langle x , y \rangle }\langle x , y \rangle \big)}_{< 0}
  \end{align*}
  which is a contradiction.

  Now if $\langle  y , y \rangle  \neq 0$, take $\alpha = i \langle
  x , y \rangle $ to get a similiar contradiction, which makes $\Re(
  \overline{\alpha} \langle x , y \rangle) = 0$
\end{proof}

\begin{definition}
  If $\langle  \cdot , \cdot \rangle $ is a positive semidefinite
  sesquilinear form, then
  \begin{align*}
    \|x\| =  \langle x , x \rangle^{\frac{1}{2}}
  \end{align*}
  is a seminorm.

  If $\langle \cdot , \cdot \rangle $ is positive definite, then $x
  \to \|x\|$ is a norm.
\end{definition}

\begin{theorem}[Cauchy-Schwarz]
  If $\langle \cdot , \cdot \rangle $ is a positive semidefinite
  sesquilinear form on $\mathcal{H}$, then for $x, y \in \mathcal{H}$
  \begin{align*}
    |\langle x , y \rangle | \le \|x\| \|y\|
  \end{align*}
\end{theorem}
\begin{proof}
  If $\|y\| = 0$, then previous proposition takes care of the proof.
  If not, choose $ \alpha =  \frac{\langle x , y \rangle }{ \langle y
  , y \rangle } $ and consider
  \begin{align*}
    0 &\le \langle x - \lambda y , x- \lambda y \rangle \\
  & = \|x\|^2 - 2 \Re \big( \lambda \langle y , x \rangle \big)) +
  |\lambda|^2 \langle y , y \rangle \\
  &= \|x\| - 2 \frac{\langle x , y \rangle^2}{\|y\|^2} +
  \frac{\langle x , y \rangle^2}{\|y\|^2} \\
  &= \|x\|^2 - \frac{|\langle x , y \rangle |^2}{\|y\|^2}
\end{align*}
which gives our inequality.
\end{proof}



% TeX_root = ../main.tex

\marginnote{\scriptsize 22/04/2025 }

\section{Tensor Products of C* Algebras}
\begin{definition}
  Recall that a C* algebra is a Banach space with a continuous
  multiplication turning it into an algebra such that $\|ab\| \le
  \|a\|\|b\|$ and $\|a^*\| = \|a\|$, and an isometric involution $*$
  such that $\|a^*a\| = \|a\|^2$.
\end{definition}

Given two algebras $\mathcal{A, B}$, the algebraic tensor product
admits a canonical algebra structure
\begin{align*}
  (a \otimes b)(a^\prime \otimes b^\prime) = aa^\prime \otimes bb^\prime
\end{align*}
and same holds if it is a $*$-algebra with
\begin{align*}
  (a \otimes b )^* = a^* \otimes b^*
\end{align*}

\begin{definition}
  Let $\mathcal{A, B}$ be two unital C* algebras. By a C* norm on the
  algebraic tensor product, we mean a norm satisfying
  \begin{enumerate}[label=(\arabic*)]
    \item $\|a \otimes b\| \le \|a\|\|b\|$
    \item $\|z^* z\| \le \|z\|^2, \ \forall z \in A \otimes B$
  \end{enumerate}
\end{definition}

\begin{example}
  Consider $C(X) \otimes C(Y)$, where $X, Y$ are compact Hausdorff
  spaces. Then $C(X) \otimes C(Y)$ sits inside $C(X \times Y)$
  densely (Stone-Weierstrass) via
  $f_i \otimes g_i \to ((x, y) \to f_i(x) g_i(x))$
\end{example}

\begin{example}
  Similarly $B(H) \otimes B(K) \to B(H \otimes_2 K)$.
\end{example}

\begin{lemma}
  Every $*$-homomorphisms between C* algebra is contractive
\end{lemma}
\begin{proof}
  Let $\phi: A \to B$ be the $*$-homomorphism. Then for all $ a \in
  \mathcal{A}$,
  \begin{align*}
    \|a\|^2 = \|a^*a\| = r(a^*a) \ge r(\phi(a^*a)) = r(\phi(a)^*
    \phi(a)) = \| \phi(a)^*\phi(a)\| = \|\phi(a)\|^2
  \end{align*}
\end{proof}

\begin{theorem}
  Let $(\mathcal{A}, \|\cdot\|)$ be a C* algebra. Suppose $
  \|\cdot\|^\prime$ is a pre-C* norm on $ \mathcal{A}$. Then $\|\cdot\|
  = \|\cdot\|^\prime$.
\end{theorem}

\begin{example}
  Let $\mathcal{A}$ be a unital C* algebra, and $n \in \mathbb{N}$.
  Then $\exists!$ C* norm on $M_n(\mathbb{C}) \otimes \mathcal{A}$
\end{example}

\begin{definition}
  Let $\mathcal{A}, \mathcal{B}$ be unital C* algebra. Define
  $\forall z \in A \otimes B$,
  \begin{align*}
    \|z\|_{\textrm{max}} &= \sup\{ \|z\| \ : \ \|\cdot\| \textrm{ is a
    C* norm on } A \otimes B\} \\
    &= \sup \{ \|\pi(z)\| \ : \ \pi: A \otimes B \to B(H) \textrm{ is
    a representation.}  \}
  \end{align*}
\end{definition}
note that this norm we defined will have its upper bound to be the
projective tensor norm of $A, B$ viewed as Banach spaces.

% TeX_root = ../main.tex

\marginnote{\scriptsize 24/04/2025 }

\begin{theorem}
  Let $\mathcal{A, B}$ be unital C* algebras and $\mathcal{H,
  H^\prime, K, K^\prime}$ be Hilbert spaces, and $\pi: \mathcal{A}
  \to B(\mathcal{H}), \pi^\prime: \mathcal{A} \to
  B(\mathcal{H}^\prime)$, $\sigma: \mathcal{B} \to B(\mathcal{K}),
  \sigma^\prime: \mathcal{B}\to B(\mathcal{K}^\prime)$ be injective
  $*$-homomorphisms. Then $\forall z \in A \otimes B$,
  \begin{align*}
    \|(\pi \otimes \sigma)(z)\|_{B(H \otimes K)} = \|(\pi^\prime
    \otimes \sigma^\prime)(z)\|_{B(H^\prime \otimes K^\prime)}
  \end{align*}
\end{theorem}
\begin{proof}
  Let $P_n$ be sequence of finite rank projections that converge to
  $I$ in the strong operator topology. Because the result holds in
  finite dimensions, $$\|(I \otimes P_n)((\pi
  \otimes \sigma)(z))(I \otimes P_n)\|_{B(\mathcal{H} \otimes
  \mathcal{K})} = \|(I \otimes P_n)((\pi^\prime
  \otimes \sigma)(z))(I \otimes P_n)\|_{B(\mathcal{H}^\prime
  \otimes \mathcal{K})}$$
  Now taking limits as $ n \to \infty$, and repeating the same steps
  3 more times, we get our result.
\end{proof}

\begin{definition}
  The above unique norm is called the minimal tensor product of
  $\mathcal{A}$ and $\mathcal{B}$, denoted $\otimes_{\textrm{min}}$.
\end{definition}

\begin{theorem}
  $\otimes_{\textrm{min}}$ is the smallest tensor norm on
  $\mathcal{A} \otimes \mathcal{B}$.
\end{theorem}
\begin{proof}
  Assuming $\|\phi \otimes \psi\|_{(A \otimes_{\textrm{min}} B)} = \|
  \phi\| \|\psi\|$ for all $\phi \in \mathcal{A}^*$ such that
  $\phi(a^*a) > 0$ for all $a \in \mathcal{A}\setminus \{ 0 \}$.
  Similarly for $\mathcal{B}$.

  Given $a_1, a_2 \in \mathcal{A}$, define $  \langle a_1 , a_2
  \rangle  = \phi(a_2^*a_1)$. This defines an inner product on
  $\mathcal{A}$. Denote by $\mathcal{H}_{\phi}$, the Hilbert space
  completion. Denote by $\Lambda: \mathcal{A} \to \mathcal{H}_\phi$.
  And do GNS constructions.
\end{proof}

Let $\mathcal{A}_1 \leqslant \mathcal{A}_2 \leqslant \ldots \leqslant
\mathcal{A}$ be an increasing sequence of C* algebras where
$\mathcal{A}_n \cong M_n(\mathbb{C})$, and $  \overline{\cup_{n \in
\mathbb{N}}\mathcal{A}_n} = \mathcal{A}$. This implies for all C* algebra $B$,
\begin{align*}
  A \otimes_{\max} = \mathcal{A} \otimes_{\min} \mathcal{B}
\end{align*}

Discussion about the max and min tensor product of
$C_\lambda(\mathbb{F}_2)$ with $C_\rho(\mathbb{F}_2)$ are not the same.

% TeX_root = ../main.tex

\marginnote{\scriptsize 29/04/2025 }

\begin{theorem}
  Let $\mathcal{A, B}$ be unital C* algebras and let $\phi:
  \mathcal{A} \to \mathcal{B}$ be a ucp map. Define
  \begin{align*}
    \mathcal{A}_\phi := \{ a \in \mathcal{A} \ : \ \phi(a^*a) =
    \phi(a)^*\phi(a), \phi(aa^*) = \phi(a) \phi(a)^* \}
  \end{align*}
  Then $A_\phi$ is a C* subalgebra of $\mathcal{A}$, and $  \phi(xa)
  = \phi(x) \phi(a)$ and $\phi(ax) = \phi(a) \phi(x)$ for all $a \in
  \mathcal{A}_\phi, x \in \mathcal{A}$.
\end{theorem}


\printbibliography[heading=bibintoc]

\end{document}

