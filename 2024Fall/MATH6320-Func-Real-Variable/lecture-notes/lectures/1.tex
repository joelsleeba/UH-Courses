% TeX_root = ../main.tex

\chapter{}

\section{Course Info}
Bernhard Bodmann \\
bgb@central.uh.edu \\
PGH 641A \\
Tue 10-11AM, Wed 1-2PM \\

Email for organizational stuff and meet for a course related conceptual stuff

\begin{itemize}
  \item Canvas
  \item MS Teams
\end{itemize}

\textbf{Textbook :} Walter Rudin, Real \& Complex Analysis, Chapters 1-9

Midterm test, October 10, in class

Grading: 30\% HW, 30\% Midterm, 40\% Final

\section{Notations and Basic Definitions}

\begin{definition}
  Let $X$ be s set and $P(X)$ be its power set. A subset $\tau \subset P(X)$ is called a topology on $X$ provided
  \begin{itemize}
    \item $\emptyset, X \in \tau$
    \item If $E_1, E_2, \ldots E_n \in \tau$, then $\cap_{j = 1}^{n}E_j \in \tau$
    \item If $J$ is any index set and for each $j  \in  J$, $E_j \in \tau$ then $ \cup_{j \in J} E_j \in \tau$
  \end{itemize}
\end{definition}

\begin{example}
  Given a set $X$, $\{ \emptyset, X \}$ is a topology known as in-discrete topology.
\end{example}

\begin{definition}
  Let $(X, d)$ be a metric space with $d: X \times X \to \mathbb{R}^+$ satisfying positive definiteness, symmetry, and triangle inequality.
\end{definition}

\begin{definition}
  We say $E \subset X$ is open if for each $x \in E$, there is an $\epsilon \ge 0$ such that $\{ y \in X \ : d(x, y) \le \epsilon \} \subset E$
\end{definition}

\begin{example}
  Let $\tau$ be the set of all open subsets of $X$, where $(X, d)$ is a metric space, then $\tau$ forms a topology. \textcolor{red}{verify this}
\end{example}

\begin{definition}
  Let $X$ be a set and $\tau$ a topology on $X$, then we call $(X, \tau)$ a topological space. Elements of $\tau$ are called open sets.
\end{definition}

\begin{definition}
  Let $X$ be a set, $\beta \subset P(X)$ such that
  \begin{itemize}
    \item $\forall x \in X, \exists B \in \beta$ such that $x \in B$
    \item If $ x \in X, B_1, B_2 \in \beta$ and if $x \in B_1 \cap B_2$, then there is $B_3 \in \beta$ such that $x \in B_3 \subset B_1 \cap B_2$
  \end{itemize}
  Then $ \beta$ is called a basis
\end{definition}

\begin{theorem}
  If $\beta$ is a basis then, $  \tau$, the collection of all (empty or non-empty) unions of elements of $  \beta$ form a topology on $X$.
\end{theorem}
\begin{proof}
  It is clear from the definition of $\tau$ that arbitrary unions of sets in $\tau$ is again in $\tau$. Also the first property guarantees that $X \in \tau$. Since empty unions are also considered, $\emptyset \in \tau$. Hence all that remains is to show that finite intersections of sets in $\tau$ is again in $\tau$.

  Let $U_1, U_2 \in \tau$, once we show that $U_1 \cap U_2 \in \tau$, we can use induction to show $\cap_{i = 1}^{n}U_i \in \tau$ when $U_1, U_2, \ldots , U_n \in \tau$. Let $x \in U_1 \cap U_2$. Since $U_1, U_2$ are unions of elements from $\beta$, there exists $B_1, B_2 \in \beta$ such that $x \in B_1 \subset U_{1}$ and $ x \in  B_2 \subset U_2$. Then by the second property of the basis, there exists $B_x \in \beta$ with $x \in B_x \subset B_1 \cap B_2 \subset U_1 \cap U_2$. Since $x \in U_1 \cap U_2$ was arbitrary, we get \[
       U_1 \cap U_2 = \bigcap_{x \in U_1 \cap U_2} B_x
  \]
  Thus $U_1 \cap U_2 \in \tau$ and hence $\tau$ is a topology.
\end{proof}

\begin{example}
  Let $\beta = \{ (p, q) \ : p, q \in \mathbb{ Q}, p < q \} \subset P(\mathbb{R})$. Then $\beta$ is a basis and the topology generated by $\beta$ is the usual euclidean topology on $\mathbb{R}$ obtained from the metric $d(x, y) = |x -y|$.
\end{example}

\begin{example}
  Let $X = [-\infty, \infty]$ and $\beta =  \{ (a, b)  \ : \   a, b \in \mathbb{R}, a< b \} \cup \{ [-\infty, b) \ : \ b \in \mathbb{ R}  \} \cup \{ (a, \infty] \ : \ a \in \mathbb{R} \}$
   Then $\beta$ is a basis.
\end{example}

\begin{example}
  Let $J$ be a set and $\mathbb{R}^J = \{ f : J \to \mathbb{R} \}$. Let $\beta$ contain all the sets of the form $\{ f: J \to \mathbb{R} \ : \ f(j_1) \in U_1, f(j_2) \in U_2 , \ldots, f(j_n) \in U_n \}$ where $ n \in \mathbb{N}, j_1, j_2, \ldots, j_n \in J$ and $ U_1, U_2, \ldots U_n$ are open sets in $ \mathbb{R}$.

  Then $\beta$ is a basis and the topology generated by $\beta$ is called the product topology in $\mathbb{R}^J$

  If $J$ is uncountable, then this topology $ \mathbb{R}^J$ is not metrizable. \textcolor{red}{verify}.
\end{example}

\begin{definition}
  Let $X$ be a set $\mathscr{M} \subset P(X)$ is a $\sigma$-algebra, if 
  \begin{itemize}
    \item $X \in \mathscr{M}$
    \item If $A \in \mathscr{M}$, then $ A^c \in \mathscr{M}$
    \item If $A_1, A_2, \ldots, A_j, \ldots \in \mathscr{M}$, then $\cup_{j = 1}^{\infty} A_j \in \mathscr{M}$
  \end{itemize}
  Then we call $(X, \mathscr{M})$ a measurable space, and $\mathscr{M}$ contains measurable sets.
\end{definition}

\begin{theorem}
  \label{thm:sigma_algeba_generated}
  Let $X$ be a set, and $F \subset P(X)$, then there exists a unique $ \sigma$-algebra $\mathscr{M}$ such that,
  \begin{itemize}
    \item $F \subset \mathscr{M}$
    \item If $\mathscr{N}$ is a $\sigma$-algebra on $X$, and $F \subset \mathscr{N}$, then $\mathscr{M} \subset \mathscr{N}$
  \end{itemize}
  Then $  \mathscr{M}$ is called a $\sigma$-algebra generated by $F$
\end{theorem}












