% TeX_root = ../main.tex

\chapter{}
\begin{theorem}[Holder's \& Minkowski Inequality]
  Let $(X, \mathcal{M},  \mu)$ be a measure space, $f, g : X \to [0,
  \infty]$ be measurable. Then for $1 \le p < \infty$, with $1/p + 1/q
  = 1$, then \[
    \int  fg \ d \mu \le \Bigg(\int  f^p \ d \mu \Bigg)^{\frac{1}{p}}
    \Bigg(\int  g^q \ d \mu\Bigg)^{\frac{1}{q}} = \| f\|_p \|g\|_q
  \]
  and \[
    \Bigg(\int(f+g)^p \ d \mu\Bigg)^{\frac{1}{p}} \le \|f\|_p + \|g\|_p
  \]
\end{theorem}
\begin{proof}
  Let $A = \|f\|_p, B=\|g\|_p$. If $A = 0$ or $A = \infty$, or $ B =
  0$, or $ B = \infty$, we have nothing to show. Hence assume that $0
  < A, B < \infty$. Let $F(x) = \frac{f(x)}{A}, G(x) =
  \frac{g(x)}{B}$. We also define $ s, t : X \to \mathbb{R}$ as \[
    F(x) = e^{\frac{s(x)}{p}}, \quad G(x) = e^{\frac{t(x)}{q}}
  \]
  By convexity of the exponential function, we have $$e^{s/p + t/q}
  \le \frac{1}{p}e^s + \frac{1}{q}e^t$$
  In terms of $F, G$, this is \[
    F(x) G(x) \le \frac{1}{p}(F(x))^p + \frac{1}{q}(G(x))^p
  \]
  Hence integrating both sides, we get \[
    \int F(x) G(x) \ d \mu \  \le \ \frac{1}{p} \int (F(x))^p \ d \mu
    + \frac{1}{q} \int (G(x))^p \ d \mu
  \]

  Now writing this in terms of $f, g$ gives us
  \begin{align*}
    \frac{1}{AB} \int fg \ d \mu &\le \frac{1}{p} \frac{1}{A^p} \int
    f^p \ d \mu + \frac{1}{q} \frac{1}{B^q} \int g^q \ d \mu \\
    &= \frac{1}{p}\frac{1}{A^p}\|f\|_p^p + \frac{1}{q}\frac{1}{B^q} \|g\|_q^q \\
    &= 1/p + 1/q = 1
  \end{align*}
  Thus we get Holder inequality.

  For Minkowski, consider
  \begin{align*}
    (f+g)^p &= (f+g)(f+g)^{p-1} \\
    &= f(f+g)^{p-1} + g(f+g)^{p-1}
  \end{align*}
  Now integrating both sides and carefully applying Holder's inequality, we get
  \begin{align*}
    \int (f+g)^p \ d m &= \int f(f+g)^{p-1} \ d   \mu + \int
    g(f+g)^{p-1} \ d  \mu \\
    &= \Bigg(\int f^p \ d  \mu \Bigg)^p  \Bigg(\int
    (f+g)^{(p-1)q} \ d \mu \Bigg)^q + \Bigg(\int
    g^q  \ d \mu\Bigg)^{q} \Bigg(\int (f+g)^{(p-1)p} \ d \mu \Bigg)^{p} \\
    &=
  \end{align*}
  \textcolor{red}{verify}
\end{proof}

\begin{definition}
  Let $0< p< \infty$. $ f: X \to \mathbb{C}$ measurable on $(X,
  \mathcal{M}, \mu)$. We define \[
    \|f\|_p = \Big( \int |f|^p \ d \mu\Big)^p
  \]
  We also write $L^p(\mu) = \{ f : X \to \mathbb{C}  \ : \  \|f\|_p <
  \infty  \}$
\end{definition}

\begin{definition}
  Let $(X, \mathcal{M}, \mu)$ be a measure space. Let $f: X \to [0,
  \infty]$ be measurable. The essential supremeum of $f$ is \[
    \textrm{ess}\sup f = \inf \{ \alpha  \ : \  \mu(\{ f> \alpha \}) = 0  \}
  \]
\end{definition}

\begin{proposition}
  With $(X, \mathcal{M}, \mu), f$ be as above. $  \beta =
  \textrm{ess}\sup f$. Then \[
    \mu(\{ f > \beta \}) = 0
  \]
\end{proposition}

\begin{definition}
  For $(X, \mathcal{M}, \mu ), f$ as above,  \[
    \|f\|_\infty = \textrm{ess}\sup \|f\|
  \]
  and $L^\infty(\mu)$ be the set of all $f$ with $\|f\|_\infty < \infty$
\end{definition}

We add a case of Holder's inequality for $\|\cdot\|_\infty$.

\begin{theorem}
  If $(X, \mathcal{M}, \mu)$ is as usual $f, g$ measurable, $ f \in
  L^1(\mu), g \in L^\infty(\mu)$, then $ fg \in L^1(\mu)$ and \[
    \|fg\|_1 \le \|f\|_1 \|g\|_\infty
  \]
\end{theorem}
\begin{proof}
  Take $E = \{ x \in X  \ : \  |g(x)|> \|g\|_\infty \}$. Then $E$ has
  measure zero, and
  \begin{align*}
    \int |fg| \ d \mu &= \int_{X\setminus E}  |fg| \ d \mu + \int_E
    |fg| \ d \mu \\
    & \le \|g\|_\infty \int_{X\setminus E}  |f| \ d \mu \\
    & \le \|g\|_\infty \|f\|_1
  \end{align*}
\end{proof}

\begin{theorem}
  let $(X, \mathcal{M}, \mu)$ be as usual, $f, g$ measurable $ f, g
  \in L^\infty(\mu)$. Then \[
    \|f+g\|_\infty \le \|f\|_\infty + \|g\|_\infty
  \]
\end{theorem}
\begin{proof}
  Notice that
  \begin{align*}
    \{ x  :  |f(x) + g(x)| > \| f\|_\infty + \|g\|_\infty \} & \subset
    \{ x  :  |f(x)| + |g(x)| > \| f\|_\infty + \|g\|_\infty \} \\
    & \subset \{ x :  |f(x)| > \|f\|_\infty \} \ \cup \ \{ x
    : |g(x)|> \|g\|_\infty \}
  \end{align*}
  Since both the sets at the end is of measure zero. Hence we get the
  inequality.
\end{proof}

\begin{theorem}
  For each $1 \le p \le \infty$, $L^p(\mu)$ is a normed vector space
  over $\mathbb{C}$ provided we identify functions that are equal
  almost everywhere.
\end{theorem}
\begin{proof}
  Positive definiteness follows from the identification of functions
  in the space. Homogeneity follows from the definition of
  $\|\cdot\|_p$. And triangle inequality is the Minkowski inequality.
  We have shown that for the cases $1 \le p < \infty$, that
  $\|\cdot\|_p$ is a norm.
\end{proof}

\begin{lemma}
  Let $(f_n) \in L^p( \mu)$ be a Cauchy sequence in $1 \le p \le
  \infty$. Then there exists a subsequence $(f_{n_j})$ which is convergent
  pointwise almost everywhere.
\end{lemma}


