% TeX_root = ../main.tex

\chapter{}

\begin{remark}
  Suppose $ A_1 ,  A_2 , \ldots$. Consider their characteristic functions $\chi_{A_n}$ and let $\lim \sup_{k \ge n} = \chi_A$. What is $A$? \begin{align*}
    \lim \sup \chi_{A_n} &= \lim_{ n \to \infty} \sup_{k \ge N } \chi_{A_k} \\ 
    & = \lim_{n \to \infty} \chi_{\cup_{k \ge n } A_k}
  \end{align*}
\end{remark}

\begin{theorem}
  Let $(X,  \mathcal{M}, \mu)$ be a measurable space, $f, g : X \to [0, \infty]$ be measurable, then \[
    \int (f+g) \ d \mu = \int f \ d \mu + \int g \ d \mu
  \]
\end{theorem}
\begin{proof}
  For $s, t: X \to [0, \infty]$ simple and measurable, by definition \[
    \int (s+t) \ d \mu = \int s \ d \mu + \int t \ d \mu
  \]
   Considering sequences of simple measurable functions $(s_n)_{ n=1}^\infty, (t_n)_{n=1}^\infty$ such that $s_n(x) \nearrow f(x), t_n(x) \nearrow g(x)$ for each $x \in X$. Then by monotone convergence theorem \[
        \int s_n \ d \mu \to \int f \ d \mu  \quad \int t_n \ d \mu \to \int g \ d \mu
   \]
  and since $s_n(x)+t_n(x) \nearrow f(x) + g(x)$ for each $x \in X$ then again by MCT we get \[
    \int (s_n + t_n) \ d \mu \to \int (f+g) \ d \mu
  \]
\end{proof}
\begin{corollary}
  If $(f_n)_{ n=1}^\infty$ is a sequence of functions $f_n: X \to [0, \infty]$, then \[
        \int \sum_{i = 1}^{\infty}  f_n \ d \mu = \sum_{i = 1}^{\infty}  \int f \ d \mu
  \]
\end{corollary}
\begin{proof}
  Let $g_m = \sum_{n = 1}^{m} f_n$. Then $(g_m)$ forms an incrasing sequence, so \begin{align*}
    \int \sum_{n \in \mathbb{N}} f_n \ d \mu  &= \int \lim_{n \to \infty} g_m d \mu \\ 
    & = \lim_{m \to \infty} \int \sum_{i = 1}^{m} f_i \ d \mu \\ 
    \end{align*}
\end{proof}

\begin{theorem}
  If $f: [0, \infty]$ is maeasurable on $(x, \mathcal{M}, \mu)$, then $\phi: \mathcal{M} \to [0, \infty]$, \[
    \phi(E) = \int_E f d \mu
  \]
   defines a measure $\phi$ and for any $g: X \to [0, \infty]$, and for any measurable $g: X \to [0, \infty]$ \[
         \int g \ d \phi = \int g f \ d \mu
   \]
\end{theorem}
\begin{proof}
  $\phi(\emptyset) = 0$ since the integral of every simple measurable function $s \le f$ over $\emptyset$ is 0. 

  Let $(E_n)_{ n =1}^\infty$ be a disjoint sequne of sets $E = \cup_{j = 1}^{\infty}E_j$, then \[
    \phi(E) = \int f  \ d \mu = \int f \chi_{X_E} \ dx = \int f \chi_{\cup_{n = 1}^{\infty}E_n} \ d \mu = \int f (\sum_{n \in \mathbb{N}} \chi_{E_n}) \ d \mu =\sum_{n \in \mathbb{N}} \int_{E_n} f \ d \mu
  \]
  which is exactly $\sum_{n \in \mathbb{N}} \phi(E_n)$. This gives that $\phi$ is a measure.

  To see the claimed identity, we first show that \[
       \int s \ d \phi = \int sf \ d \mu
  \]
  for $s: X \to [0, \infty)$ simple measurable, with \[
    s(x) = \sum_{j = 1}^{n} \alpha_j \chi_{A_j}(x)
  \]
  Then we see that \begin{align*}
    \int s \ d \mu &= \sum_{j = 1}^{n} \alpha_i \phi(A_j) \\ 
    &= \sum_{j = 1}^{n} \alpha_j \int_{A_j} f \ d \mu \\ 
    &= \int \Big( \sum_{j = 1}^{n} \alpha_j \chi_{A_j}\Big) f \ d \mu \\ 
    & = \int sf \ d \mu
  \end{align*}

  Now for any given $g: X \to [0, \infty]$, we approximate $g$ with a simple measurable seqeunce $s_n \nearrow g$. Then by monotone functoins, we get \begin{align*}
    \int g \ d \phi & = \lim_{n \to \infty} \int s_n \ d \phi \\ 
    &= \lim_{n \to \infty} \int s_n f \ d \mu \\ 
    &= \int \lim_{n \to \infty} s_n f \ d \mu \\ 
    &=  \int \phi \ d \mu
  \end{align*}
\end{proof}

\begin{definition}
  We define the space $L^1(\mu)$ of integrable functions on a measurable functions $(X, \mathcal{M},  \mu)$ to consist of all measurable $f: X \to \mathbb{C}$ such that \[
       \int |f| \ d \mu \le \infty
  \]
\end{definition}
\begin{remark}
  If $f$ is measurable, $\mathbb{C}$ valued, such that $f = u + iv$ where $u, v$ are real valued measurable functions. Then let $u^+ = \max \{ 0, u \}, u^- = \max \{ 0, -u \}$. Then $u^+, u^-$ are measurable functions. Similarly, we get $v^+, v^-$ also to be measurable functions. Then we get $f = u^+ - u^- + i(v^+ - v^-)$ and we define the integral as \[
        \int  f \ d \mu = \int u^+ \ d \mu - \int  u^- \ d \mu + i \int  v^+ \ d \mu - i \int  v^- \ d \mu
  \]
\end{remark}
