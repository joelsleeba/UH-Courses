% TeX_root = ../main.tex

\chapter{}

\begin{theorem}[Luzin's theorem]
  Let $X$ be a locally compact Hausdorff space.
  \begin{enumerate}[label=(\arabic*)]
    \item $\mu$ is a regular measure on a $\sigma$-algebra
      $\mathcal{M}$ containing $B(X)$
    \item $f: X \to \mathbb{C}$ is measurable
    \item there is a $A \in \mathcal{M}$ such that $\mu(A) < \infty$
      and $f = 0$ on $A^c$
  \end{enumerate}
  Given $\epsilon> 0$ there is a $g \in C_c(X)$ such that $
  \mu(\{ x \in X  \ : \  f(x) \neq g(x) \}) < \epsilon$ and $\sup \{
  |g(x)|  \ : \ x \in X \} \le \sup \{ |f(x)| \ : \  x \in X \}$.
\end{theorem}
\begin{proof}
  Suppose for now $A$ is compact. (We can assume this since the
    measure is regular and we can find a compact set $K \subset A$ such
  that $f = 0$ almost everywhere in $K^c$.) We'll do the $A$ not
  compact case later.

  Choose $V$ open such that $ A \subset V$ and $\overline{V}$ is
  compact. We'll first prove the existence of the desired $g$ if $f$
  is simple. Let \[
    f = \sum_{j = 1}^{n} \alpha_j \chi_{A_j}
  \]
  where each $A_j$ is disjoint and $\cup_{j = 1}^{n}A_j = A$. Again
  each of the $\mu(A_j) \le \mu(A) < \infty$. Hence by the regularity
  of the measure there are compact sets $K_j \subset A_j$ such that $
  \mu(A_j \setminus K_j) < \frac{\epsilon}{2^{j+1}}$.

  Since $K_j$ are compact and disjoint, we can find collection of
  disjoint open sets $V_j$ such that $K_j \subset V_j$.
  \textcolor{red}{verify this, I am not sure}.

  Moreover by replacing $V_j$ with $V_j \cap V$, we can assume $V_j \subset V$.
  Now by the outer regularity of the measure, we can assume $\mu(V_j
  \setminus K_j) < \frac{\epsilon}{2^{j+1}}$. Now by Urysohn, there is a
  $g_i \in C_c(X)$ such that $\chi_{K_j} \le g_j \le \chi_{V_j}$. Let   \[
    g = \sum_{j = 1}^{n} \alpha_j g_j
  \]
  Then $g$ is continuous being the finite sum of continuous function.
  Moreover since $\cup_{j = 1}^{n}V_j \subset V$, we get
  $\textrm{supp}(g) \subset \overline{V}$. Also
  \begin{align*}
    |g(x)| & \le \max \{ |\alpha_j| \} \\
    & \max_{x \in A} | f(x)|
  \end{align*}

  Now we see that $f(x) = g(x)$ for all $x \in K_j$ and $x \in
  (A_j \cup V_j)^c$. Since $K_j \subset V_j$, the set where they possibly
  \marginnote{\scriptsize Add a diagram for ease of reasoning}
  disagree is $$D = \bigcup_{j = 1}^{n}(V_j \setminus K_j) \quad \cup
  \quad \bigcup_{j = 1}^{n}(A_j \setminus K_j)$$
  Now by the subadditivity of $\mu$, we get $\mu(D) < \epsilon$ and
  we have proved the result for $A$ compact and $f$ simple.

  Now for the case when $0\le f < 1$, let $s_n$ be the
  sequence of simple functions $0 \le s_1 \le s_2 \le \ldots \le $
  with $\lim_{n \to \infty} s_n(x) = f(x)$. Let $t_n = s_n -
  s_{n-1}$, where $s_0 = 0$. Each $t_n$ is simple and $t_n = 0$ on
  $A^c$ and by construction, we get \[
    t_n \le \frac{1}{2^{n-1}} \chi_{B_n}
  \]
  for some  set $B_n$.

  Now we use the first part of the proof on $t_n$s to get a
  corresponding $g_n \in C_c(X)$. Then $g_n$ satisfy
  \begin{enumerate}[label=(\arabic*)]
    \item
    \item
    \item
  \end{enumerate}
  Let $g = \sum_{n \in \mathbb{N}} g_n$, which converges uniformly as
  $|g_n| \le \frac{1}{2^{n-1}}$ by Wierestrass. Hence $g \in C_c(X)$
  and $  \textrm{supp}(g) \subset \overline{V}$.

  We know that $f = \sum_{n = 1}^{\infty}  t_n$  from
  the definition of $t_n$. So the set $D = \{ x \in X  \ : \  f(x)
  \neq g(x) \}$ is a subset of $\cup_{n = 1}^{\infty} \{ x \in X  \ :
  \ t_n(x) \neq g_n(X)  \}$. Now the subadditivity of $\mu$ gives
  that $\mu(D) < \epsilon$.

  Next, if $f$ is non-negative, bounded, the result follows from
  scaling $f$. Again if $f \ge 0$ is measurable and possibly
  unbounded, we have
  $\cap_{n = 1}^{\infty} \{ x \in X  \ : \  f(x) \ge n \} =
  \emptyset$. Moreover $\mu(\{ f \ge 1 \}) \le \mu(A) < \infty$.
  Hence by the continuity of the measure from above, we get $\mu(\{ f
  \ge n \}) \to 0$. Hence we can replace $f$ with $f \chi_{f< n}$ for
  some appropriate $n$.

  Now if the function is general complex, we can split it as the sum
  and difference of four non-negative measurable functions and
  continue the analysis. Finally if $A$ is not compact, we can find a
  $K \subset A$ such that $ K$ is compact and $\mu(A \setminus K)$ is
  arbitrarily small by the inner regularity of the measure $\mu$ for
  finite sets.
\end{proof}


