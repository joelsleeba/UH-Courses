% TeX_root = ../main.tex

\marginnote{\scriptsize 21/11/2024 }

\section{Reisz Representation Theorem}

\begin{theorem}[Reisz Representation Theorem]
  Let $\Lambda: \mathcal{H} \to \mathbb{C}$ be a continuous linear
  functional on a Hilbert space. Then there is a unique $y \in
  \mathcal{H}$ such that $\Lambda(x) = \langle x , y \rangle $
\end{theorem}
\begin{proof}
  Assume that $\Lambda \neq 0$. Then $M = \textrm{Ker}(\Lambda)$ is a proper
  closed linear subspace of $\mathcal{H}$. Then so is $M^\perp$. Let
  $0 \neq v, w \in M^\perp$. Then $\Lambda(v) \neq 0 \neq
  \Lambda(w)$. Then since
  \begin{align*}
    \Lambda \Big( \frac{v}{\Lambda(v)} - \frac{w}{\Lambda(w)} \Big) = 0
  \end{align*}
  forces $\frac{v}{\Lambda(v)} - \frac{w}{\Lambda(w)} \in M \cap
  M^\perp = \{ 0 \}$. Hence $v \in \textrm{span}(w)$. Thus we see
  that $M$ has co-dimension 1. i.e $M^\perp$ has dimension $1$.

  Now consider $P$, the orthogonal projection to $M$ and $Q = 1 - P$.
  Then $ x = Px + Qx$ for all $ x \in \mathcal{H}$. Hence
  \begin{align*}
    \Lambda(x) = \Lambda(Px + Qx)  = \Lambda(Qx)
  \end{align*}
  Since $M^\perp$ is a one dimensional subspace $ Qx = \alpha v$ for
  $ \alpha \in \mathbb{C}$ and $0 \neq v \in M^\perp$ with
  $\Lambda(v) = 1$. This gives that
  \begin{align*}
    \Lambda(x) &= \alpha \Lambda(v)  \\
    &= \alpha \\
    &= \alpha \langle v , \frac{v}{\|v\|^2} \rangle \\
    &= \langle \alpha v , \frac{v}{\|v\|^2} \rangle \\
    &= \langle Qx , \frac{v}{\|v\|^2} \rangle  + 0\\
    &=\langle Qx , \frac{v}{\|v\|^2} \rangle + \langle Px ,
    \frac{v}{\|v\|^2} \rangle \quad (\textrm{Since } Px \perp M^\perp) \\
    &= \langle Qx + Px , \frac{v}{\|v\|^2} \rangle \\
    &= \langle x , \frac{v}{\|v\|^2} \rangle
  \end{align*}
\end{proof}

\section{Orthonormal Sets}
\begin{definition}
  A family $\{ u_\alpha \}_{\alpha \in A}$ is called orthonormal if $
  \langle u_\alpha ,  u_\beta \rangle = \delta_{a, b}$ for each
  $\alpha, \beta \in A$.

  If $x \in \mathcal{H}$, then $\langle  x , u_\alpha \rangle $ is
  called a Fourier coefficient of $x$ relative to $u_\alpha$.
\end{definition}

We consider finite orthonormal sets first.
\begin{proposition}
  Let $\{ u_\alpha \}_{\alpha \in A}$ be an orthonormal set and $ F
  \subset A$ be finite. Let $M_F = \textrm{span}\{ u_\alpha \}_{\alpha \in F}$.
  \begin{enumerate}[label=(\alph*)]
    \item  If $\phi: A \to \mathbb{C}$, $\phi|_{A \setminus F}  = 0$,
      then there is $ y \in M_F$ such that
      \begin{align*}
        y = \sum_{\alpha \in  F} \phi(\alpha)   u_\alpha
      \end{align*}
      and $\phi(\alpha) = \langle y , u_\alpha \rangle $ for each $a
      \in A$. Also
      \begin{align*}
        \|y\|^2 = \sum_{\alpha \in  F} |\phi(\alpha)|^2
      \end{align*}

    \item If $x \in \mathcal{H}$, then
      \marginnote{ \scriptsize This says that the orthogonal projection is
      the best approximation to the subspace}
      \begin{align*}
        \Big \| x = \sum_{\alpha \in  F} \langle x , u_\alpha \rangle
        u_\alpha \Big \| \le \|x - s\| \quad \textrm{for any } s \in M_F
      \end{align*}
  \end{enumerate}
\end{proposition}
\begin{proof}
  \begin{enumerate}[label=(\alph*)]
    \item Straightforward.
    \item Let $S(x) = \sum_{\alpha \in  F} \langle x , u_\alpha
      \rangle  u_\alpha$. Note that $ \langle S(x) ,  u_\alpha
      \rangle = \langle x , u_\alpha \rangle$ for each $ \alpha \in
      F$. Thus we see that $ \langle x - S(x) , u_\alpha \rangle = 0$
      for each $ \alpha \in F$. Because $M_F = \textrm{span}\{
      u_\alpha  \ : \   \alpha \in F \}$, we see that $(x - S(x))
      \perp v$ for any $v \in M_F$. Thus $(x - S(x))\perp (S(x) -
      v)$. Thus be Pythagoras theorem, we get
      \begin{align*}
        \|x - v\|^2 &= \|x - S(x)\|^2 + \|v - S(x)\|^2 \ge \|x - S(x)\|^2
      \end{align*}
      Thus we see that $S(x)$ is the best approximation of $x$ on to the
      space $M_F$. Thus we see that $S$ is the orthogonal projection to $ M$.
  \end{enumerate}
\end{proof}

