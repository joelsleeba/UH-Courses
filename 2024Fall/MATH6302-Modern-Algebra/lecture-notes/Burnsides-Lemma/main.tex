% initial settings
\documentclass[11pt]{article}
\usepackage{geometry} % automatic papersizes, margins.
\usepackage{makeidx} % 'makeidx' make and show index
\usepackage{enumitem} % itemize, enumerate, description.
\usepackage{hyperref} % hyperlinks, cross-references.
\usepackage{xcolor} % foreground and background color management.
% Better color mixing compared to 'color'
\usepackage{graphicx} % provide options for \includegraphics. Builds
% on 'graphic'
\usepackage{caption} % better control over captions of figures and equations.
\usepackage{appendix} % extra control over appendix
\usepackage[backend=biber, style=alphabetic]{biblatex} % better than
% bibtex, people say.
\usepackage{tocbibind} % add ToC/Bibliography/Index to ToC

\usepackage{amsmath} % math symbols, matrices, cases, trig functions,
% var-greek symbols.
\usepackage{amsfonts} % mathbb, mathfrak, large sum and product symbols.
\usepackage{amssymb} % extended list of math symbols from AMS.
% https://ctan.math.washington.edu/tex-archive/fonts/amsfonts/doc/amssymb.pdf
\usepackage{amsthm} % theorem styling.
\usepackage{mathrsfs} % mathscr fonts.
\usepackage{yhmath} % widehat.
\usepackage{empheq} % emphasize equations, extending 'amsmath' and 'mathtools'.
\usepackage{bm} % simplified bold math. Do \bm{math-equations-here}
\usepackage{tikz} % for tikz diagrams
\usepackage{tikz-cd} % commutative diagrams.
\usepackage{marginnote} % For sidenotes

% geometry of paper
\geometry{
  a4paper, % 'a4paper', 'c5paper', 'letterpaper', 'legalpaper'
  asymmetric, % don't swap margins in left and right pages. as
  % opposed to 'twoside'
  centering, % to center the content between margins
  bindingoffset=0cm,
}

% hyprlink settings
\hypersetup{
  colorlinks = true,
  linkcolor = {red!60!black},
  anchorcolor = red,
  citecolor = {green!50!black},
  urlcolor = magenta,
}

% theorem styles
\theoremstyle{plain} % default; italic text, extra space above and below
\newtheorem{theorem}{Theorem}[section]
\newtheorem{proposition}{Proposition}[section]
\newtheorem{lemma}{Lemma}[section]
\newtheorem{corollary}{Corollary}[theorem]

\theoremstyle{definition} % upright text, extra space above and below
\newtheorem{definition}{Definition}[section]
\newtheorem{example}{Example}[section]

\theoremstyle{remark} % upright text, no extra space above or below
\newtheorem{remark}{Remark}[section]
\newtheorem*{note}{Note} %'Notes' in italics and without counter

% renewcommands for counters
\newcommand{\propositionautorefname}{Proposition}
\newcommand{\definitionautorefname}{Definition}
\newcommand{\lemmaautorefname}{Lemma}
\newcommand{\remarkautorefname}{Remark}
\newcommand{\exampleautorefname}{Example}

\addbibresource{articles.bib}

\begin{document}

\title{Burnside's Lemma}

% author list
\author{
  Joel Sleeba \\
  University of Houston \\
  joelsleeba1@gmail.com \\
}

\maketitle

\begin{theorem}
  Let $G$ be a group, and $G \curvearrowright A$. If we denote the
  collection of all the orbits of $A$ under the action of $G$ by
  $A/G$, and the elements of $A$ which are fixed by a $g \in G$ by $A^g$, then
  \begin{align*}
    |A/G| = \frac{1}{|G|} \sum_{ g \in G} |A^g|
  \end{align*}
\end{theorem}
\begin{proof}
  Notice that
  \begin{align*}
    \bigcup_{a \in A} G_a \times a = \{ (g, a) \in G \times A  \ : \
    g \cdot a = a \}  = \bigcup_{g \in G} g \times A^g
  \end{align*}
  Thus $\sum_{a \in A} |G_a| = \sum_{g \in G} |A^g|$. Moreover,
  notice that if $ga = b$, then $G_b = gG_a g^{-1}$. Thus $|G_a| =
  |G_b|$ if $a$ and $ b$ are in the same orbits. Also, by the
  orbit-stabilizer theorem, we know that there are $|G:G_a|$ elements
  in the orbit of $a$. Thus for each representative $r$ from the
  distinct orbits of $A$, we see that
  \begin{align*}
    \sum_{g \in G} |A^g| = \sum_{r \in A/G}|G_r||G:G_r| = \sum_{r \in
    A/G} |G| = |G| |A/G|
  \end{align*}
  Hence the theorem follows.
\end{proof}

\printbibliography[heading=bibintoc]
\end{document}
