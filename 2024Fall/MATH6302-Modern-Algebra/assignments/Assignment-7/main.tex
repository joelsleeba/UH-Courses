% initial settings
\documentclass[12pt]{exam}
\usepackage{geometry}
\usepackage{graphicx}
\usepackage{enumitem}
\usepackage[usenames,dvipsnames]{xcolor}
\usepackage[backend=biber, style=alphabetic]{biblatex}
\usepackage{url,hyperref}

\usepackage{amsmath} % math symbols, matrices, cases, trig functions,
% var-greek symbols.
\usepackage{amsfonts} % mathbb, mathfrak, large sum and product symbols.
\usepackage{amssymb} % extended list of math symbols from AMS.
% https://ctan.math.washington.edu/tex-archive/fonts/amsfonts/doc/amssymb.pdf
\usepackage{amsthm} % theorem styling.
\usepackage{mathrsfs} % mathscr fonts.
\usepackage{yhmath} % widehat.
\usepackage{empheq} % emphasize equations, extending 'amsmath' and 'mathtools'.
\usepackage{bm} % simplified bold math. Do \bm{math-equations-here}

% geometry of paper
\geometry{
  a4paper, % 'a4paper', 'c5paper', 'letterpaper', 'legalpaper'
  asymmetric, % don't swap margins in left and right pages. as
  % opposed to 'twoside'
  centering, % to center the content between margins
  bindingoffset=0cm,
}

% hyprlink settings
\hypersetup{
  colorlinks = true,
  linkcolor = {red!60!black},
  anchorcolor = red,
  citecolor = {green!50!black},
  urlcolor = magenta,
}

% theorem styles
\theoremstyle{plain} % default; italic text, extra space above and below
\newtheorem{theorem}{Theorem}[section]
\newtheorem{proposition}{Proposition}[section]
\newtheorem{lemma}{Lemma}[section]
\newtheorem{corollary}{Corollary}[theorem]

\theoremstyle{definition} % upright text, extra space above and below
\newtheorem{definition}{Definition}[section]
\newtheorem{example}{Example}[section]

\theoremstyle{remark} % upright text, no extra space above or below
\newtheorem{remark}{Remark}[section]
\newtheorem*{note}{Note} %'Notes' in italics and without counter

% renewcommands for counters
\newcommand{\propositionautorefname}{Proposition}
\newcommand{\definitionautorefname}{Definition}
\newcommand{\lemmaautorefname}{Lemma}
\newcommand{\remarkautorefname}{Remark}
\newcommand{\exampleautorefname}{Example}

\addbibresource{articles.bib}

\begin{document}

\title{MATH 6320 - Modern Algebra \\ Homework 7}

% author list
\author{
  Joel Sleeba \\
}

\maketitle
\printanswers
\unframedsolutions

\begin{questions}

  \question
  \begin{solution}
    Since $p$ is a prime and $P$ is a subgroup of $S_p$ of order $p$,
    we notice that $P$ is a cyclic subgroup with $p-1$ elements of
    $P$ having order $p$. Now let $g \in S_p$ and $h \in P$ with $|h|
    = p$. Then we claim that $|ghg^{-1}| = p$.

    Since $(ghg^{-1})^p =
    e$, we see that $|ghg^{-1}|\big| p$. Since $p$ is a prime the
    only possibilities are $|ghg^{-1}| = 1$ or $p$. If $|ghg^{-1}| =
    1$, this would force $gh = g$ and $h = e$, contradicting our
    assumption. Hence we see that $|ghg^{-1}| = p$. Therefore, we see
    that conjugation with elements of $S_p$, preserves the order of
    elements of $P$.

    Moreover, we know that since $P$ is a subgroup, every conjugate
    $gPg^{-1}$ must also be a subgroup of $S_p$ with $p$ elements.
    (That $gPg^{-1}$ has $p$ elements may be seen by assuming
      $ghg^{-1} = gkg^{-1}$ and showing $h = k$, by left and right
    multiplication with $g^{-1}$ and $g$ respectively). Since we know
    that conjugation preserves the order of elements, we know that
    each conjugate of $P$ has $p-1$ $p$-cycles.

    Also, each of the distinct conjugate groups $gPg^{-1}$ intersect
    only at the identity, otherwise if $e \neq x \in gPg^{-1} \cap
    hPh^{-1}$, since $gPg^{-1}, hPh^{-1}$ are cyclic groups of order
    $p$, we'll get $gPg^{-1} = \langle x  \rangle = hPh^{-1}$.

    If $\tau \in S_p$, we know that \[
      \tau (1\ 2 \ 3 \ldots p) \tau^{-1} = (\tau(1)\ \tau(2)\ \ldots \tau(p))
    \]
    Hence we see that any $p$ cycle can be written as a conjugate of any
    other $p$-cycle if we carefully choose $\tau$. Thus conjugates of
    $P$ contain all the $p$-cycles of $S_p$. We know that the number
    of $p$-cycles of in $S_p$ is $(p-1)!$. Moreover we
    know that the number of the conjugates of $P$ is the index of
    $N_{S_p}(P)$. Hence
    \begin{align*}
      (p-1)! &= (p-1) |S_p: N_{S_p}(P)| \\
      &= (p-1) \frac{|S_p|}{|N_{S_p|}(P)} \\
      &= (p-1) \frac{p!}{|N_{S_p}(P)|}
    \end{align*}
    which on simplification gives $|N_{S_p}(P)| = p(p-1)$
  \end{solution}

  \question
  \begin{solution}
    Since $r \in D_8$, has order $4$, if $\phi: D_8 \to D_8$ is any
    automorphism, then $\phi(r)$ must also have the same order. Hence
    the possible $\phi(r)$ are $r, r^{-1} \in D_8$. Similarly  since
    $|s| = 2$, $\phi(s)$ also  must have order $2$, which gives
    $\phi(s) \in \{ s, r^2, sr, sr^2, sr^3\}$. But since $\phi(r) \in
    \{ r , r^3  \}$, if $\phi(s) = r^2$, $\phi(D_8) = \langle  r
    \rangle $, and $\phi$ will not be an automorphism. Hence $\phi(s)
    \in \{  s, sr, sr^2, sr^3 \}$. Since $s, r$ generate $D_8$, and
    each of them have $4$ and $2$ possible options, by the counting
    argument, $\textrm{Aut}(D_8)$ can have at most 8 elements.
  \end{solution}

  \question
  \begin{solution}
    Since $D_8 \trianglelefteq D_{16}$, we see that $\phi: D_{16} \to
    \textrm{Aut}(D_8): g \to \phi_g$, where $\phi_g: h \to ghg^{-1}$
    is a well defined map. Since
    \begin{align*}
      \phi_g \phi_{g^\prime}(h) & = \phi_g(g^\prime h(g^\prime)^{-1}) \\
      & = gg^\prime h (g^\prime)^{-1} g^{-1} \\
      & = (gg^\prime) h (gg^\prime)^{-1} \\
      & = \phi_{gg^\prime}(h)
    \end{align*}
    we see that $\phi$ is a group homomorphism. Moreover, we know
    that $\textrm{Ker}(\phi) = C_{D_{16}}(D_8) = \langle r^4 \rangle =
    \{r^4, e\}$. Hence by
    the first isomorphism theorem, we see that $\phi(D_{16}) =
    \frac{D_{16}}{\langle r^4 \rangle} \cong D_8$. Hence $D_8$ is
    isomorphic to a subgroup of $\textrm{Aut}(D_8)$. But from the
    previous exercise, we see that $  \textrm{Aut}(D_8)$ can have
    atmost $8$ elements. Since $D_8$ has $8$ elements, this forces
    $D_8 \cong \textrm{Aut}(D_8)$.
  \end{solution}

  \question
  \begin{solution}
    From what we proved in the class, we know that if $H \leqslant
    G$, then $N_G(H)/C_G(H)$ is isomorphic to a subgroup of
    $\textrm{Aut}(H)$. Hence in the question, we know that
    $N_{S_p}(P)/C_{S_p}(P)$ is isomorphic to a subgroup of $\textrm{Aut}(P)$.

    Since $P$ is a cyclic group of order $p$, $P \cong
    \mathbb{Z}/p\mathbb{Z}$ and hence the number of automorphisms of
    $P$ are precisely $p-1$.

    Also $C_{S_p}(P) = P$. Since $P$ is cyclic, it is clear that $P
    \subset C_{S_p}( P)$. Conversely, without loss of genrality,
    assume that $(1\ 2 \ 3 .. p) \in P$. If $\tau \in S_p$, then \[
      \tau (1\ 2 \ 3 \ldots p) \tau^{-1} = (\tau(1)\ \tau(2)\ \ldots
      \tau(p)) = (1\ 2\ 3\  \ldots p)
    \]
    if and only if $(\tau(1)\ \tau(2)\ \ldots \tau(p))$ is a rotation
    of the $1, 2, \ldots p$, preserving the order. This happens only
    when $\tau = (1\ 2 \ 3 \ldots p)^k$ for some $k$. Hence we see
    that $C_{S_p}(P) = P$.

    Moreover, we know that $|N_{S_p}(P)| = p(p-1)$. Therefore, we see that \[
      \Bigg | \frac{N_{S_p}(P)}{C_{S_p}(P)} \Bigg| =
      \frac{|N_{S_p}(P)|}{|C_{S_p}(P)|} = \frac{p(p-1)}{p} = p-1
    \]
    Therefore we see that  $N_{S_p}(P)/C_{S_p}(P) \cong \textrm{Aut}(P)$.
  \end{solution}

  \question
  \begin{solution}
    Let $(1, k) \in C_K(H)$. Then for any $(h, 1) \in G$, $$(h, k) =
    (h\varphi(1)(1), k) = (h, 1)(1,k)  = (1, k)(h, 1) =  (1\varphi(k)(h), k)$$
    forces $\varphi(k)(h) = h$. Since this is true for all $h \in H$,
    we see that $\phi(k)$ is the trivial automorphism of $H$. Hence
    $k \in \textrm{Ker}(\phi)$.

    Conversely, if $k \in \textrm{Ker}(\phi)$, then $\phi(k)(h) = h$
    for all $h \in H$. Then for any $(h, 1) \in H$(identified as a
    subgroup of $G$) \[
      (h, 1)(1, k) = (h \varphi(1)(1), k) = (h, k) = (\phi(k)(h), k)
      = (1, k)(h, 1)
    \]
    shows that $(1, k) \in C_K(H)$. Hence $C_K(H) = \textrm{ Ker}(\varphi)$.
  \end{solution}

  \question
  \begin{solution}
    We know that $\textrm{Hol}(H) = H \rtimes_\varphi \textrm{Aut}(H)$,
    where $\varphi: \textrm{Aut}(H) \to \textrm{Aut}(H)$ is the identity map.
    \begin{parts}
      \part We notice that $H = Z_2 \times Z_2 \cong V_4$, the Klein
      4 group. Therefore, by a slight abuse of notation, let $H = V_4
      = \{ 1, a, b, c \}$. Since we know that
      any two of $a, b, c$ generate the group $V_4$ we see that
      any permutation of $a, b, c$ will be a group automorphism.
      Hence we see that $\textrm{Aut}(H) \cong S_3$. Hence we see
      that $\textrm{Hol}(Z_2 \times Z_2) \cong H \rtimes K$, where $H
      = Z_2 \times Z_2$ and $K \cong S_3$.
      Also, $|H \rtimes K| = |H \times K| = |H| \times |K| = 4 \times 6 = 24$

      \part Let $G = H \rtimes K$ act on the left cosets of $K$,
      $\tilde{K} = \{ K, aK , bK ,
      cK \}$ as \[
        (h, k)(gK) = hk(g)K
      \]
      Since every element in the coset $gK$ is of the form $(g, k)$
      for some $k \in K$, well defineness of the map
      follows. Moreover,
      \begin{align*}
        (h_1, k_1)((h_2, k_2)(gK)) &= (h_1, k_1)(h_2k_2(g)K) \\
        &= h_1 k_1(h_2 k_2(g))K \\
        &= h_1k_1(h_2)k_1(k_2(g))K \\
        &= (h_1k_1(h_2), k_1k_2)(gK) \\
        &= ((h_1, k_1)( h_2, k_2))(gK)
      \end{align*}
      and
      \begin{align*}
        (e_H, e_K)(gK) = e_He_K(g)K = K
      \end{align*}
      shows that the above defined map is indeed an action.

      Consider $\varphi: H \rtimes K \to S_{\tilde{K}}$, the associated
      permutation representation of the above action. Once we show
      that $\varphi$ is bijective, since $|\tilde{K}| = 4$, this will
      show that $H \rtimes K \cong S_4$.

      Let $(h, k) \in \textrm{Ker}(\varphi)$. Then $(h, k)gK = hk(g)K = gK$ for
      all $g \in H$. This implies $hk(g) = g$ for all $g \in H$ (This
        is because $hk = (h, 1)(1, k) = (h, k)$ as $H, K$ are
      identified as subgroup of $G$). Now let $g = e_H$.  Since $k\in
      K$ is an automorphism, this forces $k(e_H) = e_H$. Then we see
      that $h = e_H$. Substituting for $h$ in $(h, k)$, we see that
      $k(g) = g$ for all $g \in H$, which forces $k \in
      \textrm{Aut}(H)$ to be the trivial automorphism. Hence we see
      that $\textrm{Ker}(\varphi) = \{(e_H, e_K)\}$ and
      $\varphi$ is injective, hence an isomorphism. Thus $H \rtimes K
      \cong S_4$.
    \end{parts}
  \end{solution}

  \question
  \begin{solution}
    We know that since $75 = 3\times5^2$, the fundamental theorem for
    Abelian groups immediately gives two groups $Z_3 \times Z_{5^2}
    \cong Z_{75}$ and $Z_3 \times Z_5 \times Z_5$.

    Now, to find a non-Ableian group of order $75$, consider the map
    $\varphi: Z_5 \to \textrm{Aut}(Z_{15})$ defined as \[
      \varphi(r) = (1\ 2\ 3\ 4\ 5)^r
    \]
    Clearly $\varphi$ is an injective homomorphism. Then define $G =
    Z_{15} \rtimes_\varphi Z_5$. We note that $G$ is not Abelian since \[
      (1, 1)(1, 2) = (\varphi(1)(1), 2) = (2, 2) \neq (3, 2)=
      (\varphi(2)(1), 2) = (1, 2)(1, 1)
    \]
    Since $75 = 15 \times 5$, we see that $|G| = 75$.
  \end{solution}

  \question
  \begin{solution}
    Let $A$ be the given matrix. Then
    \[ A^5 =
      \begin{pmatrix}%{c c}
        0 & -1 \\
        1 & 4
      \end{pmatrix}^5
      =
      \begin{pmatrix}%{c c}
        -1 & -4 \\
        4 & 15
      \end{pmatrix}^2
      \begin{pmatrix}%{c c}
        0 & -1 \\
        1  & 4
      \end{pmatrix} =
      \begin{pmatrix}%{c c}
        4 & 1 \\
        -1 & 0
      \end{pmatrix}
      \begin{pmatrix}%{c c}
        0 & -1 \\
        1 & 4
      \end{pmatrix}
      \begin{pmatrix}%{c c}
        1 & 0 \\
        0 & 1
      \end{pmatrix}
    \]
    shows that $|A| = 5$. Now define a map $\varphi: Z_{5} \to
    \textrm{Aut}(Z_{19} \times Z_{19})$ as \[
      \varphi(r)(x, y) = A^r
      \begin{bmatrix}%{c}
        x \\
        y
      \end{bmatrix}
    \]
    Since $A \in GL_2(\mathbb{F}_2)$, $A^r \in GL_2(\mathbb{F}_2)$
    for each $r \in Z_5$ and hence is a bijection. Moreover matrix
    multiplication preserves additivity, we see that it is an
    isomorphism of $Z_{19}\times Z_{19}$. Hence $\varphi(r) \in
    \textrm{Aut}(Z_{19} \times Z_{19})$.

    Now consider the group $G = (Z_{19} \times Z_{19})
    \rtimes_\varphi Z_5$. Then \[
      ((1, 1), 1) \rtimes ((1, 2), 2) = ((1, 1) A(1, 2), 2) = ((1,
      1)(-2, 9), 2) = ((-1,10), 2 )
    \]
    but \[
      ((1, 2), 2) \rtimes ((1, 1), 1) = ((1, 2) A^2(1, 1), 2) = ((1,
      2)(-5, 19), 2) = ((-4, 2), 2)
    \]
    shows that $G$ is not Abelian. And it is evident that $|G| = 19
    \times 19 \times 5 = 1805$.

    Moreover, the fundamental theorem of Abelian groups gives us two
    other groups, $Z_{1805} \cong Z_5 \times Z_{361}$ and $Z_5 \times
    Z_{19} \times Z_{19} \cong Z_{95} \times Z_{19}$ of order 1805.
  \end{solution}

  \question
  \begin{solution}
    If $\phi: Z_2 \to \textrm{Aut}(Z_{2^n})$ is a homomorphism, then
    it is completely determined by $\phi(1)$, since $1$ generate
    $Z_2$. Moreover, since $1$ has order 2 in
    $Z_2$, $\phi(1)$ has to divide 2. Then the only
    possibilities for $\phi(1)$ are either the trivial automorphism
    or $\phi(1)$ must have order $2$.

    We also see that the automorphisms of $Z_{2^n}$ are also
    completely characterized by the image of $1$ for the same reason.
    Hence if $\sigma \in \textrm{Aut}(Z_{2^n})$ is an automorphism
    with $\sigma(1) = k$, we see that
    \begin{align*}
      \sigma^2(r) &= \sigma( \sigma(r)) \\
      &= \sigma( r \sigma(1)) \\
      &= \sigma( rk) \\
      &= rk \sigma(1) \\
      &= rk^2
    \end{align*}
    If $\sigma^2 = e$, then $\sigma^2(r) =
    r$ for all $r \in Z_{2^n}$. This forces $k^2 \equiv 1 \mod 2^n$.
    We can show that the only choices for such $k \in Z_{2^n}$ are
    $\{ 1, 2^{n-1}-1, 2^{n-1}+1, 2^n-1 \}$. That $1, 2^n - 1$
    satisfies the above equation is evident. To see if there are any
    other, Let $k = 2^{n-1} + r$, then
    \begin{align*}
      (2^{n-1} + r)^2 &= 2^{2(n-1)} + r2^n + r^2 \\
      &\equiv 2^n2^n-2 + r^2 \\
      &\equiv r^2
    \end{align*}
    Thus we see that $ r = \pm 1$  gives another solution for $k$.

    Hence there are exactly
    $4$ homomorphisms from $Z_2 \to \textrm{Aut}(Z_{2^n})$. We'll
    denote each of these 4 homomorphisms by
    $\phi_1, \phi_2, \phi_3, \phi_4$, and the their corresponding
    images $\phi_i(1) \in \textrm{Aut}(Z_{2^n})$ by $\sigma_1 ,
    \sigma_2 , \sigma_3, \sigma_4$ where each $\sigma_i$ send $1$ to
    $1, 2^{n-1} - 1, 2^{n-1} + 1, 2^{n} - 1$ respectively.

    Clearly,
    we see that each $Z_{2^n} \rtimes_{\phi_i} Z_{2}$ contains
    $2^{n+1}$ elements.
    Since $\phi_1$ is the trivial morphism, we see that $Z_{2^n}
    \rtimes_{\phi_1} Z_2 \cong Z_{2^n} \times Z_2$ by the
    representation theorem and hence Abelian. By the same reasoning none
    of the other direct products are Abelian.
  \end{solution}

\end{questions}
\printbibliography[heading=bibintoc]
\end{document}

