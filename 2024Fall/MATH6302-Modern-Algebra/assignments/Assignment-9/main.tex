% initial settings
\documentclass[12pt]{exam}
\usepackage{geometry}
\usepackage{graphicx}
\usepackage{enumitem}
\usepackage[usenames,dvipsnames]{xcolor}
\usepackage[backend=biber, style=alphabetic]{biblatex}
\usepackage{url,hyperref}

\usepackage{amsmath} % math symbols, matrices, cases, trig functions,
% var-greek symbols.
\usepackage{amsfonts} % mathbb, mathfrak, large sum and product symbols.
\usepackage{amssymb} % extended list of math symbols from AMS.
% https://ctan.math.washington.edu/tex-archive/fonts/amsfonts/doc/amssymb.pdf
\usepackage{amsthm} % theorem styling.
\usepackage{mathrsfs} % mathscr fonts.
\usepackage{yhmath} % widehat.
\usepackage{empheq} % emphasize equations, extending 'amsmath' and 'mathtools'.
\usepackage{bm} % simplified bold math. Do \bm{math-equations-here}
\usepackage{marginnote}

% geometry of paper
\geometry{
  a4paper, % 'a4paper', 'c5paper', 'letterpaper', 'legalpaper'
  asymmetric, % don't swap margins in left and right pages. as
  % opposed to 'twoside'
  centering, % to center the content between margins
  bindingoffset=0cm,
}

% hyprlink settings
\hypersetup{
  colorlinks = true,
  linkcolor = {red!60!black},
  anchorcolor = red,
  citecolor = {green!50!black},
  urlcolor = magenta,
}

% theorem styles
\theoremstyle{plain} % default; italic text, extra space above and below
\newtheorem{theorem}{Theorem}[section]
\newtheorem{proposition}{Proposition}[section]
\newtheorem{lemma}{Lemma}[section]
\newtheorem{corollary}{Corollary}[theorem]

\theoremstyle{definition} % upright text, extra space above and below
\newtheorem{definition}{Definition}[section]
\newtheorem{example}{Example}[section]

\theoremstyle{remark} % upright text, no extra space above or below
\newtheorem{remark}{Remark}[section]
\newtheorem*{note}{Note} %'Notes' in italics and without counter

% renewcommands for counters
\newcommand{\propositionautorefname}{Proposition}
\newcommand{\definitionautorefname}{Definition}
\newcommand{\lemmaautorefname}{Lemma}
\newcommand{\remarkautorefname}{Remark}
\newcommand{\exampleautorefname}{Example}

\addbibresource{articles.bib}

\begin{document}

\title{MATH 6302 - Modern Algebra\\ Homework  9}

% author list
\author{
  Joel Sleeba \\
}

\maketitle
\printanswers
\unframedsolutions
\begin{questions}

  \question
  \begin{solution}
    Let $\mathbb{F}$ be a field of characteristic $0$ and
    $\textbf{1}$ be its identity. Then the map
    \begin{align*}
      \mathbb{Z} \to \mathbb{F} : n \to n \textbf{1}
    \end{align*}
    is an injective ring homomorphism. Hence we see that $\mathbb{F}$
    contains an isomorphic copy of $\mathbb{Z}$. Thus it must contain
    an isomorphic copy of its field of fractions, $\mathbb{Q}$.

    If instead $\mathbb{F}$ has characteristic $p$, then $\mathbb{F}
    \cong \mathbb{F}_{p^n}$ for some $n$, then again we see that the
    subset $ \{ 0, 1, 2, \ldots p-1 \}$ (where $r$ means $r \textbf{1}$) is
    a subfield isomorphic to $\mathbb{F}_p$.

    Uniqueness in both of these cases follow from the fact that both
    the above subfields are generated by the identity element of the
    field, which is unique.
  \end{solution}

  \question
  \begin{solution}
    Let $T \subset R \times S$ be an ideal. We'll show that $I = \{ r
    \in R  \ : \ (r, \cdot) \in T  \}$ and $J = \{ s \in S \ :
    \  (\cdot, s) \in T \}$ are ideals in $R, S$ and $T = I \times J$.

    That $I \times J \subset T$ follows directly from the definition
    of the product of rings. Conversely if $(r, s) \in T$, then $r
    \in I, s\in J$ and thus $(r, s) \in I \times J$. Thus $T = I \times J$.

    Now we'll show that $I$ is an ideal of $R$. Let $r \in R$ and $i
    \in I$. Then there exists $j \in J$ such that $(i, j) \in I
    \times J$. Since $I \times J$ is an ideal of $R \times S$,
    \begin{align*}
      (r, 1)(i, j) = (ri, j) \in I \times J
    \end{align*}
    Thus we see that $ri \in I$. By a symmetric argument, we'll get
    that $ir \in I$. Thus $I$ is an ideal of $R$. The fact that $J$
    is an ideal of $S$ follows from a similar argument.
  \end{solution}

  \question
  \begin{solution}
    Let $A_i = (n_i)$, where $1 \le i \le k$,  be the ideal generated
    by the constant polynomials $n_i$ in $R = \mathbb{Z}_d[x]$, the
    ring of polynomials with degree less than or equal to $d$.
    Since $(n_i, n_j) = 1$ for all $i \neq j$, by Bezout's lemma the
    ideals $A_i$ and $A_j$ are co-maximal whenever $i \neq j$.
    Then Chineese remainder theorem shows that the canonical map
    \begin{align*}
      \phi: R/A \to R/A_1 \times
      R/A_2 \times \ldots \times R/A_k
    \end{align*}
    is a ring isomorphism where $A = A_1 \cap A_2 \cap \ldots \cap
    A_k$. Thus $(f_1+A_1 , f_2+A_2 ,\ldots f_k + A_k) \in  R/A_1 \times
    R/A_2 \times \ldots \times R/A_k$ has a pre-image $f + A \in
    R/A$. Notice that $A_1 \cap A_2 \cap \ldots \cap A_k = (n_1n_2
    \ldots n_k)$. Let $f \in R$ be a representative from $f + A$. We
    claim that $f$ can be chosen such that the degree of $f$ is $d$.
    In case the degree of $f$ is not $d$, take $g = n_1n_2 \ldots n_k
    x^{d} + f$, where $r$ is the degree of $f$. Clearly $g \in f +
    A$, since $n_1n_2 \ldots n_k x^{d} \in A$. Thus $g$ is the
    function which satisfy the requirements of the question.

    \marginnote{ \scriptsize \it We'll slightly abuse the
      definition of $A_i$ and $A$ by viewing it as ideals of either
    $\mathbb{Z}_d[x]$ or $\mathbb{Z}_{d-1}[x]$ as we need.}
    Now if $f_i$s are monic polynomials of degree $d$, consider the
    polynomials $\tilde{f}_i \in \mathbb{Z}_{d-1}[x]$, where
    $\tilde{f}_i$ is polynomial
    removing the $x^d$ term from $f$. Let $\tilde{f} + A$ be
    the pre-image of $(\tilde{ f}_1 + A_1, \tilde{f}_2 + A_2, \ldots,
    \tilde{f}_k + A_k)$ under $\phi$. Like we did before, we can
    choose a representative $\tilde{f}$ for $\tilde{f} + A$ with
    degree $d-1$. Let $f = x^d + \tilde{f}$. Since $f, f_i$ are monic
    polynomials, we get that
    \begin{align*}
      f - f_i = \tilde{f} - \tilde{f}_i \in (n_i) = A_i
    \end{align*}
    Thus we see that $f = f_i \mod n_i$ for each $1 \le i \le k$.
    Hence we are done.
  \end{solution}

  \question
  \begin{solution}
    Division algorithm in polynomial rings shows that for any
    polynomials $a, b \in \mathbb{F}[x]$, there exist unique
    polynomials $q, r \in \mathbb{F}[x]$ such that
    \begin{align*}
      a(x) = q(x) b(x) + r(x)
    \end{align*}
    where the $\textrm{deg}(r)< \textrm{deg}(b)$. Here $b = f$ and
    therefore every $g \in \mathbb{F}[x]$ can be written as $g(x) =
    q(x) f(x) + r(x)$ where $\textrm{deg}(r) < n = \textrm{deg}(f)$.
    Thus we see that $a + (f) = r + (f)$. Hence the number of
    elements of $\mathbb{F}[x]/(f)$ is the number of all distinct
    polynomials of degree less than $n$ in $\mathbb{F}[x]$, which is
    $q^n$, since $\mathbb{F}$ is a field of order $q$.
  \end{solution}

  \question
  \begin{solution}
    We know that for any ring $R$ and an ideal $I$ of $R$, $R/I$ is a
    field if and only if $I$ is a maximal ideal. Hence the problem
    reduces to proving $f \in \mathbb{F}[x]$ is irreducible if and
    only if $(f)$ is a maximal ideal. One way is easy, if $f$ is
    reducible as $f(x) = p(x) q(x)$, where $p, q$ are not units, then
    $(f) \subsetneq (p) \neq \mathbb{F}[x]$. Thus $(f)$ cannot be a
    maximal ideal.

    To show the converse, note that the existence of the division
    algorithm for $\mathbb{F}[x]$ makes it a Euclidean domain and
    hence a PID. Hence irreduciblity and primality coincides in
    $\mathbb{F}[x]$. Therefore if $f$ is irreducible, then $f$ is
    prime and $(f)$ is a prime ideal, which again is a maximal ideal.
  \end{solution}

  \question
  \begin{solution}
    Since $\mathbb{F}[x]$ is a Euclidean domain for every field
    \label{fields_have_infinte_primes}
    $\mathbb{F}$, we see that $p \in \mathbb{F}[x]$ is a prime if and
    only if it is irreducible. Hence the problem reduces to finding
    infinitely many irreducible elements in $\mathbb{F}[x]$.

    For the sake of contradiction, assume that $\mathbb{F}[x]$ has
    only finitely many irreducible polynomials. Let $p_1 , p_2 ,
    \ldots , p_n$ be the exhaustive list of (non-constant)
    irreducible polynomials.
    \begin{align*}
      q(x) = 1 +  \prod_{i = 1}^{n} p_i(x)
    \end{align*}
    Clearly the degree of $q(x)$ is greater than the degree of all
    $p_i$. Hence $q(x) \not\in p_i(x)$. We claim that $q(x)$ is
    irreducible contradicting our assumption. If not, since
    $\mathbb{F}[x]$ is a UFD, we'll have
    \begin{align*}
      q(x) = 1 +  \prod_{i = 1}^{n} p_i(x)= \alpha \prod_{k} p_{i_k}
    \end{align*}
    where $p_{i_k} \in \{ p_1 , p_2 , \ldots , p_n \}$ and $\alpha
    \in \mathbb{F}$. This will make each $p_{i_k}$ constant polynomials since
    \begin{align*}
      1 = p_{i_k} \Big(\prod_{i = 1, i \neq i_k}^{n} p_i -
      \prod_{k^\prime \neq k} p_{i_{k^\prime}}\Big)
    \end{align*}
    and we know that the only invertible polynomials in
    $\mathbb{F}[x]$ are the non-zero constant polynomials. This would
    contradict our assumptions on $p_i$. Hence we get that
    $\mathbb{F}[x]$ has infinitely many primes.
  \end{solution}

  \question
  \begin{solution}
    Since $\mathbb{F}[x]$ is an ED, and hence a UFD, let $p(x) =
    p_1(x)p_2(x)\ldots p_n(x)$ be a factorization of $p$
    into irreducible factors unique upto units. We claim that all the ideals of
    $\mathbb{F}[x]/(p)$ are those generated by $p_i$, like
    $(p_i , p_j, \ldots p_k)/(p)$

    From the 4th isomorphism theorem for the rings we know that if
    $I$ is an ideal of $R$, $I \subset A \subset R$ is an ideal iff
    $A/I$ is an ideal of $R/I$. Here, $R = \mathbb{F}[x]$, $I = (p)$.
    Hence $A/(p)$ is an ideal of $\mathbb{F}[x]/(p)$ if and only if
    $A$ is an ideal of $\mathbb{F}[x]$ containing $(p)$. If $p$ has a
    factorization as above we clearly see that $(p) \subset (p_i,
    p_j, \ldots p_k)$.

    Conversely, if $I$ is any ideal containing $p$, then $I$ must be
    contained in a maximal ideal say $(q)$, where $q$ is an
    irreducible polynomial in $\mathbb{F}[x]$. If $q \not\in \{ p_1,
    p_2 , \ldots , p_n \}$ then $q$ would be an irreducible factor of
    $p$ contradicting our assumption. Thus we see that $I$ must be an
    ideal generated by $p_i, p_j, \ldots p_k$. Thus fourth
    isomorphism theorem proves our assertion.
  \end{solution}

  \question
  \begin{solution}
    \begin{parts}
      \part We'll use Eisenstein criterion for $p = 2$. Since $2|a_i$
      for all $i = 3, 2, 1, 0$ and $2^2 = 4 \not|6 = a_0$ satisfies
      the Eisenstein criterion. Hence it is irreducible.
      \part Again using Eisenstein criterion with $p = 3$, we see
      that the polynomial is irreducible.
      \part Let $x^4 + 4x^3 + 6x^2 + 2x + 1 = (x+1)^4 - 2x$. Now put
      $y = x+1$ to transform the polynomial to $y^4 - 2y + 2$ which
      is clearly irreducible using Eisenstein criterion with $p = 2$.

      If it was possible to factor $(x+1)^4 - 2x$ as $(x+1)^4 - 2x = p(x)q(x)$,
      then $y^4 - 2y + 2 = p(y-1)q(y-1)$, would be reducible. Hence
      we see that the original polynomial is irreducible.
      \part
      \begin{align*}
        (x+2)^p - 2^p = \sum_{k = 1}^{p} \frac{p!}{k!(p-k)!} x^k2^{p-k}
      \end{align*}
      By Eisenstein criterion for $p$ we get that the polynomial is
      irreducible, since $p$ divides each $\frac{p!}{k!(p - k)!}$ but
      $p^2 \not| p$.
    \end{parts}
  \end{solution}

  \question
  \begin{solution}
    Since $x^2 + 1$ has no real roots, it is irreducible in
    $\mathbb{R}[x]$. Hence $(x^2 + 1)$ is a maximal ideal since
    $\mathbb{R}[x]$ is a PID. Thus $\mathbb{R}[x]/(x^2 + 1)$ is a field.
    Moreover
    \begin{align*}
      (ax + b)(cx + d) = acx^2 + (bc + ad)x + bd = (bc + ad)x + (bd -
      ac) \mod (x^2 + 1)
    \end{align*}
    shows that
    \begin{align*}
      \phi: \mathbb{C} \to \mathbb{R}[x]/(x^2 + 1) := (a + ib) \to a
      + xb + (x^2 +1)
    \end{align*}
    is a ring homomorphism. Surjectivity of $\phi$ follows from the
    division algorithm on $\mathbb{R}[x]$. Also if $\phi(a + ib) =
    \phi(c + id)$, then $a-c + x(b-d) \in (x^2 + 1)$ which forces
    $a-c + x (b-d) = 0$ and hence $a + ib = c + id$. Hence we see
    that $\phi$ is a ring isomorphism.
  \end{solution}

  \question
  \begin{solution}
    Since $\mathbb{F}_{11}[x]$ is a Euclidean domain, by the division
    \label{10}
    algorithm, every element of $\mathbb{F}_{11}[x]/(x^2 + 1)$ has a
    representative of the form $ax +b \in \mathbb{F}_{11}[x]$.
    Moreover $ax + b \in cx + b + (x^2 + 1)$ if and only if $(a-c)x +
    (b - d) \in (x^2 + 1)$ if and only if $(a-c) x + (b-d) = 0$. Thus
    we see that distinct polynomials $ax + b$ are in distinct classes
    of $\mathbb{F}_{11}[x]/(x^2  + 1)$. Since there are $11 \times 11
    = 121$ polynomials of the form $ax + b$ in $\mathbb{F}_{11}[x]$,
    we see that there are $121$ elements in
    $\mathbb{F}_{11}[x]/(x^2+1)$. That $\mathbb{F}_{11}[x]/(x^2+ 2x + 2)$
    also have 121 elements follow from the same reasoning.

    To show that the above rings are fields, it is enough to show
    that the polynomials $x^2 +1$ and $x^2 + 2x + 2$ are irreducible
    in $\mathbb{F}_{11}[x]$. This is because maximality and primality
    of ideals,  and irreducibility and primality of elements agree on PIDs.

    To show that $x^2 + 1$ is irreducible it is enough to show it has
    no roots on $\mathbb{F}_11$. For the sake of contradiction,
    assume that $a^2 + 1 = 0 \mod 11$. Then $a^2 = 10 \mod 11$ forces
    $a^2 = 11n + 10$ forces $a$ to be odd. We can verify that $a$
    cannot be either $1, 3, 5, 7, 9, 11$, exhausting every "odd"
    numbers in $\mathbb{F}_{11}$. Thus we see that $x^2 + 1$ does not
    have any root in $\mathbb{F}_{11}$. Hence $x^2 + 1$ is
    irreducible in $\mathbb{F}_{11}$.

    Since $x^2 + 2x +2 = (x+1)^2 +1$, by the above reasoning, we see
    that $x^2 + 2x + 2$ is also irreducible in $\mathbb{F}_{11}$.
    Hence the above quotient rings are indeed fields.

    Consider the map
    \begin{align*}
      \phi : \mathbb{F}_{11}[x] \to \mathbb{F}_{11}[x]:= p(x) \to p(x+1)
    \end{align*}
    We claim that the corresponding natural map
    \begin{align*}
      \tilde{\phi}: \mathbb{F}_{11}[x]/(x^2 + 1) \to
      \mathbb{F}_{11}[x]/(x^2 + 2x + 1):= [p(x)] \to [p(x+1)]
    \end{align*}
    is a well defined ring homomorphism.

    To prove the well definess of the map let $p(x) = ax + b + (x^2 +
    1)q(x)$ for some $q(x) \in
    \mathbb{F}_{11}[x]$ such that $p(x) \in [ax +b]$. Then
    \begin{align*}
      \phi(p(x)) = p(x+1) &= ax + a + b + (x^2 + 2x + 2)q(x + 1)
    \end{align*}
    shows that $\phi(p(x)) \in [\phi(ax + b)] = [ax + a + b]$. Thus
    we see that $\tilde{\phi}$ is a well defined map.

    To show that $\tilde{\phi}$ is a ring homomorphism, consider $[ax
    + b], [cx + d] \in \mathbb{F}_{11}[x]/(x^2 + 1)$. Then
    \begin{align*}
      \tilde{\phi}([ax + b] + [cx + d]) &=  \tilde{\phi}([(a+c) x + (b + d)]) \\
      &= [(a + c) x + (a +c) + (b + d)] \\
      &= [ax + a + b] + [cx + c + d] \\
      &=  \tilde{\phi}([ax + b]) +  \tilde{\phi}([cx + d])
    \end{align*}
    and
    \begin{align*}
      \tilde{\phi}([ax +b][cx + d]) &=  \tilde{\phi}([acx^2 + (ad +
      bc)x + bd]) \\
      &= \tilde{\phi}([(ad + bc)x + (bd - ac)]) \\
      &= [(ad + bc)x + (ad + bc) + (bd - ac)] \\
      &= [ac(x^2 + 2x + 2) + (ad + bc)x + (ad + bc) + (bd - ac)] \\
      &= [acx^2 + (2ac + ad + bc)x + (ad + bc + bd + ac)] \\
      &= [ax + a + b][cx + c + d]\\
      &= \tilde{\phi}([ax + b])\tilde{\phi}([cx + d])
    \end{align*}
    show that $\tilde{\phi}$ is a ring homomorphism and therefore a
    Field isomorphism.
  \end{solution}

  \question
  \begin{solution}
    \begin{parts}
      \part
      We know that
      \begin{align*}
        x^8 - 1 &= (x^4+1)(x^4-1) \\
        &= ( x^4+1)(x^2+1)( x^2-1) \\
        &=( x^4+1)(x^2+1)(x+1)(x-1)
      \end{align*}
      and
      \begin{align*}
        x^6 - 1 &= (x^3+1)(x^3-1) \\
        &= (x+1)(x^2 - x + 1)(x^2+x+1)( x-1) \\
      \end{align*}
      Since $x^2 + 1, x^2 - x + 1$, and $x^2 + x + 1$ are irreducible in
      $\mathbb{Q}[x]$ having no roots, they are irreducible in
      $\mathbb{Z}[x]$. Moreover $x^4 + 1$ does not have any rational
      roots, so if it decomposes into factors, it must decompose as a
      product of two degree two polynomials in $\mathbb{Q}[x]$. Also we
      can verify that $x^4 +1$ factors into
      \begin{align*}
        x^4 + 1 = (x^2 + 1 - 2\sqrt{x})(x^2 + 1 + 2\sqrt{x})
      \end{align*}
      in $\mathbb{Q}[x]$. Since $\mathbb{Q}[x]$ is a UFD, this
      factorization is unique upto multiplication by units. Hence we
      see that $x^4 + 1$ is irreducible in $\mathbb{Z}[x]$. Thus the
      above is the decomposition of the above polynomials into
      irreducible factors in $\mathbb{Z}[x]$.

      \part In $\mathbb{Z}/2 \mathbb{Z}[x]$, we have $x^2 + 1 =
      x^2 + 2x + 1 = (x+1)^2$, and $x-1
      = x+1$. Also $x^4 + 1 = (x^2+1)^2 - 2x^2 = (x^2 + 1)^2 = (x^2 +
      2x + 1)^2 = (x+1)^4$. Combining all of this together, we get
      \begin{align*}
        x^8 - 1 = (x+1)^8
      \end{align*}

      Similarly, $x^2 - x + 1 = x^2 + x + 1$. Since $\mathbb{Z}/ 2
      \mathbb{Z}$ is a field and $x^2 + x + 1$ has no roots in it, we
      see that $x^2 + x + 1$ is irreducible in $\mathbb{Z}/ 2
      \mathbb{Z}[x]$. Thus
      \begin{align*}
        x^6 - 1 = (x^2 + x + 1)^2(x + 1)^2
      \end{align*}

      \part In $\mathbb{Z}/ 3 \mathbb{Z}[x]$, $x^2 + 1$ is
      irreducible since it has no roots. Moreover $x^4  + 1$ is
      irreducible for the same reasons why it is irreducible in
      $\mathbb{Z}[x]$. Thus we get
      \begin{align*}
        x^8 - 1  =(x^4 + 1)(x^2 + 1)(x+2)(x+1)
      \end{align*}

      Similarly, note that $x^2 - x + 1 = x^2 + 2x + 1 = (x+1)^2$,
      $x^2 + x + 1 = x^2 + 4x + 4 = (x+2)^2$, and $x - 1 = x+2$. Thus
      we see that
      \begin{align*}
        x^6-1 = (x+2)^3(x+1)^3
      \end{align*}
    \end{parts}
  \end{solution}

  \question
  \begin{solution}
    Let $\mathbb{F}$  be a finite field. Then by
    Solution \ref{fields_have_infinte_primes}, we see that $\mathbb{F}[x]$
    has infinitely many irreducible elements. Since $\mathbb{F}$
    is finite, there are only finitely many polynomials of the form
    $ax + b$, where $a, b \in \mathbb{F}$. Therefore there must be
    non-linear irreducible polynomials in $\mathbb{F}[x]$. This shows
    that $\mathbb{F}$ is not algebraically closed.
  \end{solution}

  \question
  \begin{solution}
    Consider the polynomial $x^4 + x + 1$ in $\mathbb{F}_2[x]$. We
    claim that this is an irreducible polynomial. Since it has no
    roots, it can only be reduced into quadratic polynomials. But
    since the only irreducible quadratic polynomial is $x^2 + x + 1$,
    and $(x^2 + x +1)^2 = x^4 + x^2 +1 \neq x^4 + x +1$, we get that
    $x^4 + x + 1$ is irreducible. Thus $x^4 + x + 1$ is a maximal
    ideal and hence $\mathbb{F}_2[x]/(x^4 + x + 1)$ is a field.
    Moreover by the division algorithm and using the same reasoning
    as in solution \ref{10}, we see that $\mathbb{F}_2/(x^4 + x+1)$
    is a field of 16 elements.

    Since we know that any finite subgroup of the mulitplicative
    group of a field is cyclic, we get that multiplicative group of
    $\mathbb{F}_2[x]/(x^4 + x +1)$ is cyclic of order $15$. We can
    show that $x^3 + x^2 + 1$ is not of the order $3$ or $5$ hence
    must have order $15$. Moreover since this is a cyclic group of
    order $15$, it has $\phi(15) = 4 \times 2 = 8$ generators.
  \end{solution}

\end{questions}
\printbibliography[heading=bibintoc]
\end{document}
