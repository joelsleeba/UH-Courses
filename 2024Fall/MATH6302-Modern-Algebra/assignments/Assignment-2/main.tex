% initial settings
\documentclass[12pt]{exam}
\usepackage{geometry}
\usepackage{graphicx}
\usepackage{enumitem}
\usepackage[usenames,dvipsnames]{xcolor}
\usepackage[backend=biber, style=alphabetic]{biblatex}
\usepackage{url,hyperref}

\usepackage{amsmath} % math symbols, matrices, cases, trig functions, var-greek symbols.
\usepackage{amsfonts} % mathbb, mathfrak, large sum and product symbols.
\usepackage{amssymb} % extended list of math symbols from AMS. https://ctan.math.washington.edu/tex-archive/fonts/amsfonts/doc/amssymb.pdf
\usepackage{amsthm} % theorem styling.
\usepackage{mathrsfs} % mathscr fonts.
\usepackage{yhmath} % widehat.
\usepackage{empheq} % emphasize equations, extending 'amsmath' and 'mathtools'.
\usepackage{bm} % simplified bold math. Do \bm{math-equations-here}

% geometry of paper
\geometry{
  a4paper, % 'a4paper', 'c5paper', 'letterpaper', 'legalpaper'
  asymmetric, % don't swap margins in left and right pages. as opposed to 'twoside'
  centering, % to center the content between margins
  bindingoffset=0cm,
}

% hyprlink settings
\hypersetup{
  colorlinks = true,
  linkcolor = {red!60!black},
  anchorcolor = red,
  citecolor = {green!50!black},
  urlcolor = magenta,
  }

% theorem styles
\theoremstyle{plain} % default; italic text, extra space above and below
\newtheorem{theorem}{Theorem}[section]
\newtheorem{proposition}{Proposition}[section]
\newtheorem{lemma}{Lemma}[section]
\newtheorem{corollary}{Corollary}[theorem]

\theoremstyle{definition} % upright text, extra space above and below
\newtheorem{definition}{Definition}[section]
\newtheorem{example}{Example}[section]

\theoremstyle{remark} % upright text, no extra space above or below
\newtheorem{remark}{Remark}[section]
\newtheorem*{note}{Note} %'Notes' in italics and without counter 

% renewcommands for counters
\newcommand{\propositionautorefname}{Proposition}
\newcommand{\definitionautorefname}{Definition}
\newcommand{\lemmaautorefname}{Lemma}
\newcommand{\remarkautorefname}{Remark}
\newcommand{\exampleautorefname}{Example}

% For exercise and solutions
% \newif\ifshowsolutions
% \showsolutionstrue % Change to \showsolutionsfalse to hide solutions
%
% Custom environments for problems and solutions
% \newcounter{exercise}[chapter]
% \newenvironment{problem}[1][]
%   {\par\noindent\textbf{#1.}}
%   {\par}
%
% \newenvironment{solution}
%   {\ifshowsolutions \expandafter\solutioncontent \else \expandafter\comment \fi}
%   {\ifshowsolutions \hfill \qedsymbol \else \expandafter\endcomment \fi \vspace{1em}}
%
% \newenvironment{solutioncontent}
%   {\par\noindent\textit{Solution.}}
%   {\par}
\addbibresource{articles.bib}


\begin{document}

\title{Abstract Algebra (MATH6302), Fall 2024 \\ Homework Assignment 2}

% author list
\author{
Joel Sleeba \\
}

\maketitle
\printanswers
\unframedsolutions

\begin{questions}

  \question
  \begin{solution}
    Consider $\mathcal{A} = \{ \{ 1, 2, 3 \}, \{ 2, 5 \} \} \cup \{  \{ x \},  \ x \in \mathbb{Z} \}$. We define a relation on $\mathbb{Z}$ where we claim $x \sim y$ if there exists $A \in \mathcal{A}$ such that $x, y \in A$. We claim this is reflexive and symmetric but not transitive.
    \begin{itemize}[]
      \item (Reflexivity) Let $a \in \mathbb{Z}$. Then since $\{ a \} \in \mathcal{A}$, we get $a \sim a$ and we are done.
      \item (Symmetry) Let $x \sim y$. Then there exists some $A \in \mathcal{A}$ with $x, y \in A$. This implies $y, x \in A$ and we get $ y \sim x$.
      \item (Not Transitive) $1 \sim 2$ since $ \{ 1, 2, 3 \} \in \mathcal{A}$ and $2 \sim 5$ since $\{ 2, 5 \} \in \mathcal{A}$. But $1 \not \sim 5$ since there are no elements in $\mathcal{A}$ containing $1$ and $5$.
    \end{itemize}
  \end{solution}


  \question
  \begin{solution}
    We will show that the relation defined is reflexive, symmetric and transitive. \begin{itemize}[]
      \item (Reflexivity) $(a, b) \sim (a, b)$ since $ab - ba = 0$ for all $a, b \in \mathbb{Z}$.
      \item (Symmetry) Let $(a, b) \sim (c, d)$. Then $ad-bc = 0$, by the definition of the relation $\sim$. This implies $cb - da = 0$ by the commutativity of addition of multiplication in $\mathbb{Z}$, which is equivalent to $(c, d) \sim (a, b)$.
      \item (Transitivity) Let $(a, b) \sim (c, d)$ and $(c, d) \sim (e, f)$ where $b, d, f \neq 0$. Then $ad-bc = 0$ and $cf- de = 0$. Since $d \neq 0$ by assumption,  $adf = bcf = bed$ implies $af = be$ which gives $(a, b) \sim (e, f)$.
    \end{itemize}
  \end{solution}

  \question
  \begin{solution}
     \begin{enumerate}[label=(\alph*)]
       \item Cyclic. $C_2 \times C_5 \cong C_{10}$. $(1, 1)$ generates the whole group.
       \item Not cyclic. Assume it is with generator $(a, b)$. Now since we know that $a, b \in C_4$, we get  $a^4 = b^4 = 0$ (identity element of $C_4$). This gives that $(a, b)^4 = (0, 0)$, the identity element of $C_4 \times C_4$. Because our choice of $(a, b)$ was arbitrary, this gives that the order of every element is at most $4$. Since $C_4 \times C_4$ has 16 elements, this contradicts our assumption that it is cyclic.
       \item Cyclic. $\forall n \in \mathbb{N}, n = \underbrace{1+1+\cdots+1}_{n \textrm{ times}}$. So, $1$ generates the group
       \item Not cyclic. Assume $a \in \mathbb{Q}$ be a positive rational number and consider $\langle  a \rangle $, the subgroup generated by $a$ which is precisely $ \{ na \ : \ n \in \mathbb{Z} \}$. Now $a/2$ is again a rational  but $a/2 \notin \langle a \rangle$. Hence $\mathbb{Q}$ is not cyclic.
       \item Not cyclic. Assume it is and suppose that $(a, b) \in \mathbb{Z} \times \mathbb{Z}$ is the generator. Then $(k_1a, k_2b) \not\in \langle (a, b) \rangle$ if $(k_1, k_2) = 1$
       \item Cyclic. Since $18 = 2\times3^2$, by the primitive root theorem we get that $Z_{18}^{*}$ is cyclic.
       \item Not cyclic. Since $36 = 4 \times 3^2$, by the primitive root theorem we get that it is not cyclic.
       \item Not cyclic. Since $A \Delta A = \emptyset$ for all set $A$, unless $\mathcal{P}(S)$ contains only two elements, it won't be cyclic. But $|P(S)| = 2$ if and only if $|S| = 1$.
     \end{enumerate}
  \end{solution}


  \question
  \begin{solution}
    $\mathcal{P}(\{ 1, 2 \}) = \{  \Phi, \{ 1 \}, \{ 2 \}, \{ 1, 2 \} \}$. Moreover from last assignment, we see that $(\mathcal{P}(\{ 1, 2 \}), \Delta)$ is a group. Since it has 4 elements, are there are only two distinct groups of order 4 upto isomorphism (namely the cyclic group of order 4, and the Klein 4 group), $(\mathcal{P}(\{ 1, 2 \}), \Delta)$ must be isomorphic to either one of them. 

    It is clear that the identity element of this group must be $\Phi$, since $A \Delta \Phi = A$ for all subgroup $A$ of $\{ 1, 2 \}$. Moreover we see that $A \Delta A = (A \cup A) \setminus (A\cup A)= A \setminus A = \Phi$. Hence every element in the group $(\mathcal{P}(\{ 1, 2 \}), \Delta)$ is its own inverse. Since we know this is a property of the Klein 4 group, we get that $(\mathcal{P}(\{ 1, 2 \}), \Delta)$ is isomorphic to $V_4$, the Klein group of order 4.
  \end{solution}


  \question
  \begin{solution}
    Recall that if $G_1, G_2, \ldots , G_n$ are groups, then their direct product $\mathcal{G} = G_1 \times G_2 \times \cdots \times G_n$ is a group under the operation $(a_1, a_2, \ldots , a_n)( b_1, b_2, \ldots , b_n) = (a_1b_1, a_2b_2, \ldots , a_nb_n)$ with identity element $(e_1, e_2, \ldots, e_n)$ where each $e_j$ is the identity element in $G_j$.

    ($\implies$) Assume that $\mathcal{G}$ is an Abelian group. Let $1 \le i \le n$, we will show that $G_i$ is Abelian. Let $g, h \in G_i$. Consider the corresponding elements $\tilde{g} = (e_1, e_2, \ldots, e_{i-1}, g, e_{i+1}, \ldots  ,e_n)$ and $\tilde{h} = (e_1, e_2, \ldots, e_{i-1}, h, e_{i+1}, \ldots  ,e_n)$ in $\mathcal{G}$. Since we know that $\mathcal{G}$ is Abelian, we get $\tilde{ g}\tilde{h} = \tilde{h}\tilde{g}$. This by the definition of multiplication implies $$(e_1, e_2, \ldots, e_{i-1}, gh, e_{i+1}, \ldots  ,e_n) = (e_1, e_2, \ldots, e_{i-1}, hg, e_{i+1}, \ldots  ,e_n)$$
    which implies $gh = hg$. Now since $g, h \in G_i$ was arbitrary and $1 \le i \le n$ was arbitrary, we get that $G_i$ is Abelian for all $1 \le i \le n$.

    ($\impliedby$) Conversely, if each $G_i$ is Abelian, then for $\tilde{g} = (g_1 , g_2 , \ldots , g_n), \tilde{h} = (h_1 , h_2 , \ldots , h_n) \in \mathcal{G}$, \begin{align*}
      \tilde{g}\tilde{h} &= (g_1h_1, g_2h_2, \ldots, g_nh_n) \\
      &=(h_1g_1, h_2g_2, \ldots h_ng_n) \\
      &= \tilde{h}\tilde{g}
    \end{align*}
    which shows $\mathcal{G}$ is Abelian.
  \end{solution}

  \question
  \begin{solution}
    We will show that the relation is reflexive, symmetric, and transitive.
    \begin{itemize}[]
      \item (Reflexivity) Let $g \in C_n$, then $g^{-1}g = e \in H$ since $ e \in H$.
      \item (Symmetry) Let $g \sim h$. Then $ g^{-1}h \in H$. Since $H$ is a subgroup, $(g^{-1}h)^{-1} = h^{-1}g \in H$ which implies $ h \sim g$.
      \item (Transitivity) Let $f \sim g$ and $g \sim h$. Then $f^{-1}g, g^{-1}h \in H$ and $f^{-1}h = f^{-1}(gg^{-1})h = (f^{-1}g)(g^{-1}h) \in H$. Hence $f \sim h$.
    \end{itemize}
  \end{solution}

  \question
  \begin{solution}
    \begin{enumerate}[label=(\alph*)]
       \item We know that an $x \in \mathbb{Z}_n := \mathbb{Z}/n\mathbb{Z}$ is in $\mathbb{Z}_n^{*}$ if and only if there exists an $y \in \mathbb{Z}_n$ with $xy\equiv1\mod n$, which is equivalent to $\gcd(x, n) = 1$. Hence $$\mathbb{Z}_n^{*} = \{ x \in \mathbb{Z}_n  \ : \ \gcd(x, n) = 1  \}$$
         Then by the definition of Euler-totient function, we get that the cardinality of $\mathbb{Z}_n^{*}$ is $\phi(n)$. Specifically for this question, it would be $\phi(900)$. Since $900 = 2^23^25^2$, we get that $ \phi(900) = 2(2-1)3(3-1)5(5-1) = 240$. Hence $|G| = 240$

       \item Since $\gcd(11, 900) = 1$, we see that $11 \in \mathbb{Z}_{900}^{*}$. So $11^{-1} \in \mathbb{Z}_{900}^{*}$. To find $11^{-1}$, we use the reverse Euclidean algorithm. We see that \begin{align*}
           900 &= 11\times81 + 9 \\ 
           11 &= 9\times1 + 2 \\ 
           9 & = 2\times4 + 1
       \end{align*}
       so reversing it we get \begin{align*}
         1 &= 9 - 2\times4 \\ 
         & = 9 - (11 - 9)\times 4 \\ 
         & = 9\times5 - 11\times4 \\ 
         & = (900 - 11 \times 81) \times 5 - 11 \times 4 \\ 
         & = 900\times 5 + 11 \times (-409) \\
       \end{align*}
      So $11^{-1} \equiv -409 \mod 900 = 419 \mod 900$. Hence $11^{-1} = 419$ in $Z_{900}^{*}$. 
     \end{enumerate}
  \end{solution}


  \question
  \begin{solution}
    The equation is $5457x \equiv 3317 \mod 5885$. We find the gcd of 5885 and 5457 using the Euclidean algorithm. \begin{align*}
      5885 &= 5457 \times 1 + 428 \\ 
      5457 &= 428\times 12 + 321 \\ 
      428 &= 321\times1 + 107 \\ 
      321 &= 107\times3
    \end{align*}

    Hence we see that $\gcd(5885, 5457) = 107$ and $107|3317$ as $3317 = 107\times31$. Hence dividing the whole equation by the common denominator, we see that solving $5457x \equiv 3317 \mod 5885$ is equivalent to solving $$51 x \equiv 31 \mod 55$$
      Hence we use the euclidean algorithm again, \begin{align*}
        55 &= 51 \times 1 + 4\\ 
        51 &= 4\times 12 + 3 \\ 
        4 &= 3\times1 + 1 \\ 
        3 &= 1\times3
      \end{align*}
      Reversing this gives us \begin{align*}
        1 &= 4 - 3 \\ 
        &= 4 - (51 - 4\times12) \\ 
        &= 4\times13 - 51 \\ 
        &= (55-51)\times13 - 51 \\ 
        &= 55\times13 - 51\times14 \\
        &= 55\times13 + 51\times(-14)
      \end{align*}

      So we see that $51^{-1} \equiv -14 \mod 55$ and $-14 \equiv 41 \mod 55$. Hence the primitive modulo class of 41 is the solution for the equation $51 x \equiv 1 \mod 51$.
      So multiplying the equation $51x \equiv 31 \mod 55$ with $55^{-1} = 41$, we see that $ x \equiv 31\times 41 \mod 55 = 1271 \mod 55  = 6 \mod 55$. Hence the solution set for the original equation is $ \{ 55n + 6 \ : \ n \in \mathbb{Z} \}$
  \end{solution}

  \question
  \begin{solution}
    Since $101, 103, 107$ are coprimes to each other, chineese remainder theorem gives that there is a unique solution for $x$ in the $Z_{101\times103\times107} = \mathbb{Z}_{1113121}$.

    Now $x \equiv 43 \mod 101$ implies $x = 101k_1 + 43$. Substituting this to the next equation, we get $101k_1 + 43 \equiv 10\mod 103$ which is equivalent to $k_1 \equiv (101^{-1})70 \mod 103$. Now using the reverse Euclidean algorithm, we find the inverse of $101$ in $Z_{103}$. Since $1 = 101 \times 51 + 103 \times(-50)$, we see that $101^{-1} = 51 \in \mathbb{Z}_{103}$. Therefore $k_1 \equiv 51\times70 \equiv 68\mod 103$. Hence $k_1 = 104k_2 + 68$ and $x = 101(103k_2 + 68) + 43 = 10403k_2 + 6911$.

    Now we substitute this to the next equation to get $10403k_2 + 6911 \equiv 96 \mod 107$ which is eqiuvalent to $k_2 \equiv(10403^{-1}) 33 \mod 107$. Now similarly using the reverse euclidean algorithm, we find that $1 = 4764 \times107 + 10403(-49)$. Hence we see that $10403^{-1} = -49 \equiv 58 \mod 107$. Hence $ k_2 \equiv 58\times33 \equiv 1914 \equiv 95 \mod 107$. Therefore $k_2 = 107n + 95$ and $x = 10403(107n + 95) + 6911 = 113121n + 995196$.

    Hence the solution set of the system of equations is $ \{ 113121n + 995196,   n \in \mathbb{N} \}$.
  \end{solution}

  \question
  \begin{solution}
    Given that $10$ is a primitive root modulo 313. This gives that $\langle  10  \rangle = Z_{313}^{*}$. Moreover since we know that $313$ is a prime, $\phi(313) = 312$ and therefore $|10| = 312$. Therefore if $x = 10^a \in  \langle 10 \rangle$ with $x^3 = 1 \mod 313$, then either $x = 1$ or $|10^a| = 3$. But we know that for any cyclic group with generator $g$, $|g^a| = \frac{|g|}{(a, |g|)}$. Therefore if $$|10^a| = \frac{|10|}{(a, |10|)} = \frac{312}{(a, 312)} = 3$$
    one must have $(a, 312) = \frac{312}{3} = 104$. The only possible candidates for $a$ are $104, 208$. Hence the residue classes which satisfy the given equation are $[1], [10^{104}], [10^{208}]$. Now using the usual theatrics, we see that this is exactly $[1], [214]$ and $[98]$ respectively.
  \end{solution}

  \question
  \begin{solution}
    We see that $7^4 = 2401$ has its ones digit equal to 1. Therefore $7^{4n} = (7^4)^n = (2401)^n$ must have its ones digit equal to 1 for all $n \in \mathbb{N}$. By the same logic we see that the ones digit of $7^{4n+1}, 7^{4n+2}, 7^{4n+3}$ must be $7, 9$, and $3$ respectively. Hence to find the ones digit of $7^{7^{7^7}}$, we just need to find out the residue class of $7^{7^7} = 7^{49}$. $49 = 32 + 16 + 1$ \begin{itemize}[]
      \item $7^1 \mod 4 = 3$
      \item $7^2 \mod 4 = (7 \mod 4)(7 \mod 4) = 3^{2} \mod 4 = 1 $
      \item $7^4 \mod 4 = (7^2 \mod 4)(7^2 \mod 4) = 1^{2} \mod 4 = 1 $
    \end{itemize}
    Now since we are going to keep multiplying by 1 while finding the residue classes of 7 rasied to higher powers of 2, we conclude that $7^{16}, 7^{32}$ both lie in the residue class of $1 \mod 4$. Hence $7^{49} \mod 4 = 1\times1\times3 \mod 7 = 3$. Therefore $7^{7^{7}} = 4n +3$ for some $n \in \mathbb{N}$, and therefore by our previous reasoning, we see that the ones digit of $7^{7^{7^{7}}}$ is 3.
  \end{solution}

  \question
  \begin{solution}
    We see that 1074 = 1024 + 32+ 16 + 2. \begin{itemize}[]
      \item $8^2 \mod 211 = 64 \mod 211$
      \item $8^4 \mod 211 = 4096 \mod 211 = 87$
      \item $8^8 \mod 211 = (8^4 \mod 211)(8^4 \mod 211) = 87^2 \mod 211 = 184$
      \item $8^{16} \mod 211 = (8^8 \mod 211)(8^8 \mod 211) = 184^2 \mod 211 = 96$
      \item $8^{32} \mod 211 = (8^{16} \mod 211)(8^{16} \mod 211) = 96^2 \mod 211 = 143$
      \item $8^{64} \mod 211 = (8^{32} \mod 211)(8^{32} \mod 211) = 143^2 \mod 211 = 193$
      \item $8^{128} \mod 211 = (8^{64} \mod 211)(8^{64} \mod 211) = 193^2 \mod 211 = 113$
      \item $8^{256} \mod 211 = (8^{128} \mod 211)(8^{128} \mod 211) = 113^2 \mod 211 = 109$
      \item $8^{512} \mod 211 = (8^{256} \mod 211)(8^{256} \mod 211) = 109^2 \mod 211 = 65$
      \item $8^{1024} \mod 211 = (8^{512} \mod 211)(8^{512} \mod 211) = 65^2 \mod 211 = 5$
    \end{itemize}

    Therefore $8^{1074}\mod 211 = 5\times 143 \times 96 \times  64 \mod 211 = 4392960 \mod 211 = 151$
  \end{solution}

  \question
  \begin{solution}
    Using Bezout's lemma we see that $\gcd(a, n) = \gcd(a +kn , n)$ for all $k \in \mathbb{Z}$. Hence the statement we have to prove is equivalent to showing $a \in \mathbb{Z}_n$ is a generator for $\mathbb{Z}_n$ if and only if $(a, n) = 1$. ($a$ here is assumed to be the smallest positive integer in the corresponding residue class $[a]$, and we will continue this convention)

    ($\implies$)
    Let $a \in \mathbb{Z}_n$  with $(a, n) = d \neq 1$. Then for $k = \frac{n}{d} > 1$ (still an integer), we get $ak = \frac{an}{d}$, Since $d|a$, we get that $n|ak$ which gives that $ak \mod n = 0$. Therefore the strict subset $\{ a, 2a, \ldots, (k-1)a, 0=ka \}$ is closed under modular addition, which makes it a proper subgroup. Thus we see that $a$ cannot generate $\mathbb{Z}_n$

    ($\impliedby$) Conversely, if $(a, n) = 1$ then by Bezout's lemma there exists $k_1, k_2$ with $(k_1, k_2) = 1$ such that $ak_1 + nk_2 = 1$. which implies $ak_1 \equiv 1 \mod n$. This implies $ak_1$ is the equivalent class of $1$. Now since we know $[1]$ is a generator for $\mathbb{Z}_n$, we see that $a$ generate $ \mathbb{Z}_n$.
  \end{solution}

  \question
  \begin{solution}
    Since $103$ is a prime we see that $Z_{103}^{*}$ has 102 elements and that it is a cyclic group. Let $g$ be a generator of the group. If $g^{a}$ is any other generator for $1 \le a \le 102$, we must have $|g^a| = \frac{|g|}{(a, |g|)} = \frac{102}{(a, 102)} = 102 $ which gives $(a, 102) = 1$. There is exactly $\phi(102) = 32$ such $a$ by the definition of the Euler-totient function.
  \end{solution}

\question
\begin{solution}
  Let Graham's number $g = 3^{b_1}$. We should find out $x \in Z_{121}$ such that $x \equiv g \mod 121$. Since $(3, 121) = 1$ and $\phi(121) = 110$, we see that if we can write $b_1 = 110\times q + b_1$, then using Fermat's little theorem we'll get $g = 3^{110q + r} \mod 121 \equiv (3^{110} \mod 121)^q(3^r \mod 121) = 3^r \mod 121$.

  As a general rule of thumb, we get that if $(3, n) = 1$, then $3^a \mod n \equiv 3^{r} \mod n$, where $a = \phi(n) q + r$. Now we are at a place to proceed with our calculations.
  \begin{itemize}[]
    \item Let $g = 3^{b_1}$ be the Graham's number. Since $(121, 3) = 1$ and $\phi(121) = 110$, $3^{b_1} \equiv 3^{r_1} \mod 121$ where $r_1 = b_1 \mod 110$
    \item Now let $b_1 = 3^{b_2}$. Since $(110, 3) = 1$ and $\phi(110) = 40$, $3^{b_2} \equiv 3^{r_2} \mod 110$ where $r_2 = b_2 \mod 40$
    \item Now let $b_2 = 3^{b_3}$. Since $(40, 3) = 1$ and $\phi(40) = 16$, $3^{b_3} \equiv 3^{r_3}\mod 40$ where $r_3 = b_3 \mod 16$
    \item Now let $b_3 = 3^{b_4}$. Since $(16, 3) = 1$ and $\phi(16) = 8$, $3^{b_4} \equiv 3^{r_4}\mod 16$ where $r_4 = b_4 \mod 8$
    \item Now let $b_4 = 3^{b_5}$. Since $(8, 3) = 1$ and $\phi(8) = 4$, $3^{b_5} \equiv 3^{r_5}\mod 8$ where $r_5 = b_5 \mod 4$
  \end{itemize}

  Now since $3\mod 4 = -1$ and $b_5$ is an odd number being the odd power $3$, we see that $3^{b_5} = -1 \mod 4 = 3 \mod 4$. Hence $r_5 = 3$, Tracing the argument back, we get  \begin{itemize}[]
    \item  $r_4 = 3^{r_5} \mod 8 = 3^3 = 27 \mod 8 = 3$
    \item  $r_3 = 3^{r_4} \mod 16 = 3^3 = 27 \mod 16 = 11$
    \item  $r_2 = 3^{r_3} \mod 40 = 3^{11} \mod 40 = 27$
    \item  $r_1 = 3^{r_2} \mod 110 = 3^{27} \mod 110 = 97$
    \item  $g = 3^{r_1} \mod 121 = 3^{97} \mod 121 = 9$
  \end{itemize}

  Hence $x = 9$ is the required answer.


  % Grahams number is somewhere in the power tower of 3. Like $3^{3^{\cdot ^{\cdot ^{3}}}}$. We will denote a power tower of consisting of $n$ number of 3s by $3\uparrow n$. That is $3\uparrow 0 = 3,\ 3\uparrow1 = 3^3,\ 3 \uparrow 2 = 3^{3^{3}}$ and so on.
  %
  % We know that \begin{itemize}[]
  %   \item $3^{4k} \equiv 1 \mod 10$
  %   \item $3^{4k + 1} \equiv 3 \mod 10$
  %   \item $3^{4k+2} \equiv 9 \mod 10$
  %   \item $3^{4k+3} \equiv 7 \mod 10$
  % \end{itemize}
  %
  % Hence if $3 \uparrow n \equiv k \mod 10$, then $k$ is completely determined by the residue class of the $3\uparrow (n-1)$ in $\mathbb{Z}_4$. But again \begin{itemize}[]
  %   \item $3^{2k} \equiv 1 \mod 4$
  %   \item $3^{2k+1} \equiv 3 \mod 4$
  % \end{itemize}
  %
  % So if $3\uparrow (n-1) \equiv k_1 \mod 4$, then $k_1$ is completely determined by whether $3\uparrow (n-2)$ is even or odd. But since $3=3\uparrow0$ is odd, $3^3 = 3 \uparrow 1$ is odd and similarly by induction $3\uparrow (n-2)$ is odd. Hence we see that for $n \ge 1$, $3\uparrow (n-1) \equiv 3 \mod 4$. And finally retracing our implications, we conclude that for all $n\ge 1$, $3\uparrow n \equiv 7 \mod 10$. 
  %
  % Now if we represent Graham's number as $3 \uparrow n_o$, clearly $n_o \ge 1$. Hence the unit digit of Graham's number is 7.
\end{solution}

  \question
  \begin{solution}
    If $y = 1 \mod 9797$ that implies $(y, 9797) = 1$. Since $9797 = 101 \times 97$, this is equivalent to $(y, 97) = 1$ and $(y, 101) = 1$. Hence we look for $x^3$ which simultaneously satisfy $x^3 = 1 \mod 97$ and $x^3 = 1 \mod 101$. Since we know that 97 and 101 are primes, we get that $Z_{97}^{*}, Z_{101}^{*}$ are cyclic groups with cardinality 96 and 100 respectively. 

    Since we know that the order of an element in a group must divide the order of the group, and $3 \not |100$, we get that there are no elements of order 3 in $Z_{101}^{*}$. Hence we get that the only element in $Z_{101}^{*}$ with $x^3 = 1 \mod 101$ is $x = 1$.

    Similarly, if $g \in Z_{97}^{*}$ generate the group, and $|g^k| = 3$, then we must have $ \frac{|g|}{(k, |g|)} = \frac{96}{(k, 96)} = 3$ which gives $(k, 96) = 32$. Hence the possible values for $a$ are 32 and 64. Since we know that $5$ is a generator for $Z_{97}^{*}$, we get that the elements in $x \in Z_{97}^{*}$, with $x^3 = 1 \mod 97$ are specifically $35 = 5^{32} \mod 97$ and $61 = 5^{64} \mod 97$.

    Now to find the $x$ which satisfy $x^3 = 1 \mod 9797$, we will use the Chineese remainder theorem to solve the 3 different system of linear equations. \begin{align*}
      &x = 1 \mod 97& \quad &x = 35 \mod 97& \quad &x = 61 \mod 97 \\
      &x = 1 \mod 101& \quad  &x = 1 \mod 101& \quad &x = 1 \mod 101
    \end{align*}

    Now using the chinese remainder theorem, we get that the solutions are $1, 5758, 1516 \in Z_{9797}^{*}$

  \end{solution}

\end{questions}
\printbibliography[heading=bibintoc]
\end{document}
