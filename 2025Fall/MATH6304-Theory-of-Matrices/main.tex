\documentclass[12pt]{article}

\usepackage{geometry} % automatic papersizes, margins.
\usepackage[utf8x]{inputenc} % utf8 support for LaTeX
\usepackage{makeidx} % 'makeidx' make and show index
\usepackage{enumitem} % itemize, enumerate, description.
\usepackage{hyperref} % hyperlinks, cross-references.
\usepackage{xcolor} % foreground and background color management.
% Better color mixing compared to 'color'
\usepackage{graphicx} % provide options for \includegraphics. Builds
% on 'graphic'
\usepackage{caption} % better control over captions of figures and equations.
\usepackage{appendix} % extra control over appendix
\usepackage[backend=biber, style=alphabetic]{biblatex} % better than
% bibtex, people say.
\usepackage{tocbibind} % add ToC/Bibliography/Index to ToC

\usepackage{amsmath} % math symbols, matrices, cases, trig functions,
% var-greek symbols.
\usepackage{amsfonts} % mathbb, mathfrak, large sum and product symbols.
\usepackage{amssymb} % extended list of math symbols from AMS.
% https://ctan.math.washington.edu/tex-archive/fonts/amsfonts/doc/amssymb.pdf
\usepackage{amsthm} % theorem styling.
\usepackage{mathrsfs} % mathscr fonts.
\usepackage{yhmath} % widehat.
\usepackage{empheq} % emphasize equations, extending 'amsmath' and 'mathtools'.
\usepackage{bm} % simplified bold math. Do \bm{math-equations-here}
\usepackage{tikz} % for tikz diagrams
\usepackage{tikz-cd} % commutative diagrams.
\usepackage{marginnote} % For sidenotes

\geometry{
  a4paper, % 'a4paper', 'c5paper', 'letterpaper', 'legalpaper'
  asymmetric, % don't swap margins in left and right pages. as
  % opposed to 'twoside'
  centering, % to center the content between margins
  bindingoffset=0cm,
}

\hypersetup{
  colorlinks,
  linkcolor={blue!50!black},
  citecolor={blue!50!black},
  urlcolor={blue!80!black}
}

\theoremstyle{plain} % default; italic text, extra space above and below
\newtheorem{theorem}{Theorem}[section]
\newtheorem{proposition}[theorem]{Proposition}
\newtheorem{lemma}[theorem]{Lemma}
\newtheorem{corollary}[theorem]{Corollary}
\newtheorem{problem}[theorem]{Problem}

\theoremstyle{definition} % upright text, extra space above and below
\newtheorem{definition}[theorem]{Definition}
\newtheorem{example}[theorem]{Example}
\newtheorem{exercise}[theorem]{Exercise}
\newtheorem{question}[theorem]{Question}

\theoremstyle{remark} % upright text, no extra space above or below
\newtheorem{remark}[theorem]{Remark}
\newtheorem{answer}[theorem]{Answer}
\newtheorem*{note}{Note} %'Notes' in italics and without counter

\newcommand{\propositionautorefname}{Proposition}
\newcommand{\lemmaautorefname}{Lemma}
\newcommand{\corollaryautorefname}{Corollary}
\newcommand{\problemautorefname}{Problem}
\newcommand{\definitionautorefname}{Definition}
\newcommand{\exampleautorefname}{Example}
\newcommand{\exerciseautorefname}{Exercise}
\newcommand{\questionautorefname}{Question}
\newcommand{\remarkautorefname}{Remark}
\newcommand{\answerautorefname}{Answer}
\newcommand{\noteautorefname}{Note}

\addbibresource{~/Books/Research/research.bib}

\begin{document}
\title{MATH 6304 - Theory of Matrices}

\author{
  Joel Sleeba \\
  University of Houston \\
  joelsleeba1@gmail.com \\
}

\maketitle

\pagenumbering{roman} \setcounter{page}{2}
\tableofcontents
\pagenumbering{arabic} \setcounter{page}{1}

\pagebreak

% TeX_root = ../main.tex

\chapter{Banach Spaces}

\textbf{Textbook :} A Course in Functional Analysis, John Conway

Functional analysis is the study of Topological Vector Spaces.

\begin{definition}
  Let $X$ be a vector space (over $\mathbb{R}$ or $\mathbb{C}$). A
  seminorm on $X$ is a map $\|\cdot \|: X \to [0, \infty)$ such that
  \begin{itemize}
    \item  $\|\alpha x\| = |\alpha|\| x \|, \forall \alpha \in
      \mathbb{F}, \forall x \in X$
    \item $\|x+y\| \le \|x\| + \|y\|$
  \end{itemize}
  In addition if $\forall x\neq 0, \|x\| \neq 0$, we say $\|\cdot\|$
  is a norm on $X$
\end{definition}

Norm induces a metric $ d(x, y) = \|x-y\|$

\begin{note}
  Let $X$ be a normed space. Then the maps
  \begin{itemize}
    \item $+: X \times X \to X : (x, y) \to x+y$
    \item $\cdot: \mathbb{F} \times X \to X: (\alpha, x) \to \alpha x$
  \end{itemize}
  are continuous.
\end{note}

Hence every normed space is a topological vector space.

\begin{example}
  $\mathbb{F}^n$ with $\ell_p$-norm defined as \[
    \|(\alpha_1, \alpha_2, \ldots, \alpha_n)\|_p = \Big(\sum_{i =
    1}^{n} |a_i|^p\Big)^{ \frac{1}{p} }
  \]
\end{example}

\begin{example}
  $\mathbb{F}^n$ with $\ell_\infty$-norm defined as \[
    \|(\alpha_1, \alpha_2, \ldots, \alpha_n)\|_\infty = \max \{ |a_i| \}
  \]
\end{example}

\begin{example}
  Consider $C_{00} = \{ (a_n)_{n \in \mathbb{N}}  \ : \   a_n \in
    \mathbb{F}, \forall n \in \mathbb{N}, a_n = 0  \textrm{ except for
  finitely many }  n \in \mathbb{N}\}$ which can be identified by
  collection of functions $  f: \mathbb{N} \to \mathbb{F}$ with finite support.

  Then $$\|(a_n)\|_p = \Big(\sum_{n = 1}^{\infty} |a_n|^p\Big)^{ \frac{1}{p} }$$
  is a norm on $C_{00}$
\end{example}

\begin{proposition}
  \label{EquivalentDefnsofContinuity}
  Let $X, Y$ be normed space, and let $T: X \to Y$ be linear. Then
  the following are equivalent.
  \begin{itemize}
    \item $T$ is continuous
    \item $T$ is continuous on 0
    \item $T$ is continuous on any point $x \in X$
    \item $\exists M >0$ such that $\|T(x)\|_Y \le M \|x\|_X$ for all $x \in X$
  \end{itemize}
\end{proposition}
\begin{proof}
  ($1 \implies 2$) It is clear that if $T$ is continuous, then it is
  continuous at $0$ from the definition of continuity.

  $(2 \implies 3)$ Let $x \in X$ and $\{ x_n \}_{n \in \mathbb{N}}$
  be any sequence in $X$ that converge to $ x$. Then the sequence $\{
  y_n = x_n - x \}$ converge to zero by the algebra of limits. By the
  continuity of $T$ at zero, $\{ T(y_n) = T(x_n) - T(x) \}$ converge
  to $0$. Therefore $T(x_n) \to T(x)$. And this shows $T$ is
  sequentially continuous at $x \in X$. Since the space is a metric
  space, sequential continuity is equivalent to continuity.

  $(4 \implies 2)$ Let $x \in X$. Then $\|T(0) - T(x)\| = \|T(x)\|
  \le M \|x\| = M \|0 - x\|$. Hence $T$ is continuous at $0$.

  $(3 \implies 1)$

  $(2 \implies 4)$

\end{proof}

\begin{example}
  Let $T: \mathbb{F}^n \to \mathbb{F}^n$ be defined as $T(\alpha_1,
  \alpha_2, \ldots, \alpha_n) = (\alpha_1, 0, \ldots, 0)$. Is $T$
  convergent for any norm $ \|\cdot\|_1, \|\cdot\|_2$ in the domain and range?

  % No. Let $\|\cdot\|_1 = \|\cdot\|_$
\end{example}
\begin{proof}
  \textcolor{red}{verify}
\end{proof}

\begin{example}
  Consider identity function $I: C_{00} \to C_{00}$. Let the norm in
  domain be $ \|\cdot\|_\infty$ and that in range be $\|\cdot\|_1$.
  Is the function continuous? What if the norms in domain and range
  are switched?
\end{example}
\begin{proof}
  \textcolor{red}{verify}
\end{proof}

\begin{note}
  Let $X$ be a space with two norms $\|\cdot\|_1, \|\cdot\|_2$. When
  is the two norms topologically equivalent?

  When $\exists M, M^\prime$ such that $\|x\|_1 \le M \|x\|_2$ and
  $\|x\|_2 \le M^\prime \|x\|_1$
  Equivalently, when the identity map between the two spaces with
  their respective norms are bi-continuous. (See 4th equivalent
  statement of previous proposition)
\end{note}
\begin{theorem}
  Let $X$ and $Y$ be normed spaces, and $T: X \to Y$ be linear.
  Assume $X$ is finite dimensional. Then $T$ is continuous.
\end{theorem}
\begin{proof}
  Since $T(X) \le Y$ is finite dimensional, we may assume without
  loss of generality that $Y$ is also finite dimensional and $T$ is
  onto. Let $\{ x_1, x_2, \ldots x_n \}$ be a basis for $X$. Define
  another norm on $X$ as follows. For every $x  = \sum_{i = 1}^{n}
  \alpha_i x_i \in X$, \[
    \|x\|^\prime = \sum_{i = 1}^{n} |\alpha_i| (\|T(x_i)\| + \|x_i\|)
  \]
  \textcolor{red}{verify that this is a norm}. Then for every $x \in
  X$, we have \[
    \|T(x)\| \le \sum_{i = 1}^{n} |\alpha_i|\|T(x_i)\| \le \|x\|^\prime
  \]

  Hence $T$ is bound with respect to the norm $\|\cdot\|^\prime$ on
  $X$, since all norms are equivalent on $X$. Therefore $T$ is
  continuous w.r.t to the original norm on $X$.
\end{proof}

\begin{corollary}
  Let $X$ be a finite dimensional vector space. Then any two norms in
  $X$ are equivalent.
\end{corollary}
\begin{proof}
  Let $\{ e_1, e_2, \ldots e_n \}$ be a basis for $X$. For each $x =
  \sum_{i = 1}^{n} \alpha_i e_i \in X$, define \[
    \|x\|_\infty = \max \{ |\alpha_i| \}
  \]
  Then $\|\cdot\|_\infty$ is a norm and we'll show every norm on $X$
  is equivalent to this norm. Let $\|\cdot\|$ be an arbitrary norm on
  $X$. For each $x = \sum_{i = 1}^{n} \alpha_i e_i \in X$, we have
  \begin{align*}
    \|x\| &= \|\sum_{i = 1}^{n} \alpha_i e_i\| \\
    & \le \sum_{i = 1}^{n} |\alpha_i|\|e_i\| \\
    &\le \max \{ |\alpha_i| \} \sum_{i = 1}^{n} \|e_i\| \\
    & \le \|x\|_\infty \sum_{i = 1}^{n} \|e_i\|
  \end{align*}

  % Moreover, we have \[
  %   \|x\|_\infty \le \Big(\sum_{i = 1}^{\infty} |\alpha_i|^2\Big)^{
  % \frac{1}{2} } = \|x\|_2
  % \]
  Therefore the identity map $I: (X, \|\cdot\|_\infty) \to (X,
  \|\cdot\|)$ is continuous. Since the set $ K = \{ x \in X \ :
  \ \|x\|_\infty \le 1 \}$ is compact,  K is also compact in $(X,
  \|\cdot\|)$. Moreover if $V \subset K$ is closed
  in $\|\cdot\|_\infty$, then $V$ will also be closed
  in $ \|\cdot\|$ being compact. Hence $I|_K$ will be an open map.
  Now its an easy application of \autoref{EquivalentDefnsofContinuity}.
\end{proof}


% TeX_root = ../main.tex

\marginnote{\scriptsize 16/01/2025 }

\begin{definition}
  An ideal $I \subsetneq \mathcal{A}$ is called maximal if for any
  ideal $I \subset J$, then either $J = I$ or $J = \mathcal{A}$
\end{definition}

\begin{proposition}
  Recall that $R/I$ is a field if and only if $I$ is a
  maximal ideal of the ring $R$.
\end{proposition}

\begin{remark}
  While we consider ideals of an algebra, since the algebra has a
  vector space structure, we demand the ideal to be subspace with
  respect to the underlying linear operations
\end{remark}

\begin{lemma}
  If $\mathcal{A}$ is a (unital) Banach algebra, and $I$ is a closed
  ideal, then $\mathcal{A}/I$ is a (unital) Banach algebra.
\end{lemma}
\begin{proof}
  Since $I$ is an ideal, the ring structure of $A/I$ is well defined.
  Moreover since $\mathcal{A}$ is a Banach space and $I$ is a closed
  subspace, we know that $\mathcal{A}/I$ is a Banach space. Thus, we
  just need to verify the norm inequality
  \begin{align*}
    \|ab + I\| \le \|a + I\|\|b + I\|
  \end{align*}
  By the definition of the quotient norm for any $\varepsilon > 0$,
  there exist a $i_a, i_b \in
  I$ such that
  \begin{align*}
    \|a + i_a\| \le \|a + I\| + \varepsilon, \quad \|b + i_b\| \le
    \|b + I\| + \varepsilon
  \end{align*}
  Then,
  \begin{align*}
    \|ab + I\| &\le \|ab + i_ab + i_b a + i_aib\|  \\
    &= \|(a + i_a)( b + i_b)\| \\
    &\le \| a + i_a\| \|b + i_b\| \\
    &\le (\|a + I\| + \varepsilon)(\|b + I\| + \varepsilon) \\
    &\le \|a + I\|\|b + I\| + \varepsilon \|b  + I\| + \varepsilon
    \|a + I\| + \varepsilon^2
  \end{align*}
  Since $\varepsilon$ was chosen arbitrarily, this gives our result.
\end{proof}

\begin{lemma}
  In a unital Banach algebra, every maximal ideal is closed.
\end{lemma}
\begin{proof}
  Take a maximal ideal, take its closure, then it must be either the
  ideal itself or the whole of the algebra. If it contains the whole
  of the algebra, then the unital element must be there. Then the
  original ideal must contain invertible elements by the openness of
  the set of invertible elements. This will make the original ideal,
  the whole of the algebra, which is a contradiction.
\end{proof}

\begin{lemma}
  If $\mathcal{A}$ is a Banach algebra that is a division ring (If
  every non-zero element has an inverse), then
  $\mathcal{A} = \mathbb{C}$.
  \label{2:A=C}
\end{lemma}
\begin{proof}
  Let $0 \neq a \in \mathcal{A}$. Let $\lambda \in \sigma(a)$. Then
  $\lambda1 - a$ is not invertible. Hence $\lambda1 - a = 0$. So $a =
  \lambda 1$. Hence $\mathcal{A} = \mathbb{C}$.
\end{proof}

\begin{corollary}
  Let $\mathcal{A}$ be a unital commutative Banach algebra, and $I$
  be a maximal ideal. Then $\mathcal{A}/I = \mathbb{C}$.
\end{corollary}

\begin{lemma}
  \label{specturm_of_comm_alg_and_spectrum_of_element}
  For every $a \in \mathcal{A}$, a commutative unital Banach algebra,
  \begin{align*}
    \sigma(a) = \{ \tau(a)  \ : \   \tau \in \textrm{sp}(\mathcal{  A}) \}
  \end{align*}
\end{lemma}
\begin{proof}
  Let $\tau \in \textrm{sp}(\mathcal{A})$, then $\tau(\tau(a)1 -
  a) = 0$ and therefore $\tau(a)1 - a \in \textrm{Ker}(\tau)$,
  hence it is not invertible. Hence $\tau(a) \in \sigma(a)$.

  Conversely, let $ \lambda \in \sigma(a)$, then $\lambda 1 - a$ is
  not invertible, thus the ideal $I = \langle \lambda1 - a \rangle$
  is a proper ideal, since $r(\lambda 1 - a)$ will not be invertible
  for any $r \in \mathcal{A}$. So $1 \notin \langle \lambda1 - a
  \rangle$. By Zorn's lemma, $\langle \lambda1 -a \rangle $ is
  contained in a maximal ideal $I_\lambda$.

  Define $\tau: \mathcal{A} \to \mathbb{C} = \mathcal{A}/I_\lambda :=
  x \mapsto x + I_\lambda$. Then $\tau \in \textrm{sp}(\mathcal{A})$,
  and $\tau(a) = a + I_\lambda = \lambda + I_\lambda$ since
  $\lambda1 - a \in I_\lambda$. Now by the identification of $A/I_\lambda$
  with $\mathbb{C}$ as in \autoref{2:A=C}, we see that $\tau(a) =
  \lambda$.
\end{proof}

\begin{definition}
  Let $\mathcal{A}$ be a commutative Banach algebra. Define
  \begin{align*}
    \Phi :  \mathcal{A} \to  C(\textrm{sp}(\mathcal{A})) :=
    \Phi(a)(\tau) = \tau(a)
  \end{align*}
  for all $a \in \mathcal{A}, \tau \in \textrm{sp}(\mathcal{A})$. The
  map $\Phi$ is called the \textbf{Gelfand transform}.
\end{definition}

\begin{theorem}
  $\Phi$ is a contractive algebra homomorphism with $\| \Phi(a)\| = r(a)$.
\end{theorem}
\begin{proof}
  That $\Phi$ is contractive follows easily from
  \begin{align*}
    \|\Phi(a)(\tau)\| = \|\tau(a)\| \le \|a\|
  \end{align*}
  since $\tau$ is a contraction as proved in
  \autoref{lem:spectrum_is_compact}. Linearity and multiplicativity
  of $\Phi$ follows form the fact that every element $\tau \in
  \textrm{sp}(\mathcal{A})$ is linear and multiplicative on $\mathcal{A}$.
\end{proof}

\begin{remark}[Maximal ideals of $\mathcal{A}$ and $\textrm{sp}(\mathcal{A})$]
  Let $\mathcal{A}$ be unital commutative Banach algebra. Let $\tau
  \in \textrm{sp}(\mathcal{A})$, then $\textrm{Ker}(\tau)$ is a
  closed ideal of $\mathcal{A}$, and $A/\textrm{Ker}(\tau) \cong
  \mathbb{C}$ by the first isomorphism theorem. So
  $\textrm{Ker}(\tau)$ is a maximal ideal. The converse of this is
  also true. Natural map to the quotient space of a maximal ideal
  (which is now a filed isomorphic to $\mathbb{C}$) gives an element
  of the $\textrm{sp}(\mathcal{A})$. Hence $\textrm{sp}(\mathcal{A})$
  can be identified with the maximal ideals of $\mathcal{A}$.
\end{remark}

\begin{remark}
  Suppose $\tau, \tau^\prime \in \textrm{sp}(\mathcal{A})$, with
  $\textrm{Ker}(\tau) =\textrm{Ker}(\tau^\prime)$. Let $a \in
  \mathcal{A}$, then $\tau(a)1 - a \in \textrm{Ker}(\tau) =
  \textrm{Ker}(\tau^\prime)$ implies $\tau(a) = \tau^\prime(a)$ for
  all $a \in \mathcal{A}$.
\end{remark}

\begin{remark}
  Combining both of the above, we see that $\textrm{Ker}(\Phi)$ is
  the intersection of all maximal ideals of $\mathcal{A}$, that is
  the radical of $\mathcal{A}$.
\end{remark}

\begin{theorem}
  \label{thm:beurling}
  Let $\mathcal{A}$ be a Banach algebra. Then $\forall a \in
  \mathcal{A}$, we have
  \begin{align*}
    r(a) = \lim_{n \to \infty} \|a^n\|^{1/n}
  \end{align*}
\end{theorem}
\begin{proof}
  \textcolor{red}{verify}
\end{proof}

\begin{corollary}
  If $\|a^2\| = \|a\|^2$, then $r(a) = \|a\|$ and the Gelfand
  transform will be halal.
  \marginnote{ \scriptsize \it \textcolor{red}{not sure. Might need
  C* algebra structure}}
\end{corollary}

\begin{example}
  Let $T \in B(\mathcal{H})$ be self-adjoint. Let $\mathcal{A} =
  \overline{\textrm{span}}\{ T^n  \ : \  n \in \mathbb{N} \cup {0}
  \}$. Then $\mathcal{A}$ is a unital Banach algebra. Moreover we have
  \begin{align*}
    \|T^2\| = \|T\|^{2}
  \end{align*}
  by the self adjointness of $T$. Thus the Gelfand transform $\Phi$
  is isometric on $\mathbb{R}$-$\overline{\textrm{span}}\{T^n\}$.
\end{example}

% TeX_root = ../main.tex

\marginnote{\scriptsize 23/01/2025 }

\begin{example}
  Given an infinite dimensional Hilbert space with orthonormal basis
  $(u_n)_{n \in \mathbb{N}}$, show that $\{ u_n \}$ is not compact.
\end{example}
\begin{proof}
  Since $\|u_\alpha - u_\beta\| = \sqrt{2}$, take
  $\frac{1}{\sqrt{2}}$ radius balls around each $u_\alpha$ to get a
  collection of open balls that cover the set with no finite subcover.

  Another way to see is to use the sequential compactness criterion
  and see that the  sequence $(u_n)$ does not have any convergent subsequence.
  Since this is a metric space, sequential compactness is equivalent
  to compactness.
\end{proof}

\begin{theorem}[Every Hilbert space is $\ell^{2}(A)$]
  Let $\mathcal{H}$ be a Hilbert space, $(u_\alpha)_{\alpha \in A}$
  is an orthonormal basis, then there is a unitary map $U:
  \mathcal{H} \to \ell^{2}(A)$ such that $U(u_\alpha) = \chi_\alpha$
\end{theorem}
\begin{proof}
  We first note that by linearity, if $p \in \textrm{span}\{ u_\alpha
  \ : \ \alpha \in A \}$, then $U(p)$ is determined by $\chi_\alpha$.
  Next, by Bessel's inequality,
  \begin{align*}
    \|U(p)\|_{\ell^{2}(A)} \le \|p\|
  \end{align*}
  Hence $U$ is bounded. Hence it can be continuously extended to
  $\mathcal{H} = \overline{\textrm{span} \{ u_\alpha \}}$ as a limit
  of sequences. Also, by the equivality in the Bessel's inequality,
  we get that $U$ is an isometry, hence one-to-one.

  Now it remains to show that $U$ is onto. Given $g \in \ell^{2}(A)$,
  we know that there exists at most a countable set $ \{ \alpha_1 ,
  \alpha_2 , \ldots \} = A_0$ such that $g(\alpha_i) \neq 0$. Consider
  \begin{align*}
    h_n = \sum_{j = 1}^{n} g(\alpha_j) u_{\alpha_j}
  \end{align*}
  then,
  \begin{align*}
    u(h_n)(\alpha) =
    \begin{cases}
      g(\alpha_j), & \alpha_j \in A_0 \\
      0, & \textrm{otherwise}
    \end{cases}
  \end{align*}
  Moreover,
  \begin{align*}
    \Big \| U(h_n) - g \Big \|_{\ell^{2}(A)}^2 = \sum_{j =
    n+1}^{\infty} |g(\alpha_j)|^2 \to 0
  \end{align*}
  Now if
  \begin{align*}
    h = \sum_{j = 1}^{\infty}  g(\alpha_j) u_{\alpha_j} \in
    \mathcal{H} \quad (\textrm{ since } g \in \ell^{2}(A))
  \end{align*}
  we get
  \begin{align*}
    \|h - h_n\|^2 = \|U(h_n) - g\|^2_{\ell^{2}(A)} \to 0
  \end{align*}
  and the injectivity of $U$ shows that $U(h) = g$.
\end{proof}

\chapter{Banach Space Techniques}
\begin{definition}
  If $X$ is a real or complex normed vector space with a norm, and
  the complete in the topology induced by the norm, it is called a Banach space.
\end{definition}

\begin{definition}
  If $X, Y$ are normed vector spaces over $\mathbb{R}$ or
  $\mathbb{C}$, $\Lambda: X \to Y$ linear, then the norm of the operator
  \begin{align*}
    \|\Lambda\| = \sup \{ \|\Lambda x\|  \ : \  \|x\|< 1 \}
  \end{align*}
  If $\|\Lambda\| < \infty$, then we say that $\Lambda$ is bounded.
\end{definition}

\begin{proposition}
  Given $\Lambda: X \to Y$, a linear map between normed linear
  spaces, the following are equivalent
  \begin{enumerate}[label=(\arabic*)]
    \item $\Lambda$ is bounded
    \item $\Lambda$ is continuous
    \item $\Lambda$ is continuous at some $x_o \in X$
  \end{enumerate}
\end{proposition}
\begin{proof}
  $(1 \implies 2)$
  \begin{align*}
    \|\Lambda(x - y)\| \le \|\Lambda\| \|x - y\|
  \end{align*}
  gives $\|\Lambda\|$-Lipschitz continuity.

  $(2 \implies 3)$ Follows from the definition.

  $(3 \implies 1)$ For each $\varepsilon > 0$, there is $\delta > 0$
  such that for each $x \in X$, with $\|x - x_o\| < \delta$, then
  $\|\Lambda x - \Lambda x_o\| < \varepsilon$.
  Thus for $ \|y \| < \delta$,  by linearity of $\Lambda$, we get
  \begin{align*}
    \|\Lambda y\| = \|\Lambda(x_o + y) - \Lambda x_o\| < \varepsilon
  \end{align*}
  Again using linearity, we get for $\|y^\prime\| < 1$,
  \begin{align*}
    \|\Lambda y^\prime\| < \frac{\varepsilon}{\delta} < \infty
  \end{align*}
  Now since $\overline{ B_1(0)} \subset B_2(0)$, we see that
  $\|\Lambda\| < \frac{2\varepsilon}{\delta} < \infty$.
\end{proof}

\section{Consequence of Baire category theorem}

\begin{theorem}[Baire Category Theorem]
  \label{thm:Baire-Category-Rudin}
  If $(X, d)$ is a complete metric space, and $V_1 , V_2 , \ldots$
  are dense subsets, then
  \begin{align*}
    \bigcap_{n = 1}^{\infty} V_n
  \end{align*}
  is dense in $X$.
\end{theorem}
\begin{proof}
  We show that for any non-empty open set $W \subset X$,
  \begin{align*}
    \bigcap_{n = 1}^{\infty} V_j \cap W \neq \emptyset
  \end{align*}
  We write $B_r(x) = \{ y \in X  \ : \  d(x, y)< r \}$. Since $V_1$
  is dense and open, $V_{1} \cap W$ is open and dense in $W$. Thus we
  can find an $r_{1} > 0$ such that $\overline{B_{r_{1}}(x_1)}
  \subset W \cap V_1$. (First find an $r^\prime > 0$ such that
    $B_{r^\prime}(x_1) \subset W \cap V_{1}$. Then take $r_{1} =
  \frac{r^\prime}{2}$).

  We inductively proceed by taking $x_n \in V_n \cap
  B_{r_{n-1}}(x_{n-1})$ such that $\overline{B_{r_n}(x_n)} \subset
  V_n \cap B_{r_{n-1}}(x_{n-1})$. Without loss of generality, choose
  $0 < r_n < \frac{1}{n}$. This gives a sequence which satisfies for
  $i, j > n$ that $x_i, x_j \in B_{r_n}(x_n) \implies d(x_i, x_j) <
  2r_n < \frac{2}{n}$. Hence $x_n$ is Cauchy. By completeness $x_n
  \to x \in \overline{B_{r_n}(x_n)} \subset V_n \cap W$ for all $n$.
  Thus $x \in W$ and
  \begin{align*}
    x \in \bigcap_{n = 1}^{\infty}V_n
  \end{align*}
\end{proof}

% TeX_root = ../main.tex

\marginnote{\scriptsize 28/01/2025 }

\begin{example}
  Let $(X, \Sigma, \mu)$ be a measure space. Define $E : \Sigma \to
  B(L^{2}(X, \mu)):= A \to M_{\chi_A}$. It is easy to see that $E$
  satisfies the first 3 properties of a spectral measure.
  To verify the fourth property, let $A_1 , A_2 , \ldots \in \Sigma$
  be disjoint collection. Then
  \begin{align*}
    M_{\chi_{\cup_{n = 1}^{\infty}A_n}}(f) &= \chi_{\cup_{n =
    1}^{\infty}A_n}f = \sum_{n = 1}^{\infty} \chi_{A_n} f = \sum_{ n
    = 1}^{\infty}  M_{\chi_{A_n}}  f
  \end{align*}
  shows that the fourth property is also satisfied.
\end{example}

\begin{proposition}
  \label{prop:spectral_measure_gives_complex_measure}
  Let $E$ be a spectral measure on $(X, \Sigma, \mathcal{H})$. Then
  for every $\xi, \eta \in \mathcal{H}$.
  \begin{align*}
    E_{\xi, \eta}(A) = \langle  E(A) \xi ,  \eta \rangle
  \end{align*}
  defines a finite measure on $(X, \Sigma)$ with $\|E_{\xi, \eta}\|
  \le \|\xi\| \|\eta\|$.
\end{proposition}
\begin{proof}
  That $E_{\xi, \eta}(\emptyset) = 0$ is evident. Hence we only need
  to verify the countable disjoint additivity to show that $E_{\xi,
  \eta}$ is a measure. Let $A_1 , A_2 , \ldots \in \Sigma$ be a
  mutually disjoint collection. Then
  \begin{align*}
    E_{\xi, \eta} \Big( \bigcup_{n = 1}^{\infty}A_n \Big) &= \Big
    \langle E \Big( \bigcup_{n = 1}^{\infty} A_n \Big) \xi ,  \eta
    \Big \rangle \\
    &= \Big \langle  \sum_{n = 1}^{\infty} E(A_n) \xi ,  \eta
    \Big \rangle \\
    &= \sum_{n = 1}^{\infty} \big \langle E(A_n) \xi, \eta \big \rangle  \\
    &= \sum_{n = 1}^{\infty}  E_{\xi, \eta}(A_n)
  \end{align*}
  Taking the summation outside the inner product is justified by the
  below lemma. Thus we see that $E_{\xi, \eta}$ is a complex measure.
  Moreover, since $|\langle E(A) \xi ,  \eta \rangle| \le \|\xi\|
  \|\eta\|$, as $E(A)$ is a projection, we see that the measure is
  finite (Although this is implicit in complex measures). To see that
  $ \|E_{\xi, \eta}\| \le \|\xi\| \|\eta\|$, use the definition of
  bounded variation and the lemma below.
\end{proof}

\begin{lemma}
  Let $\{ P_i \}_{i \in I}$ be a family of pairwise orthogonal
  projections on a Hilbert space $\mathcal{H}$. Then there exist a
  unique projection $P$ on $\mathcal{H}$ such that $\forall \xi, \eta
  \in \mathcal{H}$,
  \begin{align*}
    \sum_{i \in I} \langle P_i \xi ,  \eta \rangle = \langle P \xi ,
    \eta \rangle
  \end{align*}
\end{lemma}
This means the convergence of orthogonal projections is not in the
norm sense, but rather in the sense above. For example
\begin{example}
  Consider $P_n = P_{\delta_n}$ in $B(\ell^{2}(\mathbb{N}))$, and let
  $  Q_n = \sum_{i = 1}^{n} P_i$. Then clearly $Q_i$ doesn't converge
  in norm, but rather in the above sense to the identity map in
  $B(\ell^{2}(\mathbb{N}))$.
\end{example}

\begin{lemma}[Reisz representation for sesquilinear forms]
  \label{lem:resiz_representation_for_sesquilinear}
  \marginnote{ \scriptsize \it \textcolor{red}{I'm not convinced.
  Need to verify}}
  Let $B(\cdot, \cdot)$ be a bounded sesquilinear form on a Hilbert
  space $\mathcal{H}$. Then there exist a unique $T \in
  B(\mathcal{H})$ such that for all $\xi, \eta \in \mathcal{H}$,
  \begin{align*}
    B(\xi, \eta) = \langle T \xi ,  \eta \rangle
  \end{align*}
\end{lemma}

\begin{proposition}
  \label{prop:spectral_measure_and_integration}
  \marginnote{ \scriptsize \it \textcolor{red}{verify if $\phi$ is a
  linear functional}}
  Let $E$ be a spectral measure on $(X, \Sigma, \mathcal{H})$, and
  $\phi$ be a bounded linear functional on $X$. Then
  there exist a unique $  T_\phi \in B(\mathcal{H})$ such that for
  all $\xi, \eta \in \mathcal{H}$
  \begin{align*}
    \int_X \phi \ d  E_{\xi, \eta} = \langle T_\phi \xi ,  \eta \rangle
  \end{align*}
  and we denote $T_\phi:= \int_X \phi \ d  E$
\end{proposition}
\begin{proof}
  See that the integral is sesquilinear on $\xi, \eta$ for simple
  functions. Then use
  \autoref{lem:resiz_representation_for_sesquilinear}.
\end{proof}

Note that the set $\mathcal{M}(X)$ of all bounded measurable functions on $X$,
equipped with sup norm is a Banach space. Moreover with pointwise
product and complex conjugation, it turns into a commutative C$^*$ algebra.

\begin{theorem}
  Let $E$ be a spectral measure on $(X, \Sigma, \mathcal{H})$ and
  \marginnote{ \scriptsize \it \textcolor{red}{verify if $\phi$ is a
  linear functional}}
  $\phi$ be a bounded measurable function. The map
  \begin{align*}
    \mathcal{M}(X) \to B(\mathcal{H}):= \phi \mapsto \int_ X \phi \ d E
  \end{align*}
  is a contractive $*$-homomorphism.
\end{theorem}
\begin{proof}
  To show that it is a $*$-homomorphism, observe that
  $\overline{E_{\xi, \eta}} = E_{\eta, \xi}$. This follows since
  \begin{align*}
    \overline{E_{\xi, \eta}(A)} = \overline{ \langle E(A) \xi , \eta
    \rangle } = \langle \eta , E(A) \xi \rangle = \langle E(A)  \eta , \xi
    \rangle = E_{\eta,  \xi}(A)
  \end{align*}
  Now let $T_{\overline{\phi}} \in B(\mathcal{H})$ corresponding to
  $\phi$ as in \autoref{prop:spectral_measure_and_integration}. Then
  by definition,
  \begin{align*}
    \int_X \overline{\phi} \ d  E_{\xi, \eta} &= \langle T_{\overline{
    \phi}}  \eta, \psi  \rangle
  \end{align*}
  Now the left integral is equal to
  \begin{align*}
    \overline{\int_X  \phi \ d E_{\eta, \xi}} = \overline{\langle
    T_{\phi} \eta ,  \xi \rangle } = \langle \xi ,  T_\phi \eta
    \rangle = \langle T_\phi^* \xi ,  \eta \rangle
  \end{align*}
  Thus we get that $T_{\overline{\phi}} = T_{\phi}^*$ and hence the
  map preserve the involution.

  To show that the map is multiplicative, we need to show that
  \begin{align*}
    \int_X \phi \psi \ d  E = \int_X \phi \ d  E \circ \int_X \psi \ d  E
  \end{align*}
  Notice that by \autoref{prop:spectral_measure_and_integration}, we'll
  done if we show that for all $\xi, \eta \in \mathcal{H}$,
  \begin{align*}
    \int_X \phi \psi \ d E_{\xi, \eta} = \int_X \phi \ d  E_{(\int_X
    \psi \ d E) \xi, \eta}
  \end{align*}
  But for this, it is enough to show the equivalence of the measures
  $\psi dE_{\xi, \eta}$ and $E_{(\int_X \psi \ d E)\xi, \eta }$. That
  is for all $A \in \Sigma$, we need to show that
  \begin{align}
    \label{eq:4}
    \Big \langle E(A)\Big(\int_X  \psi \ d  E\Big)\xi ,  \eta \Big
    \rangle = \int \psi \chi_A \ d  E_{\xi, \eta}
  \end{align}
  But again, the left hand side of the inner product is
  \begin{align*}
    \Big \langle \Big( \int_X \psi \ d E\Big) \xi ,  E(A) \eta \Big \rangle
  \end{align*}
  since $E(A)$ is a projection. Again by
  \autoref{prop:spectral_measure_and_integration}, we see that the
  above inner product is
  \begin{align*}
    \int_X \psi \ d E_{\xi, E(A) \eta}
  \end{align*}
  Then \autoref{eq:4} reduces to showing
  \begin{align*}
    \int_X \psi \ d E_{\xi, E(A) \eta} = \int \psi \chi_A \ d  E_{\xi, \eta}
  \end{align*}
  Again using the same reasoning, it is enough to show that the
  measures are the same. That is for any $B \in \Sigma$, we must have
  \begin{align*}
    E_{\xi , E(A) \eta}(B) =     \int_X \chi_A \chi_B \ dE_{\xi, \eta}
  \end{align*}
  But this is equivalent to
  \begin{align*}
    \langle E(A) \xi ,  E(B) \eta \rangle = \langle E(A)E(B) \xi ,
    \eta \rangle =  \langle  E(A \cap B) \xi , \eta \rangle = \int_X
    \chi_{A\cap B} \ d E_{ \xi, \eta} = E_{\xi, \eta}(A \cap B)
  \end{align*}
  which is true by a property of the Spectral measure.
\end{proof}

% \begin{definition}
%   A C$^*$ algebra is a Banach *-algebra satifying the norm equality
%   $\|a^*a\| = \|a\|^2$, for all $a \in \mathcal{A}$.
% \end{definition}

% TeX_root = ../main.tex

\marginnote{\scriptsize 09/09/2025 }

\begin{theorem}
  Let $F \subset M_n$ be a family of diagonalizable matrices, then
  $F$ is a commuting family if and only if it is simultaneously diagonalizable.
\end{theorem}
\begin{proof}
  It is an easy exercise to show that a simultaneously diagonalizable
  family is commuting.

  We prove the converse by induction over $n$. For $n = 1$, there's
  nothing to prove. Assume that this is true for all $F^\prime
  \subset M_k$, where $k < n$. If each $A \subset F$ is of the form
  $A = \lambda I$, again nothing to prove. Thus, assume $A$ is
  diagonalizable with eigenvalues $\lambda_1 , \lambda_2 , \ldots ,
  \lambda_r, r \ge 2$, and assume $AB = BA$, for each $B \in F$.
  Then $A, B$ are simultaneously diagonalizable by previous theorem
  and hence without loss of generality, assume that
  $A$ is diagonal.

  \textcolor{red}{verify rest from the lecture notes}
\end{proof}

\begin{remark}
  Given $C \in M_n(\mathbb{C})$, we can think of function associated
  with $C$, as $ \langle x , y \rangle   \to \langle Cx, y\rangle$.
  Notice that this function preserves all information about $C$,
  since we can find the individual matrix entries.
  We can also associate $Q_c : x \to \langle Cx , x \rangle$. We
  recall $Q_c$ determines $C$ as
  \begin{align*}
    \langle Cx, y \rangle = \frac{1}{4}\sum_{j = 1}^{4} i^j\|Q_c(x + i^j
    y)\|^2, \quad ( i = \sqrt{-1})
  \end{align*}
\end{remark}

\subsection{Hermitian, normal, and unitary matrices}
\begin{definition}
  Let $A \in M_{n, m}$, then the adjoint $A^* \in M_{m, n}$ satisfies
  $\langle  Ax , y \rangle  = \langle x , A^*y \rangle $ for each $x
  \in \mathbb{C}^m, y \in \mathbb{C}^n$.

  If for $A \in M_n$, $A = A^*$, then we say that $A$ is Hermitian or
  self-adjoint. If $A = - A^*$, the it is called skew-hermitian.

  If $A \in M_n$, then $A = B+ iC$, where $B = \frac{1}{2} (A + A^*),
  C = \frac{1}{2i} (A - A^*)$ have the property $B = B^*, C = C^*$.
  Here $ B$ is called the real part of $A$, and $iC$ is called the
  imaginary part of $A$.
\end{definition}

\begin{proposition}
  \begin{align*}
    A = A^* \iff (iA) = -(iA)^*
  \end{align*}
\end{proposition}

\begin{proposition}
  A matrix $A \in M_n$ is Hermitian if and only if for all $x \in
  \mathbb{C}^n$, $ \langle Ax , x \rangle  \in \mathbb{R}$.
\end{proposition}
\begin{proof}
  If $A$ is Hermitian, then
  \begin{align*}
    \overline{\langle Ax , x \rangle } & = \langle x , Ax \rangle =
    \langle A^*x , x \rangle  = \langle Ax , x \rangle
  \end{align*}
  shows that $\langle  Ax , x \rangle  \in \mathbb{R}$.

  Conversely, assume that $\langle  Ax , x \rangle  \in \mathbb{R}$
  for all $x \in \mathbb{C}^n$. Let $ A = B + iC$, where $B= B^*, C =
  C^*$. Then
  \begin{align*}
    \langle Ax , x \rangle  &= \underbrace{\langle Bx , x
    \rangle}_{\in \mathbb{R}}  + i \underbrace{\langle Cx , x
    \rangle}_{\in \mathbb{R}}  \\
  \end{align*}
  We conclude that $\langle  Cx , x \rangle  = 0$ for all $x$. Now
  using polarization identity, we get that $C = 0$.
\end{proof}

We also consider and equivalent of unimodular numbers.

\begin{definition}
  Let $A \in M_n$. $A$ is unitary if $A^*A = I = AA^*$.
\end{definition}

\begin{proposition}
  Let $A \in M_n$. The following are equivalent.
  \begin{enumerate}[label=(\arabic*)]
    \item $A$ is unitary.
    \item The columns of $A$ form an orthonormal basis.
    \item Rows of $A$ form an orthonormal basis.
    \item A preserves the norm for each $x \in \mathbb{C}^n$. That is
      $\|Ax\| = \|x\|$.
    \item $A$ preserve the inner product. That is $\langle  Ax , Ay
      \rangle = \langle x , y \rangle$
  \end{enumerate}
\end{proposition}
\begin{proof}
  You know,
\end{proof}

% TeX_root = ../main.tex

\marginnote{\scriptsize 04/02/2025 }

\begin{corollary}
  Let $X, Y$ be Banach spaces. If $T: X \to Y$ is linear, bounded map
  and $ T$ is one-to-one and onto, then $T^{-1}$ is bounded.
\end{corollary}
\begin{proof}
  Use the fact the open mapping theorem gives that $T^{-1}$ is
  continuous, and that continuity is boundedness in linear spaces.
\end{proof}

\begin{theorem}[Closed graph theorem]
  Let $X, Y$ be Banach spaces, then the graph of $T$, defined as
  $G(T) = \{ (x, Tx)  \ : \  x \in X \} \subset X \times Y$, under
  the norm $\|(x, y)\| = \| x\|_X + \|y\|_Y$ is closed if and only if
  $T$ is bounded.
\end{theorem}
\begin{proof}
  Refer back to functional analysis notes.
\end{proof}

\section{Applications of Banach-Steinhaus}

Let $C_{per}([-\pi, \pi])$ denote continuous functions $f: [-\pi,
\pi] \to \mathbb{C}$ such that $f(\pi) = f(-\pi)$. Since each
$C_{per}([-\pi, \pi]) \subset L^{2}([-\pi, \pi])$, each such $f$ has
a Fourier series. Let
\begin{align*}
  c_n = \frac{1}{2\pi} \int_{-\pi}^{ \pi}  f(t) e^{int} \ dt
\end{align*}
and
\begin{align*}
  s_n(t) = \sum_{ j = -n}^{n} c_j e^{ijt}
\end{align*}
We know that $s_n \to f$ in $L^2$.

But what about pointwise convergence. Let
\begin{align*}
  D_n = \sum_{j = -n}^{n} e^{ijt} =
  \frac{\sin((n+\frac{1}{2})t)}{\sin(\frac{t}{2})}
\end{align*}
Observe that
\begin{align*}
  \frac{1}{2\pi} \int_{-\pi}^{ \pi}  f_n(t)D_n(x-t) \ dx &= \sum_{j =
  -n}^{n} \Big( \frac{1}{2\pi} \int_{-\pi}^{ \pi} f(x) e^{-ijt} \ dt
  \Big) e^{ijx} \\
  &= \sum_{j = -n}^{n} c_j e^{ijx} = s_n(x)
\end{align*}
Choose linear functionals $\Lambda_n: C_{per}([-\pi, \pi]) \to
\mathbb{C}$ defined as
\begin{align*}
  \Lambda_n(f) = \frac{1}{2\pi} \int_{-\pi}^{ \pi}  f(t)D_n(-t) \ dt = s_n(0)
\end{align*}
Putting sup norm on $C_{per}([-\pi, \pi])$, we get that $\Lambda_n$
is linear, bounded, with
\begin{align*}
  |\Lambda_n(f)| &= \Big|\frac{1}{2\pi} \int_{-\pi}^{ \pi}
  f(t)D_n(-t) \ dt \Big| \\
  &\le \frac{1}{2 \pi}\|f\|_\infty\|D_n\|_1
\end{align*}
\textcolor{red}{Read the rest from Rudin}


\printbibliography[heading=bibintoc]

\end{document}
