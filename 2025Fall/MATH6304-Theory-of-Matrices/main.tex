\documentclass[12pt]{report}

\usepackage{geometry} % automatic papersizes, margins.
\usepackage{makeidx} % 'makeidx' make and show index
\usepackage{enumitem} % itemize, enumerate, description.
\usepackage{hyperref} % hyperlinks, cross-references.
\usepackage{xcolor} % foreground and background color management. Better color mixing compared to 'color'
\usepackage{graphicx} % provide options for \includegraphics. Builds on 'graphic'
\usepackage{caption} % better control over captions of figures and equations.
\usepackage{appendix} % extra control over appendix
\usepackage[backend=biber, style=alphabetic]{biblatex} % better than bibtex, people say.
\usepackage{tocbibind} % add ToC/Bibliography/Index to ToC

\usepackage{amsmath} % math symbols, matrices, cases, trig functions, var-greek symbols.
\usepackage{amsfonts} % mathbb, mathfrak, large sum and product symbols.
\usepackage{amssymb} % extended list of math symbols from AMS. https://ctan.math.washington.edu/tex-archive/fonts/amsfonts/doc/amssymb.pdf
\usepackage{amsthm} % theorem styling.
\usepackage{mathrsfs} % mathscr fonts.
\usepackage{yhmath} % widehat.
\usepackage{empheq} % emphasize equations, extending 'amsmath' and 'mathtools'.
\usepackage{bm} % simplified bold math. Do \bm{math-equations-here}
\usepackage{tikz} % for tikz diagrams
\usepackage{tikz-cd} % commutative diagrams.
\usepackage{marginnote} % For sidenotes

\geometry{
  a4paper, % 'a4paper', 'c5paper', 'letterpaper', 'legalpaper'
  asymmetric, % don't swap margins in left and right pages. as opposed to 'twoside'
  centering, % to center the content between margins
  bindingoffset=0cm,
}

\hypersetup{
    colorlinks,
    linkcolor={blue!50!black},
    citecolor={blue!50!black},
    urlcolor={blue!80!black}
}

\theoremstyle{plain} % default; italic text, extra space above and below
\newtheorem{theorem}{Theorem}[section]
\newtheorem{proposition}{Proposition}[section]
\newtheorem{lemma}{Lemma}[section]
\newtheorem{corollary}{Corollary}[theorem]
\newtheorem{problem}{Problem}[section]

\theoremstyle{definition} % upright text, extra space above and below
\newtheorem{definition}{Definition}[section]
\newtheorem{example}{Example}[section]
\newtheorem{exercise}{Exercise}[section]

\theoremstyle{remark} % upright text, no extra space above or below
\newtheorem{remark}{Remark}[section]
\newtheorem*{note}{Note} %'Notes' in italics and without counter 

\newcommand{\propositionautorefname}{Proposition}
\newcommand{\lemmaautorefname}{Lemma}
\newcommand{\corollaryautorefname}{Corollary}
\newcommand{\problemautorefname}{Problem}
\newcommand{\definitionautorefname}{Definition}
\newcommand{\exampleautorefname}{Example}
\newcommand{\remarkautorefname}{Remark}
\newcommand{\noteautorefname}{Note}

\addbibresource{~/Books/Research/research.bib}


\begin{document}
    \title{MATH 6304 - Theory of Matrices}

    \author{
      Joel Sleeba \\
      University of Houston \\
      joelsleeba1@gmail.com \\
    }

    \maketitle

    \pagenumbering{roman} \setcounter{page}{2}
    \tableofcontents
    \pagenumbering{arabic} \setcounter{page}{1}

    % TeX_root = ../main.tex

\marginnote{\scriptsize 26/08/2025 }

Office Hours : Tuesday 10 - 11 AM, Wednesday 1 - 2 PM

\section{Introduction}

Let $A$ be a $m \times n$ matrix and $D$ a diagonal $n \times n$
matrix with entries $d_1 , d_2 , \ldots , d_n$,
\begin{align*}
  A =
  \begin{bmatrix}%{c c c c}
    | & | & \cdots &  | \\
    a_1 & a_2 & \cdots &  a_n \\
    | & | & \cdots &  | \\
  \end{bmatrix}
\end{align*}
Then
\begin{align*}
  AD =
  \begin{bmatrix}%{c c c c}
    | & | & &  | \\
    d_1a_1 & d_2a_2 & \cdots &  d_na_n \\
    | & | & &  | \\
  \end{bmatrix}
\end{align*}
and if
\begin{align*}
  B =
  \begin{bmatrix}%{c c c c}
    - & b_1 & - \\
    - & b_2 & - \\
    & \vdots &  \\
    - & b_n & - \\
  \end{bmatrix}
\end{align*}
then
\begin{align*}
  DB =
  \begin{bmatrix}%{c c c c}
    - & d_1b_1 & - \\
    - & d_2b_2 & - \\
    & \vdots &  \\
    - & d_nb_n & - \\
  \end{bmatrix}
\end{align*}

\textcolor{red}{Do the same for upper triangular matrices and add
some context to the multiplications}.

Notice that every time you left multiply, you play with column, and
when you right multiply, you play with the rows.

\begin{exercise}
  Let $A$ be an $n \times n$ matrix over $\mathbb{C}$. Let $\omega =
  e^{ \frac{2\pi i}{n}}$. Then prove that
  \begin{align*}
    A' = \frac{1}{n} \sum_{k = 0}^{n} (U^*)^kAU^k
  \end{align*}
  preserve all the diagonal entries of $A$ and kills the rest of
  entries. That is $A^\prime = \text{Diag}(A)$
\end{exercise}

\subsection{Review of Linear Algebra}

\begin{itemize}
  \item Rank-Nullity Theorem
  \item Orthogonality
  \item Orthogonal projection is the closest point on the subspace
    from the given vector.
\end{itemize}

    % TeX_root = ../main.tex

\marginnote{\scriptsize 28/08/2025 }

\begin{definition}
  If $A \in M_n$, $A = (a_{i, j})_{i, j = 1}^n$, we let
  \begin{align*}
    \rm{trace}(A) = \sum_{j = 1}^{n} a_{jj}
  \end{align*}
  The determinant is
  \begin{align*}
    \textrm{det}(A) = \sum_{ \sigma \in  S_n} \textrm{sgn}(\sigma)
    a_{1, \sigma(1)} a_{2, \sigma(2)} \ldots a_{n, \sigma(n)}
  \end{align*}
\end{definition}

\begin{remark}
  If $A = [a_1  a_2  \ldots  a_n]$, then $det(A) = f( a_1 , a_2 ,
  \ldots , a_n)$ is the only function that is linear in each $a_i$,
  alternating (swapping columns doesn't alter the value), and
  normalized ($det(I) = 1$).

  This is useful to show that for $A, B \in M_n$, $det(AB) = det(A) det(B)$.

  Moreover if $A =
  \begin{bmatrix}
    B & C \\
    0 & D
  \end{bmatrix}$, then $det(A) = det(B) det(D)$.

  We also have
  \begin{align*}
    det(A) = \sum_{i, j = 1}^{n} (-1)^{i + j} a_{i, j} det(A_{i, j})
  \end{align*}
  Where $A_{i, j}$ is the submatrix with $i$th row and $j$th column
  removed from $A$.
\end{remark}

\subsection{Eigenvalues and Eigenvectors}

\begin{definition}
  Eigenvalue, Eigenvector, Spectrum of a matrix
\end{definition}

\subsection{Similarity}

\begin{definition}
  A matrix $B \in M_n$ is similar to $A \in M_n$, if there is an
  invertible $S \in M_n$ such that $B = S^{-1}AS$. This defines an
  equivalence relation.
\end{definition}

\begin{theorem}
  If $A, B \in M_n$ are similar. Then their characteristic polynomial
  $P_A = P_B$.
\end{theorem}

\begin{remark}
  characteristic polynomial i snot characteristic upto similarity, because
  \begin{align*}
    A =
    \begin{pmatrix}%{c c}
      0 & 1 \\
      0 & 0
    \end{pmatrix} \text{ and } B =
    \begin{pmatrix}%{c c}
      0 & 0\\
      0 & 0
    \end{pmatrix}
  \end{align*}
\end{remark}

    % TeX_root = ../main.tex

\marginnote{\scriptsize 02/09/2025 }

\subsection{Diagonazability}

Can we find conditions for diagonalizability.

\begin{theorem}
  Let $A \in M_n(\mathbb{C})$, $p_A(t) = \prod_{j = 1}^{n} (t -
  \lambda_j)$, and $\lambda_i \neq \lambda_j$ for $j \neq k$, then
  $A$ is diagonalizable.
\end{theorem}
\begin{proof}
  We'll show that there's a linearly independent set of $n$
  eigenvectors. Let $x_j \in \mathbb{C}^n$ such that $Ax_j =
  \lambda_j x_j$. If $\{x_1 , x_2 , \ldots , x_n  \}$ were linearly
  dependent, then there is a linear combination
  \begin{align*}
    \alpha_1x_{j_1} + \alpha_2x_{j_2} + \ldots + \alpha_rx_{j_r} = 0
  \end{align*}
  with $ r \le n$, and all $\alpha_j \neq 0$. Let $r$ be smallest
  such $r \le n$, and assume with possible renumbering that $j_i =
  i$. Then applying $ A$ to the linear combination gives us
  \begin{align*}
    A( \alpha_1x_{1} + \alpha_2x_{2} + \ldots + \alpha_nx_{n})
    =\alpha_1 \lambda_1 x_{1} + \alpha_2 \lambda_2 x_{2} + \ldots +
    \alpha_n \lambda_n x_{n}  = 0
  \end{align*}
  multiplying the previous equation with $\lambda_r$ and then
  subtracting gives us
  \begin{align*}
    \alpha_1 (\lambda_1 - \lambda_r) x_{1} + \alpha_2 (\lambda_2 -
    \lambda_r) x_{2} + \ldots + \alpha_r (\lambda_r - \lambda_r) x_{r}  = 0
  \end{align*}
  which contradicts the minimality of $r$.
\end{proof}

Unfortunately this is just a sufficient condition, as it excludes the
following matrix.

\begin{align*}
  \begin{bmatrix}%{c c c}
    0 & 0 & 0\\
    0 & 0 & 0\\
    0 & 0 & 1
  \end{bmatrix}
\end{align*}

\begin{definition}
  If for $A \in M_n(\mathbb{C})$,
  \begin{align*}
    p_A(t) = (t - \lambda_1)^{m_1} ( t - \lambda_2)^{m_2} \ldots (t -
    \lambda_r)^{m_r}
  \end{align*}
  then we say that $\lambda_j$ has algebraic multiplicty $m_j$. We
  call $\textrm{null}(\lambda_jI - A)$, the geometric multiplicity of
  $\lambda_j$
\end{definition}

\begin{lemma}
  If $A \in M_n$ has eigenvalue $\lambda$, and $ p_A(t) = (t -
  \lambda)^m q(t)$, with $q(\lambda) = 0$, then $r =
\textrm{nul}(\lambda I - A)) \le m$
\end{lemma}
\begin{proof}
Choose a basis $\{ x_1 , x_2 , \ldots , x_r \}$ of veignevectors,
spanning $E_\lambda = \{ x \in \mathbb{C}^n \ : \ Ax = \lambda x \}$.
Complete it to a basis $\{ x_1 , x_2 , \ldots , x_n \}$ of $
\mathbb{C}^n$.Let $S = [x_1 , x_2 , \ldots , x_n]$.

Then $AS = [ \lambda x_1 , \lambda x_2 , \ldots  \lambda x_r,
y_{r+1}, \ldots y_n]$ with some vectors $y_{r+1} , \ldots , y_n$.
Then $S^{-1}AS = $\textcolor{red}{verify}, and we get
\begin{align*}
  \textrm{det}(tI - A) & = \textrm{det}(tI - S^{-1}AS) \\
  &= ( t - \lambda)^r \textrm{det}(t - C)
\end{align*}

Thus we conclude that algebraic multiplicity of $\lambda$ is at least
equal to $r$.
\end{proof}

\begin{remark}
See that the sum of all the algebraic multiplicity of the eigenvalues
of $A \in M_n(\mathbb{C})$ is $n$.
\end{remark}

\begin{theorem}
The matrix $A \in M_n(\mathbb{C})$ is diagonalizable if and only if
the algebraic and geometric multiplicities are equal for each eigenvalue.
\end{theorem}
\begin{proof}
We note that given two eigenvalues $\lambda_j \neq \lambda_k$, then
their eigenspaces $E_{i}, E_{j}$ intersect trivially. Thus if $\{ v_1
, v_2 , \ldots , v_{r_1} \}$ and $\{ u_1 , u_2 , \ldots , u_{r_2} \}$
form a basis for $E_{\lambda_1}$ and $E_{\lambda_2}$ respectively,
then $\{ v_1 , v_2 , \ldots , v_{r_1}, u_1 , u_2 , \ldots , u_{r_2}
\}$ is linearly independent. Iterating this way, we get a basis for
$E_1 + E_2 \ \ldots  + E_n$ with dimension $r = \sum_{i = 1}^{k} r_i$.

If algebraic and geometric multiplicities equal then $r = n$, and we
have a basis of eigenvectors. Otherwise if $r < n$, then we do not
have such a basis of eigenvectors. And since existence of a basis of
eigenvectors characterizes diagonalizability, this characterizes
diagonalizability.
\end{proof}

Next lecture, we'll look when multiple matrices can be simultaneously
diagonalizable with the same $S$ matrix.

    % TeX_root = ../main.tex

\marginnote{\scriptsize 04/09/2025 }

\subsection*{Warm Up}
Assume we know that $A$ is diagonalizable. Let $p_0, p_1 , p_2 ,
\ldots , p_n \in \mathbb{C}$ and consider
\begin{align*}
  B: = P(A) = p_0 I + p_1 A + p_2A^2 + \ldots + p_nA^n
\end{align*}
Is $B$ diagonalizable?
\begin{proof}
  Yes. Because if $A = S^{-1}DS$, then $A^n = S^{-1}D^n S$, and therefore
  \begin{align*}
    B = S^{-1} (p_0 I  + p_1D + p_2D^2 + \ldots + p_nD^n)S
  \end{align*}
\end{proof}

\subsection{Simultaneous diagonalization}

\begin{theorem}
  Let $A, B$ be diagonalizable. Then $AB = BA$ if and only if they
  are simultaneously diagonalizable by the same $S$.
\end{theorem}
\begin{proof}
  Let $D_A = S^{-1} A S$, and $B^\prime = S^{-1} B S$, where $D_A$ is
  a diagonal matrix. Without loss of generality, assume that common
  eigenvalues appear together in $D_A$. If not choose $S$ with an
  additional permutation of the rows.

  Assuming $AB = BA$, we get
  \begin{align*}
    D_AB^\prime &= S^{-1} ASS^{-1}BS \\
    &= S^{-1}ABS \\
    &= S^{-1}BAS \\
    &= S^{-1}BSS^{-1}AS \\
    &= B^\prime D_A
  \end{align*}
  If $B^\prime = [b_{i, j}^\prime]_{i, j = 1}^n$, then by $D_A
  B^\prime = B^\prime D_A$, from the diagonal structure of $D_A$, we get
  \begin{align*}
    \tilde{\lambda}_ib_{i, j}^\prime = b_{i, j}^\prime \tilde{\lambda}_j
  \end{align*}
  where $\tilde{\lambda}_i$ is the $i$-th diagonal entry on $D_A$.
  So, we have
  \begin{align*}
    (\tilde{ \lambda}_i - \tilde{\lambda}_j) b_{i, j}^\prime = 0
  \end{align*}
  which shows that if $\tilde{ \lambda}_i \neq \tilde{\lambda}_j$,
  then $b_{i, j}^\prime = 0$. Thus we get that
  \begin{align*}
    B^\prime =
    \begin{bmatrix}%{c c c c}
      B_1^\prime &  &  & \\
      & B_2^\prime &   & \\
      &  &  & \ddots  & \\
      &  & &   & B_r^\prime \\
    \end{bmatrix}
  \end{align*}

  Since $B$ and $B^\prime$ are diagonalizable, so is each
  $B_i^\prime$. Taking matrices $T_1 , T_2 , \ldots , T_r$ that
  diagonalize $B^\prime_1 ,  B^\prime_2 , \ldots , B^\prime_r$
  respectively, let
  \begin{align*}
    T =
    \begin{bmatrix}%{c c c c}
      T_1&  &  &  \\
      & T_2 &  &  \\
      &  & \ddots &  \\
      &  &  & T_4 \\
    \end{bmatrix}
  \end{align*}
  Then,
  \begin{align*}
    T^{-1} B T =
    \begin{bmatrix}
      T_1^{-1}B_1^\prime T_1 &  &  & \\
      & T_2^{-1}B_2^\prime T_2 &   & \\
      &  &  & \ddots  & \\
      &  & &   & T_r^{-1} B_r^\prime T_r \\
    \end{bmatrix} =
    \begin{bmatrix}%{c c c c}
      D_1^\prime&  &  &  \\
      &  D_2^\prime&  &  \\
      &  &  \ddots&  \\
      &  &  & D_r^\prime \\
    \end{bmatrix}
  \end{align*} where each $D_i^\prime$ is a diagonal block.
  Also,
  \begin{align*}
    T^{-1} D_A T =
    \begin{bmatrix}
      T_1^{-1}\lambda_1 I T_1 &  &  & \\
      & T_2^{-1}\lambda_2 IT_2 &   & \\
      &  &  & \ddots  & \\
      &  & &   & T_r^{-1} \lambda_r I T_r \\
    \end{bmatrix} = D_A
  \end{align*}
  This implies $D_A = T^{-1} S^{-1} A S T$, and $D_B = T^{-1} S^{-1}
  B ST$ are both diagonal.

  \textcolor{red}{Converse is left as an exercise}
\end{proof}

Next, we consider simultaneous diagonalization for a family of matrices.
\begin{definition}
  A family $F \subset M_n$ is a commuting family if for each $A, B
  \in F$, $AB = BA$.
\end{definition}

\begin{definition}
  A subspace $W \subset \mathbb{C}^n$ is called an $A$-invariant
  subspace for some $A \in M_n$ if $Aw \in W$ for all $w \in W$. If
  $F \subset M_n$, then $W$ is called $F$-invariant if for each $A
  \in F$, $W$ is $A$-invariant.
\end{definition}

\begin{lemma}
  If $W \subset \mathbb{C}^n$ is $A$-invariant for some $A \in M_n$,
  and suppose that $\textrm{dim}(W) \ge 1$, then there is an $x \in W
  \setminus \{  \textbf{0} \}$ such that $Ax = \lambda x$.
\end{lemma}
\begin{proof}
  We consider $B:= A|_W$. Then $B: W \to W$ has an eigenvector since
  it has atleast one eigenvalue by the fundamental theorem of algebra.
\end{proof}

\begin{lemma}
  If $F \subset M_n$ is a commuting family, then there exists an $x
  \in \mathbb{C}^n$ such that for each $A \in F$, $Ax = \lambda_A x$.
\end{lemma}
\begin{proof}
  Choose $W$ to be an $F$-invariant subspace of minimum, non-zero
  dimension. Existence of $W$ is guaranteed since we can choose $W =
  \mathbb{C}^n$.

  Next, we show that any $ x \in W \setminus \{ \textbf{0} \}$ is an
  eigenvector for each $ A \in \mathbb{F}$. Assume this is not true.
  Then there is  a $ y\in W \setminus \{ \textbf{0} \}$, and an $ A
  \in F$, such that $Ay \not\in \mathbb{C}y$. Since $W$ is
  $A$-invariant by the setup, by previous lemma, we get that there is
  a $x \in W \setminus \{ \textbf{0} \}$ such that $Ax = \lambda_x x$
  for some $\lambda \in \mathbb{C}$.

  Let $ W_0 = \{ z \in W \ : \ Az = \lambda z \}$. By $y \notin W_0$,
  we get that $W_0 \neq W$. But for any $B \in F$, by invariance of
  $W_0$, $Bx \in W$, and for $u \in W_0$,
  \begin{align*}
    A(Bu) = B(Au) =  \lambda Bu
  \end{align*}

  We observe $Bu \in W_0$, thus $B$ maps $W_0$ to $W_0$, so $ W_0$ is
  $ F$-invariant. We have derived a contradiction with the minimality of $W$.
\end{proof}

\begin{remark}
  This implies that commuting families have at least one common eigenvector
\end{remark}

\begin{definition}
  A simultaneously diagonalizable family is a family $F \subset M_n$
  such that there exists $S \in M_n$ for which $S^{-1} A S$ is
  diagonal for each $A \in F$
\end{definition}



    \printbibliography[heading=bibintoc]

\end{document}

