% TeX_root = ../main.tex

\marginnote{\scriptsize 11/09/2025 }

\subsection*{Warm up}

\begin{enumerate}[label=(\arabic*)]
  \item Let $A \in M_n(\mathbb{C})$. For each $  x \in \mathbb{C}^n$,
    $ \langle Ax , x \rangle  = 0$ if and only if $A = 0$.
  \item Let $A \in M_n(\mathbb{R})$ and $A = A^*$, then for each $x
    \in \mathbb{R}^n$, $\langle Ax , x \rangle  = 0$ if and only if $A = 0$
\end{enumerate}

\begin{remark}
  If $A$ is Hermitian and unitary, then $AA^* = A^2 = I$, so each
  eigenvalue $ \lambda = \pm 1$.
\end{remark}

\begin{example}
  For $y \in \mathbb{C}^n\setminus \{ 0 \}$, $ H = I - 2
  \frac{y^*y}{\|y\|^2}$ is called a Householder transform
\end{example}
\begin{exercise}
  Check that this is Hermitian and satisfies $H^2 = I$.
\end{exercise}

\section{Shur Triangulizatoin and consequences of unitary equivalence}

\begin{definition}
  We say that $A, B \in M_n(\mathbb{C})$ are unitarily equivalent if
  there is $U \in M_n$, $U^*U = I$ such that $B = U^*AU$.
\end{definition}

\begin{remark}
  because of $U^{-1} = U^*$, if $A$ and $B$ are unitarily equivalent,
  then there are similar. The converse is not true in general.
\end{remark}

\begin{question}
  How can we tell if $A, B$ are unitarily equivalent?
\end{question}

We first look for necessary conditions.
\begin{proposition}
  If $A, B$ are unitarily equivalent, then
  \begin{align*}
    \sum_{i, j = 1}^{n} |a_{i, j}|^2 = \sum_{i, j = 1}^{n} |b_{i,j}|^2
  \end{align*}
\end{proposition}
\begin{proof}
  We observe that
  \begin{align*}
    \sum_{i, j = 1}^{n} |a_{i, j}|^2 = \sum_{i = 1}^{n} [AA^*]_{i, i}
    = \textrm{tr}(AA^*)
  \end{align*}
  The same is true for $B=U^*AU$. So,
  \begin{align*}
    \sum_{i, j = 1}^{n} |b_{i, j}|^2 &= \textrm{tr}(BB^*) \\
    &= \textrm{tr}(U^*AU (U^*AU)^*) \\
    &= \textrm{tr}(U^*AA^*U) \\
    &= \textrm{tr}(AA^*)
  \end{align*}
\end{proof}

We compare with a result that characterizes unitary equivalence

\begin{theorem}[Carl Pearcy, 1962]
  If all words $w_1$ on the alphabet $\{ A, A^* \}$, and $w_2$ on the
  alphabet $\{ B, B^* \}$ of length upto $2n^2$ have
  \begin{align*}
    \textrm{tr}(w_1) = \textrm{tr}(w_2)
  \end{align*}
  then $A, B$ are unitarily equivalent.
\end{theorem}

\begin{theorem}[Shur's Triangulization Theorem]
  Let $A in M_n$ having eigenvalues $\lambda_1 , \lambda_2 , \ldots ,
  \lambda_n$, then thre is a unitary $U \in M_n$ such that $A =
  UTU^*$, where $T$ is an upper triangular matrix with $\lambda_1 ,
  \lambda_2 , \ldots , \lambda_n$ in the diagonal.
\end{theorem}

\begin{remark}
  Remember that $T$ is in genearl not unique. For example,
  \begin{align*}
    A =
    \begin{bmatrix}%{c c}
      1 & 1\\
      0 & 1
    \end{bmatrix} \quad \textrm{and} \quad
    B =
    \begin{bmatrix}%{c c}
      1 & -1\\
      0 & 1
    \end{bmatrix}
  \end{align*}
\end{remark}

\begin{proof}[Proof of Shur's Triangulization Theorem]
  We'll use induction on $n \in \mathbb{N}$. Assume this is true for
  matrices $M_{n-1}$. Given $A \in M_n$ with eigenvalues $\lambda_1 ,
  \lambda_2 , \ldots , \lambda_n$. Choose an eigenvector $x$
  corresponding to an eigenvalue $\lambda_1$, and take $\|x\| = 1$.

  Applying the Gram-Schmidt process, we complement $\{x\}$ to an
  orthonormal basis $\{ x, z_2 , z_3 , \ldots , z_n \}$, Let $U_1 = [x,
  z_1 , z_2 , \ldots , z_n]$. Then $U_1$ is unitary and $Ax = \lambda_1 x$, and
  \begin{align*}
    U_1^*AU_1 &= U_1^*[\lambda_1x , *] \\
    &=
    \begin{bmatrix}%{ c c}
      \lambda_1 & * \\
      0 & B
    \end{bmatrix}
  \end{align*}

  Now $B \in M_{n-1}$ and it follows from induction argument.
\end{proof}

\begin{definition}
  A matrix $A \in M_n$ is called normal if $A^*A = A^*A$.
\end{definition}
\begin{example}
  Hermitian, skew-Hermitian, and Unitary matrices are normal.
\end{example}

\begin{proposition}
  A matrix $A \in M_n$ is normal if and only if for each $x \in
  \mathbb{C}^n$, $\|Ax\| = \|A^*x\|$.
\end{proposition}
\begin{proof}
  Normality is $AA^* - A^*A = 0$. Thus for all $x \in \mathbb{C}^n$,
\end{proof}
