% TeX_root = ../main.tex

\marginnote{\scriptsize 28/08/2025 }

\begin{definition}
  If $A \in M_n$, $A = (a_{i, j})_{i, j = 1}^n$, we let
  \begin{align*}
    \rm{trace}(A) = \sum_{j = 1}^{n} a_{jj}
  \end{align*}
  The determinant is
  \begin{align*}
    \textrm{det}(A) = \sum_{ \sigma \in  S_n} \textrm{sgn}(\sigma)
    a_{1, \sigma(1)} a_{2, \sigma(2)} \ldots a_{n, \sigma(n)}
  \end{align*}
\end{definition}

\begin{remark}
  If $A = [a_1  a_2  \ldots  a_n]$, then $det(A) = f( a_1 , a_2 ,
  \ldots , a_n)$ is the only function that is linear in each $a_i$,
  alternating (swapping columns doesn't alter the value), and
  normalized ($det(I) = 1$).

  This is useful to show that for $A, B \in M_n$, $det(AB) = det(A) det(B)$.

  Moreover if $A =
  \begin{bmatrix}
    B & C \\
    0 & D
  \end{bmatrix}$, then $det(A) = det(B) det(D)$.

  We also have
  \begin{align*}
    det(A) = \sum_{i, j = 1}^{n} (-1)^{i + j} a_{i, j} det(A_{i, j})
  \end{align*}
  Where $A_{i, j}$ is the submatrix with $i$th row and $j$th column
  removed from $A$.
\end{remark}

\subsection{Eigenvalues and Eigenvectors}

\begin{definition}
  Eigenvalue, Eigenvector, Spectrum of a matrix
\end{definition}

\subsection{Similarity}

\begin{definition}
  A matrix $B \in M_n$ is similar to $A \in M_n$, if there is an
  invertible $S \in M_n$ such that $B = S^{-1}AS$. This defines an
  equivalence relation.
\end{definition}

\begin{theorem}
  If $A, B \in M_n$ are similar. Then their characteristic polynomial
  $P_A = P_B$.
\end{theorem}

\begin{remark}
  characteristic polynomial i snot characteristic upto similarity, because
  \begin{align*}
    A =
    \begin{pmatrix}%{c c}
      0 & 1 \\
      0 & 0
    \end{pmatrix} \text{ and } B =
    \begin{pmatrix}%{c c}
      0 & 0\\
      0 & 0
    \end{pmatrix}
  \end{align*}
\end{remark}
