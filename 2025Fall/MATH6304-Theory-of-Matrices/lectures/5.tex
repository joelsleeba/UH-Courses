% TeX_root = ../main.tex

\marginnote{\scriptsize 09/09/2025 }

\begin{theorem}
  Let $F \subset M_n$ be a family of diagonalizable matrices, then
  $F$ is a commuting family if and only if it is simultaneously diagonalizable.
\end{theorem}
\begin{proof}
  It is an easy exercise to show that a simultaneously diagonalizable
  family is commuting.

  We prove the converse by induction over $n$. For $n = 1$, there's
  nothing to prove. Assume that this is true for all $F^\prime
  \subset M_k$, where $k < n$. If each $A \subset F$ is of the form
  $A = \lambda I$, again nothing to prove. Thus, assume $A$ is
  diagonalizable with eigenvalues $\lambda_1 , \lambda_2 , \ldots ,
  \lambda_r, r \ge 2$, and assume $AB = BA$, for each $B \in F$.
  Then $A, B$ are simultaneously diagonalizable by previous theorem
  and hence without loss of generality, assume that
  $A$ is diagonal.

  \textcolor{red}{verify rest from the lecture notes}
\end{proof}

\begin{remark}
  Given $C \in M_n(\mathbb{C})$, we can think of function associated
  with $C$, as $ \langle x , y \rangle   \to \langle Cx, y\rangle$.
  Notice that this function preserves all information about $C$,
  since we can find the individual matrix entries.
  We can also associate $Q_c : x \to \langle Cx , x \rangle$. We
  recall $Q_c$ determines $C$ as
  \begin{align*}
    \langle Cx, y \rangle = \frac{1}{4}\sum_{j = 1}^{4} i^j\|Q_c(x + i^j
    y)\|^2, \quad ( i = \sqrt{-1})
  \end{align*}
\end{remark}

\subsection{Hermitian, normal, and unitary matrices}
\begin{definition}
  Let $A \in M_{n, m}$, then the adjoint $A^* \in M_{m, n}$ satisfies
  $\langle  Ax , y \rangle  = \langle x , A^*y \rangle $ for each $x
  \in \mathbb{C}^m, y \in \mathbb{C}^n$.

  If for $A \in M_n$, $A = A^*$, then we say that $A$ is Hermitian or
  self-adjoint. If $A = - A^*$, the it is called skew-hermitian.

  If $A \in M_n$, then $A = B+ iC$, where $B = \frac{1}{2} (A + A^*),
  C = \frac{1}{2i} (A - A^*)$ have the property $B = B^*, C = C^*$.
  Here $ B$ is called the real part of $A$, and $iC$ is called the
  imaginary part of $A$.
\end{definition}

\begin{proposition}
  \begin{align*}
    A = A^* \iff (iA) = -(iA)^*
  \end{align*}
\end{proposition}

\begin{proposition}
  A matrix $A \in M_n$ is Hermitian if and only if for all $x \in
  \mathbb{C}^n$, $ \langle Ax , x \rangle  \in \mathbb{R}$.
\end{proposition}
\begin{proof}
  If $A$ is Hermitian, then
  \begin{align*}
    \overline{\langle Ax , x \rangle } & = \langle x , Ax \rangle =
    \langle A^*x , x \rangle  = \langle Ax , x \rangle
  \end{align*}
  shows that $\langle  Ax , x \rangle  \in \mathbb{R}$.

  Conversely, assume that $\langle  Ax , x \rangle  \in \mathbb{R}$
  for all $x \in \mathbb{C}^n$. Let $ A = B + iC$, where $B= B^*, C =
  C^*$. Then
  \begin{align*}
    \langle Ax , x \rangle  &= \underbrace{\langle Bx , x
    \rangle}_{\in \mathbb{R}}  + i \underbrace{\langle Cx , x
    \rangle}_{\in \mathbb{R}}  \\
  \end{align*}
  We conclude that $\langle  Cx , x \rangle  = 0$ for all $x$. Now
  using polarization identity, we get that $C = 0$.
\end{proof}

We also consider and equivalent of unimodular numbers.

\begin{definition}
  Let $A \in M_n$. $A$ is unitary if $A^*A = I = AA^*$.
\end{definition}

\begin{proposition}
  Let $A \in M_n$. The following are equivalent.
  \begin{enumerate}[label=(\arabic*)]
    \item $A$ is unitary.
    \item The columns of $A$ form an orthonormal basis.
    \item Rows of $A$ form an orthonormal basis.
    \item A preserves the norm for each $x \in \mathbb{C}^n$. That is
      $\|Ax\| = \|x\|$.
    \item $A$ preserve the inner product. That is $\langle  Ax , Ay
      \rangle = \langle x , y \rangle$
  \end{enumerate}
\end{proposition}
\begin{proof}
  You know,
\end{proof}
