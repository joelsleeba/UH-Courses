% TeX_root = ../main.tex

\marginnote{\scriptsize 28/08/2025 }

See Burgers equations and diffusion equation from the notes.
Reasons for boundary condition and initial condition.

\section{Boundary Contitions}
\begin{definition}
  Dirichlet boundary condition is when the prescribed value of
  unknown $u$ at the boundary $\partial \Omega$. For example when $u$
  is the temperature of the metal rod, and the temperature along the
  boundary is known.
\end{definition}

\begin{definition}
  Neumann boundary condition is when $\nabla \cdot \vec{n} = g(x)$,
  where $x \in \partial \Omega$. When $ g(x) = 0$, we say it is
  \textbf{homogeneus Neumann boundary condition}.
\end{definition}

\begin{definition}
  Robin boundary condition is when $u + \nabla \cdot \vec{n} = g(x)$,
  where $x \in \partial \Omega$.
\end{definition}

\begin{definition}
  No flux boundary condition is when $\vec{J}\cdot\vec{n} = 0$, where
  $ x \in \partial \Omega$.
\end{definition}

\section{Steady States}

A steady state solution is a solution ot
\begin{align*}
  \partial_t u + \nabla\cdot J = Q
\end{align*}
that does not depend on time. So it solves $\nabla \cdot J = Q$.
If we have $u$ to be the concentration of chemical species, then $J =
- D \nabla u$. So $\nabla \cdot J = Q \implies D \Delta u = Q$ if $D$
is a constant.

\section{Euler's Equation}
Let $\vec{u} = \vec{u}(x, t)$ be the velocity of a fluid for $(x, t)
\in \Omega \subset \mathbb{R}^3 \times [0, T]$, $T > 0$.

Using conservation of mass $R \subset \Omega \subset \mathbb{R}^3$,
where $\rho = \rho(x, t)$ is the density of the fluid.
\begin{align*}
  \frac{d}{dt} \int_R \rho \ d \nu = - \int_{\partial R}
  \vec{J}\cdot\vec{n} \ d A = - \int_{\partial R}  \rho \vec{u} \cdot
  \vec{n} \ d A
\end{align*}
By divergence theorem, we have
\begin{align*}
  \frac{\partial \rho}{\partial t} + \nabla(\rho \vec{u}) = 0
\end{align*}
Assuming the fluid is incompressible, we have $\rho$ is a constant, hence
\begin{align*}
  \nabla \cdot \vec{u} = 0
\end{align*}
