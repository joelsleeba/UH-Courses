% TeX_root = ../main.tex

\chapter{Derivation of PDEs}

\marginnote{\scriptsize 26/08/2025 }
\section{Conservation Laws}
These are balanced laws that express the fact that some quantity is
balanced throughout a process.

Let $u(x, t)$ be the density of some quantity (mass, heat, energy,
bacteria). Given a region $R \subset \mathbb{R}^n$, then
\begin{align*}
  \int_R u(x, t) \ d  x
\end{align*}
represents the total amount of quantity inside $R$ at time $t$.

\begin{definition}
  The quantity (represented by $u$) is then locally conserved if it
  is only gained or lost either through domain boundaries or because
  of sources or sinks in the domain.
\end{definition}

\begin{definition}
  The \textbf{flux} of $u(x, t)$ is the direction and rate of flow of
  this quantity per unit area. It has a magnitude and a direction and
  is therefore a vector field $\vec{J}$.
\end{definition}

\begin{remark}
  Flux is defined so that $\vec{J} \cdot \vec{n}dA$ represent the
  amount of quantity $u$ flowing across a small area $dA$ per unit time.
\end{remark}

\begin{remark}
  $\vec{J}$ has the unit of density times velocity.
\end{remark}

Track $u$, area, time on same domain $\Omega$, and assume we have a
source term $Q(x, t) > 0$. Conservation of $u$ on a region $R \subset
\Omega$ then implies
\begin{align*}
  \frac{d}{dt}\int_R u(x, t) \ d  t = \int_R \frac{\partial
  u}{\partial t} \ d x = - \int_{\partial R} \vec{J} \cdot \vec{n}
  \ d x + \int_R Q(x, t) \ dx
\end{align*}
Using divergence theorem, we can rewrite this as
\begin{align*}
  \int_R \frac{\partial u}{\partial t} \ d x = - \int_{R} \nabla
  \cdot \vec{J} \ d x + \int_R Q(x, t) \ dx
\end{align*}
which gives
\begin{align*}
  \int_R \frac{\partial u}{\partial t} + \nabla \cdot \vec{J}  - Q(x,
  t) \ dx = 0
\end{align*}
Since $R$ can be chosen arbitrarily, and the variables $u, J, Q$ are
"nice enough" (sufficiently smooth) this forces
\begin{align}
  \label{eq:continuity_eq}
  \frac{\partial u}{\partial t} + \nabla \cdot \vec{J} - Q = 0
\end{align}
for all points in space and time.

$\autoref{eq:continuity_eq}$ called the continuity equation is the
standard form of all conservation equations and depending on the
context different $\vec{J}$ will play different roles.

\section{Review of Vector Calculus}

\begin{theorem}[Gauss-Green Theorem]
  Suppose $u \in C^1(\overline{\Omega})$, and let $\vec{x} \in \Omega
  \subset \mathbb{R}^m$ with $u_{x_i}: = \frac{\partial u}{\partial x_i}$, then
  \begin{align*}
    \int_{\Omega}  u_{x_i} \ d x = \int_{\partial \Omega}   u n_i \ d x
  \end{align*}
\end{theorem}

\begin{theorem}[Divergence Theorem]
  \begin{align*}
    \int_\Omega \nabla \cdot \vec{J} \ d  x = \int_{\partial \Omega}
    \vec{J} \cdot \vec{n} \ d x
  \end{align*}
\end{theorem}

\begin{theorem}[Green's First Identity]
  \begin{align*}
    \int_{\Omega}(u \Delta v + \nabla u \cdot \nabla v) \ d x =
    \int_{\partial \Omega}   u \frac{\partial v}{\partial \vec{n}} \ dS
  \end{align*}
\end{theorem}

\begin{theorem}[Green's 2nd identity]
  \begin{align*}
    \int_{\Omega}(u \Delta v - v \Delta u) \ d x = \int_{\Omega}   u
    \frac{\partial v}{\partial \vec{n}} - v \frac{\partial
    u}{\partial \vec{n}} \ dS
  \end{align*}
\end{theorem}

\begin{remark}

\end{remark}

\begin{theorem}
  Suppose $f : \Omega \times [a, b] \to  \mathbb{C}$ with $a < b \in
  \mathbb{R}$ and $f(\cdot, t): \Omega \to \mathbb{C}$ is integrable
  for each $t \in [a, b]$. Let $F(t) = \int_\Omega  f(x, t) \ d
  \mu(x)$, where $\mu$ denote a measure for $\Omega$. Suppose
  $\frac{\partial f}{\partial t}$ exists and $g \in L^{1}(\mu)$ such
  that $\Big|\frac{\partial f}{\partial t} (x, t)\Big| < g(x)$, then
  $F$ is differentiable and
  \begin{align*}
    F^\prime(t) = \int_\Omega  \frac{\partial f}{\partial t}(x, t) \ d \mu(x)
  \end{align*}
\end{theorem}

\section{Examples}

\subsection{Advection (Convection, Drift)}

Advection or drift where a bulk of the quantity is carried along the
medium in a given velocity.

Suppose $u(x, t)$ is the density of some quantity that is being
transported with velocity $\vec{v}$. Then
\begin{align*}
  \vec{J} = \vec{v} u
\end{align*}
\begin{example}[Simple transport]
  Assume $\vec{v}$ is a constant. Then  \autoref{eq:continuity_eq} becomes
  \begin{align*}
    \frac{\partial u}{\partial t} + \vec{v} \ \nabla u  - Q = 0
  \end{align*}
  In case $Q = 0$ and $ \Omega \subset \mathbb{R}$, then we get
  \begin{align*}
    \frac{\partial u}{\partial t} + \vec{v} \ \frac{\partial u}{\partial t} = 0
  \end{align*}
\end{example}
