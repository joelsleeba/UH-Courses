% initial settings
\documentclass[12pt]{exam}
\usepackage{geometry}
\usepackage{graphicx}
\usepackage{enumitem}
\usepackage[usenames,dvipsnames]{xcolor}
\usepackage[backend=biber, style=alphabetic]{biblatex}
\usepackage{url,hyperref}

\usepackage{amsmath} % math symbols, matrices, cases, trig functions,
% var-greek symbols.
\usepackage{amsfonts} % mathbb, mathfrak, large sum and product symbols.
\usepackage{amssymb} % extended list of math symbols from AMS.
% https://ctan.math.washington.edu/tex-archive/fonts/amsfonts/doc/amssymb.pdf
\usepackage{amsthm} % theorem styling.
\usepackage{mathrsfs} % mathscr fonts.
\usepackage{yhmath} % widehat.
\usepackage{empheq} % emphasize equations, extending 'amsmath' and 'mathtools'.
\usepackage{bm} % simplified bold math. Do \bm{math-equations-here}

% geometry of paper
\geometry{
  a4paper, % 'a4paper', 'c5paper', 'letterpaper', 'legalpaper'
  asymmetric, % don't swap margins in left and right pages. as
  % opposed to 'twoside'
  centering, % to center the content between margins
  bindingoffset=0cm,
}

% hyprlink settings
\hypersetup{
  colorlinks = true,
  linkcolor = {red!60!black},
  anchorcolor = red,
  citecolor = {green!50!black},
  urlcolor = magenta,
}

% theorem styles
\theoremstyle{plain} % default; italic text, extra space above and below
\newtheorem{theorem}{Theorem}[section]
\newtheorem{proposition}{Proposition}[section]
\newtheorem{lemma}{Lemma}[section]
\newtheorem{corollary}{Corollary}[theorem]

\theoremstyle{definition} % upright text, extra space above and below
\newtheorem{definition}{Definition}[section]
\newtheorem{example}{Example}[section]

\theoremstyle{remark} % upright text, no extra space above or below
\newtheorem{remark}{Remark}[section]
\newtheorem*{note}{Note} %'Notes' in italics and without counter

% renewcommands for counters
\newcommand{\propositionautorefname}{Proposition}
\newcommand{\definitionautorefname}{Definition}
\newcommand{\lemmaautorefname}{Lemma}
\newcommand{\remarkautorefname}{Remark}
\newcommand{\exampleautorefname}{Example}

\addbibresource{~/Books/Research/research.bib}

\begin{document}

\title{MATH 6303 - Modern Algebra \\ Homework  2}

% author list
\author{
  Joel Sleeba \\
}

\maketitle
\printanswers
\unframedsolutions

\begin{questions}

  \question
  \begin{solution}
    We notice that the roots of $x^4 - 2$ are $\sqrt[4]{2},
    -\sqrt[4]{2}, i \sqrt[4]{2}, -i \sqrt[4]{2}$. Thus the splitting
    field of $x^4 - 2$ is a subfield of $\mathbb{Q}(\sqrt[4]{2}, i)$.
    Moreover the splitting field must contain $\sqrt[4]{2}, i = (i
    \sqrt[4]{2})(\sqrt[4]{2})^{-1}$. Thus we see that the splitting
    field is precisely $\mathbb{Q}(\sqrt[4]{2}, i)$.

    Now to find the degree of the splitting field, we observe that
    $x^4 - 2$ is irreducible by the Eisenstein criteria. Hence
    \begin{align*}
      [\mathbb{Q}(\sqrt[4]{2}, i): \mathbb{Q}] =
      [\mathbb{Q}(\sqrt[4]{2}, i) : \mathbb{Q}(\sqrt[4]{2})]
      [\mathbb{Q}(\sqrt[4]{2}): \mathbb{Q}] = 2 \times 4 = 8
    \end{align*}
  \end{solution}

  \question
  \begin{solution}
    We notice that the roots of $x^4 + 2$ are $\sqrt[4]{2}\omega_8^1
    , \sqrt[4]{2}\omega_8^3 , \sqrt[4]{2}\omega_8^5 ,
    \sqrt[4]{2}\omega_8^7$, where $\omega_n$ is a primitive nth root
    of unity.
    Clearly the splitting field must contain $\sqrt[4]{2} \omega_8^1$ and
    $\omega_8^2 = \omega_4$, since $\omega_4 = \omega_8^2 =
    (\sqrt[4]{2}\omega_8^1)^{-1}
    \sqrt[4]{2} \omega_8^3$.
    Moreover any field which contain $\sqrt[4]{2}\omega_8, \omega_4$ will
    contain all the other roots. Hence we see that the splitting
    field of $x^4 - 2$ is
    $\mathbb{Q}(\sqrt[4]{2}\omega_8, \omega_4)$.

    Without loss of generality, assume that $\omega_8 =
    \frac{1+i}{\sqrt{2}}$, and $\omega_4 = i$. As $\omega_4 =
    \omega_8^2$, clearly $\mathbb{Q}(\sqrt[4]{2}\omega_8, \omega_4) \subset
    \mathbb{Q}(\sqrt[4]{2}, \omega_8)$. Since
    \begin{align*}
      \omega_4 = i = (1+i) - 1 = (\sqrt[4]{2})^2
      \frac{(1+i)}{\sqrt{2}} - 1 = \sqrt[4]{2}^2 \omega_8 - 1
    \end{align*}
    we get that
    \begin{align*}
      \sqrt[4]{2} = \frac{\omega_4+1}{\sqrt[4]{2}\omega_8} \ \in \
      \mathbb{Q}(\sqrt[4]{2}\omega_8, \omega_4)
    \end{align*}
    and
    \begin{align}
      \label{eq:2}
      \omega_8 = \frac{(1+i)}{\sqrt{2}} =
      \frac{1+\omega_4}{\sqrt[4]{2}^2} \ \in
      \ \mathbb{Q}(\sqrt[4]{2}\omega_8, \omega_4)
    \end{align}
    Hence, we see that $\mathbb{Q}(\sqrt[4]{2}\omega_8, \omega_4) =
    \mathbb{Q}(\sqrt[4]{2}, \omega_8)$ is the splitting field of $x^4 - 2$

    Again, clearly $\mathbb{Q}(\sqrt[4]{2}, \omega_4) \subset
    \mathbb{Q}(\sqrt[4]{2}, \omega_8)$ as $\omega_8^2 = \omega_4$.
    But \autoref{eq:2} gives the converse and hence
    $\mathbb{Q}(\sqrt[4]{2}, \omega_4) =
    \mathbb{Q}(\sqrt[4]{2}, \omega_8)$. Thus, the splitting field of
    $x^4 + 2$ is again $\mathbb{Q}(\sqrt[4]{2}, i)$, and from the previous
    question we see that the degree of the extension of again $8$.
  \end{solution}

  \question
  \begin{solution}
    Since we are well aware of the roots of the polynomial $x^2 + x +
    1$ to be $\omega, \omega^2$, where $\omega = e^{i
    \frac{2\pi}{3}}$. We see that the roots of the polynomial $x^4 +
    x^2 + 1$ are $\pm \omega, \pm \omega^2$, where $\omega = e^{i
    \frac{\pi}{3}}$. Thus we see that the splitting field of the
    polynomial $x^4 + x^2 + 1$ is $\mathbb{Q}(e^{i \frac{\pi}{3}})$.

    Now since the degree of the extension is the same as
    the degree of the minimal polynomial in $\mathbb{Q}[x]$ for $e^{i
    \frac{\pi}{3}}$, we look for the minimal polynomial of $e^{i
    \frac{\pi}{3}}$. We know that to be $x^2 - x + 1$. Hence we see
    that the splitting field of $x^4 + x^2 + 1$ is of degree $2$ over
    $\mathbb{Q}$.
  \end{solution}

  \question
  \begin{solution}
    We notice that the roots of $x^6 - 4$ are $\pm \sqrt[3]{2}, \pm
    \sqrt[3]{2} \omega, \pm \sqrt[3]{2} \omega^2$, where $\omega =
    e^{i \frac{2\pi}{3}}$. Thus the splitting field of $x^6 - 4$ is
    $\mathbb{Q}(\sqrt[3]{2}, \omega)$.

    Now to find the degree of the splitting field, we observe that
    $[\mathbb{Q}(\sqrt[3]{2}): \mathbb{Q}] = 3$ as $x^3 - 2$ is
    irreducible in $\mathbb{Q}$. Moreover $x^2 + x +1$ is irreducible
    in $\mathbb{Q}(\sqrt[3]{2})$ as the polynomial only have complex
    roots, $\omega, \omega^2$. Hence we get that
    $[\mathbb{Q}(\sqrt[3]{2}, \omega): \mathbb{Q}(\sqrt[3]{2})] = 2$.
    Now using the tower law, we get the degree of the splitting
    field to be $6$.
  \end{solution}

  \question
  \begin{solution}
    Let $x \in \mathbb{F}_{p^s}$ be an $n$-th root of unity
    where $n = p^km$ with $\gcd(p, m) = 1$. Then
    \begin{align*}
      x^{p^k m} = (x^m)^{p^k} = 1
    \end{align*}
    Since we'll show that $x \to x^p$ is a Field isomorphism in the
    next question, we see that this implies $x^m = 1$. Thus every
    $n$-th root of unity is an $m$-th root of unity. Converse is easy
    to see as if $x^m = 1$, then $x^n = (x^m)^{p^k} = 1$. Thus $n$-th
    roots of unity in $\mathbb{F}_{p^s}$ are precisely the $m$-th
    roots of unity. Thus every $n$-th root of unity are precisely the
    roots of the polynomial $f(x) = x^m - 1$. As $\gcd(m, p) = 1$,
    the only root of $D_f(x) = mx^{m-1}$ is $0$, and $0$ is not a
    root of $f$, we see that $f$ is separable. Hence $f$ has $m$
    distinct roots, which gives a proof for the statement.

  \end{solution}

  \question
  \begin{solution}
    Since we have shown in class that $(a + b)^p = a^p + b^p$ for
    fields of characteristic $p$, and $(ab)^p = a^pb^p$ by the
    commutativity of the ring operation, we see that the map $\phi: x
    \mapsto x^p$ is a field endomorphism on $\mathbb{F}_{p^n}$. Again
    since $ \mathbb{F}_{p^n}$ is an integral domain, $x^p = 0$ forces $x
    = 0$. Hence the map is an injective endomorphism.
    Since an injective endomorphism between finite spaces are
    bijective, $x \mapsto x^p$ is a field automorphism.

    Now let $\phi^m : \mathbb{F}_{p^n} \to \mathbb{F}_{p^n} := x
    \mapsto x^{p^m}$ for some $m \in \mathbb{N}$. We'll show that
    $\phi^m$ is the identity map if and only if $n|m$. Since we
    know that the multiplicative group of $\mathbb{F}_{p^n}$ has $p^n
    - 1$ elements, from group theory, we get that $x^{p^n - 1} = 1$
    for all $x \in \mathbb{F}_{p^n}^*$. Thus $x^{p^n} = x$ for all $ x
    \in \mathbb{F}_{p^n}$. Thus, $\phi^n$ is the identity map, and
    $\phi^{kn}$ is an identity map for all $k \in \mathbb{N}$.

    Since we know that the mulitiplicative group
    of a finite field is cyclic, let $F_{p^n}^* = \langle x_0
    \rangle$. Now, if for some $m \in \mathbb{N}$, $x^{p^m} = x$
    for all $x \in \mathbb{F}_{p^n}$, then this would force $x_0^{p^m
    - 1} = 1$, while $|x_0| = p^n - 1$. Thus $(p^n - 1)|(p^m - 1)$,
    which happens only if $n|m$
  \end{solution}

\end{questions}
\printbibliography[heading=bibintoc]
\end{document}
