% TeX_root = ../main.tex

\marginnote{\scriptsize 30/01/2025 }

\begin{theorem}
  Let $X$ be a compact Hausdorff space and $\mathcal{H}$ be a
  Hilbert space, and $\Psi: C(X) \to B(\mathcal{H})$ is a
  $*$-homomorphism (representation). Then there exist a unique
  spectral measure on $(X, \mathcal{B}, \mathcal{H})$ ($\mathcal{B}$
  being the Borel sigma algebra on $X$), such that
  \begin{align*}
    \Psi(f) = \int_X  f \ d E
  \end{align*}
\end{theorem}
\begin{proof}
  For every $\xi, \eta \in \mathcal{H}$, the map $ \nu : C(X) \to
  \mathbb{C}  :=  f \mapsto \langle  \Psi(f) \xi ,  \eta \rangle $ is
  a bounded linear functional on $C(X)$, hence by the reisz
  representation theorem, there exists a unique measure $\mu_{\xi,
  \eta}$ on $X$ such that
  \begin{align*}
    \langle \Psi(f) \xi ,  \eta \rangle  = \int f \ d \mu_{\xi,
    \eta} \quad f \in C(X)
  \end{align*}
  Now for any $\xi, \eta \in \mathcal{H}$, the linear map
  \begin{align*}
    \mathcal{M}(X) \to  \mathbb{C} := \phi
    \mapsto \int \phi \ d \mu_{\xi, \eta}
  \end{align*}
  is a bounded sesquilinear form about $\xi, \eta$ with the operator
  norm $\|\xi\|\|\eta\|$, which is attained when $\phi = \chi_X$. Hence
  there exist a unique $T_\phi \in B(\mathcal{H})$ such that
  \begin{align*}
    \langle T_\phi \xi ,  \eta \rangle = \int \phi \ d \mu_{\xi, \eta}
  \end{align*}

  Now we show that the map $\mathscr{F}: \mathcal{M}(X)\to B(\mathcal{H})  :=
  \phi \mapsto    T_\phi$ is a $*$-representation of
  $\mathcal{M}(X)$. Linearity is obvious, though
  \textcolor{red}{verify}. Argument for $T_{\overline{ \phi}} =
  T_{\phi}^*$ is as in the last theorem.
  Observe that for every $f \in C(X)$, $T_f = \Psi(f)$.
  \textcolor{red}{Take inner product with $\xi, \eta$}.
  \marginnote{ \scriptsize \it \textcolor{red}{verify the rest}}

  To show multiplicativity of the above map, let $f \in C(X), \phi
  \in \mathcal{M}(X)$. Then $\forall \xi, \eta \in
  \mathcal{H}$,
  \begin{align*}
    \marginnote{ \scriptsize \it \textcolor{red}{verify the 2nd step}}
    \langle T_{\phi f}(\xi), \eta \rangle  &= \int \phi f \ d \mu_{\xi, \eta} \\
    &= \int \phi \ d \mu_{\Psi(f) \xi, \eta} \\
    &= \langle T_\phi \Psi(f) \xi ,  \eta \rangle\\
    &= \langle  T_\phi T_f \xi ,  \eta \rangle
  \end{align*}
  shows that $\mathscr{F}$ is multiplicative if one of the functions
  are in $C(X)$.
  More generally, if $\phi, \psi \in \mathcal{M}(X)$, choose a net
  $f_i \in C(X)$  such that $f_i$ converges to $\phi$ weak * in
  $C(X)^{**} = \mathcal{M}(x)$, and $\|f_i\| \to \|\phi\|$
  (\textcolor{red}{This is
      guaranteed by Goldstein's
  theorem}) and $\|f_i\| \le \|\phi\|$. Now for any $\xi, \eta \in
  \mathcal{H}$, we have
  \begin{align*}
    \langle T_{\phi, \psi} \xi ,  \eta \rangle = \int \phi \psi \ d
    \mu_{\xi, \eta} = \lim_{i} \int \phi f_i \ d \mu_{\xi, \eta} =
    \lim_i \langle T_{\phi f_i} \xi ,  \eta \rangle = \lim_i \langle
    T_\phi T_{f_i} \xi , \eta \rangle
  \end{align*}
  which converge to $\langle T_\phi T_\psi \xi ,  \eta \rangle$.
  \textcolor{red}{verify}. Thus $\mathscr{F}$ is a $*$-representation
  of $\mathcal{M}(X)$.

  Now define $E : \Sigma \to  B(\mathcal{H}) :=   A \mapsto
  T_{\chi_A}$. Obviously $E$ satisfy all the first three properties
  of a spectral measure easily.

  See also that the map $E$ is countably additive. Let $(A_n)$ be a
  family of pairwise disjoint measurable sets of $X$
  and let $A = \cup_{n = 1}^{\infty}A_n$. Then for every $\xi, \eta
  \in \mathcal{H}$,
  \begin{align*}
    \Big \langle E \big(\bigcup_{n \in \mathbb{N}} A_n \big) \xi ,
    \eta \Big \rangle &= \int  \chi_{\cup_{n = 1}^{\infty}A_n} \ d
    \mu_{ \xi, \eta} = \mu_{\xi, \eta}\Big( \bigcup_{n =
    1}^{\infty}A_n \Big) = \sum_{n \in \mathbb{N}} \mu_{\xi , \eta}(A_n)
  \end{align*}

\end{proof}
